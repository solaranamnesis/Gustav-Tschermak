\documentclass[a4paper, 11pt, oneside, polutonikogreek, german]{article}
\usepackage{gfsbaskerville}
% Load encoding definitions (after font package)
\usepackage[LGR,T1]{fontenc}
\usepackage{textalpha}

\usepackage{listings}
\lstset{basicstyle=\ttfamily}

% Babel package:
\usepackage{babel}

% With XeTeX$\$LuaTeX, load fontspec after babel to use Unicode
% fonts for Latin script and LGR for Greek:
\ifdefined\luatexversion \usepackage{fontspec}\fi
\ifdefined\XeTeXrevision \usepackage{fontspec}\fi

% "Lipsiakos" italic font `cbleipzig`:
\newcommand*{\lishape}{\fontencoding{LGR}\fontfamily{cmr}%
		       \fontshape{li}\selectfont}
\DeclareTextFontCommand{\textli}{\lishape}

\usepackage{float}
\usepackage{graphicx}
\graphicspath{ {./figures/} }
\usepackage[figurename=]{caption}
\usepackage{booktabs}
\setlength{\emergencystretch}{15pt}
\usepackage{fancyhdr}
\usepackage{microtype}

\begin{document}
\begin{titlepage} % Suppresses headers and footers on the title page
	\centering % Centre everything on the title page
	\scshape % Use small caps for all text on the title page

	%------------------------------------------------
	%	Title
	%------------------------------------------------
	
	\rule{\textwidth}{1.6pt}\vspace*{-\baselineskip}\vspace*{2pt} % Thick horizontal rule
	\rule{\textwidth}{0.4pt} % Thin horizontal rule
	
	\vspace{0.75\baselineskip} % Whitespace above the title

        {\LARGE Die mikroskopische Beschaffenheit der Meteoriten erläutert durch photographische Abbildungen} % Title
	
	\vspace{0.75\baselineskip} % Whitespace below the title
	
	\rule{\textwidth}{0.4pt}\vspace*{-\baselineskip}\vspace{3.2pt} % Thin horizontal rule
	\rule{\textwidth}{1.6pt} % Thick horizontal rule
	
	\vspace{1\baselineskip} % Whitespace after the title block
	
	%------------------------------------------------
	%	Subtitle
	%------------------------------------------------
	
	{herausgegeben von G. Tschermak.} % Subtitle or further description
	
	\vspace*{1\baselineskip} % Whitespace under the subtitle
	
	%------------------------------------------------
	%	Editor(s)
	%------------------------------------------------

 	{\small Die Aufnahmen von J. Grimm in Offenburg.\\ 25 Tafeln mit 100 mikrophotographischen Abbildungen.} % Subtitle or further description

	\vspace{1\baselineskip} % Whitespace before the editors

    %------------------------------------------------
	%	Cover photo
	%------------------------------------------------
	
	%\includegraphics[scale=1]{cover}
	
	%------------------------------------------------
	%	Publisher
	%------------------------------------------------
		
	\vspace*{\fill}% Whitespace under the publisher logo
	
	Stuttgart 1885. % Publication year
	
	{\small E. Schweizerbart'sche Verlagshandlung (E. Koch).} % Publisher

	\vspace{1\baselineskip} % Whitespace under the publisher logo

    Internet Archive Online Edition  % Publication year
	
	{\small Attribution NonCommercial ShareAlike 4.0 International } % Publisher
\end{titlepage}
\setlength{\parskip}{1mm plus1mm minus1mm}
\setcounter{tocdepth}{3}
\setcounter{secnumdepth}{3}
\tableofcontents
\clearpage
\section{Ankündigung.}
\paragraph*{}
Da in dem jetzt abgeschlossenen Werk von Herrn Prof. Tschermak nur die Meteorsteine Berücksichtigung finden, so haben die Herren A. Brezina und E. Cohen es übernommen, eine wünschenswerthe Ergänzung durch ähnliche Behandlung der Meteoreisen zu liefern. Mit der Bearbeitung derselben ist schon begonnen worden, und die bisherigen Aufnahmen, welche natürlich im reflectirten Licht stattfinden müssen, haben durchaus befriedigende Resultate geliefert. Es wird das Haupfgewicht darauf gelegt werden, alle wichtigeren Structurformen zur Darstellung zu bringen, welche beim Aetzen polirter Platten hervortreten. Ferner werden die Reichenbach'schen Lamellen, die Art des Auftretens accessorischer Gemengteile, Veränderungen in der Structur in der Nähe der Brandrinde u. s. w. zur Darstellung gelangen.

Unterzeichnete Verlagshandlung hofft, dass die erste Lieferung bis Ostern 1886 erscheinen kann. Auch für dieses Werk sind zunächst drei Lieferungen in Aussicht genommen.

Stuttgart, 15. Juli 1885.

\bigskip

E. Schweizerbart'sche Verlagshandlung (E. Koch).
\clearpage
\section{Vorwort.}
\paragraph*{}
Nur Wenigen ist es vergönnt, die Meteoriten genauer kennen zu lernen und doch ist es der Wunsch Vieler sich mit den wichtigsten Eigenschaften dieser Körper vertraut zu machen, welche aus fernen Himmelsräumen zur Erde gelangt, uns über einen Teil der Sternenwelt unmittelbare Nachricht bringen. Die Meteoriten, diese Splitter untergegangener oder vielleicht noch kreisender planetarischer Massen, geben uns eine früher ungeahnte Gelegenheit, fremde kosmische Körper mit den Händen zu greifen, zu wägen, zu messen, zu zerlegen, mit allen Mitteln der Mineralogie und Chemie zu prüfen. Was wir an ihnen wahrnehmen, erlaubt uns wichtige Kombinationen mit den Resultaten astronomischer Beobachtung sowie direkte Vergleiche mit der Steinrinde unseres Planeten. Diese bilden die Grundlage interessanter und bedeutungsvoller Schlüsse, die weit über den Bereich unserer Wohnstätte hinausreichend unser Wissen und Vermuten über die früheren Zustände sowohl der Erde als auch der ganzen Sternenwelt erheblich zu vervollkommnen geeignet sind.

Der Versuch, die Kenntnis von der Zusammensetzung dieser merkwürdigen Körper einem größeren Kreise zugänglich zu machen, dürfte daher von Vielen gebilligt werden. Für eine umfangreiche systematische Darstellung scheint es mir aber etwas zu frühe, weil manche der einfacheren Fälle noch nicht genügend untersucht sind. Dagegen halte ich es für nützlich, jenen Weg einzuschlagen, welcher die Grundlage für die fernere schriftliche Verständigung bietet, nämlich die Publikation guter Abbildungen. Solche mangeln für einige Abteilungen der Meteoriten gänzlich, die vorhandenen aber sind in verschiedenen Werken und Zeitschriften verstreut. Schon die Sammlung der bisher veröffentlichten Zeichnungen wäre vorteilhaft, besser aber ist es, wenn es gelingt, nicht bloß das bisher dargestellte in getreuen Bildern wiederzugeben, sondern alles überhaupt Wichtige in gleicher Weise zu illustrieren.

Derlei Bilder können entweder die äußere Form und die Beschaffenheit der Oberfläche schildern oder das Gefüge und die mineralogische Zusammensetzung der Meteoriten darstellen. Das erstere hat eine besondere Bedeutung für die Naturgeschichte der Meteoriten, das letztere ein allgemeineres Interesse, weil nicht nur der Bestand jener kosmischen Körper zur Anschauung gebracht, sondern auch der Vergleich mit der Textur der Gesteine und der Ausbildung der tellurischen Minerale ermöglicht wird. Demnach mag der Versuch gerechtfertigt erscheinen, durch Auswahl guter Beispiele das Gefüge und den Bestand jener Meteoriten, welche die Prüfung im durchfallenden Lichte gestatten, auf die bestmögliche Weise zur Anschauung zu bringen. Ich glaubte deshalb eine dahin gerichtete Anfrage des Herrn Prof. E. Cohen in Straßburg, welcher sich durch die Herausgabe der "Sammlung von Mikrophotographien zur Veranschaulichung der mikroskopischen Struktur von Mineralien und Gesteinen" so sehr verdient gemacht hat, in Betracht nehmen und der Aufforderung des Herrn J. Grimm, welcher jene Bilder in so vorzüglicher Ausführung herstellte, sowie jener des Herrn Verlegers E. Koch entsprechen zu sollen, obgleich ich im Augenblicke schon durch die Herausgabe eines Lehrbuches der Mineralogie und andere Arbeiten in Anspruch genommen bin und obgleich mir die Schwierigkeit der Beschaffung des Materials keine geringe zu sein schien.

Da manche der in Betracht kommenden Meteoriten durch Kauf nicht zu erhalten sind, so war ich in mehreren Fällen auf die Bereitwilligkeit der Herren Museumsvorstände und Besitzer von Meteoritensammlungen hingewiesen. Es freut mich, sagen zu können, dass ich von mehreren Seiten durch leihweise Überlassung von Präparaten (P), durch Überlassung von Splittern als Geschenk (G) oder im Tausche (T) in zuvorkommender Weise unterstützt wurde, und zwar von folgenden Herren, denen ich hiermit den gebührenden Dank ausspreche:

Herr Direktor Dr. S. Aichhorn in Graz (T), Se. Excellenz Hr. Geheimrat Freiherr v. Braun in Wien (G), Hr. A. Fauser in Pest (G), Hr. Oberbergdirektor Dr. C. W. v. Gümbel in München (P), Hr. Intendant Dr. F. v. Hochstetter in Wien (P), durch freundliche Vermittlung des Hrn. Dr. A. Brezina, Hr. Professor Dr. J. A. Krenner in Pest (G), Hr. Professor A. v. Lasaulx in Bonn (P), Hr. Nevil-Story Maskelyne in Salthrop (P), Hr. Oberbergrat M. Websky in Berlin (P), Hr. Oberbergrat v. Zepharovich in Prag (G), Hr. Professor F. Zirkel in Leipzig (P). Für die eifrige Mithilfe bei der Bearbeitung des Materials bin ich Hrn. Dr. Max Schuster zu vielem Danke verpflichtet.

Die mir vorliegende Aufgabe habe ich so aufgefasst, dass es vor allem nötig sei, Bilder zusammen zu stellen, welche das Gefüge und die Gemengteile von Repräsentanten aller Abteilungen der Meteorsteine darstellen, dass hingegen die Wiedergabe der mikroskopischen Verhältnisse von weniger allgemeiner Bedeutung vorläufig wegbleiben könne. Die vorliegenden Tafeln sollen demnach dem Besitzer gleichsam eine systematische Präparatensammlung ersetzen, ihm die Beurteilung der Meteorsteine nach ihrer inneren Beschaffenheit ermöglichen und im vorkommenden Falle die Bestimmung der Gemengteile erleichtern.

Für die vorliegende Sammlung wurden drei Hefte zu acht Tafeln in Aussicht genommen, welche zugleich mit dem Texte alle Abteilungen der Meteoriten mit Ausnahme der silikatfreien Eisenmassen in systematischer Folge behandeln.

Wien im Mai 1883.

G. Tschermak.
\clearpage
\section{Allgemeines über die Beschaffenheit der Meteoriten.}
\subsection{Äußere Form.}
\paragraph*{}
Die Gestalt der Meteoriten ist keine regelmäßige, sie zeigt vielmehr immer nur zufällige Begrenzungen und zwar solche, welche beim Zersprengen eines Gesteins von richtungsloser Struktur entstehen. Die Meteoriten haben also die Form von Bruchstücken. Die Kanten sind jedoch häufig abgerundet, die Unebenheiten öfters ausgeglichen. Fast alle Meteoriten sind mit einer dunklen Rinde überzogen, welche die Merkmale einer Schmelzung oft auch einer Trift an sich trägt and gewöhnlich als Schmelzrinde bezeichnet wird. An den Meteoreisen hat die Rinde das Ansehen and die Beschaffenheit des Hammerschlages, sie ist zugleich Oxydationsrinde. Die Abrundung, Trift und Rindenbildung betrachtet man als Folgen des Widerstandes der Luft beim Eindringen der mit planetarischer Geschwindigkeit ankommenden Meteoriten und als das Resultat der Erhitzung, welche durch die Zusammendrückung der Luft entsteht.

\subsection{Gefüge.}
\paragraph*{}
Das gröbere Gefüge oder die Struktur der Meteoriten ist verschieden, jedoch nicht sehr mannigfaltig. Viele Meteoreisen haben keine Struktur. Sie bestehen aus einem einzigen Individuum wie jenes von Braunau, oder sie stellen einen Kristallstock dar, wie die Eisen von Agram, Toluca. Andere sind jedoch körnig, wie die von Zacatecas, Rasgatà. Jene, welche den Übergang zu den Meteorsteinen bilden, indem sie Körner und Kristalle von Silikaten eingelagert enthalten, haben oft ein porphyrartiges Ansehen, weil die Silikate in einer Grundmasse von Eisen zu schwimmen scheinen, z. B. die von Pallas entdeckte Masse von Krasnojarsk. Die nächste Stufe bilden solche, die viele Silicatkörner in einem zarteren Eisenschwamm zeigen, wie die Stücke von Hainholz.

Manche Meteoreisen bieten den Charakter einer Breccie dar, indem sie Bruchstücke von Silikatgestein umschließen, z. B. jene von Tula, Copiapo. Unter den Meteorsteinen kommen öfters Breccien vor. Die grauen Bruchstücke sind durch eine dunklere Masse verkittet, wie in den Steinen von Dacca, St. Mesmin, Rutlam. Zuweilen ist diese Struktur undeutlich, wie in dem Stein von Juvinas, der bloß den Wechsel grobkörniger und feinkörniger Masse wahrnehmen lässt. Die Grundmasse, welche die Trümmer verbindet, ist mitunter schwärzlich, halbglasig und zeigt Spuren oder auch deutliche Kennzeichen einer Fluidaltextur, wie in den Durchschnitten der Steine von Orvinio und Chantonnay. Von der deutlichen Breccienstruktur bis zum gleichartigen Ansehen gibt es allerlei Übergänge, die angeschliffen entweder bloß feine dunkle wirr verlaufende Linien oder aber eine wolkige Zeichnung, endlich ein marmoriertes Ansehen darbieten. Haidinger hat auf diese Strukturverhältnisse, besonders aber auf die Breccien- und Tuffstruktur aufmerksam gemacht.

Sehr häufig bestehen die Steine aus kleinen Bruchstücken, aus Splittern und aus Gesteinspulver. Solche Massen haben den Charakter eines vulkanischen Tuffs. Die Splitter sind bald ungefähr gleichartig (Shalka), bald auffallend ungleichartig (Loutolaks). Bisweilen unterscheidet sich die feste Grundmasse stark von den eingelagerten Körnern, so dass ein porphyrartiges Aussehen entsteht, wie in dem Stein von Goalpara.

Manche von den steinartigen Meteoriten erscheinen kristallinisch-körnig, z. B. jene von Chassigny, Shergotty, Ibbenbühren, doch gibt es Übergänge zur Tuffstruktur, so dass derselbe Stein von einem Beobachter für kristallinisch von dem andern als klastisch bezeichnet wird, z. B. der von Stannern. Dichte halbglasige Massen sind selten (Tadjera). Sie haben Ähnlichkeit mit der dunklen Grundmasse einiger früher genannten Breccien.

Ein Gefüge, welches an der Mehrzahl der Steine bald in auffallender Weise, bald weniger deutlich ausgesprochen vorkommt, ist das chondritische. Kügelchen und überhaupt rundliche Körper, welche bald aus einem einzigen Kristallindividuum, bald aus mehreren bestehen, öfters auch aus verschiedenen Gemengteilen zusammengesetzt sind, bilden das Gestein fast allein (Borkut), oder sie lagern unverletzt, öfters auch zersplittert in einer lockeren bis festen Tuffmasse (Ausson). Die kugeligen Gebilde, welche gewöhnlich kleiner als erbsengroß sind, werden von mir als Erstarrungsprodukte angesehen, oder, um es ungefähr anzudeuten, als erstarrte Tropfen. Die Grundmasse ist bisweilen schwarzgefärbt, z. B. in Renazzo, Grosnaja. Einige Steine zeigen eine fast gleichartige schwarze durch Kohle gefärbte Masse, wie jene von Cold Bokkeveld.

Das chondritische Gefüge ist nur den Meteoriten eigentümlich, den tellurischen Felsarten aber fremd. Die Sphärulite in manchem Obsidian und Perlit sind zwar auch Silikatkügelchen, doch zeigen sie eine radiale Faserung um ihr Centrum, während die Chondren, wenn sie überhaupt faserig sind, eine exzentrische Faserung besitzen, überdies ist hier und dort die Grundmasse von verschiedenem Gefüge.

In manchen Steinen bemerkt man schwarze Klüfte, deren Füllung wiederum der schwarzen Grundmasse in den früher bezeichneten Breccien entspricht. Die Wände an der Kluft erscheinen bisweilen gegeneinander verschoben. Beim Zerspringen erfolgt öfters eine Trennung nach diesen Klüften und die entstehenden Flächen sind striemig und metallisch glänzend, sie zeigen Harnische (Pultusk, Mocs, Murcia).

Die tuffartigen Meteorsteine sind fein porös. Flüssigkeiten werden begierig aufgesogen. Die Steine verhalten sich aber häufig so, als ob sie erhitzt oder gefrittet worden wären. Sie klingen beim Anschlagen wie die Backsteine. Selten sieht man größere Poren, wie in dem Stein von Richmond. In jenem von Juvinas bemerkt man in den kleinen Hohlräumen Kristalle. In beiden Fällen erhalt man den Eindruck, als ob die Erscheinung von einer nachträglichen Erhitzung des Steines herrührte.

\subsection{Gemengteile.}
\paragraph*{}
Die homogenen Teile, welche die Meteoriten zusammensetzen, bieten keine große Mannigfaltigkeit.

\emph{Eisen}. Reines Eisen und nickelhaltiges Eisen bilden die Hauptmasse des Meteoreisens, ferner die Grundmasse oder den Eisenschwamm der Übergangsglieder zu den Meteorsteinen. In den letzteren sind auch oft noch zusammenhängende Eisenfäden vorhanden, endlich ist das Eisen bloß in Form getrennter Partikelchen verteilt.

\emph{Grafit} erscheint in knollenförmigen Einschlüssen in manchen Meteoreisen und umgibt manchmal die Troilit.

\emph{Schreibersit} (Phosphornickeleisen, Rhabdit) ist oft in unregelmäßigen, scharfkantigen, tafelförmigen oder in nadelförmigen Einschlüssen im Meteoreisen verteilt.

\emph{Troilit} und \emph{Magnetkies}. In den Eisenmassen bildet das Einfach-Schwefeleisen FeS, welches Haidinger Troilit nannte, kleinere oder größere knollige, öfters auch tafelförmige Einschlüsse. In den Steinen ist das Schwefeleisen etwas heller gefärbt. Seine Zusammensetzung entspricht dem Magnetkies und hat auch dessen Form. Nach Brezina hat auch der Troilit die Magnetkiesform. Seltenere Schwefelverbindungen sind der dichte schwarze Daubreelit, welcher Chrom, Eisen und Schwefel enthält und der Oldhamit CaS, endlich der Osbornit.

\emph{Chromit}. In vielen Meteoriten in Körnern oder oktaedrischen Kristallen enthalten.

Selten ist der \emph{Magnetit}, bisher bloß im Stein von Shergotty, und der Tridymit (Asmanit) bisher nur in den Massen von Breitenbach und Rittersgrün.

\emph{Olivin}. In ziemlich großen Kristallen und Körnern, in einigen porphyrartigen Meteoreisen und Übergangsgliedern, ferner in allen Chondriten und einigen anderen Meteoritenarten. Für sich allein bildet er den Stein von Chassigny.

\emph{Bronzit}. Sowohl die fast farblose oder weiße eisenarme Verbindung Enstatit als auch der eisenhaltige Bronzit, endlich auch das von den Mineralogen als Hypersthen bezeichnete eisenreiche Glied ist in den Meteoriten repräsentiert. Die Verbreitung ist die gleiche wie beim Olivin, auch bildet der Bronzit für sich eine Meteoritenart.

\emph{Augit}. Kristalle von der Form des Augits, aber von geringerem Kalkgehalte als der tellurische Pyroxen, wurden in mehreren Steinen beobachtet. In der Form größerer und kleinerer Körnchen ist er auch in anderen ziemlich verbreitet. Bronzit und Augit werde ich öfters als "Pyroxen" zusammenfassen.

\emph{Anorthit}. In mehreren Steinen wurden Kristalle und Körner von Anorthit als wesentlicher Gemengteil erkannt. In vielen Steinen finden sich Körner von \emph{Plagioklas}, der wohl nicht immer die Zusammensetzung des Anorthits haben dürfte. Merkwürdig ist der

\emph{Maskelynit} von Plagioklas-Zusammensetzung, jedoch von einfacher Lichtbrechung.

\emph{Glas}. Außer den kristallisierten und kristallinischen Gemengteilen kommen öfters auch glasartige vor. Die einen sind farblos, dem Maskelynit ähnlich, die anderen farbig, meist braun, öfters mit Anfängen der Entglasung. Diese dürften vorzugsweise Magnesiasilikat enthalten.

\emph{Kohle} und schmelzbare, in Alkohol auflösliche Kohlenwasserstoffe wurden in schwarzen Meteoriten nachgewiesen, als Seltenheit ein Carbonat von der Beschaffenheit des Breunnerits. Das in solchen Steinen gefundene Wasser halten Viele für nachträglich aufgenommen, weil die übrigen Meteoriten kein Wasser enthalten, ferner die gefundenen Sulfate für sekundäre Bildungen, hervorgerufen durch die Verwitterung des enthaltenen Schwefeleisens.

Es bleibt noch zu bemerken, dass in manchen Schriften solche Bestandteile von Meteoriten angegeben wurden, welche nicht nachgewiesen sind, wie z. B. Blei, Eisenkies, Leucit, Schwefel, oder welche mit neuen Namen belegt wurden, sich später aber als etwas Bekanntes erwiesen, wie Shepardit, Piddingtonit.

\subsection{Einteilung.}
\paragraph*{}
G. Rose hat eine Einteilung der Meteoriten nach den herrschenden Gemengteilen vorgeschlagen, die mit den Meteoreisen beginnt und mit denjenigen Steinen schließt, welche die meiste Ähnlichkeit mit tellurischen Felsarten haben.\footnote{Beschreibung und Einteilung der Meteoriten. Berlin, 1864.} Diese Folge verläuft ungefähr im selben Sinne, wie die Abnahme des spezifischen Gewichtes und ist insofern eine geologische zu nennen, als nach der Parallele, welche von Daubrée zwischen der Zusammensetzung der Meteoriten und jener unseres Planeten gezogen wurde, zuerst der angenommene metallische Kern der Erde als die älteste Bildung, hierauf die spezifisch leichteren Silikatmassen als die jüngeren Bildungen in Betracht kommen.\footnote{Etudes recentes sur les Météorites. Journal de savants 1870.} Wenn die Meteoriten Splitter sind, welche von einem oder von mehreren kleinen planetarischen Körpern herrühren, so wird man sich jedes solche kleine Gestirn ähnlich wie die Erde gebaut denken, also in der Vorstellung eine metallische Kugel mit einer Silikatrinde konstruieren, welche letztere aus Erstarrungsprodukten und aus Tuffmassen bestand.\footnote{Tschermak. Die Bildung der Meteoriten und der Vulkanismus. Sitzungsber. d. Wiener Akad. Bd. 71. Abt. 2. (1875.)}

Die Rose'sche Einteilung lasst sich demnach als eine solche betrachten, welche den gegenwärtigen Vorstellungen von der Bildungsfolge Rechnung trägt und kann daher als eine natürliche bezeichnet werden. Trotzdem werde ich hier nicht diese, sondern die umgekehrte Anordnung befolgen, weil es mir für den vorliegenden Fall zweckmäßiger scheint, mit jenen Meteoriten zu beginnen, welche die meisten Anknüpfungspunkte an die bekannten Felsarten darbieten.

Die Abteilungen werden folgende sein:
\begin{enumerate}
    \item Calciumreiche Steine, arm an gediegenem Eisen.  
    \item Magnesiumreiche Steine, arm an gediegenem Eisen.  
    \item Magnesiumreiche chondritische Steine mit gediegenem Eisen.  
    \item Eisen mit Silikaten.  
    \item Meteoreisen.
\end{enumerate}
\paragraph*{}
Innerhalb dieser Abteilungen werden den Grundsätzen der Petrographie gemäß einzelne Meteoritenarten unterschieden, deren jede ein besonderes Gemenge oder eine eigentümliche Struktur darbietet. Zu den von G. Rose aufgestellten Arten sind noch einige hinzugekommen, welche ich schon in dem Verzeichnisse von 1872 angegeben habe.\footnote{Mineralogische Mitteilungen 1872, p. 165.} Durch meine letzten Untersuchungen ist die Zusammensetzung einiger schon früher unterschiedener Abteilungen genauer bestimmt, ferner sind neue bekannt geworden, so dass in einigen Fällen eine Umstellung und Neubenennung erforderlich wurde.

Die bis jetzt bekannten Meteoritenarten sind:
\begin{enumerate}
    \item Die wesentlichen Gemengteile sind Pyroxen und Plagioklase. Die Rinde ist glänzend.
    \begin{itemize}
        \item \emph{Eukrit} G. Rose. Augit und Anorthit, statt des letzteren auch Maskelynit.
        \item \emph{Howardit} G. Rose. Augit, Bronzit, Anorthit.
    \end{itemize}
    \item Pyroxen und Olivine bilden die wesentlichen Gemengteile. Die Rinde ist wenig glänzend bis matt, ebenso in den folgenden Abteilungen.
    \begin{itemize}
        \item \emph{Bustit} Autor. Diopsid und Enstatit.
        \item \emph{Chladnit} G. Rose. Enstatit mit wenig Anorthit.
        \item \emph{Diogenit} Aut. Bronzit.
        \item \emph{Amphoterit} Aut. Bronzit und Olivin.        
        \item \emph{Chassignit} G. Rose. Olivin.
    \end{itemize}
    \item Bronzit, Olivin, Eisen als wesentliche Gemengteile.
    \begin{itemize}
        \item \emph{Chondrit} G. Rose. Textur chondritisch.
    \end{itemize}
    \item Eisen, netzförmig, darin Silikate: Plagioklas, Olivin, Pyroxen, Troilit.
    \begin{itemize}
        \item \emph{Grahamit} Aut. Plagioklas, Bronzit, Augit im Eisen.
        \item \emph{Siderophyr} Aut. Bronzit im Eisen.
        \item \emph{Mesosiderit} G. Rose. Bronzit, Olivin im Eisen.
        \item \emph{Pallasit} G. Rose. Olivin im Eisen.
    \end{itemize}
    \item Eisen mit untergeordnetem Troilit, Schreibersit \emph{etc.}
    \begin{itemize}
        \item \emph{Meteoreisen}.
    \end{itemize}
\end{enumerate}
\paragraph*{}
Den Namen Shalkit, welchen G. Rose für das Gemenge von Bronzit und Olivin vorgeschlagen hatte, habe ich aufgegeben, weil für den Meteorstein von Shalka widersprechende Resultate bekannt wurden. Stattdessen will ich die Bezeichnung Amphoterit vorschlagen, ferner für die aus Bronzit allein bestehenden Steine den Namen Diogenit.\footnote{Nach Diogenes von Apollonia, welcher zuerst klare Vorstellungen über den kosmischen Ursprung und die siderische Natur der Meteoriten aussprach.} Die neue Meteoritenart unter 4. musste auch durch eine Bezeichnung unterschieden werden, wofür ich Grahamit\footnote{Nach dem Chemiker Graham benannt, welcher den in Meteoreisen absorbiert enthaltenen Wasserstoff entdeckte.} wählte, endlich waren auch die Massen von Breitenbach und Rittersgrün als eine besondere Art (Siderophyr) hervorzuheben.
\clearpage
\section{Beschreibung der dargestellten Arten.}
\subsection{}
\subsubsection{Eukrit.}
\paragraph{}
Hierher gehören die Steine von Juvinas, Jonzac, Stannern, Petersborough, Konstantinopel. Sie sind wesentlich Gemenge von Augit und Anorthit. Am besten ist der Stein von Juvinas untersucht.

Das Gefüge ist, wie schon G. Rose a. a. O. bemerkte, an verschiedenen Stellen sehr verschieden indem kristallinisch-kleinkörnige und undeutlich feinkörnige Gemenge wechseln. Die letzteren erscheinen unter dem Mikroskop kristallinisch bis tuffartig. Das Ganze hat also einen undeutlich breccienartigen Charakter. (Taf. 1, Fig. 1, 3.)

Der Anorthit ist deutlich auskristallisiert. Die eingeschlossenen Kristalle zeigen scharfe Umrisse, die in kleinen Drusenräumen sitzenden haben eine durch Vorherrschen von M = (0 1 0) tafelförmige Gestalt, an welcher noch die Flachen T, \emph{l}, P, \emph{x} zu bemerken sind.

Die Kristalle sind teils wasserhell, teils durch Einschlüsse getrübt und weiß. Bei starker Vergrößerung erkennt man teils rundliche Glaseinschlüsse teils feine nadelförmige Gestalten, welche den Randzonen parallel angeordnet erscheinen. Das optische Verhalten ist im übrigen dasselbe, wie an dem Anorthit vom Vesuv.

Die eingeschlossenen Kristalle sind meist zwillingsartig gebaut oder sie erweisen sich als Zwillingsstöcke von komplizierter Zusammensetzung. Ihre Größe beträgt oft 2 mm. Im Dünnschliff erscheinen dieselben im auffallenden Lichte bläulich, im durchfallenden bräunlich. Sie sind ungemein reich an sehr kleinen Einschlüssen, die meistens farblos und nur selten braun gefärbt befunden werden. Die meisten Einschlüsse sind rundlich, manche aber auch langgestreckt, alle zeigen sehr schmale Kontouren. Zuweilen haben sie eine Libelle, einige wenige enthalten ein schwarzes Körnchen. Auf das polarisierte Licht üben sie keine Wirkung. Ihre Anordnung entspricht immer der äußeren Begrenzung, die länglichen sind meistens zur Längsausdehnung der Kristalle parallel gestreckt. Man darf sonach die rundlichen wie die gestreckten als Glaseinschlüsse ansehen. Einschlüsse mit breiter Kontur, welche sich als Gasporen erkennen lassen, sind selten. (Taf. 1, Fig. 1; Taf. 2, Fig. 2, 3, 4.)

Die Augitkristalle, welche in den Drusenräumen auftreten, sind braunschwarz. Die Form ist dieselbe wie bei manchem Diopsid, indem gewöhnlich \emph{c} = (0 0 1), \emph{u} = (1 1 1), \emph{o} = (2$^{\prime}$ 2 1), \emph{a} = (1 0 0), \emph{m} = (1 1 0) und \emph{b} = (0 1 0) auftreten.

Die Auslöschung auf der Längsfläche gibt einen Winkel von 52° 10$^{\prime}$, während derselbe Winkel für Diopsid von Ala = 51° 6$^{\prime}$ n. Des Cl. Blättchen parallel \emph{a} = 100 geben das Bild einer optischen Axe in ähnlicher Lage wie der Diopsid von Ala.

Die Schnitte parallel \emph{a} und \emph{b} lassen die Kristalle aus unzähligen dünnen Lamellen parallel \emph{c} aufgebaut erscheinen. An manchen Stellen kann man fast mit Sicherheit erkennen, dass eine wiederholte Zwillingsbildung nach \emph{c} die Ursache ist, also genau so wie bei der entsprechenden schaligen Zusammensetzung der Diopside. Die schwarze Farbe des Augits rührt von zahllosen Einschlüssen her, welche meist schwarz seltener braun sind, teils nadelförmig teils rundlich geformt erscheinen. Die braunen rundlichen erweisen sich als Glaseinschlüsse, was die nadelförmigen sind, ließ sich nicht bestimmen. Die rundlichen und staubartigen sind vorzugsweise der Endfläche \emph{c} = 0 0 1 parallel angeordnet, die nadelförmigen der aufrechten Axe parallel gelagert.

Die im Gestein eingeschlossenen Augite sind unvollkommene Kristalle oder Körner ohne geradlinige Umrisse, voll von den eben genannten Einschlüssen und häufig durch Quersprünge gegliedert. An den Wänden der Sprünge zeigen sich die Einschlüsse oft so sehr vermindert, dass man die blass bräunliche Farbe des reinen Diopsids wahrnimmt. Dagegen sind die Sprünge mit schwarzer Masse erfüllt. Es sieht so aus, als ob die Substanz der Einschlüsse aus den Wänden in die Klüfte gewandert wäre. (Taf. 1, Fig. 1, 3; Taf. 3, Fig. 1.)

Ein fernerer Gemengteil ist jenes gelbe Silikat, welches schon G. Rose in der Form kleiner Blättchen wahrnahm. Dieselben finden sich hie und da in der Grundmasse, an manchen Punkten ragen sie in die Drusenräume oder setzen durch diese hindurch. Im Dünnschliffe zeigt sich, dass dieselben aus winzigen Körnchen von blass bräunlicher Farbe bestehen und dass sie an vielen Stellen dieselbe feine Lamellentextur wie der vorbeschriebene Augit besitzen. Die Schmelzbarkeit zu schwarzem Glase stimmt zu dieser Ähnlichkeit, so dass man wohl kaum irre gehen wird, diese blassen Körnchen für Diopsid ohne schwarze Einschlüsse zu halten. Die Blättchen erscheinen demnach wie Pseudomorphosen nach einem nicht angebbaren Silikat. Im Dünnschliff zeigt sich aber, dass das gelbe Silikat nicht nur in dieser Form, sondern auch in verschiedenen körnigen Partikeln in der Grundmasse unregelmäßig verteilt sei und die Maschen zwischen den kleinen Anorthitkristallchen der Grundmasse ausfülle.

Obwohl man sich den Vorgang nicht leicht erklären kann, so macht doch alles dies den Eindruck, als ob die früher tuffartige Grundmasse umgeschmolzen wäre, wobei wieder Anorthit, andererseits aber gereinigter Diopsid auskristallisiert wären. Die gelben Blättchen wären dann Paramorphosen. Die Umschmelzung der porösen Grundmasse würde auch die Entstehung der vielen kleinen Drusenräume begreiflich machen. Von opaken Gemengteilen, die aber nur in geringer Menge vorhanden sind, kennt man Magnetkies, Chromit und Nickeleisen.

Den Magnetkies hat G. Rose in Drusenräumen kristallisiert gefunden und die Form bestimmt, ferner auch das sparsame Vorkommen von Nickeleisen beobachtet. Ich fand außerdem eisenschwarze Körnchen, welche das Verhalten des Chromits zeigen.

Der Meteorit von Stannern ist dem vorigen sehr ähnlich, doch zeigt er schon eine ausgesprochene Tuffstruktur.\footnote{G. Rose a. a. O. und Tschermak. Mineralogische Mitteilungen 1872, p. 83.} An demselben Stücke sieht man deutlich körnige, kleine strahlige und fast dichte Splitter und Trümmer nebeneinander. Einzelne Steine sind körnig wie der von Juvinas, andere sind viel dunkler gefärbt von feinkörnigem bis dichtem Gefüge.

Der Anorthit und der Augit haben dieselben Eigenschaften wie im Stein von Juvinas, nur sieht man scharfe Kristallumrisse seltener, dagegen häufig eine Verwachsung von beiden Mineralen, wobei dieselben oft als abwechselnde Platten erscheinen. Das gelbe Silikat und der Magnetkies sind auch zu erkennen. (Taf. 2, Fig. 1; Taf. 3, Fig. 3.)

Die Meteoriten von Jonzac sind jenem von Juvinas ungemein ähnlich, der von Petersborough in Tennessee nähert sich in seiner Beschaffenheit nach G. Rose dem Stein von Stannern.

Zu den Eukriten ist ferner noch der Meteorit von Shergotty zu zählen, den ich vor Jahren beschrieb.\footnote{Sitzungsberichte der Wiener Akad. Bd. 65. Abt. 1, p. 122 and Tschermaks Mineralog. Mitteil. 1872, p. 87.} Derselbe ist ein deutlich körniges Gemenge, wesentlich bestehend aus gelblich grauen matten Körnchen und Prismen, welche als ein Augit bestimmt wurden, ferner aus wasserhellen glasigen Körnchen und Säulchen, welche auf kein bisher bekanntes Mineral zu beziehen sind und von mir als Maskelynit bezeichnet wurden.

Der Augit verhält sich in durchfallendem Lichte wie der vulkanische Augit, er erscheint lichtbraun, fast ganz frei von Einschlüssen. Eine Zwillingsbildung nach a ist ziemlich häufig, jedoch ohne Wiederholung. Die Trübung, welche ihm ein fremdartiges Aussehen verleiht, bleibt ihm teilweise auch im durchfallenden Lichte. Dieselbe rührt von ungemein feinen unregelmäßigen Sprüngen her, welche schließen lassen, dass der Augit eine mechanische Veränderung erfahren habe. Durch diese Beschaffenheit ist er von dem tellurischen Augit verschieden, ebenso durch die chemische Zusammensetzung, da er weniger Kalk enthält. Der Maskelynit ist vollkommen farblos und wasserhell. Er zeigt im Dünnschliffe meist langgestreckte Umrisse und parallel der Länge feine Linien, so dass er im gewöhnlichen Lichte ganz und gar den Eindruck von Plagioklas hervorruft. Die chemische Zusammensetzung entspricht gleichfalls einem Plagioklas aus der Labradoritreihe. Im polarisierten Lichte löscht er aber vollkommen aus, erweist sich also einfach brechend. Die Sprünge in demselben entsprechen einem deutlich muscheligen Bruche wie bei einem Glas. Demnach verhält sich der Maskelynit wie ein durch Schmelzung oder überhaupt durch bloße mechanische Veränderung in den amorphen Zustand übergeführter Labradorit.

Einschlüsse sind in demselben öfters zu bemerken. Sie sind ganz unregelmäßig geformt und bestehen aus Augit und aus Magnetit.

Stellenweise zeigen sich weiße trübe Partikel, welche nur trüber Maskelynit sind, ferner sehr sparsam ein gelbes Silikat, welches ungemein kleine Körnchen bildet und sich optisch zweiaxig verhält.

Von opaken Gemengteilen wurden zwei erkannt. Der eine ist Magnetit, welcher hier zum ersten male als Gemengteil eines Meteoriten gefunden wurde, der zweite, sehr spärlich verbreitete, Magnetkies.

\subsubsection{Howardit.}
\paragraph{}
Nach den bis jetzt bekannten Untersuchungen sind hierher zu rechnen die Meteoriten von Massing, Loutolaks, Białystok, Le Teilleul, Nobleborough, Francfort. Darunter sind die beiden zuerst angeführten am besten bekannt.

Der Meteorit von Loutolaks hat ein tuffartiges Gefüge. In einer erdigen lockeren grauen Grundmasse liegen Splitter und Körner von grüngelber weißer und schwarzer Farbe, ferner auch kleine Bruchstücke eines Gemenges, welches leicht als Eukrit zu erkennen ist.

Das Ganze hat den Charakter eines vulkanischen Tuffs, indem Splitter von verschiedenen Mineralen, wie sie sonst nicht in demselben kristallinischen Gestein zusammenvorkommen, beisammen liegen and bloß unregelmäßige Begrenzungen, selten aber Spuren von Kristallumrissen wahrzunehmen sind. Ich konnte unter den durchsichtigen Gemengteilen dreierlei Anorthite, viererlei Augite, ferner Bronzit unterscheiden.

Der Anorthit findet sich in den genannten kleinen Eukritbruchstücken mit denselben Eigenschaften wie in dem Stein von Stannern. Die kleinen runden Glaseinschlüsse sind in derselben Form und Verteilung vorhanden. Derjenige Anorthit oder überhaupt Plagioklas, welcher ohne Verwachsung mit Augit in Splittern verbreitet ist, erscheint entweder dem vorigen gleich oder er enthält große dunkle Einschlüsse von Glas oder Grundmasse, oder aber er ist fast ganz frei von Einschlüssen.

Der Augit, welcher in den Eukritbruchstücken enthalten ist, hat dieselben Eigenschaften, wie jener in dem Stein von Stannern. Man sieht braune Körner, bisweilen mit schwarzen Linien und schwarz gefüllten Sprüngen, vorwiegend aber gelben körnigen Augit mit feinschaliger Zusammensetzung. Jene Augitsplitter und Körner, welche in großer Menge in der Grundmasse liegen, sind entweder den beiden vorigen gleich oder sie sind mehr grünlich gefärbt und von ausgezeichnet feinschaliger Zusammensetzung nach 0 0 1. An manchen dieser Körner lässt sich auch die entsprechende Zwillingsbildung erkennen, an anderen beobachtet man viel schwarze nadelförmige parallel gelagerte Einschlüsse. Die vierte Form des Augits erscheint in größeren Splittern von sehr blass bräunlicher Farbe ohne schalige Zusammensetzung.

Der Bronzit bildet sehr blass grünlich gefärbte größere Splitter fast ohne Einschlüsse. Die gerade Auslöschung und das faserige Wesen charakterisieren diesen Gemengteil hinreichend. Um aber vollständig sicher zu gehen, habe ich die gelbgrünen Körner, welche nach der mikroskopischen Prüfung als Bronzit bestimmt wurden, noch besonders geprüft, weil sie früher für Olivin angesehen worden waren. Ich erhielt aber die Spaltbarkeit des Bronzits. Bei der Behandlung des Pulvers mit konzentrierter Salzsäure wurde dasselbe nur sehr wenig angegriffen.

Für Olivin halte ich einzelne kleine Splitter in der Grundmasse, ferner vermute ich denselben in jenen Gemengen, welche als kleinkörnige Gesteinsplitter vorkommen und oft reich an beigemengten schwarzen Körnchen sind. Letztere bilden einen Teil der schon mit freiem Auge wahrnehmbaren dunklen Körner und Splitter. Die anderen erwiesen sich als gleichartig mit jenen, welche auch im Stein von Stannern auftreten und feinkörniger bis dichter Eukrit sind.

Einige kleine pechschwarze Körner sind wohl als Chromit anzusehen. Dass eine sehr geringe Menge von Magnetkies und von gediegen Eisen vorhanden sei, geht schon aus den Beobachtungen von Partsch und G. Rose hervor. (Taf. 4, Fig. 1, 2, 4.)

Die angeführten Beobachtungen wurden an einem Exemplar gemacht, welches ich von Hrn. Prof. Wiik, also aus der besten Quelle erhielt, und welches mit dem Exemplar des Wiener Hofmuseums vollständig übereinstimmte. Beim Vergleich mit G. Roses Resultaten stellt sich heraus, dass die von diesem Forscher für Olivin gehaltenen gelbgrünen Körner von mir als Augit und Bronzit bestimmt wurden. Ein Umstand, welcher früher die richtige Beurteilung des Meteoriten erschwerte, ist eine von Berzelius ausgeführte Analyse, nach welcher der Stein größtenteils aus Olivin bestünde. Die Analysen von Arppe (Rammelsberg, D. chem. Nat. d. Meteoriten 1870), welche der von mir angegebenen Zusammensetzung vollkommen entsprechen, haben jedoch jene irrtümliche Bestimmung beseitigt.

Der Stein von Massing ist dem vorigen sehr ähnlich. Ich konnte dies an dem kleinen Präparate, welches mir von Hrn. Oberbergrat v. Gümbel überlassen wurde, genügend sicher erkennen. Der Stein ist ebenfalls ein Tuff in dem sowohl Kristallsplitter, als auch kleine Bruchstücke dichten Gesteins durch eine erdige Grundmasse verbunden sind. Unter den Splittern sieht man Anorthit vom gleichen Aussehen und mit den gleichen Einschlüssen, wie in dem Stein von Loutolaks. Der Augit ist in derselben Weise vertreten in braunen, gelben, sowie in den grünlichgrauen Splittern mit feinschaligem Baue.

Der Bronzit hat dasselbe Ansehen, doch kommen öfters Kristalle mit gut erhaltener Form vor, wo von einer auf Taf. 4, Fig 3 dargestellt ist. Vereinzelt finden sich aber auch stengelige Splitter, ähnlich jenen, welche in den Chondriten so gewöhnlich sind.

Die kleinen Gesteinsbruchstücke sind auch von ungefähr gleicher Art und auch ungemein dicht, so dass hier die Gegenwart von Bronzit nur beiläufig zu bestimmen ist. Chromit und Magnetkies erscheinen auch in derselben Weise, wie im vorigen Meteorit. Mit diesem Befunde stimmen die Beobachtungen Gümbels bis auf die Deutung der grünlichen Splitter als Olivin sehr gut überein und die Analyse Schwagers harmoniert ebenfalls mit demselben.\footnote{Sitzungsberichte der bayrischen Akademie. 1878. 1.}

Der Meteorit von Białystok ist nach G. Rose dem vorigen sehr ähnlich, jener von Le Teilleul, welcher in dem Verzeichnis des Pariser Museums zu den Howarditen gerechnet wird, scheint mir, nach dem im Wiener Hofmuseum liegenden Stücke zu urteilen, bestimmt dazu zu gehören.
\clearpage
\subsection{}
\subsubsection{Bustit.}
\paragraph{}
Das Gemenge von Diopsid und Enstatit ist bisher bloß durch den Stein von Busti bei Goruckpur (gefallen am 2. Dezember 1852) repräsentiert. Maskelyne hat denselben untersucht, die einzelnen Gemengteile gemessen und analysirt.\footnote{Proceedings of the Royal Society 18. 146.} Das Gefüge ist beinahe kristallinisch, doch unterscheidet man Kristalle und größere Splitter, welche in einer aus feinen Splittern bestehenden Grundmasse liegen. Flight hat eine Abbildung des ganzen Steines veröffentlicht, welche die ungleichartige Mengung deutlich wahrnehmen lasst.\footnote{Geological Magazine, September 1875.} Der Stein hat keine Rinde.

Der Diopsid, welcher das herrschende Mineral ist, erscheint im auffallenden Lichte grau bis violett. Maskelyne konnte an Körnern die Prismenzone bestimmen und auch eine Pyramidenfläche erkennen. Dieser Gemengteil ist meist auffallend durch seine feinschalige Zusammensetzung nach 1 0 0, welche oft mit wiederholter Zwillingsbildung nach dieser Fläche verbunden ist. Außer dieser Blätterung, welche dem Diallag entspricht, ist öfters auch noch eine schalige Zusammensetzung nach 0 0 1 mit einer deutlichen Zwillingsbildung nach dieser Fläche wahrnehmbar. Die oft reichlichen Einschlüsse sind schwarz und bald nadelförmig, der ersten Lamellierung parallel gelagert, bald rundlich. Sie sind die Ursache der violetten Färbung.

Der Enstatit lässt öfters scharfe Begrenzungen wahrnehmen. Maskelyne konnte nur die Prismenzone bestimmen. Derselbe unterscheidet dreierlei Enstatite, den grauen undurchsichtigen, den graulichweißen durchscheinenden und den farblosen wasserhellen Enstatit. Im Dünnschliff erkennt man ebenfalls verschiedene Arten. Der graue führt eine große Anzahl von Glaseinschlüssen mit sich, welche bisweilen eine fixe Libelle haben. Sie zeigen sehr oft einen polygonalen Umriss und erscheinen als negative Kristalle, die mit einem blass bräunlichen Glase erfüllt sind. Wenn viele solche Einschlüsse vorhanden sind, ist der Enstatit trübe. Es gibt aber auch völlig farblose Splitter, die ganz frei von Einschlüssen sind.

Außer diesen beiden Gemengteilen fanden sich untergeordnet Plagioklas, Oldhamit, Nickeleisen, Osbornit. Der Plagioklas wird von Maskelyne nicht angeführt, er ist auch nur spärlich vorhanden, doch konnte ich die farblosen Splitter, welche fast frei von Einschlüssen sind und keine Zwillingslamellen erkennen lassen, mit großer Wahrscheinlichkeit auf Plagioklas beziehen, weil dieser Gemengteil mit dem Plagioklas im Stein von Bishopville im übrigen völlig übereinstimmt. Der Oldhamit CaS ist nur in einem Teile des Steines in rundlichen Körnern von tesseraler Spaltbarkeit vorhanden. Das Nickeleisen ist nur in geringer Menge, noch spärlicher der Osbornit enthalten. Der letztere zeigt Oktaeder und die Reaktionen auf Schwefel, Calcium und Titan oder Zirconium. Die Einschlüsse im Diopsid dürften nach Maskelyne Osbornit sein.

Die Bilder Fig. 1 und 2 auf Taf. 5 sind aus einem Präparat erhalten, welches mir Herr Nevil-Story Maskelyne bereitwilligst zur Benützung überließ.

\subsubsection{Chladnit.}
\paragraph{}
Auch dieses Gemenge ist bisher nur in einem einzigen Meteoriten und zwar in jenem von Bishopville gefunden worden. Der Stein ist grobkörnig und besteht zum größten Teil aus schneeweißem lockeren Enstatit. G. Rose bemerkte auch noch andere weiße Körnchen, vermochte sie jedoch nicht zu bestimmen. Nach meinen Beobachtungen gehören dieselben zum Plagioklas. Der dritte Gemengteil ist Magnetkies. Der Stein hatte eine marmorierte Rinde, teils farblos, teils schwarz, weiß, bläulich und grau.

Der Enstatit bildet meist große, aber auch kleine Körner. An einem der letzteren konnte ich scharfe Umrisse wahrnehmen. Der Schnitt ging ungefähr parallel \emph{a} = 1 0 0. Die Endigung des Kristalles war dreiflächig, eine Fläche entsprach der Zone \emph{p} \emph{a},  die beiden andern den Zonen \emph{u} \emph{b}.\footnote{Hier und im Folgenden sind für Bronzit und Enstatit \emph{a} = \emph{b} bei v. Rath, \emph{u} = \emph{u} bei v. Rath. \emph{b} = \emph{a} bei v. Rath, \emph{p} = \emph{k} bei v. Rath.} Die Körner sind von vielen feinen unregelmäßigen Sprüngen durchsetzt, abgesehen von den Spaltrissen, welche beim Präparieren entstehen. Einschlüsse sind nur in geringer Menge vorhanden und bestehen aus opaken Körnchen, seltener aus schwarzen Nadeln.

Der Plagioklas ist meistens mit den kleinen Enstatitkörnern verbunden. Niemals beobachtete ich eine regelmäßige Begrenzung. Die Umrisse sind rundlich lappig oder gestreckt. Im polarisierten Lichte sieht man bisweilen eine sehr deutliche Zwillingstextur, indem entweder breite Lamellen in Wechselstellung erscheinen oder aber manche Körner aus ungemein schmalen Lamellen zusammengesetzt sind, so dass dieselben zwischen gekreuzten Nicols äußerst fein liniiert erscheinen. Die übrigen Körner haben eine einfache, gewöhnlich aber eine undulöse Auslöschung, manche sind aus mehreren kleinen Körnchen zusammengesetzt. Auf das Verhalten im polarisierten Lichte gründet sich die Bestimmung als Plagioklas. Der Versuch, einzelne Körnchen für weitere Prüfung aus dem Gemenge zu sondern, misslang nicht nur wegen ihrer Kleinheit, sondern auch deshalb, weil dieselben weder durch die Farbe noch durch den Glanz Vom Enstatit unterschieden werden können.

Der Plagioklas zeigt stellenweise Schlieren und zarte Trübung, in welchem Falle derselbe im durchfallenden Lichte bräunlich erscheint. Kleine opake Einschlüsse sind selten, dagegen kommen größere, oft spindelförmige Bronziteinschlüsse nicht selten vor. Der Magnetkies bildet größere und kleinere Körner, die in den vorliegenden Stücken von einem braunen, durch Einwirkung der Luft entstandenen Hof umgeben sind.

Mit der angegebenen mikroskopischen Beschaffenheit stimmt die Analyse Rammelsbergs,\footnote{Monatsberichte der Berliner Akademie. 1861, p. 895.} welche außer den Bestandteilen des Enstatits auch Tonerde, Kalk and Alkalien in geringer Menge angibt, vollkommen überein.

G. Rose gibt auch noch geringe Mengen von Nickeleisen and ein schwarzes Mineral an, welches hie und da feine Kluftausfüllungen bildet. Beim Zerbrechen erhielt ich auf solchen Klüften glänzende Harnische, ähnlich wie in später anzuführenden Meteoriten, in welchen diese aus Eisen, Magnetkies and Silikatschmelze bestehen.

Da der Enstatit keine anderen Erscheinungen darbietet als der Bronzit in folgenden Meteoriten, so wurde in den Fig. 3 und 4 auf Tafel 5 vorzugsweise der Plagioklas zur Anschauung gebracht.

\subsubsection{Diogenit.}
\paragraph{}
Diese Abteilung ist von der vorigen mineralogisch wenig verschieden, da der wesentliche Gemengteil dem Bronzit oder Hypersthen entspricht, Gattungen, welche mit dem Enstatit durch Übergänge verbunden sind. Die Trennung erfolgt also nur wegen des bedeutenden Gehaltes an Eisenoxydul. Hierher gehören die Steine von Manegaum, Ibbenbühren und wohl auch der von Shalka.

Das Innere der beiden ersteren ist hell graulichgelb mit größeren lichtgelbgrünen Körnern. Sowohl diese Körner als auch die übrige Masse bestehen aus einem Bronzit mit 20.5 resp. 17.5 Eisenoxydul. In dem Stein von Manegaum fand Maskelyne auch geringe Mengen von Chromit und gediegen Eisen.\footnote{Philosophical Transactions 160. p. 189. (1870.)} Im Stein von Ibbenbühren konnte v. Rath nur Bronzit erkennen, abgesehen von wenigen Einschlüssen.\footnote{Monatsberichte der Berliner Akademie 1872, p. 27. Pogg. Ann. 146. p. 474.}

Der Meteorit von Ibbenbühren ist ungemein gleichartig. (Taf. 6. Fig. 2.) Der Bronzit bildet große und kleine Körner fast ohne eine Spur von regelmäßiger Form. Im polarisierten Lichte zeigen manche Individuen eine zarte Streifung, jedoch ohne ausgesprochenen Zwillingscharakter. Zuweilen zeigen sich auch einzelne sehr dünne Lamellen von schiefer Auslöschung eingeschlossen. Dieselben dürften einem Augit angehören. Einschlüsse sind nur sparsam vorhanden, und zwar teils rotbraune Glaseinschlüsse, teils opake Körnchen, welche Magnetkies und Chromit sein dürften. Auch sieht man bisweilen sehr schmale schwarze Klüfte, deren Füllung eine braunschwarze Masse ist. An einer Stelle beobachtete ich zwischen den Bronzitkörnern ein farbloses Mineral, aus einem Aggregat kleiner Körnchen bestehend, welche Zwillingsbildungen ähnlich denen der Plagioklase wahrnehmen ließen, doch nicht so ausgesprochen, dass die Bestimmung sicher wäre. Es könnte auch Tridymit sein.

Der Stein von Shalka zeigt in einer hellgrauen etwas zerreiblichen Masse größere grünlichgraue Körner von Bronzit und schwarze Körner von Chromit. Im Dünnschliff lässt sich erkennen, dass alles Durchsichtige Bronzit ist und die großen Körner desselben, die bisweilen Kristallumrisse zeigen, in einer Grundmasse von Bronzitsplittern liegen. (Taf. 6, Fig. 1.) Der Bronzit enthält öfters braune Glaseinschlüsse oder opake Körnchen. Manche der letzteren sind nach den Sprüngen im Bronzit angeordnet, dürften also erst nachträglich abgesetzt worden sein. Es ist mir wahrscheinlich, dass die letzteren aus Magnetkies bestehen. Beim Behandeln des Meteoriten mit Säure wird in der Tat etwas Schwefelwasserstoff entwickelt. Grüngelbe Körnchen, die man hie und da beobachtet, hielt G. Rose (a. a. O. p. 125) für Olivin, daher dieser Forscher den Stein von Shalka als ein Olivingemenge definierte, wofür die Abteilung Shalkit aufgestellt wurde. Maskelyne vermochte jedoch keinen Olivin zu finden, auch mir gelang es nicht, solchen nachzuweisen. Ich isolierte einzelne der gelbgrünen Körnchen, fand jedoch die Spaltbarkeit des Bronzits und bei der Behandlung mit conc. Salzsäure nur eine sehr geringe Zersetzung. Somit ist nur erwiesen, dass außer dem herrschenden grünlichgrauen Bronzit auch gelbgrüner untergeordnet vorkommt, was bei der Tuffstruktur des Meteoriten begreiflich ist.

\subsubsection{Amphoterit.}
\paragraph{}
Von dieser Abteilung kennt man bisher nur einen Meteoriten, nämlich den von Manbhoom in Bengalen (22. Dez. 1863). Derselbe ist ein grüngelbliches körniges Gemenge, in welchem der Bronzit und Olivin fast die gleiche Farbe zeigen. Außer diesen sind auch zahlreiche Körner von Magnetkies und wenige Körner von Eisen bemerkbar.

Bei einem Versuche, welchen ich vor Jahren ausführte, erhielt ich ungefähr 33\% in Säure Unlösliches, welches als Bronzit erkannt wurde. In einem Dünnschliffe, welchen ich damals herstellen ließ und welchen ich aus dem k. k. Hofmuseum zur Benützung erhielt, lässt sich körniger Olivin, von vielen Sprüngen durchzogen und arm an Einschlüssen als Hauptgemengteil and Bronzit in länglichen bis rundlichen Körnern von etwas faserigem Ansehen leicht erkennen. Beide sind blassgrün. Außer diesen sind aber auch farblose Körnchen hie und da eingestreut, welche sich optisch so verhalten, wie der später beim Chondrit zu beschreibende Plagioklas. Die rundlichen opaken Körner sind Magnetkies, einige längliche Eisen. (Taf. 6, Fig. 3.)

\subsubsection{Chassignit.}
\paragraph{}
Auch diese Art ist bisher nur durch einen Meteoriten, den Stein von Chassigny repräsentiert. Nach G. Rose bildet dieser eine kleinkörnige, fast gleichartige etwas zerreibliche Masse von grünlichgelber ins Graue ziehender Farbe. Vauquelin fand schon, dass das Pulver von Salzsäure unter Gallertbildung zersetzt wird und fand bei der Analyse die Verhältnisse des Olivins.

Im Dünnschliffe sieht man blass gelbgrüne, beiläufig gleichgroße Körner, die fast überall enge aneinanderschließen und jene für den meteoritischen Olivin oft so charakteristischen gröberen und feineren Sprünge zeigen. (Taf. 6, Fig. 4.) Sie enthalten nur wenige bräunliche Glaseinschlüsse. Zwischen den Olivinkörnern bleiben hie und da kleine, oft dreiseitige Zwischenräume, die mit farblosem oder braunem Glase ausgefüllt sind. Diese Glaspartikel erscheinen oft als das Zentrum von radial in den Olivin verlaufenden Sprüngen. Bei stärkerer Vergrößerung bemerkt man in dem Glas oft sehr viele farblose Körnchen oder zierliche Nadeln, welche Doppelbrechung zeigen oder auch braune Kriställchen. Es ist also in vielen derselben schon eine Entglasung eingetreten.

Chromit, oft in deutlichen Oktaedern, ist ungefähr gleichförmig eingestreut. Bisweilen liegt ein kleiner Chromitkristall mitten in einem Glaspartikel. Der Schliff ist Eigentum des k. k. Min. Hofmuseums.
\clearpage
\subsection{}
\subsubsection{Chondrit.}
\paragraph{}
Hierher gehört die große Mehrzahl der steinartigen Meteoriten, womit gesagt ist, dass die Meteorsteine sowohl der Textur als dem Bestande nach meistens gleichartig sind.

In Bezug auf die Gemengteile wiederholen die Chondrite den Bestand der Amphoterite, da sie hauptsächlich aus Olivin und Bronzit bestehen, jedoch enthalten sie außerdem auch Eisen und Magnetkies in erheblicher Menge, sowie untergeordnet Chromit. Akzessorisch kommen aber auch die Minerale der Eukrit, nämlich Plagioklas und Augit vor, öfters auch farbloses Glas (dem Maskelynit ähnlich), braunes Glas, ein doppelbrechender noch nicht bestimmter Gemengteil und ein bisher nicht erwähnter, nämlich Kohle.

Das feinere Gefüge ist wesentlich durch das Vorkommen der Chondren charakterisiert. Dasselbe schwankt aber zwischen den folgenden Extremen

a. Vollkommen chondritisch, wenn von Grundmasse fast nichts zu bemerken ist, die Chondren fast allein herrschen.

b. Tuffartig, erdig, wofern die aus kleinen Splittern bestehende Grundmasse vorwiegt und zuweilen auch Gesteinbruchstücke vorkommen.

c. Halbglasig, wenn der Stein eine dichte schimmernde Masse darstellt.

d. Kristallinisch, wenn der Meteorit fast gänzlich aus festgefügten Körnern besteht und die Chondren sehr zurücktreten.

Die Grundmasse besteht aus Splittern oder Körnern, von welchen die ersteren sich nicht immer genauer bestimmen lassen. Unter den Splittern sind häufig auch Bruchstücke von Chondren deutlich erkennbar. Außer den durchsichtigen Partikelchen kommen immer auch opake vor, welche sich als Eisen, Magnetkies, zuweilen auch als Chromit erkennen lassen. Wenn Kohle auftritt, so ist sie meistens in der Grundmasse gleichförmig verteilt, so dass dieselbe schwarz erscheint. Auch der Magnetkies verbreitet sich zuweilen als Imprägnation stellenweise in der Grundmasse.

Außer den Splittern und Körnern, welche sich auf Olivin und Bronzit, zuweilen auch auf Augit beziehen lassen, finden sich in der Grundmasse oft auch kleine rundliche farblose Körner, die entweder einfach brechend sind, so dass man die Wahl hätte, sie als Glas oder Maskelynit anzusprechen, oder doppelbrechend sind und gewöhnlich eine undulöse Auslöschung zeigen, so dass man mit einiger Wahrscheinlichkeit Plagioklas annehmen darf. Manche dieser Körnchen zeigen aber im polarisierten Lichte Zwillingslamellen in Wechselstellung in genau derselben Weise wie die triklinen Feldspate, daher die Bestimmung als Plagioklas wohl eicher ist. Da letztere Körnchen dieselbe Form, Größe und Verwachsung mit der Umgebung zeigen wie die vorigen, so ist es wahrscheinlich, dass alle drei Arten substantiell gleich, also die doppelbrechenden auf Plagioklas, die einfach brechenden auf Maskelynit zu beziehen sind. Alle diese Körnchen und Körnergruppen sind mit der Umgebung innig verwachsen und schließen oft Partikelchen der anderen Silikate ein, sie sind niemals zersplittert. Daraus lässt sich schließen, dass sie später gebildet seien, und zwar nach der Ablagerung des Gesteintuffs.

Außer diesen Körnern finden sich in der Grundmasse auch scharfkantige größere Körner, welche gleichfalls mit der Grundmasse innig verbunden erscheinen. Sie sind fast farblos, lassen Spuren einer unvollkommenen Spaltbarkeit, öfters auch viele zarte, krumm verlaufende Sprünge erkennen. Zwischen gekreuzten Nicols geben sie nur geringe Aufhellung, einen grauen Farbenton, jedoch niemals schönere Interferenzfarben. Durch dieses Verhalten, die Farblosigkeit und die zarten Sprünge sind sie von allen übrigen durchsichtigen Gemengteilen leicht zu unterscheiden. Im konvergenten Lichte lässt sich konstatieren, dass sie zweiaxig sind. Ich konnte diese Körner bisher mit keinem bekannten Mineral identifizieren.

Die Chondren, welche schon von Reichenbach\footnote{Poggendorffs Annalen, Band 111, p. 353.} und G. Rose als merkwürdige Bildungen hervorgehoben wurden, sind eine charakteristische Form, in welcher dieselben Körper, die auch in der Grundmasse vorkommen, einzeln oder gemengt auftreten. Ihr Zusammenvorkommen ist aber ein anderes, als jenes der Gemengteile in echt kristallinischem Gestein, indem nicht Chondren von bestimmter Beschaffenheit nebeneinander liegen, sondern die verschiedenartigsten im selben Gemenge angetroffen werden, in der Art wie die verschiedenartigsten Minerale in einem Tuff beisammen liegen. Beispiele geben Taf. 7, Fig. 1 bis 4.

Die Größe der Chondren ist variabel, manchmal werden solche von Walnussgroße beobachtet, zuweilen wiederum solche von staubartiger Kleinheit. Am häufigsten sind sie etwa hirsekorngroß. Die Oberfläche ist meistens etwas rau bis höckerig, seltener glatt (an harten faserigen Kügelchen).

Die äußere Form der Chondren ist verschieden und wechselt in einem und demselben Meteoriten. Von den vollkommen runden Chondren bis zu den unförmlichen Stücken lassen sich alle Übergänge wahrnehmen.

Manche Chondren erscheinen kugelrund. (Tafel 7 und 8.) Sind solche Chondren sehr fest und die Grundmasse locker, so können erstere leicht herausgenommen werden, worauf sie eine Runde Höhlung hinterlassen. Die lockeren zerbrechen dagegen oft, wenn man sie zu isolieren versucht. Die festen runden Chondren zeigen oft eine merkwürdige Deformation. Sie bieten dann runde Aushöhlungen dar, so, als ob sich an dem noch weichen Kügelchen ein anderes hartes abgeformt hätte.\footnote{Zuerst in dem Stein von Tieschitz beobachtet. Denkschriften d. Kais. Ak. d. Wiss. z. Wien. Math. naturw. Kl. Bd. 39. p. 187.} Auf Tafel 7, Fig. 3 sind die Durchschnitte zweier solcher Kügelchen dargestellt. Andere Chondren sind abgeplattet oder länglichrund, wieder andere zeigen Vorsprünge und Einbuchtungen. An diese schließen sich bezüglich der Form jene an, welche lappig oder fetzenartig erschienen. Derlei Chondren lassen sich meistens nicht unverletzt aus der Grundmasse nehmen, daher sich die Gestalt meist nur aus den Umrissen des Durchschnittes ergibt. Auf Tafel 7 sind Beispiele solcher Formen gegeben. An jene Chondren, welche wie abgerundete Splitter aussehen, schließen sich endlich jene, welche als größere Gesteinstücke mit rundlichen Kanten erscheinen und so das Extrem der Chondrenbildung darstellen, wie in dem Stein von Alexinaé. Die Kontouren der Chondren sind nicht immer scharf, daher bei der mikroskopischen Beobachtung die Grenze gegen die Grundmasse ganz oder teilweise undeutlich erscheint. Bei flüchtiger Betrachtung werden viele Chondren leicht übersehen und mit Grundmasse verwechselt, scharfe Kristalle als der letzteren zugehörig betrachtet, während sie einem porphyrischen Chondrum angehören \emph{etc.}

Da die Gemengteile der Chondren im Allgemeinen dieselben sind, wie jene der Grundmasse, so zeigt sich auch öfters wenig Unterschied in der Farbe der beiden, besonders bei den grauweißen, wie Milena, Alfianello, häufig aber sind die Chondren desselben Steines ungleich gefärbt, die einen weiß oder grau, die anderen braun oder schwarz und auch die Grundmasse unterscheidet sich öfters durch die Farbe von jenen, besonders in den Steinen mit schwarzer Grundmasse, wie Renazzo.

Die Textur der Chondren ist mannigfaltig. Manche bestehen aus einem einzigen Kristallindividuum, sind monosomatisch, wofern man von den darin vorkommenden Einschlüssen absieht. Viele bestehen aus mehreren Individuen derselben Art, sind polysomatisch und erscheinen körnig oder blätterig, stängelig, faserig. Die gemischten, aus mehrerlei Gemengteilen bestehenden sind wiederum körnig, blätterig, faserig oder aber porphyrisch. Bisweilen zeigt sich in solchen auch zweierlei Textur, indem z. B. ein Teil des Chondrums blätterig oder körnig, der andere faserig erscheint. Die dichten Chondren sind wohl meistens zu den gemischten zu rechnen, doch mögen auch einfache darunter vorkommen.

Obwohl die Zusammensetzung der Chondren wenig mannigfaltig erscheint, so wird doch die mikroskopische Bestimmung öfters schwierig und bei sehr kleinen Individuen zuweilen unsicher, da namentlich der Olivin der Chondrite dem Bronzit oft ungemein ähnlich ist. In der Mehrzahl der Fälle kann man aber auch dann, wenn keine Formen erkennbar sind, aus dem Charakter der Spaltlinien und der Oberfläche des Schliffes einen sicheren Schluss ziehen.

Nach den Gemengteilen und der Mischung angeordnet, ergeben sich folgende Arten des Baues der Chondren. \emph{Olivinchondren}. Monosomatische Kügelchen kommen öfters vor. Nur wenige sind frei von Einschlüssen, so dass sie ein wahres kugelförmiges Individuum darstellen. Andere sind in regelmäßiger Weise von Einschlüssen durchsetzt und erscheinen als gefächerte Kugeln. Um den Charakter derselben richtig aufzufassen, geht man von der Beschaffenheit der Olivinkristall aus, wie solche in den porphyrischen Chondren häufig scharf ausgebildet vorkommen. An diesen erkennt man im Innern öfters eine schalige oder überhaupt lückenhafte Bildung, indem sich dort eine Glasmasse ausbreitet, die in den Schnitten bald eine ungefähr sichelförmige Figur, bald mehrere solche in paralleler Stellung aufeinander folgende Figuren ergibt.

Beispiele solcher Kristalle sind in Fig. 1 auf Taf. 9 zu sehen, oberhalb ein Kristall mit einem großen Glaseinschluss in der Mitte und zwei kleineren solchen Einschlüssen, entsprechend dem schichtenartigen Baue, unterhalb ein größerer gefächerter Kristall mit mehreren tafelförmigen im Bilde ungefähr horizontalen Glaseinschlüssen. Das Glas erscheint hier und in vielen der folgenden Bilder dunkelgrau bis schwarz, während es tatsächlich braun und durchsichtig ist. In den Steinen von Borkut und von Knyahinya wurden öfter solche Kristalle beobachtet, welche aus einer vollkommen geschlossenen gleich dicken Rinde, im Inneren aber aus einem Fachwerk bestehen, welches mit der Rinde zusammenhängt und aus mehreren parallel der Fläche \emph{b} = 0 1 0 gelagerten Lamellen besteht. Zwischen diesen Lamellen ist ein braunes Glas oder eine in Entglasung begriffene Masse eingelagert, welche letztere eine Neigung zur körnigen bis faserigen Ausbildung verrät. Die Lamellenbildung im Inneren der Kristalle ist bisweilen wohl noch deutlich, jedoch nicht mehr so gleichartig, wie im letzteren Falle. Dies zeigt Fig. 3 auf derselben Tafel in einem Kristalldurchschnitte, welcher durch zwei große Glaseinschlüsse unterbrochen und einerseits geöffnet ist. Die Einschlüsse erscheinen geweihförmig, indem sie die Räume zwischen zackenförmig vorspringenden Lamellen ausfüllen.

Mit den Kristallen, welche aus einer Rinde und im Inneren aus gleichorientierten Lamellen bestehen, zwischen denen Glasmasse eingeschlossen ist, kommen die gefächerten Olivinkugeln überein, wovon eine in Fig. 2 auf Taf. 10 dargestellt ist. Die Rinde ist einheitlich gebildet und löscht gleichzeitig mit sämtlichen Olivinlamellen aus. Da in diesem Beispiele der Olivin gelbgrün, das Glas hellbraun gefärbt ist, so hebt sich im Bilde das Glas wenig ab. Wenn die Lamellen sich nicht ununterbrochen durch das Innere erstrecken, sondern immer nur auf kurze Strecken fortsetzen, so erscheint die zwischengelagerte Glasmasse netzartig wie in Fig. 4 auf Taf. 9. Die Glasmasse, welche in geeigneten Schnitten solcher Kugeln Streifen oder Netze bildet, ist seltener hellfarbig und durchsichtig, häufiger dunkelbraun oder trübe durchsichtig. Oft ist die Glasmasse spärlich und die Grenze gegen die Lamellen unscharf, so dass die Streifen und Netze bloß als eine graue Zeichnung erscheinen. Sehr häufig ist aber die Füllmasse teilweise oder fast ganz entglast. In letzterem Falle besteht dieselbe aus Glas und vielen feinen Körnchen oder Fasern. Statt des braunen Glases ist die Füllmasse in manchen Fällen ein körniger Plagioklas oder auch ein farbloses Glas (Maskelynit).

Makroskopisch sind jene Olivinkügelchen, welche durchgehende Lamellen enthalten, sehr auffallend, weil sie beim Zerschlagen Täfelchen liefern und einen vollkommen spaltbaren Gemengteil vermuten lassen. Die Angaben von Feldspat in manchen älteren Publikationen dürften sich hierauf beziehen. Ohne genauere Untersuchung würde man auch jetzt noch derlei Kugeln eher für Bronzit als für Olivin halten. G. Rose hat die parallelen Streifen, welche viele Kugeln im durchfallenden Lichte zeigen, schon beobachtet und abgebildet. Die von ihm benutzten Präparate waren jedoch, wie ich mich durch die Güte des Herrn Oberbergrates Websky überzeugen konnte, viel zu dick, als dass er den Unterschied zwischen Olivin und Glasmasse hätte wahrnehmen können.

Wenn der Schnitt, welcher Olivinkugeln von der angegebenen Beschaffenheit trifft, schief gegen die Ebene der Lamellen gerichtet ist, so wird die Regelmäßigkeit des Baues weniger deutlich hervortreten, besonders in den Kügelchen mit netzartig verteilter Glasmasse. Hierher gehört das in Fig. 4 auf Taf. 8 gegebene Bild, in welchem eine Olivinkugel mit ungewöhnlich dicker Rinde dargestellt ist und die Färbung, welche das netzartig verteilte Glas dem Inneren erteilt, hervorgehoben wird. Wenn im Inneren keine parallelen Lamellen auftreten und demzufolge das Glasnetz unregelmäßig erscheint, so gibt der Durchschnitt oft eine gekröseartige Textur wie in Fig. 1 auf Taf. 11.

Sowohl an den Kristallen von Olivin, als auch an den Kugeln ist öfters eine Einseitigkeit der Ausbildung bemerklich, wofür auf Taf. 9 die Figuren 3 und 4 Beispiele geben. Unter den monosomatischen Olivinkügelchen zeigen sich bisweilen auch solche, die zwar auch aus einem lückenhaft gebildeten Kristall bestehen, aber eine ganz andere Ausbildung desselben zeigen, indem derselbe innen kompakt, nach außen aber skelettartig oder strauchartig geformt ist und hier in den Lücken reichliche Glasmasse beherbergt. Öfters werden auch solche Kugeln beobachtet, welche gleichförmig lückenhaft gebildete Individuen sind, indem sie nach Art der gestrickten Formen aus unzähligen Stäbchen bestehen, welche scharenweise nach derselben Richtung gestreckt sind. Das ganze Skelett löscht gleichzeitig aus. Die Lücken sind durch Glas erfüllt. Ein Beispiel gibt Fig. 3 auf Taf. 10. Hier weicht nur ein kleiner Teil des Kügelchens in seiner optischen Orientierung von der Hauptmasse ab. Im Stein von Mezö-Madaras kommen monosomatische Kügelchen vor, in welchen die feinen Stäbchen der gestrickten Bildung deutlich drei auf einander senkrechte Richtungen verfolgen, welche gemäß der Auslöschung zugleich die Richtungen der Kristallaxen sind.

Die polysomatischen Olivinchondren sind mannigfaltig. Die einen reihen sich an die vorbeschriebenen gefächerten Kugeln an, indem sie aus mehreren Systemen paralleler Tafeln mit zwischenliegender Glasmasse bestehen. Fig. 4 auf Taf. 10 gibt ein hierhergehöriges Beispiel. Andere schließen sich insofern an, als sie innen einheitlich gebildet sind und hier aus abwechselnden Lamellen von Olivin und Glas bestehen, nach Außen aber polysomatisch sind, indem die Rinde aus vielen Individuen zusammengesetzt ist. Fig. 2 auf Taf. 11. Derlei Kugeln haben oft eine dunkle Rinde, indem sich daselbst Einschlüsse von Eisen und Magnetkies massenhaft einstellen. Die Bedeckung der Olivinkügelchen mit Eisen und Magnetkies, ferner die Durchtränkung der Rinde seitens dieser opaken Begleiter ist eine häufige Erscheinung, daher viele Olivinkügelchen, wie schon G. Rose bemerkte, beim Zerschlagen eine dunkle Rinde zeigen.

Viele Olivinkugeln sind porphyrisch, indem sie deutliche Kristalle in glasiger seltener feinkörniger Grundmasse zeigen. Manche haben eine dicke Rinde und schließen sich in dieser Hinsicht an die zuvor besprochenen Gebilde an, Fig. 3 auf Taf. 8 und 11, während andere kaum eine Andeutung von Rinde wahrnehmen lassen. Fig. 4 auf Taf. 7. In letzter Figur bat man ein ausgezeichnetes Beispiel der porphyrischen Struktur. Die Kristalle sind bald kompakt, bald von Glaseinschlüssen durchzogen, wie schon früher bemerkt wurde. Nach den Durchschnitten zu urteilen, ist die Form eine einfache, meistens bloß aus \emph{m} = (1 1 0), \emph{b} = (0 1 0), \emph{k} = (0 2 1) bestehende, wie an den Kristallen in den Olivinschlacken. Zwillinge wurden nicht beobachtet. Alle diese Kristalle sind von vielen feinen Rissen durchsetzt. Das optische Verhalten bietet nichts ungewöhnliches. Zuweilen sieht man in der Glasgrundmasse der Kügelchen unvollendete Olivinkristall in der Form von gabeligen Mikrolithen von zierlichen farnkrautähnlichen Gestalten, von netzartigen Häufchen oder von gröberen skelettartigen Bildungen, aus Täfelchen und Stäbchen bestehend, welche sich rechtwinkelig anordnen. Zu den letzteren gehört der in Fig. 1 Taf. 10 gegebene Durchschnitt. Die fadenförmigen oder nadelförmigen Glaseinschlüsse, welche sich hier zeigen, stimmen der Lage nach mit jenen überein, welche in manchen kompakten Kristallen bei stärkerer Vergrößerung als Stäbchen wahrgenommen werden. Obwohl Glaseinschlüsse im Olivin so außerordentlich häufig sind, so finden sich doch Libellen, welche einem eingeschlossenen Dampfe entsprechen, sehr selten. Das Beispiel einer Libelle gibt Fig. 3 auf Taf. 18.

An die porphyrischen Olivinkugeln schließen sich durch allmählige Übergänge verbunden die körnigen an. Wenn die Glasgrundmasse abnimmt, schließen sich die Kristalle enge aneinander an, doch zeigen sich noch scharfe Umrisse, wie in Fig. 2 auf Taf. 7, bei verschwindender Grundmasse bemerkt man an den Körnern selten mehr eine bestimmte Form: Fig. 2 auf Taf. 8. Diese Figur gibt auch ein Beispiel dafür, dass die körnigen Kugeln oft nach außen zu reich an Eisen und Magnetkies erscheinen und demnach eine dunkle Rinde darbieten. Häufig sind, besonders in den kohligen Chondriten höchst feinkörnige Chondren von rundlicher oder lappiger Form, welche nach der Ähnlichkeit des Gefüges mit den früher bezeichneten auch als Olivinchondren anzusehen sind. Kleinkörnige bis feinkörnige Chondren sind in Fig. 1 und 3 Taf. 7 sowie in Fig. 2 auf Taf. 20 dargestellt.

Unter den porphyrischen bis körnigen Olivinchondren sind jene merkwürdig, welche neben den Kristallen und Körnern auch eine monosomatische Olivinkugel mit Glasnetz enthalten. Solche wurden in den Steinen von Dhurmsala und Mezö-Madaras beobachtet. Fig. 1 auf Taf. 8 gibt ein Beispiel. Von der Anschauung ausgehend, dass die Chondren erstarrte Tropfen sind, wird man die Erscheinung dadurch erklären können, dass man sich vorstellt, ein kleiner schon erstarrter Tropfen sei durch einen noch flüssigen großen Tropfen umhüllt und eingeschlossen worden. Die Umhüllung einer kleinen Kugel durch eine größere wurde makroskopisch von G. Rose in dem Stein von Krasnoj-Ugol und von mir im Stein von Mocs beobachtet.

Manche der körnigen Chondren sind im auffallenden Lichte dunkel bis schwarz gefärbt und zeigen sich bei der mikroskopischen Prüfung so reich an Körnern von Eisen und Magnetkies, dass nur wenige durchsichtige Stellen übrig bleiben. Viele der schwarzen Punkte dürften auch auf Chromit zu beziehen sein.

Opake Einschlüsse sind in Olivinchondren jeder Art häufig und die Menge derselben nimmt in den körnigen gegen die Oberfläche gewöhnlich zu. Diese Einschlüsse sind vorwiegend Eisen und Magnetkies. Das Eisen erscheint oft in kleinen Kügelchen, während der Magnetkies gewöhnlich Körnchen von unbestimmter Gestalt bildet. In geringerer Menge ist Chromit verbreitet, welcher kleine schwarze Körnchen oder staubartige Häufchen darstellt.

\emph{Bronzitchondren}. Die Mannigfaltigkeit der Ausbildung ist hier geringer als bei den vorigen, die Chondren sind meist stengelig bis faserig. Monosomatische Chondren wurden nicht konstatiert. Die großen Bronzitindividuen, welche in den Chondriten beobachtet werden, sind immer mit körniger Masse verbunden, sind nur Teile einer Kugel oder gehören keiner deutlich erkennbaren Kugel an. Solche Individuen haben zuweilen eben erkennbare Kristallumrisse wie in Fig. 1 Taf. 12. Die Einschlüsse haben nichts charakteristisches. Teils sind es opake Körnchen oder Kügelchen, teils Glaseinschlüsse, welche eiförmig, fadenförmig gestaltet sind, zuweilen auch negative Kristalle ausfüllen wie die dunkelbraunen Einschlüsse, welche in Fig. 2 auf derselben Tafel dargestellt sind. Große Glaseinschlüsse, wie solche im Olivin vorkommen, fehlen im Bronzit gänzlich, Libellen wurden niemals beobachtet.

Die Bronzitchondren bestehen zuweilen aus wenigen großen Kristallen oder Körnern, zwischen welchen eine geringe Menge von Glas liegt. Nur selten ist die Glasmasse im Inneren solcher Chondren beträchtlich, wie in dem Fig. 3 auf Taf. 12 dargestellten Falle. Die großen Kristalle solcher aus wenigen Individuen bestehenden Kugeln löschen bisweilen nicht einheitlich aus, indem einzelne langgestreckte schmale Teile in der Auslöschung von dem Hauptindividuum etwas abweichen. Zuweilen treten kreuzförmige Durchwachsungszwillinge auf.

Die meisten Bronzitchondren sind stengelig bis faserig wie die Beispiele auf Taf. 13, 14, ferner Fig. 4 auf Taf. 7 zeigen. Die Anordnung der Stengel oder Fasern ist eine exzentrische. Im Dünnschliffe kommen freilich auch Schnitte vor, welche eine konzentrische Anordnung zeigen, doch tritt dieser Fall nur ein, wenn die längsten Stengel oder Fasern senkrecht getroffen werden. Öfters finden sich solche Chondren, welche zwei Systeme von exzentrischer Faserung zeigen, in den knolligen Chondren sieht man bisweilen auch mehrere solche Teile. Manche Bronzitchondren erscheinen wirrfaserig, besonders jene von lappigem Durchschnitte. Unter den wirrfaserigen sind jene auffallend, welche im Durchschnitte eine gitterartige Zeichnung darbieten, weil auf einem faserigen Grunde gröbere Stengel in verschiedenen Richtungen sich kreuzend hervortreten. Fig. 4 auf Taf. 12 gibt ein Beispiel. Die meisten Chondren haben eine Rinde. Je feiner die Fasern der Chondren sind, desto deutlicher tritt im Allgemeinen die Rinde hervor, welche immer aus vielen Individuen besteht. Es ist vielleicht nicht überflüssig, zu bemerken, dass die Olivinchondren, welche aus abwechselnden Lamellen von Olivin und Glas bestehen, Durchschnitte liefern, zu Täuschungen Veranlassung geben und für parallelstengelige Aggregate von Bronzit gehalten werden können.

Häufig sind die runden harten braunen Chondren von feinfaseriger Textur, welche eine glatte Oberfläche haben und sich leicht aus der Grundmasse herauslösen lassen, wofern diese nicht sehr fest ist. Viele derselben erscheinen fast trübe und lassen nur in sehr dünnen Präparaten bei stärkerer Vergrößerung die faserige Textur erkennen. Oft bestehen sie noch zum Teil aus braunem Glase und gewähren den Eindruck einer unvollständigen Entglasung. Alle haben eine Rinde, welche heller gefärbt und zuweilen auffallend dick ist. Fig. 4 auf Taf. 13 gibt ein Beispiel, in welchem die Faserung noch deutlich erkennbar, die Rinde dünn ist. In Fig. 3 auf Taf. 7 hat man das Bild einer fast dicht erscheinenden eingedrückten Kugel mit dicker Rinde, in Fig. 2 auf derselben Tafel das einer fast dicht erscheinenden braunen Kugel mit sehr dünner Rinde.

Von durchsichtigen Einschlüssen findet sich in den Bronzitchondren bloß ein dunkelbraunes Glas, von opaken Eisen in Kügelchen, Körnchen und Flittern, Magnetkies in Körnern oder staubartig verteilt. Von den schwarzen Einschlüssen könnte einiges auf Chromit zu beziehen sein.

\emph{Olivin-Bronzitchondren}. Viele Chondren sind nicht einfach, sondern bestehen aus den beiden Hauptgemengteilen der Chondrite, aus Olivin und Bronzit. Die einen sind körnige Mischungen mit gleicher Ausbildung der beiden Gemengteile, in welchem Falle die Bronzitkörner kaum durch das mehr faserige Ansehen von den anderen unterschieden werden. Zuweilen mischt sich noch ein wenig Augit in dünnen Prismen hinzu, wie in dem Stein von Renazzo. Andere Chondren bestehen zum Teil aus einer körnigen Masse von Olivin, zum Teil aber aus dem exzentrisch faserigen Bronzit oder die Kugel ist einerseits porphyrisch und enthält daselbst Olivinkristall in glasiger bis feinkörniger Grundmasse, anderseits wird sie von strahlig-faserigem Bronzit gebildet. Dieser nimmt zuweilen überhand und dann bestehen die Chondren vorwiegend aus faserigem Bronzit, in welchem Olivinkörner eingestreut liegen. Unter den porphyrischen Chondren kommen oft solche vor, in welchen der Olivin größere Kristalle bildet, die Grundmasse aber von kleineren Bronzitkristallen und Glas gebildet wird. Da der Bronzit die Zwischenräume einnimmt, so muss man schließen, der Olivin sei früher auskristallisiert, sei das ältere Mineral, während der Bronzit später gebildet wurde. Ein Beispiel gibt Fig. 4 auf Taf. 11. Der Bronzit, welcher von Glasmasse umgeben ist, bietet oft kreuzförmige Durchwachsungszwillinge dar, wie solche von Becke in den mineralog.-petrogr. Mitt. Bd. 7 pag. 95 beschrieben wurden. Nach später anzuführenden Beobachtungen erscheinen in dem Glase, welches in vielen Olivinchondren die Zwischenmasse bildet, infolge beginnender Entglasung viele feine Nadeln, welche wohl richtig für Bronzit zu halten sind. Solche Chondren werden demnach zu den eben angeführten gehören und deren früheren Zustand darstellen.

\emph{Augitchondren}. In manchen Chondriten finden sich Kugeln mit Kristallen oder Körnern von grünlichgrauer Färbung, welche dem Bronzit ähnlich sind, jedoch im polarisierten Lichte Erscheinungen zeigen, die auf Augit schließen lassen. In den Steinen von Renazzo und Knyahinya sind solche Kugeln häufiger als in der Mehrzahl der Chondrite. S. Taf. 15. Die Durchschnitte der Kristalle und Körner, welche günstig liegen, erscheinen im polarisierten Lichte aus Lamellen von abwechselnder Stellung und schiefer Auslöschung zusammengesetzt, was einer wiederholten Zwillingsbildung entspricht. Die Lamellen sind weder so scharf begrenzt noch so gleichförmig dick wie in den Plagioklasen, vielmehr oft absätzig oder auch etwas gekrümmt. In dem Stein von Knyahinya bemerkt man den Parallelismus der Lamellen mit den feinen Spaltlinien, daher die Zwillingsebene parallel 1 1 0 oder 1 0 0 anzunehmen ist, wovon die letztere Lage als die beim tellurischen Augit gewöhnliche die wahrscheinlichere ist. Aus der Auslöschungsschiefe ließ sich kein bestimmtes Resultat entnehmen, da nicht so viele Durchschnitte beobachtet wurden, um einerseits das Zwillingsgesetz zu bestimmen, anderseits wenn das letztere als herrschend angenommen wird, die Auslöschungsschiefe auf 0 1 0 aus den extremen Zahlen herzuleiten. In dem Stein von Knyahinya sind die Augitbündel in den Chondren häufig durcheinander gewachsen, während sie in dem Steine von Renazzo einzeln nebeneinander liegen und oft durch Glas getrennt sind, so dass eine körnige bis porphyrische Struktur entsteht. Manche der Chondren enthalten spreuförmigen Augit, an dem keine bestimmten Kristallumrisse zu bemerken sind, wohl aber wiederum die Zusammensetzung aus Zwillingslamellen wahrgenommen wird. Im Stein von Renazzo ist der Augit von Körnern begleitet und schließt auch Körner ein, welche nach dem Verhalten im polarisierten Lichte und dem Mangel an deutlichen feinen Spaltrissen als Olivin bestimmt wurde.

\emph{Plagioklas-Chondren}. Es ist eine seltene Erscheinung, dass die Hauptmasse einer Kugel aus jenem Gemengteil besteht, welcher nach seinem optischen Verhalten als Plagioklas bestimmt wurde. Ein hierher gehöriger Fall ist auf Taf. 16 in Fig. 4 dargestellt. Hier bildet körniger Plagioklas die Zwischenmasse, welche die Räume zwischen parallelen Lamellen und Stäbchen von Olivin ausfüllt. Dieser gibt dem Gefüge den Charakter, obwohl der Plagioklas bei weitem überwiegt. Während in den meisten Fällen die hauptsächlich aus Olivin bestehenden Fächerkugeln als Zwischenmasse Glas oder feinkörnige Grundmasse, in wenigen Fällen Plagioklas enthalten, ist hier das Verhältnis völlig umgekehrt; der Plagioklas tritt vor und das Fächerwerk des Olivins bildet ein zartes Gerippe. Die Figur erinnert an den auf Taf. 10 in Fig. 2 abgebildeten Durchschnitt, jedoch ist daselbst die Zwischenmasse ein Glas, welches an Menge hinter dem Olivin zurückbleibt.

\emph{Glas-Chondren}. Zuweilen kommen Kugeln vor, welche bloß aus Glas bestehen, häufiger solche, in welchen Glas vorherrscht und die Kristallbildungen zurücktreten. Beide Fälle wurden in dem Stein von Mezö-Madaras beobachtet, der letztere ist auf Taf. 18 in Fig. 1 dargestellt. Das bräunliche Glas bildet die Hauptmasse, in welcher lange unvollkommen ausgebildete Olivinkristall verstreut sind. Diese verraten durch ihre Gabelung die Tendenz zur Bildung gefächerter Individuen. Außerdem erkennt man als Entglasungsprodukte farnkrautähnliche Mikrolithe, einzelne feine doppelbrechende Nadeln und netzförmige Häufchen von rechtwinkelig angeordneten Nadeln derselben Art. Letztere dürften auf Bronzit zu beziehen sein. Halbglasige Chondren werden in vielen Steinen angetroffen. Oft ist die Entglasung ziemlich gleichförmig vorgeschritten, wofür Fig. 4 auf Taf. 17 ein Beispiel gibt. Die im durchgehenden Lichte blassblaue Kugel besteht nur zum Teile aus Glas, im übrigen aus doppelbrechenden Körnchen und Flittern ohne scharfe Umrisse, die gleichsam mit der Grundmasse verfließen. Das Entglasungsprodukt und die dünne doppelbrechende Rinde scheinen aus Olivin zu bestehen, die Grundmasse dürfte ein Feldspatglas sein. Eine fast vollständige Entglasung bieten auch die öfters vorkommenden braunen Bronzitkügelchen, welche eine ungemein zarte exzentrische Faserung besitzen, die erst im polarisierten Lichte deutlicher wird und welche bisweilen auch noch sternförmige Flocken von Mikrolithen enthalten, wie solche in Fig. 1 auf Taf. 19 erscheinen.

Sowie in den Glaskugeln, zeigt sich auch in der Zwischenmasse der Olivinchondren die Entglasung durch Entstehung feiner Nadeln sehr häufig. Einen hierher gehörigen Fall gibt Fig. 2 auf Taf. 18. Da sich von der Bildung der Nadeln bis zur Einlagerung deutlicher Bronzitkristalle die Übergänge beobachten lassen, so hat man die Nadeln als die Anfänge von Bronzitkristallen anzusehen, das braune Glas vorwiegend als Bronzitglas zu betrachten. Gewöhnlich bilden sich aber in der Zwischenmasse keine größeren Kristalle, vielmehr führt die Entglasung bloß zur Bildung einer trüben aus unzähligen Mikrolithen bestehenden Masse. Die Zwischenmasse ist also in den meisten Chondren trübe durchscheinend. Öfters besteht die Zwischenmasse aus einem dunkelbraunen bis beinahe schwarzen Glase, welches fast undurchsichtig ist. Nicht selten ist ein Teil der Zwischenmasse lichtbraun, ziemlich durchsichtig und zum Teil entglast, im übrigen tiefbraun, fast undurchsichtig.

\emph{Schwarze Chondren}. In den Chondriten finden sich nicht selten schwarze, im Bruche matte Kugeln, welche Olivin und außer diesem öfters auch Bronzit enthalten und durch eine große Menge von Einsprenglingen, die im durchfallenden Lichte schwarz erscheinen, so dunkel gefärbt sind. Die Einsprenglinge sind Eisen, Chromit, Magnetkies oder schwarzes Glas. Derlei schwarze Kugeln kommen zugleich mit anderen vor, welche im auffallenden Lichte ziemlich dunkel gefärbt erscheinen und fallen daher nicht besonders auf. Beispiele dafür bieten die Steine von Knyahinya, Mezö-Madaras, Lancé, Renazzo. Verschieden von diesen sind aber jene tiefschwarzen Kugeln, welche im Bruche Glasglanz bis Fettglanz haben, in den weißlichen Chondriten einzeln vorkommen und für diese charakteristisch sind. Die Steine von Alfianello, Chateau Renard, Milena, Mocs liefern gute Beispiele. Diese schwarzen Kugeln bestehen hauptsächlich aus Maskelynit oder aus körnigem Plagioklas. Sie sind gegen die Oberfläche zu durchsichtig und farblos und enthalten hier nur wenige schwarze Einsprenglinge. Im Inneren aber sind sie voll von eckigen bis rundlichen, im durchfallenden Lichte schwarzen Körnchen. Letztere gehören mindestens zu einem Teile dem Magnetkies an, da bei auffallendem Lichte mehrere Stellen den Glanz und die braune Farbe zeigen, welche dem Magnetkies zukommen. Ein hierher gehöriges Beispiel ist auf Taf. 17 in Fig. 3 dargestellt.

\emph{Eisenchondren}. Kugeln, die wesentlich aus Eisen bestehen, sind in den Chondriten nicht häufig. Vollkommen runde Chondren dieser Art werden in den Steinen von Renazzo, Mezö-Madaras, Borkut, Dhurmsala beobachtet. Die Eisenkugeln in Renazzo haben öfters eine schwache unvollständige Rinde, welche von braunem Glase oder Bronzit gebildet wird. Sie stellen sich, da viele Silikatkugeln desselben Steines einen runden Eisenkern einschließen und dieser bald klein ist, bald an Menge die Silikate übertrifft, als der Endpunkt einer Reihe von eisenhaltigen Chondren dar. Rundliche Eisenklümpchen, welche oft mit Magnetkies verbunden erscheinen, sind in vielen Chondriten, z. B. in jenen von Mocs, Barbotan, Lucé, Klein-Wenden enthalten. Sie bestehen, wie G. Rose zeigte (p. 87), bald aus einem, bald aus mehreren Individuen, was nach dem Ätzen der Durchschnitte leicht erkannt wird. In vielen Chondriten kommen Olivinkugeln mit Rinden vor, in welchen letzteren das Eisen in Gestalt feiner Tröpfchen verteilt ist und der Rinde ein dunkles Ansehen gibt. Ein Beispiel hat man auf Taf. 8 in Fig. 2. Dies führt zu der nicht seltenen Erscheinung einer gänzlichen Einhüllung von Olivinchondren durch eine Eisenschale, wie dies in Fig. 3 auf Taf. 19 zu sehen ist. Die Eisenhülle ist in solchen Fällen teils kompakt, teils schwammig.

Chondren von Magnetkies habe ich in den Chondriten nicht beobachtet, obwohl öfters größere Klümpchen vorkommen. Während also von Olivin als dem schwerste schmelzbaren, folglich am frühesten erstarrenden Gemengteil am häufigsten Chondren gebildet werden und der Bronzit sich anschließt, sind die Chondren des bei der Abkühlung später erstarrenden Nickeleisens seltener und von dem erst bei verhältnismäßig niederen Temperaturen erstarrenden Magnetkies wurden gar keine Chondren gebildet.

Gruppen von Chondren wurden bisher in keinem Meteoriten wahrgenommen, doch zeigen sich als eine seltene Erscheinung Doppelchondren, wovon Fig. 2 auf Taf. 19 ein Beispiel gibt. Hier sind zwei monosomatische Olivinkugeln in paralleler Stellung so verwachsen, dass die größere die Einbuchtung enthält, in welcher die kleinere liegt. Dieses Vorkommen erinnert an jene meist aus feinfaserigem Bronzit bestehenden Kugeln, die eine Einbuchtung zeigen, welche wie die Abformung einer zweiten Kugel aussieht (vergl. Fig. 3 auf Taf. 7).

Wie schon früher bemerkt wurde, sieht man außer den vollständigen Chondren fast immer auch Bruchstücke derselben und zwar am deutlichsten in solchen Steinen, deren Chondren scharfe Umrisse zeigen, während in jenen, deren Chondren sich wenig von der Grundmasse abheben, auch das Vorkommen von Bruchstücken schwer zu konstatieren ist. Der ungeübte Beobachter kommt oft in Versuchung, die Bruchstücke für ursprüngliche Bildungen zu halten und ihre Formen unrichtig zu deuten. Dies kann bei den blätterigen Olivinchondren eintreten, deren Bruchstücke als blätterige Tafeln erscheinen, auch bei den exzentrisch-radialfaserigen Bronzitchondren, deren Bruchstücke nicht selten spitz-pyramidale Formen erkennen lassen und eine Ähnlichkeit mit Hagelkörnern darbieten. Beispiele liefern die Steine von Mezö-Madaras, Barbotan, Knyahinya. Auch die Splitter der porphyrischen, körnigen und dichten Chondren können zu Täuschungen Veranlassung geben, da sie wie Einschlüsse einer fremden Gesteinsart aussehen. Die zusammengehörigen Bruchstücke derselben Kugel findet man fast niemals neben einander, woraus zu schließen ist, dass die Zersplitterung schon vor der Ablagerung des Meteoritentuffs stattfand. Zerschlagene Chondren sind in den Steinen von Barbotan, Cabarras City, Chateau Renard, Knyahinya, Pultusk, Tipperary und vielen anderen häufig. Ein Beispiel von Chondrensplittern gibt Fig. 4 auf Taf. 19.

Die \emph{Grundmasse} der Chondrite oder jene Masse, welche außer den deutlich erkennbaren Chondren und Chondrensplittern vorhanden ist, bedingt nächst diesen durch ihre wechselnde Menge und ihre Beschaffenheit das äußere Ansehen der Chondrite.

Viele Steine, deren Chondren scharf ausgebildet sind und eine bedeutende Festigkeit besitzen, enthalten fast gar keine Grundmasse. Sie bestehen fast nur aus Chondren und aus deren Bruchstücken, ihr Ansehen ist demnach ein vollkommen chondritisches. Das spärliche Bindemittel besteht vorzugsweise aus Flittern und Körnchen von Eisen und Magnetkies und was außerdem von staubförmiger Silikatmasse hinzukommt, ist bei mehreren dieser Steine durch die opake Beimengung dunkel gefärbt, so dass die Chondren und Splitter umso deutlicher hervortreten. Ein ausgezeichnetes Beispiel ist der Stein von Borkut, in welchem die Chondren meist vollkommen rund und sehr fest sind, während das in geringer Menge vorhandene Bindemittel wenig zusammenhält, demnach die Chondren leicht auseinanderfallen, ferner der leicht zerreibliche Stein von Ornans, dessen Chondren von staubartiger Kleinheit und nur lose verbunden sind. In anderen Steinen wie in dem von Mezö-Madaras ist das spärliche Bindemittel fester und ziemlich dunkel gefärbt (s. Taf. 7, Fig. 2 und Taf. 19, Fig. 1 und 4). Auch in dem Stein von Tieschitz erscheint die Grundmasse dunkel und noch reichlicher vorhanden (s. Taf. 7, Fig. 3). Wenn das Bindemittel nicht so dunkel und dabei spärlich entwickelt ist, so erscheinen die Chondren und Splitter enge aneinandergepresst, wie im Stein von Knyahinya. An vielen Punkten zeigt sich dann kein Bindemittel. Wo aber solches vorhanden ist, bemerkt man öfter die Anzeichen eines später entstandenen Kittes in der Form von kleinen farblosen doppelbrechenden Pünktchen und Häufchen, welche undulöse Auslöschung und im polarisierten Lichte dasselbe Aussehen darbieten wie der Plagioklas in den früher bezeichneten Plagioklaskugeln. Diese Pünktchen und Häufchen erfüllen Lücken in der Bindemasse, sind mit dieser innig verschmolzen, schließen Körnchen von Olivin ein und sind häufig mit Magnetkies verbunden, ganz ebenso wie in den später zu bezeichnenden weißgrauen tuffartigen Chondriten, nur sind sie sparsamer entwickelt. Außerdem sieht man in der Bindemasse zuweilen jenes farblose Silikat, welches nicht mit dem Plagioklas übereinstimmt und welches in der Tafelerklärung als ein dem Monticellit ähnlicher Gemengteil bezeichnet ist. Auf Taf. 14 ist in den Fig. 3 und 4 das Auftreten desselben in dem Stein von Knyahinya charakterisiert. Hier schmiegt sich dasselbe an die Chondrensplitter an und umgibt dieselben zum Teile. Eine genauere Bestimmung der qualitativen Zusammensetzung dieses sparsam vertretenen Gemengteilen ist mir, wie gesagt, bis jetzt nicht gelungen. Zu den Steinen mit hellem festem aber nicht reichlich vorhandenem Bindemittel gehören außer jenem von Knyahinya auch die von Dhurmsala, Chateau Renard, Cabarras City, Tipperary u. a. m. In manchen, wie beispielsweise in dem von Ensisheim, wechselt helles Bindemittel mit dem dunklen, daher die Masse auf dem polierten Durchschnitte marmoriert aussieht.

Einige Chondrite haben vollständig das Aussehen eines klastischen Gesteines mit Tuffgrundmasse, z. B. der Stein von Alexinaé (Sokobanja). Eine aschgraue bis gelblichgraue Masse von erdigem Bruche, die aber nicht locker, sondern ziemlich fest ist und beim Anschlagen fast wie ein Backstein klingt, schließt nicht nur Chondren und deren Splitter, sondern auch kleinere und größere bis 15 cm lange eckige Bruchstücke eines dunkleren gefrittet aussehenden chondrischen Gesteines, ferner scharfkantige Bruchstücke von körnigem Magnetkies ein. Bei der mikroskopischen Beobachtung erscheint die Hauptmasse vorzugsweise aus Chondrensplittern zusammengesetzt und die Grundmasse aus dem entsprechenden Staube. Auch hier stellen sich die farblosen doppelbrechenden Punkte und Häufchen ein, von welchen aber manche zwischen gekreuzten Nicols zwickelförmige, zwillingsartig verbundene Individuen erkennen lassen, ähnlich wie der Tridymit im Rittersgrüner Meteoriten, so dass man in diesem Falle an der Bestimmung als Plagioklas irre wird. Die gefrittet aussehenden Gesteinsbruchstücke zeigen unter dem Mikroskope Chondren, welche mit der undeutlich körnigen Grundmasse verfließen und in den enthaltenen Kristallindividuen reichliche, wahrscheinlich sekundär gebildete Glaseinschlüsse, wie z. B. in dem auf Taf. 12 in Fig. 2 abgebildeten Bronzit. Auch der Stein von Siena zeigt an vielen Stellen ein deutlich klastisches Gefüge, da sowohl Chondren und deren Splitter als auch Bruchstücke von gefrittetem und solche von schwarz imprägniertem Gestein vorkommen. An die deutlich klastischen Chondrite mit heller Grundmasse schließen sich die weißgrauen Chondrite, welche eine ziemliche Reihe, wie z. B. die Steine von Alfianello, Girgenti, Mauerkirchen, Milena, Mocs, Tourinnes la Große umfassen. In der hellen matten tuffartigen, aber ziemlich festen Masse sieht man mit freiem Auge bald häufiger, bald seltener deutliche Kugeln, ferner kleine bis größere Körner von Magnetkies und Flitter bis Klümpchen von Eisen. Mikroskopisch zeigt die Masse wenige deutliche Chondren und fast gar keine deutlichen Chondrensplitter, im Übrigen ein undeutlich körniges Haufwerk, in welchem alle Körnchen von Sprüngen durchzogen sind und in welchem die einzelnen Partikel durch ihr Gefüge doch wieder an Teile von Chondren erinnern. Die Bilder in Fig. 3 auf Taf. 16, Fig. 2 auf Taf. 17, Fig. 1 und 2 auf Taf. 21 geben eine Vorstellung von der Undeutlichkeit dieses Gemenges, in welchen bloß stellenweise die Chondrentextur erkennbar ist. Magnetkies und Eisen sind allenthalben zerstreut. Wenn der Magnetkies eine Kugel umhüllt, wie in Fig. 1 auf Taf. 11 oder einen Kristall umgibt wie in Fig. 1 auf Taf. 12, so treten dieselben ausnahmsweise scharf aus der Umgebung hervor. In der Grundmasse erblickt man nicht selten jene farblosen doppelbrechenden Punkte und Häufchen, welche meist nur eine undulöse Auslöschung und bloß hie und da im polarisierten Lichte die abwechselnden Streifen zeigen, welche die Plagioklase charakterisieren. Fig. 2 auf Taf. 16 stellt eines jener Körnchen im Stein von Mocs dar, in welchem ich zuerst die Zwillingsstreifung an diesem Gemengteil der Chondrite bemerkte. Die einzelnen Körnchen und die Häufchen sind mit der Grundmasse innig verbunden, sie verzweigen sich meistens in derselben, füllen Lücken aus, schließen Grundmasse in der Form kleiner Körnchen ein (Fig. 3 auf Taf. 16), verhalten sich also in dem Gemenge wie eine zuletzt entstandene Imprägnation. Sie siedeln sich besonders häufig in der Umgebung der Chondren an und verbinden dort die feinste Grundmasse. Wahrscheinlich ist der Plagioklas auch unmerklich zwischen den Körnern und Splittern der Grundmasse verteilt und bedingt zum Teile die Festigkeit des Ganzen.

Am häufigsten beobachtet man den Plagioklas in den weißgrauen Chondriten, wie in jenen von Alfianello, Girgenti, Mauerkirchen, Milena, Mocs, Tourinnes la Große, aber auch in den grauen Chondriten, wie in jenen von Aigle, Ausson, Chantonnay, Dhurmsala, Ensisheim, Groß-Divina, Knyahinya, Lissa, Mezö-Madaras, New-Concord, Pultusk, in den klastischen wie Alexinaé, Siena und in den kristallinisch aussehenden wie Erxleben, Murcia ist er vertreten. In dem letzteren fand ich den Plagioklas sparsam in scharfkantigen Splittern (s. Taf. 16, Fig. 1). Von den farblosen Körnern und Häufchen, welche das zuvor beschriebene Vorkommen zeigen, sind jedoch manche einfachbrechend. Da dieser isotrope Gemengteil genau dieselbe Form und Verteilung zeigt wie der Plagioklas und im gewöhnlichen Lichte denselben Eindruck macht wie dieser, so glaubte ich aus dieser auffallenden Gleichheit der äußeren Form auf eine Gleichheit der chemischen Zusammensetzung schließen und denselben für Maskelynit halten zu dürfen. In dem Chondrit von Alfianello kommen sowohl doppelbrechende als auch isotrope Körner vor, endlich auch solche, die nur teilweise aufhellen, so dass hier ein Übergang vom Plagioklas zum Maskelynit vorzuliegen scheint. Dies führt dazu, die isotropen Körner als umgeschmolzenen, also durch Erhitzung isotrop gewordenen Plagioklas anzusehen, wovon schon früher (pag. 7) gesprochen wurde. Dies wird noch dadurch bekräftigt, dass die farblosen Körner in der Schmelzrinde des plagioklashaltigen Chondrits von Mocs isotrop sind. In dem Chondrit von Alfianello, der die Bilder Fig. 1 und 2 auf Taf. 17 geliefert hat, zeigt der Maskelynit im gewöhnlichen Lichte öfters feine untereinander parallele Striche, welche durch das ganze Korn laufen, in größeren Körnern bloß eine Strecke weit anhalten und Kanten parallel sind, welche an dem Korn wahrgenommen werden. Diese Striche sind bloß durch eine Verschiedenheit der Lichtbrechung hervorgerufen, da sie nicht bei jeder Richtung des durchgehenden Lichtes auftreten. Sie erinnern an die Lamellierung im Plagioklas. Der Maskelynit ist viel seltener als der Plagioklas. In dem Chondrit von Chateau Renard ist bloß Maskelynit und kein Plagioklas zu erkennen. So wie der Plagioklas und der Maskelynit sind auch das Eisen und der Magnetkies in den meisten Chondriten mit der Grundmasse innig verbunden. Sie schmiegen sich an die Körnchen derselben, füllen Lücken aus, umhüllen die Chondren, verhalten sich also wie eine später gebildete Imprägnation.

In den grauweißen Chondriten beobachtet man auch am häufigsten die schwarzen Klüfte, welche im Querschnitte als Adern erscheinen, sowie auch jene breiteren gangförmigen Massen, welche beiden Bildungen jedoch in den grauen Chondriten ebenfalls oft vorkommen. Im Dünnschliffe sind die Adern im durchfallenden Lichte schwarz. Sie durchsetzen das Gestein bald in gerader Richtung, bald sind sie unregelmäßig gekrümmt, bald einzeln, bald verästelt und netzartig. Sie weichen meist den harten Chondren aus und vereinigen sich gern mit den Körnern von Magnetkies (Fig. 1 auf Taf. 22). Eine Verschiebung der Wände habe ich daran selten beobachtet. Im auffallenden Lichte sieht man im Inneren der Adern äußerst zarte Eisenblätter, welche der Richtung der Adern parallel liegen, im Querschnitte also wie ungemein dünne Fäden erscheinen, überdies auch zuweilen Tröpfchen von Eisen und Magnetkies. Die Hauptmasse ist schwarz, fast matt, spröde. Sie scheint eine halbglasige Masse zu sein. Öfters schließt sie Splitter des Gesteines ein. Beim Zerschlagen trennt sich das Gestein öfters nach den Klüften und auf diesen erblickt man sodann einen ziemlich deutlich glänzenden striemigen Harnisch, welcher durch die genannten Eisenblätter hervorgerufen ist. In dem Stein von Ställdalen sind Adern, welche Harnische liefern, sehr häufig. In einem Stücke des Steines von Murcia fand ich Klüfte, die vollständig mit Eisen gefüllt sind, welches im Querbruche die tesserale Spaltbarkeit erkennen lässt. In manchen Exemplaren der Chondrite findet sich ein enges Netz von Adern oder es erscheint eine breitere gangförmige schwärzliche Masse, wie ich eine solche mit scharfer Begrenzung in dem Stein von Orvinio beobachtete.\footnote{Die Trümmerstruktur der Meteoriten von Orvinio und Chantonnay. Sitzungsber. d. Wiener Ak. Bd. 70 Abt. 1. November 1874.} Hier sind hellere Bruchstücke der chondritischen Masse von einer dunklen dichten Masse, welche Eisenblättchen enthält and Fluidalstruktur zeigt, umgeben und durch diese verbunden. Auch in dem Stein von Chantonnay sind Bruchstücke des Chondrits durch eine schwarze Bindemasse vereinigt und durch dieselbe zum Teil imprägniert. Die schwarze Masse zeigt hier ein feines unregelmäßiges Netz von Eisen. Die benachbarte Silikatmasse hat viel Maskelynit und an einigen Stellen sind Splitter von Olivin, welche von der Masse umschlossen erscheinen, teilweise oder ganz verglast und verhalten sich isotrop. Unter den Steinen von Mocs sind solche, die von einer schwarzen Masse gangartig durchsetzt werden, nicht selten.\footnote{Sitzungsber. d. Wiener Ak. Bd. 85 Abt. 1. März 1882.} Die Mächtigkeit der letzteren beträgt bis 19 mm. Die schwarze Masse ist dicht, halbglasig, spröde. Im durchfallenden Lichte erscheinen darin viele Splitter des Nebengesteines. Die Grenze gegen das Nebengestein ist zuweilen scharf, doch zeigt sich oft ein allmähliger Übergang durch Imprägnation des letzteren (Fig. 2 auf Taf. 22). Wo die schwarze Masse kompakt erscheint, bemerkt man im auffallenden Lichte viele Kügelchen von Eisen, auch rundliche langgestreckte Eisenklümpchen, endlich feine, der Längsrichtung des Ganges entsprechend gestreckte Eisenfäden, welche die Querschnitte dünner Eisenblätter sind. Diese Eisenfäden bringen den Eindruck einer Fluidalstruktur hervor. Von den Eisenklümpchen gehen öfter feine Eisenadern aus, welche die Gangmasse quer durchsetzen. Diese Adern endigen bisweilen in leere Querklüfte. Solche leere Sprünge zeigen sich auch in den entsprechenden Massen der Chondrite von Chantonnay und Orvinio. Statt des Eisens bildet auch der Magnetkies Kügelchen in der schwarzen Masse. Diese besteht nebst Eisen und Magnetkies auch aus einem schwarzen Glase. Weil dieselbe sehr viele Splitter des Nebengesteins enthält, so zeigt sie im Bruche nur geringen Glanz. Die Gangmasse ist demnach zum geringen Teil eine Injektion, die aus Eisen, Magnetkies und Glas besteht, zum größeren Teil eine Imprägnation der Grundmasse des Chondrits. Die gangartigen Massen stehen mit den Adern in Verbindung, welche sich als Apophysen der vorigen darstellen. Das Ganze macht den Eindruck, als ob sich die chondritische Masse an ihrer Ablagerungsstätte durch rasche Erhitzung zerklüftet und als ob sie eine Schmelze bis in die feinsten Kapillarspalten aufgesogen hätte. Unter den vielen Exemplaren der Steinregen von Pultusk\footnote{Zeitschr. d. niederrh. Ges. f. Natur- und Heilkunde zu Bonn 1868.} und von Mocs sind einzelne gefunden worden, welche sich im Bruche durch schwarze Farbe und größere Härte von den übrigen unterscheiden, so dass es den Anschein hatte, als ob diese Stücke aus einer fremden Masse bestünden. Dieselbe stimmt jedoch vollständig mit der eben geschilderten gangartigen Masse überein.

Mit dieser nahe verwandt ist auch die schwarze dichte Grundmasse einiger Meteorite, z. B. jene in dem merkwürdigen Stein von Goalpara, der im Durchschnitte ein porphyrisches Aussehen zeigt, weil Enstatitkristalle in einer unvollkommen chondritischen Olivinmasse liegen, die zugleich löcherig ist (s. Taf. 20, Fig. 3). Eine schwarze, fast halbglasige Masse imprägniert die feine Grundmasse, bildet die Wände der Löcher und Spalten, umgibt die Häufchen der Olivinkörner, dringt zwischen diese ein und verästelt sich daselbst in den feinsten Ausläufern. Die schwarze Masse enthält nach meinen Beobachtungen ein feines Netz von kristallinischem Eisen, ferner Magnetkies, Kohle und ein durch Säure zersetzbares Glas. Der Stein von Richmond enthält ebenfalls zwischen den Chondren und Splittern eine schwarze fast halbglasige Grundmasse, die sich in feinen Verästelungen bis in die feinsten Klüfte zwischen den Körnchen der Silikate verbreitet. Auch dieser Stein hat kleine Löcher, doch sind deren Wände etwas drusig. Der Stein von Tadjera von dichtem bis halbglasigem Bruche scheint eine ähnliche Grundmasse zu besitzen. Ich konnte denselben nicht näher untersuchen. Die eben besprochenen schwarzen Imprägnationen weisen auf eine Veränderung der chondritischen Masse durch Erhitzung, wobei Kohlenwasserstoffe eine Rolle gespielt haben dürften. In dem Stein von Goalpara beträgt der Gehalt an Kohlenwasserstoffen 0.85 pz.

An die Steine mit schwarzer Imprägnation reihen sich diejenigen mit matter schwarzer kohligen Grundmasse, die kein Eisen enthält. Die einen derselben, wie die Steine von Renazzo, Grosnaja bestehen vorzugsweise aus Chondren, außerdem aus harter Grundmasse und sind fester, während die anderen, welche zumeist aus einer weicheren Masse bestehen, locker erscheinen. Von den letzteren habe ich nur den Stein von Cold Bokkeveld untersuchen können. In allen diesen schwarzen kohligen Meteoriten sind die Chondren glasreich und ziemlich mannigfaltig, indem außer porphyrischen Olivinkugeln auch Bronzit-Olivinkugeln von verschiedenartiger Bildung und auch Augitkugeln vorkommen, charakteristisch für mehrere dieser Steine ist das Auftreten von feinkörnigen lappigen Chondren, welche im Durchschnitte trübe und filzähnlich erscheinen. Sie bestehen wahrscheinlich aus Olivin (s. Taf. 20, Fig. 1 und 2).

Die körnigen Chondrite, welche noch zu besprechen sind und zu denen die Steine von Cleguerec, Erxleben, Klein-Wenden, Pilistfer, Stauropol u. a. gehören, sind in ihrer mikroskopischen Beschaffenheit den weißgrauen Chondriten insofern ähnlich, als sie nicht viele deutliche Chondren enthalten. Diese sind aber makroskopisch kaum zu erkennen, da sie mit der Grundmasse verschmolzen erscheinen. Die Grundmasse erscheint u. d. M. verschmolzen körnig, die einzelnen Körner scheinen oft miteinander und mit den unscharf begrenzten Chondren zu verfließen (s. Taf. 20, Fig. 4). Das makroskopisch-körnige Aussehen der Bruchfläche wird nur zum Teile durch die Spaltflächen der Körner, zum Teile aber auch durch die zerrissenen Eisenpartikel und Magnetkieskörner hervorgebracht. Die Körner und Kristalle der Grundmasse und der Chondren und zwar die Olivine wie die Bronzite sind ungemein reich an Glaseinschlüssen (s. Taf. 18, Fig. 4). Der Charakter dieser Einschlüsse und das verschmolzene Aussehen der ganzen Masse berechtigen zu der Vermutung, dass die hier beobachteten Glaseinschlüsse nicht ursprüngliche, sondern durch eine nachträgliche Erhitzung entstandene seien. Die körnig aussehenden Chondrite wären demnach als gefrittete Gesteine anzusehen. Früher wurde schon bemerkt, dass in den eminent klastischen Chondriten wie in jenen von Alexinaé und Siena gefrittet aussehende Bruchstücke vorkommen. Das mikroskopische Bild derselben ist demjenigen fast gleich, welches die eben genannten körnigen Chondrite liefern. Dasselbe Verfließen der Chondren mit der Grundmasse und der Körner in der letzteren, dieselbe Häufigkeit der Glaseinschlüsse, welche ungemein oft negativen Kristallen entsprechen. Es fehlen bloß die Eisenpartikelchen, welche in den körnigen Chondriten häufig sind.

Zum Schlusse ist noch eine Bemerkung über die Rinde der Chondrite anzufügen. Diese ist äußerlich schwarz bis braun oder grau, fast matt und zeigt nur selten fettglänzende wie gefirnisst aussehende Punkte dort, wo Plagioklas, Maskelynit oder Augit angeschmolzen sind. Einzelne Punkte haben auch das Aussehen des Hammerschlages, wenn Eisenpartikel an die Oberfläche treten und manche erscheinen mit einem braunen Pulver überzogen, wo freiliegender Magnetkies abbrannte. Auf die Formen der Rinde, gemäß welcher an manchen Exemplaren eine Brust- und Rückenseite, sowie ein Schlackensaum unterschieden werden können, gehe ich hier nicht ein, da nur Makroskopisches und Bekanntes zu wiederholen wäre. Der Bau der Rinde, welcher zuerst von Brezina beschrieben wurde, ist merkwürdig. Die Rinde besteht oft aus drei wohl unterscheidbaren Gliedern, welche im Durchschnitte des Steines Zonen bilden (s. Taf. 21, Fig. 1 und 2). Die äußerste Rinde oder eigentliche Schmelzrinde ist glasig. Sie erscheint zum Teile schwarz, undurchsichtig, zum Teil aus einem braunen, selten einem farblosen Glase zusammengesetzt. Das braune Glas wird man von Olivin und Bronzit, das farblose von Plagioklas oder Maskelynit ableiten, da selbes nur in der Rinde solcher Chondrite vorkommt, welche die letzteren enthalten. Die schwarze Farbe ist den Resten von Magnetkies und dem Eisenoxyduloxyd zuzuschreiben. Die zweite Zone oder Saugzone besteht aus den Gemengteilen des Gesteins und stellenweise aus einer geringen Menge zwischen den feinen Klüften eingeklemmter schwarzer, brauner bis farbloser Masse. Diese Zone ist also durchsichtig. In dem Chondrit von Mocs enthält sie Maskelynit, während der übrige Stein Plagioklas und keinen Maskelynit enthält. Die dritte, innerste oder Imprägnationszone ist am breitesten. Sie zeigt wiederum die unveränderten Gemengteile des Meteoriten, jedoch sind die Silikate mit einer großen Menge schwarzer Masse imprägniert. Demnach erscheint hier die schwarze Masse von durchsichtigen Körnchen durchsprenkelt. Die schwarze Masse der Imprägnationszone zeigt im auffallenden Lichte immer viele sehr kleine gelbe Flitter, welche auf Magnetkies bezogen werden können, dagegen selten feine Adern von metallischem Eisen. Durch diese Beschaffenheit unterscheidet sich die schwarze Masse der Rinde von jener in den Adern und gangartigen Injektionen der Chondrite. An Steinen, welche wenig porös sind, wie der von Knyahinya, fehlt öfters die zweite und dritte Zone und es ist bloß die glasige Schmelzrinde zu bemerken. Dass auf die Schmelzrinde eine durchsichtige, nur wenig imprägnierte Zone folgt, ist dadurch zu erklären, dass hier die Schmelze dünnflüssig war, folglich durch die poröse heiße Masse rasch aufgesogen und weitergeführt wurde. In der dritten Zone hat sich sodann die Schmelze in dem kühleren Teile der Kruste angesammelt und ist hier erstarrt. Schnitte, welche ungefähr parallel der Schmelzrinde durch die Kruste geführt wurden, ergaben Resultate, welche den vorigen entsprechen. Man sieht wiederum die durchsichtige unmerklich imprägnierte Saugzone und die dicke dunkle Imprägnationszone aufeinanderfolgen (s. Taf. 21, Fig. 3). Zum Vergleiche mit den Chondriten wurden auch Präparate aus der Kruste des Eukrits von Juvinas versucht, jedoch gelangen bloß Parallelschliffe, die ein blasiges braunes Glas mit Splittern und Kristallen von Plagioklas darboten (s. Taf. 21, Fig. 4).\footnote{Von den Schriften, welche die mikroskopische Beschaffenheit der Chondrite behandeln, mögen hier noch folgende angeführt werden: Alfianello, Foullon, Sitzungsber. d. Wiener Ak. Bd. 88. 1. 433. Chondren, Sorby, Nature Bd. 15. p. 495. Gümbel, Sitzungsber. bayr. Ak. 1875 p. 313 und 1878 p. 14. Chondrite und Meteoriten überhaupt, Tschermak, Sitzungsber. Wien. Ak. 88. 1. 347. Wadsworth. Mem. Mus. Comp. Zoology. 11. Part. 1. Glas, Lasaulx, Sitzungsber. Niederrhein. Ges. 1882. Juli 3. Gopalpur, Tschermak, Sitzungsber. Wiener Ak. Bd. 65. 1. Februar 1872. Goalpara, T., ebend. 62. 2. Dez. 1870. Grosnaja, T., Mineralog. petrogr. Mitt. 1. p. 153. Knyahinya, Kenngott, Sitzungsber. Wiener Ak. Bd. 59. 2. Mai 1869. Lancé, Drasche, Tschermaks Min. Mitt. 1875. p. 1. Mocs, Tschermak, Sitzungsber. Wiener Ak. 85. 1. 195; Rinde, Brezina, ebendas. p. 335. St. Denis Westrem, Prinz, Leg Météorites tombées en Belgique Bruxelles 1885. Tieschitz, Makowsky und Tschermak, Denkschr. Wien. Ak. Bd. 39, p. 187. Zsadany, Cohen, Verh. d. naturhist. med. Vereines Heidelberg. 2. 2.}
\clearpage
\subsection{}
\subsubsection{Grahamit.}
\paragraph{}
Dieses Gemenge, welches sich fast wie eine Mischung von Eisen und Howardit verhält, ist bis jetzt nur durch die Massen von der Sierra de Chaco und von Mejillones vertreten. In manchen Sammlungen erscheint erstere mit der spezielleren Fundortangabe Vaca Muerta. Makroskopisch bemerkt man ein Netz von Eisen, das nach G. Rose aus vielen Individuen besteht, welche Widmannstädten'sche Figuren zeigen. Dasselbe ist von dunklen Körnern von Magnetkies (Troilit) begleitet. Mit den Fäden des Netzes verschmolzen, treten hie und da Eisenkugeln auf, die bis 5 mm im Durchmesser haben. Sie bestehen auch aus mehreren Individuen. Eingebettet in dem Netze sind Körner und Splitter, selten Kugeln von Silikaten, welche dunkelgrün bis braun, zuweilen auch weiß erscheinen. Die Silikatmasse erscheint körnig. Unter d. M. ist das Gefüge der durchsichtigen Teile wohl auch meistens körnig, doch zeigt es sich an manchen Stellen deutlich klastisch, indem Splitter von Plagioklas und anderen Silikaten von Eisen umgeben sind (s. Taf. 22, Fig. 3), selten chondritisch, da einzelne undeutliche Chondren vorkommen. Ein ziemlich großer Teil der Silikate ist Plagioklas, welcher meist Körner mit undeutlichen Kristallumrissen bildet. Im polarisierten Lichte gibt er prächtige Farben und erscheint aus ziemlich breiten Zwillingslamellen zusammengesetzt, deren Auslöschungsrichtungen auf ein dem Anorthit nahestehendes Glied hinweisen. Sehr viele dieser Plagioklase zeichnen sich durch reichlich eingestreute bräunliche kristallisierte bis rundliche Einschlüsse aus, welche durch ihre Größe und Form auffallen (s. Taf. 22, Fig. 4). Dieselben haben gewöhnlich ungefähr 0,007 mm Länge und 0,003 mm Breite, doch kommen auch solche von 0,013 mm Durchmesser vor. Eine bestimmbare Kristallform wurde daran nicht erkannt. Alle sind doppelbrechend. Die einen haben das Ansehen von Prismen und sind oft mit der längsten Axe den Plagioklaslamellen parallel gelagert, geben gerade Auslöschung, in den ungefähr quadratischen Querschnitten diagonale Auslöschung, was zugleich mit den Spaltrissen auf Bronzit hindeutet. Andere zeigen eine ungefähr monokline Form wie Titanit, außerdem gibt es viele von rundlicher Form. Die Plagioklase, welche als Bruchstücke auftreten, sind häufig frei von diesen Einschlüssen. Dieselben zeigen auch schmälere Zwillingslamellen als die körnigen. Der Plagioklas erscheint auch zuweilen dicht und enthält in diesem Falle viele sehr kleine rundliche Glaseinschlüsse (Taf. 23, Fig. 2).

Der pyroxenische Gemengteil ist zumeist Bronzit von grünlichgrauer Färbung, welcher im Längsschnitte ein faseriges Ansehen hat. Dieser bildet Körner, die zuweilen mit freiem Auge sichtbar und isolierbar sind, ferner undeutliche Kristalle, die mit Eisen umgeben oder mit Olivin oder Plagioklas verwachsen sind. Die Einschlüsse im Bronzit sind teils Körner von Magnetkies, teils wenig deutliche negative Kristalle, die von schwarzer Masse erfüllt sind oder rundliche kleine bis staubartige opake Körnchen. Feine Blättchen von schiefer Auslöschung, welche parallel 1 0 0 eingeschaltet vorkommen, sind wohl auf Augit zu beziehen (s. Taf. 23, Fig. 1). Seltener als der Bronzit ist brauner Augit, welcher vollkommen klar erscheint und in der Form von Körnern ohne deutliche Kristallflächen auftritt. Er hat das Ansehen des Augits mancher Basalte, zeigt aber ebenso wenig wie die anderen Gemengteile eine Zuwachsschichtung (Taf. 23, Fig. 2).

Der Olivin ist ebenso stark oder stärker vertreten als der Bronzit. Meistens sind die Körner desselben ziemlich klein, innig miteinander verwachsen, trübe und voll staubartiger Einschlüsse; zuweilen aber sind die Körner klar und bilden große Kugeln und mit freiem Auge erkennbare Individuen. Einige derselben haben einen chondritischen Bau, indem ein rundliches Kristallindividuum an der Oberfläche schwarz imprägniert, außerdem aber von einer trüben körnigen Olivinrinde umgeben ist, welche voll staubartiger Einschlüsse erscheint. Die Silikate der Umgebung solcher den Chondren entsprechenden Gebilde sind gewöhnlich Bruchstücke (s. Taf. 23, Fig. 3). Derselbe Olivin zeigt bisweilen eine merkwürdige Beschaffenheit, welche mich anfänglich dazu verleitete, darin ein anderes Silikat zu vermuten. Parallel zu zwei aufeinander senkrechten, den Auslöschungen parallelen Richtungen liegen ungemein feine graue oder braune Nadeln, welche in scharfe Spitzen endigen. Dieselben gehen vom Rande oder von den Sprüngen des Olivinkornes aus, welche mit Magnetkies und rotbrauner Masse, die ein Oxydationsprodukt ist, erfüllt sind. Da die Nadeln in großer Zahl vorhanden sind, so bilden sie feine Parallelgitter und Kreuzgitter am Rande und neben den Klüften. Die grauen Nadeln scheinen Kanäle zu sein, welche mit einem hellfarbigen Glase gefüllt sind, während in den braunen entschieden jenes Oxydationsprodukt, welches in den Klüften vorhanden, enthalten ist (s. Taf. 23, Fig. 4).

An einigen wenigen Stellen des Gemenges finden sich auch farblose Partikel, welche im polarisierten Lichte dieselben zwickelartigen Individuen und im Ganzen dasselbe Ansehen darbieten, wie der Tridymit des Rittersgrüner Meteoriten. Da die Masse an mehreren Punkten eine klastische Beschaffenheit hat, so dürfte das Vorkommen von Tridymit neben Olivin nicht für ein ursprüngliches zu halten sein. Endlich findet sich in der Silikatmasse stellenweise auch bräunliches Glas in geringer Menge, worin feine grüne Nadeln von rhombischer Form auftreten. Diese dürften für Bronzit zu halten sein.

\subsubsection{Siderophyr.}
\paragraph{}
Die Masse von Rittersgrün enthält in einem Schwamm von Eisen ein körniges Gemenge, in welchem nach den Untersuchungen von Maskelyne und v. Lang Bronzit and Asmanit enthalten sind. Der Bronzit bildet nicht selten deutliche Kristalle mit vielen glätten Flächen, sonst aber Körner, an welchen übrigens auch öfters einzelne deutliche Flächen auftreten. Er hat eine grüne Farbe und deutliche Spaltbarkeit nach dem Prisma. Er zeigt keinen deutlichen Pleochroismus und wenige Einschlüsse. Diese sind Körner von Troilit und rundliche durchsichtige doppelbrechende Körnchen, welche ich nicht genauer bestimmen konnte (s. Taf. 25, Fig. 3). Mit dem Bronzit verwachsen zeigen sich körnige Häufchen jenes farblosen Gemengteilen, welcher Asmanit genannt worden, der aber nach den Untersuchungen von Winkler and Weisbach and nach dem optischen Verhalten auf Tridymit zu beziehen ist. Derselbe lässt öfter die Form sechsseitiger Täfelchen erkennen, zeigt im polarisierten Lichte in dünnen Schichten keine deutlichen Farben, dagegen eine Zusammensetzung aus Individuen in mindestens drei verschiedenen Stellungen. Die einzelnen Individuen erscheinen häufig zwickelförmig und hakenförmig, seltener breit-tafelförmig, sonst leistenförmig, geben eine zu den Seitenkanten schiefe Auslöschung und verhalten sich optisch zweiaxig bei schiefer Stellung der Mittellinie gegen die größte Fläche der Blättchen (s. Taf. 25, Fig. 1 and 2). Das optische Verhalten ist demnach dasselbe wie jenes beim Tridymit, welcher auch im spez. Gewicht mit diesem Gemengteil übereinstimmt.

\subsubsection{Mesosiderit.}
\paragraph{}
Das Gemenge von Eisen mit Olivin and Bronzit erscheint in einfachster Form in dem Meteoriten von Lodran, welchen ich vor längerer Zeit beschrieb.\footnote{Sitzungsber. d. Wiener Ak. Bd. 61. Abt. 2. April 1870.} Das Netz von Eisen ist in demselben so fein, dass es in dieser Beziehung den Übergang zu den körnigen Chondriten herstellt; jedoch ist von einer Chondrenbildung nichts darin zu bemerken. Der Olivin bildet Körner oder deutliche Kristalle, die öfters ebene Flächen darbieten; dieselben sind oberflächlich blaugrau bis berlinerblau gefärbt and äußerlich mit Chromitstaub überzogen, innen aber von hellgrüner Farbe. Sprünge im Innern sind mit Körnchen eines opaken Gemengteilen besetzt, was ich für eine sekundäre Erscheinung halte. Der Bronzit hat eine spargelgrüne bis gelbgrüne Farbe, deutliche prismatische Spaltbarkeit und enthält eiförmige Einschlüsse von Plagioklas, haarförmige opake Nadeln parallel der Prismenzone und opake rundliche Körner, wahrscheinlich von Chromit. Außer Eisen, Olivin und Bronzit sind noch untergeordnet Körner von Troilit und oktaedrische Kristalle von Chromit in dem Gemenge enthalten.

Der Mesosiderit von Hainholz zeigt bald ein feineres, bald ein gröberes Eisennetz von körniger Textur, darin ein körniges Silikatgemenge mit Troilit, stellenweise aber auch große Körner und Kristalle von Olivin liegen. Reichenbach gibt einen Kristall von 4,5 cm Länge an, ferner auch ziemlich große Kugeln. Die Olivinkörner sind am Rande mit der Grundmasse verwachsen und erscheinen klar, doch enthalten manche derselben auch rundliche Einschlüsse von Troilit (s. Taf. 24, Fig. 3). Stellenweise kommen Körner von jener Beschaffenheit und mit denselben braunen Nadeln vor, wie solche in der Masse von der Sierra de Chaco beobachtet wurden and deren eines auf Taf. 23 in Fig. 4 abgebildet ist. Der Bronzit bildet kleinere Körner als der Olivin und zeigt keine deutlichen Kristallumrisse. Einschlüsse sind häufig. Sie bestehen aus opaken Körnern und braunen durchsichtigen Glaseiern. Stellenweise zeigen sich in dem Gemenge Körner von Plagioklas mit breiten Zwillingslamellen, bald frei von Einschlüssen, bald reichlich erfüllt. Augit ist nur hie and da vertreten. Er bildet Körner von feinschaliger Zusammensetzung und grauer Farbe, welche letztere durch viele staubartige Einschlüsse hervorgebracht wird. Diese sind teils braune Glaseier, teils opake Körner. Alle diese in größeren Körnern vorkommenden Silikate sind mit einer Grundmasse umgeben, welche zum Teile aus größeren rundlichen von Staub erfüllten Olivinkörnern, so wie aus den übrigen schon genannten Gemengteilen in bunter Verwachsung und aus zwischengeklemmtem braunem Glase besteht (s. Taf. 24, Fig. 4). Die Masse von Hainholz weist demnach dieselben Gemengteile in ähnlicher Ausbildung auf, wie der Meteorit von der Sierra de Chaco doch mit dem Unterschiede, dass in der Masse von Hainholz der Plagioklas zurücktritt.

Der Meteoritenfall von Estherville, welcher einen Schwarm von vielen kleinen und einigen großen Exemplaren zur Erde brachte, ist wohl auch hierher zu rechnen. Viele der kleinen Stücke bestehen bloß aus Eisen, andere nur aus Silikatmasse, die übrigen aus beiden zugleich. In den großen Exemplaren sieht man auch beide vereinigt. Denkt man sich alle Stücke des Schwarmes zu einer Gesteinsmasse vereinigt, so gäbe dies ein grobes unregelmäßiges Gemenge von Eisen und körniger Silikatmasse. Das Eisen kommt nach L. Smith auch in der Form von Knollen innerhalb der Silikatmasse vor. Die Analyse gab diesem Beobachter außer Eisen einen durch Säure zersetzbaren Anteil von der Zusammensetzung des Olivins, einen unzersetzbaren Anteil von der Zusammensetzung des Bronzits und in geringer Menge die Bestandteile des Troilits und Chromits.\footnote{Comptes rend. Bd. 90. pag. 960.} G. vom Rath beobachtete in der Silikatmasse große Körner von Olivin eingeschlossen, ferner kleine Drusenräume, worin die kristallisierten Erhabenheiten messbare Kanten bildeten. In der Masse fand er auch farblose durchsichtige Körner, stellenweise mit Kristallflächen. Ob dieselben einem Plagioklas zugehören, lässt er dahingestellt.\footnote{Sitzungsber. d. Niederrhein. Ges. zu Bonn. Ber. v. 8. Nov. 1880.} Der Dünnschliff der Silikatmasse lässt, abgesehen von den großen Individuen des Olivins eine grüne körnige Masse wahrnehmen, in welcher als Grundlage ein kleinkörniger von vielen Einschlüssen staubiger Olivin und in diesem schwebende Kristalle und Körner von Bronzit zu unterscheiden sind. Der Bronzit hat teils das gewöhnliche Ansehen und enthält wenige Einschlüsse, teils aber ist er durch einen feinen Staub getrübt und zeigt außerdem noch größere Glaseinschlüsse. Diese trüben Körner haben makroskopisch ein ungewöhnliches Ansehen. Sie sind fettglänzend und erscheinen durch die Trübung heller gefärbt als die übrigen Gemengteile. Smith hat solche Körner besonders untersucht und eine Zusammensetzung gefunden, nach welcher dieselben zu zwei Dritteln aus Bronzit-, zu einem Drittel aus Olivinsubstanz bestehen. Er hielt sie demnach für einen besonderen Gemengteil, den er als Peckhamit bezeichnete. Durch die Güte des Herrn N. H. Winchell in Minneapolis habe ich sowohl eine Probe des Silicatgemenges mit einigen fettglänzenden Körnern, als auch ein größeres Korn von Peckhamit erhalten. Letzteres zeigte die prismatische Spaltbarkeit des Bronzits, gab aber auch Spaltflachen, die auf Krystallflachen des Olivins bezogen werden konnten. Das optische Verhalten war fast dasselbe wie das des Bronzits. Ein Schliff parallel einer prismatischen Spaltflache gab das Bild in Fig. 2 auf Taf. XXIV. Der ganze Dünnschliff ist durch einen feinen Staub getrübt und enthalt außerdem größere Einschlüsse von zweierlei Art. Die einen sind dunkelbraune bis schwarze Kugeln, die anderen stabförmige oder spindelförmige lichtgefärbte Glaseinschlüsse, welche negativen Krystallen entsprechen und gleichgefärbte runde Glaseinschlüsse. Ein Blick auf das Bild genügt zu erkennen, dass ein Gemenge vorliegt, welches bei der Analyse kein Resultat gibt, welches einem einfachen Gemengteil entspricht. Da nun die getrübten Bronzite in dem Silicatgemenge denselben Charakter zeigen wie der oben geschilderte Peckhamit und da alle Übergange vom reinen Bronzit zum Peckhamit vorkommen, so mochte ich diesen für einen Bronzit halten, welcher durch die große Menge von Einschlüssen getrübt und fettglänzend erscheint. An manchen Stellen des Gemenges erblickt man farblose durchsichtige Krystalle und Gruppen von Plagioklas, welche breite Zwillingslamellen darbieten, bald frei von Einschlüssen sind, bald wieder solche kristallisierte Einschlüsse wie die Masse von der Sierra de Chaco enthalten, bald durch viele sehr kleine runde Glaseinschlüsse staubig erscheinen. Taf. 24, Fig. 1 gibt das Bild einer Stelle, wo der Plagioklas mit Olivin und Bronzit verwachsen ist. Troilit und Chromit kommen in Körnern allenthalben in der Silikatmasse vor.

\subsubsection{Pallasit.}
\paragraph{}
In den Massen von Krasnojarsk, Brahin, Bitburg, Atacama bildet Meteoreisen die Grundmasse in der Form eines groben Netzes, worin Olivinkristall eingeschlossen sind. G. Rose und v. Kokscharow haben die Kristalle von Krasnojarsk sorgfältigen Messungen unterzogen. Als Nebengemengteile treten überall Troilit und Chromit, zumeist in Verbindung mit dem Eisen auf. In der Masse von Brahin beobachtet man stellenweise auch Splitter von Olivinkristallen in der Eisengrundmasse. Der Olivin ist klar und durchsichtig, mit Ausnahme jenes in der Masse von Atacama, worin der Olivin von unzähligen feinen Sprüngen durchsetzt wird und nach diesen krummflächige fettig glänzende Ablösungen bildet. In derselben Masse bemerkt man auch viele netzartig verbreitete schwarze Klüfte, welche durch den Olivin, die breiteren auch durch das Eisen hindurchsetzen und mit einem schwarzen Glase gefüllt sind. In sehr dünnen Schichten erscheint dieses braun. In mikroskopischer Beziehung bieten die Olivine nichts Auffallendes, außer den von G. Rose im Olivin der Pallasmasse wahrgenommenen Röhren, welche ich auch in dem Olivin der Brahiner Masse bemerkte. Wo diese Röhren in größerer Anzahl vorkommen, sind sie alle einander parallel und bringen bei der Beobachtung mit freiem Auge einen weißlichen Schiller oder einen bläulichen Lichtschein hervor. Es sind nach der aufrechten Axe gestreckte Kanäle von rundlichem bis vierseitigem Querschnitte, welche nach meinem Dafürhalten negativen Kristallen entsprechen und bald mit einem farblos erscheinenden, bald mit einem tiefbraunen Glase gefüllt sind (s. Taf. 25, Fig. 4).
\clearpage
\section{Schlussbemerkung.}
\paragraph*{}
Die bisher bekannten Meteoritenarten, von welchen alle mit Ausnahme des Meteoreisens kurz beschrieben wurden, bieten bestimmte Eigentümlichkeiten der Struktur und der mikroskopischen Beschaffenheit dar, welche hier nochmals übersichtlich hervorgehoben werden mögen.

Bezüglich der Eigenschaften der Kristalle und Individuen überhaupt, ist die Häufigkeit der Glaseinschlüsse zu bemerken. Der Olivin mit seinen oft enormen Glasmassen steht obenan und auch die zuweilen vorkommende staubartige Verteilung des Glases, wie im Olivin des Grahamits und Mesosiderits ist eine besondere Erscheinung. Zunächst steht der Plagioklas, der oft solche Einschlüsse zeigt und im Eukrit selbe in so feiner Verteilung enthält, dass sie auch bei starker Vergrößerung nicht mehr einzeln erkannt werden, jedoch eine zarte Trübung veranlassen, welche im auffallenden Lichte eine bläuliche, im durchfallenden eine gelbliche Farbe hervorrufen. Bronzit und Augit sind ärmer an Glas gegenüber dem Olivin, jedoch sind dieselben auch bisweilen von einem Glasstaub durchsetzt und der Augit im Eukrit beherbergt merkwürdige linear angeordnete Einschlüsse von dunkelbraunem Glase. Obwohl aber Glaseinschlüsse allenthalben zu sehen sind, so erscheinen doch Dampfporen selten und gerade im Olivin, der die größten Glaseinschlüsse darbietet, findet sich nur sehr selten eine fixe Libelle. Flussigkeitseinschlüsse sind, wie schon Sorby anführte, nirgends zu beobachten. Dieses vollständige Fehlen gibt den ersten Hinweis darauf, dass bei der Bildung der meisten Meteoriten eine Mitwirkung des Wassers ausgeschlossen war. Dem entspricht auch die vollständige Abwesenheit wasserhaltiger Silikate. In dem kohligen Meteoriten von Orgueil sind allerdings wasserhaltige Salze gefunden worden. Wenn hier der Wassergehalt ein ursprünglicher ist, so sind derlei kohligen Meteorite von anderer Bildung als die übrigen and gehören im geologischen Sinne einer späteren Bildungsepoche an.

An den Kristallen habe ich niemals Zuwachsschichten unterscheiden können, wie solche in den vulkanischen Felsarten am Augit and Plagioklas häufig wahrgenommen werden. Die einzige Verschiedenheit im Inneren ist die öfters beobachtete Abnahme der Glaseinschlüsse in der Rinde, welche beim Olivin and Plagioklas konstatiert wurde. Bemerkenswert ist anderseits die Häufigkeit der schaligen und der wiederholt zwillingsartigen Zusammensetzung beim Augit der Meteorite and die lagenförmige Anordnung der dunklen Einschlüsse im Augit der Eukrit, welche die schwarzen Streifen hervorruft.

Eine Eigentümlichkeit der gewöhnlichen Meteorsteine bilden die Chondren, welche durch ihre Textur von allen ähnlichen tellurischen Bildungen abweichen. Nicht nur der Olivin und Bronzit, sondern auch die übrigen in größerer Menge vorkommenden Gemengteile außer dem Magnetkies bilden Chondren, unter denen die aus Glas bestehenden besonders hervorzuheben sind. Die bunte Zusammensetzung, die Glaseinschlüsse und Glaskugeln, das Vorkommen von Kugeln mit Einbuchtungen, die Vereinigung von Chondren und deren Splitter beweisen, dass die Chondren sich nicht in der kompakten Gesteinsmasse als eine den Magnesiasilikaten eigentümliche Erstarrungsform gebildet haben und sprechen für die schon eingangs erwähnte Ansicht, nach welcher die Chondren rasch erstarrte Tropfen sind, deren viele infolge der großen Sprödigkeit zerbrachen.

Zu den bemerkenswerten Erscheinungen gehört die oft vorkommende, bis ins Feinste gehende Durchklüftung der Silikate. Die Kristallindividuen sind meistens von unzähligen feinen Sprüngen durchzogen, am auffallendsten jene der tuffartigen Chondrite, am wenigsten die kristallinisch aussehenden und die vorwiegend aus Eisen bestehenden Massen mit eingesprengten Silikaten, aber auch unter diesen zeigt eine und zwar jene von Atacama die Durchklüftung des Olivins. Demnach bieten alle diese Meteorite bei der mikroskopischen Untersuchung den Anblick von Massen, welche durch rasche Temperaturänderungen bis zu den kleinsten Splittern zersprengt and zerklüftet worden sind.

Aus den Beschreibungen geht hervor, dass bei den Meteoriten die Trümmerstruktur häufig sei, sehr viele bald deutlich, bald undeutlich klastisch sind und dass eine Anzahl der Meteorsteine ein vollständig tuffartiges Ansehen haben. Auch in den eisenreichen Massen, wie in jenen von Brahin, Atacama, der Sierra de Chaco sind Bruchstücke von Kristallen verbreitet. Diese Erscheinungen stimmen mit der Ansicht von einer allgemein vulkanischen Bildung der Meteorite und entsprechen der zuvor gedachten Herkunft der Chondren.

In der Grundmasse der Meteorsteine macht sich öfters ein Bindemittel bemerkbar, welches Plagioklas oder Maskelynit, in den kristallinisch oder gefrittet aussehenden auch braunes Glas ist. Auch Eisen und Magnetkies erscheinen als letzte Bildungen und als Imprägnation der Grundmasse. In den feinsten Klüften der Kristalle sieht man öfters Ansiedelungen von opaken Körnern and Ästchen, von denen manche als Magnetkies zu erkennen sind. In den schwarzen Kluftfüllungen und den gangartigen schwarzen Massen treten wiederum Eisen und Magnetkies in Flasern und Kügelchen, umgeben von schwarzem Glase auf. In der an die schwarzen gangartigen Bildungen grenzenden Silikatmasse wurde die Umwandlung von Plagioklas und von Olivin in isotrope Körner, also eine Verglasung beobachtet. Diese Erscheinungen umfassen die Merkmale von Imprägnationen, Frittungen und Injektionen, welche eine nachträgliche Veränderung der Silikatmassen durch Erhitzung bedeuten.

Eine Besonderheit der Meteoriten ist die dunkle Rinde, deren äußere schlackige Beschaffenheit und deren innere Gliederung samt den Verglasungserscheinungen eine oberflächliche Erhitzung der einzelnen Exemplare beweist.

Die aufgezählten Eigenschaften bedingen, abgesehen von der Rinde, einen Habitus der Meteoriten, durch welchen sie von den tellurischen Felsarten verschieden und in den meisten Fällen leicht unterscheidbar sind. Es gibt kein tellurisches Gestein, welches mit einer Meteoritenmasse verwechselt werden könnte, selbst wenn die mineralogische Zusammensetzung beider quantitativ dieselbe wäre. Die Gemengteile der Meteoriten sind zwar größtenteils der Gattung nach den Gemengteilen tellurischer Gesteine gleich, doch sind sie der Art nach von denselben verschieden. Selbst das Eisen von Ovifak, welches nach den Beobachtungen von Steenstrup als ein tellurisches anzusehen ist, unterscheidet sich durch Textur und Zusammensetzung von den bekannten Meteoreisen und auch die im Basalt bei Ovifak beobachteten, mit Eisen verbundenen Silikatgemenge lassen sich nicht mit dem Eukrit der Meteorite identifizieren, da von der eigentümlichen halbklastischen Struktur der letzteren abgesehen auch die Hauptgemengteile, der Plagioklas und der Augit, sich durch die Einschlüsse und deren Anordnung als Bildungen eigener Art charakterisieren.

Mai 1885.
\clearpage
\pagestyle{fancy}
\fancyhf{}
\rhead{Tafel 1.}
\section{Tafeln 1 bis 25. Zu jeder derselben eine Tafelerklärung.}
\subsection{Erklärung der Tafel 1.}
\paragraph{}
Figur 1. Gibt ein Bild von der Zusammensetzung des Meteoriten von Juvinas. Die gestreckten Kristalle, welche bei Anwendung von polarisiertem Lichte aus farbigen Lamellen zusammengesetzt erscheinen, sind Anorthit, die braunen oft schwarz gestreiften Körner aber Augit. Braunschwarze bis rabenschwarze Körnchen werden auf Chromit bezogen.

Figur 2. Zum Vergleiche mit den Meteoriten ist hier auch ein Bild von jenem Gestein aufgenommen, welches als Einschluss im Basalt von Ovifak in Grönland gefunden wurde, und aus Anorthit, Augit, Nickeleisen und einem dem Hisingerit ähnlichen Silikat besteht. Zuerst wurden diese eukritischen Einschlüsse, sowie die im Basalt und lose gefundenen Eisenklumpen von Nordenskiöld für Meteoriten gehalten, gegenwärtig werden dieselben den Beobachtungen Steenstrups zufolge von den meisten Forschern für tellurisch erklärt. Im Bilde treten die langgestreckten Anorthitkristall deutlich hervor, die dunkleren Körner sind Augit, die vollständig schwarzen Teile Nickeleisen mit Rinde von einem braunschwarzen hisingeritartigen Mineral. Weder der Anorthit noch der Augit zeigen solche Einschlüsse wie die entsprechenden Kristalle der Meteoriten.

Figur 3. Gibt den Charakter vieler Stellen im Eukrit von Juvinas wieder. Zuerst fallen die großen Körner von Augit auf, da sie durch schwarze Einschlüsse wie liniiert aussehen. Sie zeigen hellfarbig begrenzte Quersprünge, die oft mit einer schwarzen Masse gefüllt sind. Auf der linken Seite bemerkt man ein Bruchstück von farblosem Anorthit mit feinen punktartigen Einschlüssen. Diese gröberen Stücke von Augit und Anorthit erscheinen durch eine feinkristallinische Masse verbunden, deren Beschaffenheit oben am Rande des Bildes wahrzunehmen ist. Längliche Anorthitkriställchen bilden Maschen, in welchen ein gelbes Silikat liegt. Dieses hat auch die Eigenschaften des Augits.

Figur 4. Stellt eine für die Bildungsgeschichte des Eukrits von Juvinas wichtige Stelle dar. Breite Kristalle von Anorthit und Körner von dunklem Augit bilden ein gröberes Gemenge. Der gelbe feinkörnige Augit erscheint in zwei Streifen, welche Durchschnitte von Lamellen sind. Letztere können als Pseudomorphosen gedeutet werden. Auch in der Grundmasse ist gelber kleinkörniger Augit verteilt.
\clearpage

\cfoot{\thepage}
\vspace*{\fill}
\begin{figure}[H]
\centering
\includegraphics[width=\textwidth,keepaspectratio]{figs/1-1.png}
\caption{\small Figur 1 --- Eukrit von Juvinas. Eine Stelle von gröberer Textur im polaris. Lichte. Vergrößerung 75.}
\end{figure}
\vspace*{\fill}
\clearpage

\rhead{Tafel 1.}
\vspace*{\fill}
\begin{figure}[H]
\centering
\includegraphics[width=\textwidth,keepaspectratio]{figs/1-2.png}
\caption{\small Figur 2 --- Eisenhaltiger tellurischer Eukrit von Ovifak im polaris. Lichte. Vergrößerung 75.}
\end{figure}
\vspace*{\fill}
\clearpage

\rhead{Tafel 1.}
\vspace*{\fill}
\begin{figure}[H]
\centering
\includegraphics[width=\textwidth,keepaspectratio]{figs/1-3.png}
\caption{\small Figur 3 --- Eukrit von Juvinas. Eine Stelle mit zweierlei Textur. Vergrößerung 75.}
\end{figure}
\vspace*{\fill}
\clearpage

\rhead{Tafel 1.}
\vspace*{\fill}
\begin{figure}[H]
\centering
\includegraphics[width=\textwidth,keepaspectratio]{figs/1-4.png}
\caption{\small Figur 4 --- Eukrit von Juvinas. Mit Durchschnitten von Lamellen eines gelben Silikates. Vergrößerung 75.}
\end{figure}
\vspace*{\fill}
\clearpage
\rhead{Tafel 2.}
\subsection{Erklärung der Tafel 2.}
\paragraph{}
Figur 1. Ist das Bild eines Anorthitindividuums, welches in der tuffartigen Masse des Eukrits von Stannern liegt. Vom früheren Kristallumriss ist nur an der linken Seite etwas zu erkennen. Die langen Linien liegen parallel M = (0 1 0) und zeigen die Grenzen der Zwillingslamellen an, welche hier ungewöhnlich zahlreich sind. Diese Grenzen sind öfters mit kleinen rundlichen Glaseinschlüssen und länglichen Gasporen besetzt. Die kurzen, zu den vorgenannten beiläufig senkrechten Linien sind linear angeordnete längliche Glaseinschlüsse oder auch Sprünge. Die beiden anderen Systeme kurzer Linien, welche den Prismenflächen T = (1 1$^{\prime}$ 0) und \emph{l} = (1 1 0) parallel sind, haben denselben Charakter wie die vorigen.

Figur 2. Liefert ein Bild von der gewöhnlichen Beschaffenheit des Innern der Anorthitkristalle im Eukrit von Juvinas. Es ist das Ende einer gehäuften Zwillingsgruppe dargestellt Man bemerkt unzählige kleine rundliche Glaseinschlüsse, welche zuweilen perlschnurartig aneinandergereiht sind und überhaupt eine Tendenz zu reihenförmiger Anordnung zeigen. Die Mehrzahl derselben sind submikroskopisch und bedingen den bläulichen Farbenton der Anorthitblättchen im auffallenden, den blass braunen im durchfallenden Lichte.

Figur 3. Zeigt die innere Beschaffenheit mancher Anorthitkristalle im Stein von Juvinas, vieler im Eukrit von Stannern an. Die rundlichen Glaseinschlüsse sind in geringer Zahl vorhanden, dagegen treten viele langgestreckte Glaseinschlüsse auf, die im Bilde verschiedene Richtungen zeigen, da die Anorthitpartie aus mehreren Individuen zusammengesetzt ist. Die Grundmasse besteht aus kleinkörnigem Augit und opaken Körnern von Magnetkies.

Figur 4. Um den für Anorthit charakteristischen Zwillingsbau darzustellen, wurde auch ein Bild aufgenommen, welches das abgebrochene Ende eines Kristalls im polarisierten Lichte gesehen darstellt. Die Zwillingslamellen sind von sehr ungleicher Dicke. Einschlüsse parallel M und auch ungefähr senkrecht dazu gelagert sind bemerklich, ebenso zwei Sprünge in der letzteren Richtung. Das Kristallbruchstück ist von körnigem Augit und Magnetkies umgeben.
\clearpage

\rhead{Tafel 2.}
\vspace*{\fill}
\begin{figure}[H]
\centering
\includegraphics[width=\textwidth,keepaspectratio]{figs/2-1.png}
\caption{\small Figur 1 --- Anorthit in dem Eukrit von Stannern. Vergrößerung 160.}
\end{figure}
\vspace*{\fill}
\clearpage

\rhead{Tafel 2.}
\vspace*{\fill}
\begin{figure}[H]
\centering
\includegraphics[width=\textwidth,keepaspectratio]{figs/2-2.png}
\caption{\small Figur 2 --- Anorthit mit vielen rundlichen Glaseinschlüssen. Eukrit von Juvinas. Vergrößerung 160.}
\end{figure}
\vspace*{\fill}
\clearpage

\rhead{Tafel 2.}
\vspace*{\fill}
\begin{figure}[H]
\centering
\includegraphics[width=\textwidth,keepaspectratio]{figs/2-3.png}
\caption{\small Figur 3 --- Anorthit mit langgezogenen Glaseinschlüssen. Eukrit von Juvinas. Vergrößerung 160.}
\end{figure}
\vspace*{\fill}
\clearpage

\rhead{Tafel 2.}
\vspace*{\fill}
\begin{figure}[H]
\centering
\includegraphics[width=\textwidth,keepaspectratio]{figs/2-4.png}
\caption{\small Figur 4 --- Kristallbruchstück von Anorthit. Eukrit von Juvinas. Polaris. Licht. Vergrößerung 75.}
\end{figure}
\vspace*{\fill}
\clearpage

\rhead{Tafel 3.}
\subsection{Erklärung der Tafel 3.}
\paragraph{}
Figur 1. Ein Durchschnitt des braunen Augits, welcher einen Hauptgemengteil der Eukrit von Juvinas und Stannern bildet. Die zahlreichen dunklen Streifen, welche im Bilde nach rechts geneigt sind, bestehen aus braunen bis schwarzen Körnern und aus braunen Glaseinschlüssen. Sie lagern der Endfläche (0 0 1) parallel. Untergeordnet erscheint ein zweites Streifensystem, welches gegen links geneigt ist. Überdies machen sich Sprünge bemerkbar, welche nach zwei Richtungen verlaufen und der prismatischen Spaltbarkeit entsprechen.

Figur 2. Der gelbe körnige Augit, welcher schon auf Taf. 1, Fig. 4 in streifenförmigen Durchschnitten dargestellt wurde, welcher sich aber auch öfters zwischen den Krystallen von Augit und Anorthit mit unbestimmten Umrissen ausbreitet oder zwischenklemmt, bei stärkerer Vergrößerung. An mehreren Stellen ist die feinschalige Zusammensetzung der Korner parallel (0 0 1) bemerklich, welche in den Mineralen der Diopsidreihe so häufig beobachtet wird. Kleine opake Korner sind in der körnigen Masse oft schwarmweise verteilt. Die Umgebung bilden Krystalle von Anorthit und ein Korn von Magnetkies.

Figur 3. Um die charakteristische Textur der Meteoriten von Stannern, welche aber auch in jenen von Juvinas stellenweise zu beobachten ist anzudeuten, wurde ein Bild aufgenommen, in dem die Korner und Splitter des farblosen Anorthits von der gleichförmig aussehenden aus Augitsplittern bestehenden Grundmasse umgeben erscheinen. In dem großen Individuum von Anorthit bemerkt man als Beispiel der hier gewöhnlichen Erscheinung einen Einschluss von Augit. Nirgendsfindensich Krystallumrisse. Die Augitkörner lassen häufig die feinen schwarzen Streifen erkennen, welche in Fig. 1 vergrößert dargestellt sind.

Figur 4. Gibt ein Bild des gleichförmigen Gemenges von Augit und Maskelynit in dem Stein von Shergotty. Der Augit zeigt keine schalige Zusammensetzung, bloß hie und da Zwillinge nach (1 0 0), ferner auch keine Krystallformen , sondern füllt den Raum zwischen den Krystallen des Maskelynits. Der letztere ist farblos, zeigt einfache Lichtbrechung, hat jedoch meistens gestreckte Formen ähnlich denen der Feldspate. Die Endigung der Krystalle, welche früher von mir als tesseral gedeutet wurden, erscheint im Schnitte oft rechtwinkelig. Alle die hellen Stellen im Bilde bis an dessen Grenze beziehen sich auf Maskelynit. Man bemerkt ferner einige schwarze dem Magnetit entsprechende Punkte und Flecke.
\clearpage

\rhead{Tafel 3.}
\vspace*{\fill}
\begin{figure}[H]
\centering
\includegraphics[width=\textwidth,keepaspectratio]{figs/3-1.png}
\caption{\small Figur 1 --- Brauner Augit im Eukrit von Juvinas. Vergrößerung 250.}
\end{figure}
\vspace*{\fill}
\clearpage

\rhead{Tafel 3.}
\vspace*{\fill}
\begin{figure}[H]
\centering
\includegraphics[width=\textwidth,keepaspectratio]{figs/3-2.png}
\caption{\small Figur 2 --- Gelber körniger Augit im Eukrit von Juvinas. Vergrößerung 250.}
\end{figure}
\vspace*{\fill}
\clearpage

\rhead{Tafel 3.}
\vspace*{\fill}
\begin{figure}[H]
\centering
\includegraphics[width=\textwidth,keepaspectratio]{figs/3-3.png}
\caption{\small Figur 3 --- Eukrit von Stannern Tuffcharakter. Vergrößerung 75.}
\end{figure}
\vspace*{\fill}
\clearpage

\rhead{Tafel 3.}
\vspace*{\fill}
\begin{figure}[H]
\centering
\includegraphics[width=\textwidth,keepaspectratio]{figs/3-4.png}
\caption{\small Figur 4 --- Eukrit von Shergotty. Gleichförmig kristallinisch. Vergrößerung 65.}
\end{figure}
\vspace*{\fill}
\clearpage

\rhead{Tafel 4.}
\subsection{Erklärung der Tafel 4.}
\paragraph{}
Figur 1. Um die klastische Beschaffenheit der Howardit zur Anschauung zu bringen, ist hier eine Stelle aus dem Stein von Loutolaks bei schwacher Vergrößerung dargestellt. Splitter von Kristallen und Bruchstücke von kleinkörnigem Gestein lagern in einer pulverigen Grundmasse. Der Splitter, dessen Bild in der Mitte erscheint, gehört dem farblosen Anorthit an. Oberhalb desselben zeigen sich größere Splitter von Augit; an dem einen rechts die geradlinige Spur einer Krystallfläche, an dem anderen die feinen Nadeln, welche in dem Plättchen schief gegen die Oberfläche aufsteigen. Unterhalb des Anorthits macht sich ein feinkörniges Gesteinsbruchstück bemerklich, worin zufolge der Beobachtung bei stärkerer Vergrößerung Bronzit der herrschende Bestandteil zu sein scheint. Nebenan lagern Splitter mit schwacher Linienzeichnung, den Bronzit zugehörig, rechts ein kleines Augitkörn mit Gitterzeichnung. Die schwarzen Körper gehören dem Magnetkies, gediegenem Eisen und wohl auch dem Magnetit an.

Figur 2. Als charakteristisch für Howardit ist hier ein aus Anorthit und Augit bestehendes Bruchstück abgebildet. Der Anorthit erscheint völlig gleich demjenigen in den Eukriten von Juvinas und Stannern, der Augit ist wiederum von zweierlei Beschaffenheit. Der in den Maschen des Anorthits liegende ist gelblich und zeigt stellenweise eine feinschalige Zusammensetzung, der andere Augit, z. B. der im Bilde oberhalb sichtbare, ist braun und enthält viele schwarze staubartige Körnchen, welche sich öfters linear anordnen. In den Sprüngen dieses Augits bemerkt man eine schwarze Füllung ferner im Anorthit und Augit, sowie zwischen denselben öfters schwarze größere Körner. Demnach wiederholen sich in diesen Bruchstücken die im Eukrit gewöhnlichen Erscheinungen.

Figur 3. Manche der Bronzite im Howardit von Loutolaks und Massing zeigen geradlinige Umrisse, wovon dieses Bild ein Beispiel gibt. Der Schnitt geht nahezu parallel der prismatischen Spaltfläche 1 1$^{\prime}$ 0. Die parallelen Risse sind der aufrechten Axe parallel. Die Umrisse zur Linken lassen auf die Längsfläche (0 1 0), ferner auf das am Hypersthen bekannte horizontale Prisma \emph{d} = (0 2 1) schließen. Zarte Quersprünge sind mit Pünktchen besetzt, größere schwarze runde oder eckige Einschlüsse unregelmäßig verteilt. Um den Bronzit lagern Augitpartikel und ein großes schwarzes mattes Korn.

Figur 4. Unter den Augitsplittern des Howardits von Loutolaks haben viele das Ansehen des vulkanischen Augits unserer Gesteine. Sie erscheinen bräunlich und gelblich. Andere aber, die eine blasse ins Grünliche fallende Farbe zeigen, sind durch eine feinschalige Zusammensetzung nach (0 0 1) ausgezeichnet. Ein Beispiel gibt diese Figur, die einen sehr unregelmäßig geformten Splitter von solchem Augit von Sprüngen durchzogen darstellt. Derselbe ist von kleinen Splittern des braunen und des gelblichen Augits umgeben.
\clearpage

\rhead{Tafel 4.}
\vspace*{\fill}
\begin{figure}[H]
\centering
\includegraphics[width=\textwidth,keepaspectratio]{figs/4-1.png}
\caption{\small Figur 1 --- Howardit von Loutolaks mit sehr ausgesprochenem Tuffcharakter. Vergrößerung 65.}
\end{figure}
\vspace*{\fill}
\clearpage

\rhead{Tafel 4.}
\vspace*{\fill}
\begin{figure}[H]
\centering
\includegraphics[width=\textwidth,keepaspectratio]{figs/4-2.png}
\caption{\small Figur 2 --- Eukritbruchstück im Howardit von Loutolaks. Vergrößerung 160.}
\end{figure}
\vspace*{\fill}
\clearpage

\rhead{Tafel 4.}
\vspace*{\fill}
\begin{figure}[H]
\centering
\includegraphics[width=\textwidth,keepaspectratio]{figs/4-3.png}
\caption{\small Figur 3 --- Bronzit im Howardit von Massing, Schnitt beiläufig parallel 1 1' 0. Vergrößerung 160.}
\end{figure}
\vspace*{\fill}
\clearpage

\rhead{Tafel 4.}
\vspace*{\fill}
\begin{figure}[H]
\centering
\includegraphics[width=\textwidth,keepaspectratio]{figs/4-4.png}
\caption{\small Figur 4 --- Bruchstück von Augit im Howardit von Loutolaks. Vergrößerung 200.}
\end{figure}
\vspace*{\fill}
\clearpage

\rhead{Tafel 5.}
\subsection{Erklärung der Tafel 5.}
\paragraph{}
Figur 1. Zeigt eine Stelle im Stein von Busti, wo der Enstatit einen deutlichen Umriss darbietet. Man sieht im unteren Teile des Bildes einen Kristall mit einer dachförmigen Endigung, welche wahrscheinlich dem Doma \emph{p} = (1 0 2) entspricht. Der Schnitt ist aber der optischen Prüfung zufolge zur aufrechten Axe schief nach aufwärts geneigt, daher der innere Winkel schärfer ist als wenn der Schnitt parallel zu \emph{b} = 0 1 0 wäre. Oberhalb und links erscheinen ebenfalls Splitter von Enstatit und zwar mit großen Glaseinschlüssen, rechts ein Splitter von Plagioklas, etwas schlierig, fast ohne Einschlüsse.

Figur 2. Ist von einer Stelle genommen, wo der Diopsid vorwiegt. Die kleineren Splitter in der Mitte des Feldes zeigen eine äußert feine Lamellentextur. Sie enthalten ungemein kleine Einschlüsse. Der große Diopsid im unteren Teile des Bildes zeigt einen Rest der ursprünglichen Kristallausbildung. Er ist reich an feinen schwarzen staubartigen Einschlüssen, enthält aber auch kleine Glaseinschlüsse. Im durchfallenden Lichte zeigen beide Diopside einen grauvioletten Farbenton, welcher aber dort fehlt, wo die Einschlüsse zurücktreten. Rechts ist wieder ein Splitter von Plagioklas zu bemerken.

Figur 3. Stellt eine kleinkörnige Partie des Steines von Bishopville dar. Im unteren Teile des Bildes ist ein größeres Korn von Enstatit erkennbar, links ein Durchschnitt durch eine farblose Plagioklastafel, an welche sich ein Korn von Magnetkies anfügt. Das übrige ist ein Gemenge von Enstatit- und Plagioklaskörnern.

Figur 4. Gibt die Erscheinung wieder, welche eines von den größeren Plagioklaskörnern desselben Steines zwischen gekreuzten Nicols darbietet. Die Zwillingslamellen sind ziemlich breit, die Spaltbarkeit ist nicht deutlich ausgesprochen. Links hat man Enstatit von kleinkörnigem Gefüge.
\clearpage

\rhead{Tafel 5.}
\vspace*{\fill}
\begin{figure}[H]
\centering
\includegraphics[width=\textwidth,keepaspectratio]{figs/5-1.png}
\caption{\small Figur 1 --- Bustit von Busti bei Gorukpur. Enstatit und Plagioklas. Vergrößerung 65.}
\end{figure}
\vspace*{\fill}
\clearpage

\rhead{Tafel 5.}
\vspace*{\fill}
\begin{figure}[H]
\centering
\includegraphics[width=\textwidth,keepaspectratio]{figs/5-2.png}
\caption{\small Figur 2 --- Diopsid und Plagioklas im Stein von Busti. Vergrößerung 65.}
\end{figure}
\vspace*{\fill}
\clearpage

\rhead{Tafel 5.}
\vspace*{\fill}
\begin{figure}[H]
\centering
\includegraphics[width=\textwidth,keepaspectratio]{figs/5-3.png}
\caption{\small Figur 3 --- Chladnit von Bishopville. Enstatit, Plagioklas, Magnetkies. Vergrößerung 75.}
\end{figure}
\vspace*{\fill}
\clearpage

\rhead{Tafel 5.}
\vspace*{\fill}
\begin{figure}[H]
\centering
\includegraphics[width=\textwidth,keepaspectratio]{figs/5-4.png}
\caption{\small Figur 4 --- Plagioklas im Chladnit von Bishopville im polaris. Lichte. Vergrößerung 75.}
\end{figure}
\vspace*{\fill}
\clearpage

\rhead{Tafel 6.}
\subsection{Erklärung der Tafel 6.}
\paragraph{}
Figur 1. Gibt den Charakter des Meteoriten von Shalka wieder. Kin größerer Bronzitkristall, durch den Schnitt in einer zu den Kristallaxen schiefen Richtung getroffen, ist von einer körnigen Masse von Bronzit umgeben. In dem großen Bronzit beobachtet man Längsrisse der prismatischen Spaltbarkeit entsprechend, sowie einzelne grobe Sprünge, welche gegen die feinen Risse beiläufig senkrecht oder schief gerichtet sind. Kleine rundliche braune Glaseinschlüsse machen sich am unteren Ende des Kristallschnittes bemerklich.

Figur 2. Der Diogenit von Ibbenbühren, welcher meist aus größeren Körnern von Bronzit zusammengesetzt ist, enthält zwischen diesen auch kleinkörnige Teile. Aus einem solchen ist die Figur entnommen. Sie zeigt den Bronzit rechts in Längsschnitten, links aber in Querschnitten, welche durch das Schleifen nach der prismatischen Spaltbarkeit zersprungen sind, teils wohl auch schon vordem zersprungen waren.

Figur 3. Eine Stelle aus dem Amphoterit von Manbhoom, welche das körnige Gemenge von Olivin, Bronzit und Magnetkies erkennen lässt. Der Olivin bildet unregelmäßige Körner mit krummen Sprüngen. Der Bronzit im Bilde rechts hat eine zarte Faserung nach der Spaltbarkeit. Die größeren und kleineren opaken Körner entsprechen dem Magnetkies.

Figur 4. Eignet sich sehr gut, den Charakter des gleichförmig-körnigen Olivins vieler Meteorsteine darzustellen. Die Körner haben keine regelmäßige Begrenzung und sind von vielen gröberen Sprüngen, welche ganz unregelmäßig verlaufen, ferner aber auch von unzähligen feinen, oft schwach gekrümmten Sprüngen, die beiläufig parallel verlaufen, durchzogen. Die letztere Anordnung der feinen Sprünge macht öfters die Unterscheidung von Bronzit und Olivin recht schwierig. Zwischen den Olivinkörnern sieht man hie und da kleine trübe Partikel, zuweilen auch farblose durchsichtige Körnchen, z. B. im Bilde rechts oben, eingeklemmt. Häufig sind Körnchen von Chromit mit Andeutungen tesseraler Formen.
\clearpage

\rhead{Tafel 6.}
\vspace*{\fill}
\begin{figure}[H]
\centering
\includegraphics[width=\textwidth,keepaspectratio]{figs/6-1.png}
\caption{\small Figur 1 --- Bronzit in dem Stein von Shalka. Vergrößerung 160.}
\end{figure}
\vspace*{\fill}
\clearpage

\rhead{Tafel 6.}
\vspace*{\fill}
\begin{figure}[H]
\centering
\includegraphics[width=\textwidth,keepaspectratio]{figs/6-2.png}
\caption{\small Figur 2 --- Diogenit von Ibbenbühren. Bronzit. Vergrößerung 80.}
\end{figure}
\vspace*{\fill}
\clearpage

\rhead{Tafel 6.}
\vspace*{\fill}
\begin{figure}[H]
\centering
\includegraphics[width=\textwidth,keepaspectratio]{figs/6-3.png}
\caption{\small Figur 3 --- Amphoterit von Manbhoom. Olivin, Bronzit, Magnetkies. Vergrößerung 200.}
\end{figure}
\vspace*{\fill}
\clearpage

\rhead{Tafel 6.}
\vspace*{\fill}
\begin{figure}[H]
\centering
\includegraphics[width=\textwidth,keepaspectratio]{figs/6-4.png}
\caption{\small Figur 4 --- Chassignit von Chassigny. Olivin, Chromit. Vergrößerung 80.}
\end{figure}
\vspace*{\fill}
\clearpage

\rhead{Tafel 7.}
\subsection{Erklärung der Tafel 7.}
\paragraph{}
Figur 1. Der Chondrit von Lancé eignet sich wegen der Dunkelheit der Grundmasse und der Kleinheit der Einschlüsse zur Darstellung des chondritischen Charakters. Das Bild zeigt links unterhalb den Durchschnitt eines rundlichen oberhalb abgeplatteten Chondrums von einem Sprung durchsetzt. Es ist monosomatisch. Rechts davon hat man Durchschnitte von zwei zusammengesetzten (polysomatischen) Kügelchen. Im Übrigen sieht man sowohl kleine körnige Knöllchen mit vielen opaken Einschlüssen, als auch längliche Fetzen, endlich kleine Splitter, welche bald aus einem bald aus mehreren Individuen bestehen.

Figur 2. Hier ist das Zusammenvorkommen verschiedenartiger Chondren im Stein von Mezö-Madaras dargestellt. In der Mitte zeigt sich der Durchschnitt eines porphyrischen Kügelchens. Die Rinde ist mit Körnern von Magnetkies gemengt, das Innere besteht aus Olivinkörnern in halbglasiger Grundmasse. Rechts unterhalb erscheint der Durchschnitt eines dichten Chondrums (Bronzit), in welchem die Fasertextur eben erkennbar ist. Die übrigen Chondren sind meist körnig. Oberhalb des zentralen Kügelchens ist auch ein zwischengeklemmter körniger Splitter wahrzunehmen.

Figur 3. Die hier wiedergegebene Stelle aus dem Stein von Tieschitz lässt in dunkler Grundmasse rechts einen Teil von einer undeutlich faserigen Kugel (Bronzit) erkennen, welche eine blasse Rinde besitzt und an der gegen die Mitte des Bildes gerichteten Seite ausgehöhlt erscheint. Im oberen Teile ist ein schaliges Kügelchen (Bronzit) zu bemerken, welches gleichfalls eine Konkavität besitzt. Im Übrigen sind einige körnige Kügelchen, sowie lappige Durchschnitte (Olivin) sichtbar, wovon die letzteren unregelmäßig geformten Chondren entsprechen. Fast in der Mitte des Bildes hat man den scharf gezeichneten Splitter einer strahligen Kugel (Bronzit).

Figur 4. In diesem Bilde, welches dem Stein von Homestead, Iowa City (2. Febr. 1875) entnommen ist, hat man rechts den Durchschnitt eines strahligen Kügelchens (Bronzit), welches die exzentrische Lage des Radiationspunktes deutlich zeigt, links aber ein porphyrisches Kügelchen, in dem scharf gezeichnete Olivinkristall von einer trüben halbglasigen Grundmasse getragen werden. Oberhalb schmiegt sich an beide ein körniges Olivinchondrum, unterhalb sind es Splitter und ein unregelmäßig gestaltetes Chondrum. Die schwarz erscheinenden Stellen werden von Eisen und Magnetkies eingenommen.
\clearpage

\rhead{Tafel 7.}
\vspace*{\fill}
\begin{figure}[H]
\centering
\includegraphics[width=\textwidth,keepaspectratio]{figs/7-1.png}
\caption{\small Figur 1 --- Chondrit von Lancé. Kügelchen, Knöllchen, Fetzen und Splitter in einer schwarzen Grundmasse. Vergrößerung 65.}
\end{figure}
\vspace*{\fill}
\clearpage

\rhead{Tafel 7.}
\vspace*{\fill}
\begin{figure}[H]
\centering
\includegraphics[width=\textwidth,keepaspectratio]{figs/7-2.png}
\caption{\small Figur 2 --- Chondrit von Mezö-Madaras. Ein porphyrisches, ein dichtes und mehrere körnige Kügelchen. Vergrößerung 75.}
\end{figure}
\vspace*{\fill}
\clearpage

\rhead{Tafel 7.}
\vspace*{\fill}
\begin{figure}[H]
\centering
\includegraphics[width=\textwidth,keepaspectratio]{figs/7-3.png}
\caption{\small Figur 3 --- Chondrit von Tieschitz. Kügelchen, davon zwei mit Eindrücken, Knöllchen, Splitter. Vergrößerung 75.}
\end{figure}
\vspace*{\fill}
\clearpage

\rhead{Tafel 7.}
\vspace*{\fill}
\begin{figure}[H]
\centering
\includegraphics[width=\textwidth,keepaspectratio]{figs/7-4.png}
\caption{\small Figur 4 --- Chondrit von Homestead. Ein porphyrisches und ein strahliges Kügelchen. Vergrößerung 65.}
\end{figure}
\vspace*{\fill}
\clearpage

\rhead{Tafel 8.}
\subsection{Erklärung der Tafel 8.}
\paragraph{}
Figur 1. Der Stein von Dhurmsala, dem das Bild entnommen ist, enthält viele größere Chondren von länglichrunder Form. Ein solches ist hier dargestellt um zu zeigen, dass bisweilen, wenn auch selten, in einem großen Chondrum ein kleines eingeschlossen ist. Das große Chondrum ist unvollkommen porphyrisch. Kristalle, Körner und das kleine Chondrum, alle aus Olivin bestehend, sind von einer dichten, bei stärkerer Vergrößerung feinkörnig erscheinenden Grundmasse umgeben, zum Teile aber sitzen die Körner hart aneinander. Das kleine Chondrum ist monosomatisch.

Figur 2. Die Figur stellt ein kleinkörniges Chondrum aus dem Stein von Seres dar. Dasselbe ist von länglich runder, etwas abgeplatteter Form und besteht hauptsächlich aus gelblichgrünem Olivin, dessen Körner oft Glaskügelchen und opake Einschlüsse führen. Zwischen denselben ist nur eine geringe Menge amorpher Grundmasse bemerkbar. Selten ist ein bräunliches Bronzitkörnchen zu sehen. Die Rinde ist voll von Magnetkies und Eisen, auch im Innern des Chondrums treten zwei größere Körner von Magnetkies auf.

Figur 3. Die in den Chondren enthaltene Glasmasse ist bisweilen in auffallender Menge vorhanden. Hier ist ein solcher Fall aus dem Stein von Lancé abgebildet. Die Rinde des Kügelchens ist geschlossen, aus Olivinkörnern zusammengesetzt. Das Innere besteht aus braunem Glas, welches wiederum Olivinkörner einschließt. Der Olivin enthält Kügelchen von braunem Glas und Magnetkies eingeschlossen.

Figur 4. Das hier dargestellte Olivinkügelchen im Stein von Tieschitz zeigt an allen Punkten gleichzeitige Auslöschung, entspricht also einem einzigen Kristall und hat Ähnlichkeit mit einem Kernkristall. Die blass grünliche Hülle enthält viele braune Glaskügelchen, auch opake Körnchen. Der braune Kern besteht aus Olivin und aus braunem Glase. Der Olivin ist eine Fortsetzung der Hülle, doch ist er hier lückenhaft gebildet, aus krummen und lappigen, im Bilde beiläufig horizontalen Wänden bestehend, welche nur geringe Zwischenräume lassen. Diese sind mit braunem Glase gefüllt.
\clearpage

\rhead{Tafel 8.}
\vspace*{\fill}
\begin{figure}[H]
\centering
\includegraphics[width=\textwidth,keepaspectratio]{figs/8-1.png}
\caption{\small Figur 1 --- Ein porphyrisches Chondrum, Körner und ein kleineres Chondrum einschließend. Stein von Dhurmsala. Vergrößerung 8.}
\end{figure}
\vspace*{\fill}
\clearpage

\rhead{Tafel 8.}
\vspace*{\fill}
\begin{figure}[H]
\centering
\includegraphics[width=\textwidth,keepaspectratio]{figs/8-2.png}
\caption{\small Figur 2 --- Körniges Chondrum in dem Stein von Seres. Vergrößerung 65.}
\end{figure}
\vspace*{\fill}
\clearpage

\rhead{Tafel 8.}
\vspace*{\fill}
\begin{figure}[H]
\centering
\includegraphics[width=\textwidth,keepaspectratio]{figs/8-3.png}
\caption{\small Figur 3 --- Kügelchen mit Glaseinschluss, aus dem Chondrit von Lancé. Vergrößerung 160.}
\end{figure}
\vspace*{\fill}
\clearpage

\rhead{Tafel 8.}
\vspace*{\fill}
\begin{figure}[H]
\centering
\includegraphics[width=\textwidth,keepaspectratio]{figs/8-4.png}
\caption{\small Figur 4 --- Monosomatisches Kügelchen mit dunklerem Kern und blasser Hülle. Chondrit von Tieschitz. Vergrößerung 160.}
\end{figure}
\vspace*{\fill}
\clearpage

\rhead{Tafel 9.}
\subsection{Erklärung der Tafel 9.}
\paragraph{}
Figur 1. In einem porphyrischen Kügelchen des ausgezeichnet chondritischen Steines von Borkut sind mehrere Olivinkristall umgeben von einer Matrix, welche aus braunem Glase besteht. Die Olivine schließen viel von der Grundmasse ein. In dem oberen Kristall, welcher beiläufig parallel 1 0 0 durchschnitten erscheint und die scharfe Kante 0 2 1 : 0 2$^{\prime}$ 1 darbietet, sind außer dem großen Einschluss in der Mitte, welcher sich wie ein negativer Kristall verhält, noch andere Glaseinschlüsse zu sehen, welche, wenn auch unvollkommen, einen schaligen Bau hervorrufen. Unter diesem Kristall ist ein großes Individuum zu sehen, welches auch noch deutliche Kristallumrisse erkennen lässt. Dasselbe ist von mehreren tafelförmigen parallel gelagerten Glaseinschlüssen durchzogen, so dass der Kristall gefächert erscheint. Links ist ein Olivinkristall mit sehr wenig Einschlüssen wiedergegeben.

Figur 2. Ein Olivinkristall, welcher beiläufig parallel 0 1 0 durchschnitten ist und weniger scharfe Umrisse zeigt. Im Inneren sieht man radialfaserige Aggregate, welche eine andere Auslöschung zeigen als die Umgebung und welche Bronzit zu sein scheinen, außerdem feine gestreckte Glaseinschlüsse. Das Ganze erscheint im Innern als eine unregelmäßige Verwachsung von Olivin mit Bronzit, nach außen zu aber als ein gleichartiger Kristall.

Figur 3. Schiefer Durchschnitt durch einen Olivinkristall. Die Richtung des Schnittes ist der Zone 1 0 0 : 0 1 0 genähert. Auf der rechten Seite ist der Kristall geöffnet und von der Grundmasse aus dringen zwei geweihförmige Glaseinschlüsse herein, die bezüglich einer horizontalen Linie beiläufig symmetrisch angeordnet sind. Der Kristall erscheint demnach hier in mehrere Wände geteilt, welche gegen eine dickere Mittelwand symmetrisch gestellt sind. Links ist der Kristall einheitlich gebaut. Die vertikale Kante scheint der aufrechten Axe parallel zu sein. Die Umgebung besteht größtenteils aus Olivinkörnern.

Figur 4. Ein merkwürdiges Individuum von Olivin, welches die Form eines Pilzes nachahmt. Der Stiel zur Linken wird von einem undeutlich ausgebildeten Kristall dargestellt, welcher ungefähr dieselbe Stellung hat, wie der oberste Kristall in Fig. 1. Der Hut wird von einem monosomatischen Kügelchen gebildet, welches im Innern von einem braunen Glasnetz durchzogen ist. Der Olivin erscheint hier in viele Platten und Stäbchen zerteilt, während die Rinde vollkommen zusammenhängend ist. Wie schon aus der Lage der Spaltrisse zu entnehmen ist, bilden Hut und Stiel ein Individuum. In der Tat geben alle Teile der ganzen Pilzform gleichzeitige Auslöschung. Der Doppelkörper ist umgeben von braunem Glas und vielen aneinander gedrängten Olivinkristallen. Das Bild stellt, wie der Vergleich mit Fig. 1 auf voriger Tafel zeigt, einen Teil des dort abgebildeten porphyrischen Olivinknollens dar.
\clearpage

\rhead{Tafel 9.}
\vspace*{\fill}
\begin{figure}[H]
\centering
\includegraphics[width=\textwidth,keepaspectratio]{figs/9-1.png}
\caption{\small Figur 1 --- Olivinkristall in einem glasreichen Kügelchen des Steines von Borkut. Vergrößerung 160.}
\end{figure}
\vspace*{\fill}
\clearpage

\rhead{Tafel 9.}
\vspace*{\fill}
\begin{figure}[H]
\centering
\includegraphics[width=\textwidth,keepaspectratio]{figs/9-2.png}
\caption{\small Figur 2 --- Olivinkristall mit Bronzit- und Glaseinschlüssen im Chondrit von Aigle. Vergrößerung 70.}
\end{figure}
\vspace*{\fill}
\clearpage

\rhead{Tafel 9.}
\vspace*{\fill}
\begin{figure}[H]
\centering
\includegraphics[width=\textwidth,keepaspectratio]{figs/9-3.png}
\caption{\small Figur 3 --- Olivinkristall mit einem geweihförmigen Glaseinschluss im Chondrit von Seres. Vergrößerung 160.}
\end{figure}
\vspace*{\fill}
\clearpage

\rhead{Tafel 9.}
\vspace*{\fill}
\begin{figure}[H]
\centering
\includegraphics[width=\textwidth,keepaspectratio]{figs/9-4.png}
\caption{\small Figur 4 --- Monosomatisches Olivinkügelchen links mit einem gleichorientierten Fortsatz. Chondrit von Dhurmsala. Vergrößerung 35.}
\end{figure}
\vspace*{\fill}
\clearpage

\rhead{Tafel 10.}
\subsection{Erklärung der Tafel 10.}
\paragraph{}
Figur 1. In einem gemischten Kügelchen, welches aus Olivin, Bronzit und braunem Glas besteht und in dem Stein von Tieschitz beobachtet wurde, erscheint der Olivin entweder in geschlossenen Kristallen mit vielen langgezogenen Glaseinschlüssen oder in Kristallskeletten. Das Bild zeigt den größeren Teil eines solchen skelettartigen Individuums aus mäanderförmig angeordneten Leisten bestehend, welche fadenförmige, an den Enden öfters köpfige braune Glaseinschlüsse enthalten. Die Leisten, welche nach den Kristallaxen gestreckt sind, umschließen eine tiefbraune Glasmasse und sind von Glas, Olivinkristallen und radialfaserigem Bronzit umgeben. Früher, als mir noch weniger Erfahrung zur Seite stand, hatte ich diese Leisten für Bronzit gehalten (Denkschr. d. Wiener Akad. Bd. 39. pag. 197. Fig. 7).

Figur 2. Das Vorkommen von blätterigen Olivinkügelchen, welche man in den Chondriten öfters findet, wird hier durch ein deutliches Beispiel aus dem Stein von Mezö-Madaras dargestellt. Das Kügelchen besteht im Inneren aus parallel gestellten und optisch gleich orientierten blassgelbgrünen Tafeln von Olivin und aus hellbraunem Glas, welches die Zwischenräume ausfüllt. Der Schnitt ist ungefähr senkrecht gegen die Ebene der Platten geführt. Nach Außen ist das Kügelchen von einer Olivinrinde geschlossen, welche mit den Platten des Inneren gleichzeitig auslöscht, so dass das Ganze, vom Glas abgesehen ein einziges von Fächern durchzogenes Individuum darstellt, einem netzartig ausgebildeten Kristall entsprechend.

Figur 3. Während in dem vorigen Falle das Kügelchen aus Platten zusammengesetzt ist, sind es hier vorzugsweise Stäbchen von Olivin, welche das Innere bilden. Dieselben scheinen nach den drei Kristallaxen gestreckt zu sein, da sie alle gleichzeitig auslöschen und an denselben drei Richtungen ausgesprochen sind. Nur der linke untere Teil der Kügelchen zeigt eine andere Auslöschung als das übrige und dem entsprechend eine andere, vierte Richtung der Stäbchen. Zwischen den Stäbchen ist wiederum ein hellbraunes Glas verbreitet, auch ist an vielen Stellen des Umrisses eine dünne Rinde bemerkbar, welche mit den Stäbchen zugleich auslöscht. Am unteren Rande rechts, wo keine Rinde wahrnehmbar ist, erscheint das Kügelchen unvollständig und so als ob ein Stück abgebrochen wäre. In diesem Kügelchen erscheint demnach der Olivin größtenteils als ein einziges netzartig ausgebildetes Individuum, entsprechend den gestrickten Formen. Die schwarzen Unterbrechungen im Bilde des Kügelchens sind durch sekundär entstandene Klüfte mit brauner Füllung hervorgebracht. Die Umgebung wird von Olivinkörnern und von Magnetkies gebildet.

Figur 4. Ein großes polysomatisches Olivinkügelchen, aus mehreren Systemen von annähernd parallelen Tafeln von Olivin und von Glas zusammengesetzt. Jedes der Systeme zeigt eine andere einheitliche Auslöschung und erscheint bei stärkerer Vergrößerung ähnlich dem inneren Teile des in Fig. 2 dargestellten Kügelchens, doch finden sich im Glase hie und da doppelbrechende Nadeln, welche für Bronzit zu halten sind. Vier Systeme herrschen vor. Die geraden Begrenzungslinien derselben lassen auf ebene Zusammensetzungsflächen schließen. Am Rande wird an vielen Stellen eine dünne Olivinrinde bemerkbar, welche mit dem zugehörigen Plattensystem gleichzeitig auslöscht. Die Umgebung besteht aus Körnchen und Kügelchen, die vorwiegend Olivin sind und aus Magnetkies.
\clearpage

\rhead{Tafel 10.}
\vspace*{\fill}
\begin{figure}[H]
\centering
\includegraphics[width=\textwidth,keepaspectratio]{figs/10-1.png}
\caption{\small Figur 1 --- Skeletartiger Olivin mit fadenförmigen Glaseinschlüssen in dem Chondrit von Tieschitz. Vergrößerung 160.}
\end{figure}
\vspace*{\fill}
\clearpage

\rhead{Tafel 10.}
\vspace*{\fill}
\begin{figure}[H]
\centering
\includegraphics[width=\textwidth,keepaspectratio]{figs/10-2.png}
\caption{\small Figur 2 --- Gefächertes Kügelchen aus abwechselnden Lamellen von Olivin und Glas bestehend. Chondrit von Mezö-Madaras. Vergrößerung 200.}
\end{figure}
\vspace*{\fill}
\clearpage

\rhead{Tafel 10.}
\vspace*{\fill}
\begin{figure}[H]
\centering
\includegraphics[width=\textwidth,keepaspectratio]{figs/10-3.png}
\caption{\small Figur 3 --- Olivin von gestrickter Form in einem Kügelchen des Steines von Homestead. Vergrößerung 200.}
\end{figure}
\vspace*{\fill}
\clearpage

\rhead{Tafel 10.}
\vspace*{\fill}
\begin{figure}[H]
\centering
\includegraphics[width=\textwidth,keepaspectratio]{figs/10-4.png}
\caption{\small Figur 4 --- Olivinkügelchen aus mehreren Systemen abwechselnder Olivin- und Glaslamellen bestehend. Chondrit von Knyahinya. Vergrößerung 25.}
\end{figure}
\vspace*{\fill}
\clearpage

\rhead{Tafel 11.}
\subsection{Erklärung der Tafel 11.}
\paragraph{}
Figur 1. Durchschnitt eines Olivinkügelchens mit dicker, durchsichtiger Rinde, welche, von kleinen Abweichungen abgesehen, gleichzeitig mit dem Olivin des Inneren auslöscht. Der letztere bietet in diesem Schnitte eine gekröseartige Zeichnung, da das braune Glas ziemlich unregelmäßig verteilt ist. Äußerlich erscheint das Kügelchen mit einer zusammenhängenden undurchsichtigen Schichte bedeckt, welche aus Körnchen von Magnetkies und auch von Eisen besteht. Im Wesentlichen herrscht Ähnlichkeit mit dem auf Tafel 8 in Fig. 4 dargestellten Objekte.

Figur 2. Auch dieses Olivinkügelchen ist mit Ausnahme der äußersten Rinde monosomatisch. Im Inneren zeigen sich viele flache Stäbchen von Olivin, die zum größeren Teile nach einer im Bilde vertikalen Richtung ausgedehnt und einander parallel gelagert sind und nur zum kleineren Teile einer anderen Richtung folgen, welche die vorige unter einem schiefen Winkel schneidet. Die Stäbchen liegen in einer Grundmasse von braunem Glase. Die Rinde des Kügelchens ist von zweierlei Beschaffenheit. Der innere durchsichtigere Teil gibt gleichzeitig mit den Stäbchen im Inneren einheitliche Auslöschung. Er enthält viele kleine runde Einschlüsse von Magnetkies. Der äußere Teil erscheint aus vielen sehr verschieden orientierten Körnchen zusammengesetzt, welche meistens aus Olivin, im übrigen aus Magnetkies bestehen.

Figur 3. Die Rinde dieses Olivinkügelchens wird von mehreren kurzen, dicken, verschieden orientierten Individuen gebildet und umschließt eine Masse braunen Glases, welches mehrere Olivinkristall verbindet. Diese bieten z. T. langgestreckte z. T. kurze Durchschnitte dar, welche letztere die kranzförmige Anordnung der Rinde wiederholen. Die Umgebung bilden Körner und Kügelchen von Olivin, solche von Bronzit, ferner braunes Glas und Magnetkies.

Figur 4. Diese Figur zeigt die gewöhnliche Anordnung der Olivin- und Bronzitkristalle in den glasführenden Chondren. Der Olivin bildet meist breite, der Bronzit schmale gestreckte Individuen. An manchen Stellen sind vierseitige Querschnitte der letzteren zu bemerken und im unteren Teile der Figur erscheint auch ein Durchdringungszwilling von Bronzit in der Gestalt eines schiefen Kreuzes abgebildet. Die Bronzite schmiegen sich öfters an die Olivine an und drängen sich in den Kanälen zwischen den Olivinkristallen zusammen. Alle diese Kristalle sind in einem braunen Glase eingebettet.
\clearpage

\rhead{Tafel 11.}
\vspace*{\fill}
\begin{figure}[H]
\centering
\includegraphics[width=\textwidth,keepaspectratio]{figs/11-1.png}
\caption{\small Figur 1 --- Ein monosomatisches Olivinkügelchen mit dicker Rinde in dem Chondrit von Alfianello. Vergrößerung 60.}
\end{figure}
\vspace*{\fill}
\clearpage

\rhead{Tafel 11.}
\vspace*{\fill}
\begin{figure}[H]
\centering
\includegraphics[width=\textwidth,keepaspectratio]{figs/11-2.png}
\caption{\small Figur 2 --- Olivinkügelchen, im Innern monosomatisch, mit dicker Rinde in dem Stein von Mezö-Madaras. Vergrößerung 70.}
\end{figure}
\vspace*{\fill}
\clearpage

\rhead{Tafel 11.}
\vspace*{\fill}
\begin{figure}[H]
\centering
\includegraphics[width=\textwidth,keepaspectratio]{figs/11-3.png}
\caption{\small Figur 3 --- Polysomatisches Olivinkügelchen mit vielem Glas und dicker Rinde in dem Chondrit von Seres. Vergrößerung 160.}
\end{figure}
\vspace*{\fill}
\clearpage

\rhead{Tafel 11.}
\vspace*{\fill}
\begin{figure}[H]
\centering
\includegraphics[width=\textwidth,keepaspectratio]{figs/11-4.png}
\caption{\small Figur 4 --- Olivin in größeren, Bronzit in kleineren Kristallen und Glasgrundmasse in einem großen Kügelchen des Steines von Knyahinya. Vergrößerung 70.}
\end{figure}
\vspace*{\fill}
\clearpage

\rhead{Tafel 12.}
\subsection{Erklärung der Tafel 12.}
\paragraph{}
Figur 1. Ein Beispiel des Vorkommens einzelner Bronzitkristalle in einer feinkörnig aussehenden Partie des Mocser Steines. Durch die Begrenzung von Magnetkies werden die Enden des Kristalls deutlicher hervorgehoben. Der Schnitt ist beiläufig parallel 1 0 0. Die obere dachförmige Grenze entspricht der gewöhnlich durch die Flächen \emph{e} = (1 2 2) hervorgebrachten stumpfen Endigung. Außer den feinen Spaltlinien gemäß der Spaltbarkeit nach dem aufrechten Prisma geben sich auch noch andere Trennungen zu erkennen. Die Einschlüsse sind teils durchsichtige, teils opake Körnchen, selten Glaskügelchen. Um die Spaltlinien stärker hervortreten zu lassen, wurde das Bild bei starker Blendung aufgenommen Der Bronzitkristall hat links einen Fortsatz und grenzt dort an Bronzitkörner, rechts aber zum Teil an körnigen Olivin mit Plagioklas.

Figur 2. Ein Bronzitindividuum in einem der harten braunen halbglasigen Steinsplitter, welche in dem tuffartigen Chondrit von Alexinaé so häufig sind. Der Schnitt ist der optischen Beobachtung zufolge beiläufig parallel 0 1 0 geführt, also ungefähr senkrecht zur negativen Mittellinie. Die zahlreichen großen Einschlüsse sind sehr auffallend. Sie erfüllen negative Kristalle und bestehen aus Magnetkies, braunem Glase und einem Gemenge dieser beiden. Untergeordnet finden sich doppelbrechende Körnchen verschiedener Orientierung eingeschlossen, welche wohl auf Olivin zu beziehen sind. Der Bronzit ist beiderseits von Magnetkies eingefasst.

Figur 3. Von den Bronzitchondren mit großen unregelmäßig verbundenen Individuen gibt dieser Durchschnitt einen öfter vorkommenden Fall an. Das längliche Körperchen hat eine etwas zackige Begrenzung, die einer rauen Oberfläche entspricht. Die Rinde ist vollkommen kompakt, im Inneren erscheint eine ziemlich große Menge braunen Glases eingeschlossen, welches auch kleine Körnchen von Magnetkies enthält.

Figur 4. Für die Darstellung einer in manchen Chondren wiederkehrenden eigentümlichen Textur des Bronzits wurde ein Beispiel aus dem Stein von Knyahinya benutzt. Dickere Stäbchen von Bronzit bilden ein unregelmäßiges Gitter, dessen Öffnungen mit parallelfaserigem bis radialfaserigem Bronzit erfüllt sind. Im unteren Teile des Bildes hat man eine Grenze des Kügelchens, auf der rechten Seite einen birnförmigen Einschluss von Magnetkies, welcher im Schliffe etwas ausgebrochen ist.
\clearpage

\rhead{Tafel 12.}
\vspace*{\fill}
\begin{figure}[H]
\centering
\includegraphics[width=\textwidth,keepaspectratio]{figs/12-1.png}
\caption{\small Figur 1 --- Längsschnitt eines Bronzitkristalls in dem Chondrit von Mocs. Vergrößerung 160.}
\end{figure}
\vspace*{\fill}
\clearpage

\rhead{Tafel 12.}
\vspace*{\fill}
\begin{figure}[H]
\centering
\includegraphics[width=\textwidth,keepaspectratio]{figs/12-2.png}
\caption{\small Figur 2 --- Teil eines Bronzitlängsschnittes mit vielen Einschlüssen. Stein von Alexinaé. Vergrößerung 160.}
\end{figure}
\vspace*{\fill}
\clearpage

\rhead{Tafel 12.}
\vspace*{\fill}
\begin{figure}[H]
\centering
\includegraphics[width=\textwidth,keepaspectratio]{figs/12-3.png}
\caption{\small Figur 3 --- Polysomatisches Bronzitkügelchen mit Glaseinschluss im Chondrit von Pultusk. Vergrößerung 70.}
\end{figure}
\vspace*{\fill}
\clearpage

\rhead{Tafel 12.}
\vspace*{\fill}
\begin{figure}[H]
\centering
\includegraphics[width=\textwidth,keepaspectratio]{figs/12-4.png}
\caption{\small Figur 4 --- Teil eines Bronzitkügelchens von gitterartig stengeliger Textur im Stein von Knyahinya. Vergrößerung 70.}
\end{figure}
\vspace*{\fill}
\clearpage

\rhead{Tafel 13.}
\subsection{Erklärung der Tafel 13.}
\paragraph{}
Figur 1. Die radialstengelige Textur vieler Bronzitchondren wird hier durch ein Beispiel aus dem Stein von Tipperary illustriert. Das Gefüge ist bei der Betrachtung mit freiem Auge radialfaserig und zwar ist die Faserung exzentrisch. Viele der im Bilde sichtbaren Säulchen sind gerade, andere aber etwas gebogen. An den breiteren sind die feinen Spaltlinien entsprechend dem aufrechten Prisma und die welligen querverlaufenden Trennungen deutlich. An der Grenze des Kügelchens erscheint ein Ansatz von körnigem Bronzit.

Figur 2. Durchschnitt eines harten braunen Bronzitkügelchens aus dem Chondrit von Gnadenfrei, welchen ich Hrn. Prof. v. Lasaulx verdanke. Ein Teil der Stengelchen ist vom Zentrum des Kügelchens aus radial gerichtet, die Mehrzahl folgt aber anderen Richtungen. Die Spaltlinien und querverlaufenden Trennungen treten auch hier allenthalben hervor. Stellenweise zeigen sich unregelmäßig geformte Beimengungen von Magnetkies und von braunem Glase.

Figur 3. Dieser anfänglich wirrstengelig erscheinende Durchschnitt besteht nach der optischen Prüfung aus vier parallelstengeligen Bronzitbündeln, deren jedes einheitlich auslöscht. Der eine Teil zur Linken mit einer deutlichen Gitterzeichnung erscheint als ein querdurchschnittenes Bündel parallel verwachsener Individuen, der Teil rechts oben entspricht nahezu einem Längsschnitt durch ein solches Bündel, während die beiden übrigen Teile als schiefe Durchschnitte solcher Verwachsungen anzusehen sind. Das Bild vereinigt mehrere nicht selten vorkommende Arten von Durchschnitten des Bronzits in den Chondriten. Der Umriss des Kügelchens erscheint oberhalb stark ausgründet. Die Umgebung wird von dunkler Grundmasse und verschiedenen Chondren gebildet.

Figur 4. Durchschnitt eines dunkelbraunen festen Bronzitkügelchens, welche in den Chondriten häufig sind. Die Ebene des Durchschnittes liegt quer gegen die feine Faserung. An vielen Stellen des Umrisses bemerkt man eine sehr dünne Rinde von körniger Textur. Die nächstgelegene innere Schichte ist grau, etwas durchsichtig und bietet an vielen Stellen eine sehr feine Gitterzeichnung dar. Ganze Partien haben einheitliche Auslöschung. Der innere Teil ist fast undurchsichtig, wahrscheinlich in Folge einer Beimischung opaker Körnchen. Der Umriss ist ebenfalls ausgründet. In der Umgebung zeigen sich vorherrschend Körner und Kristalle von Olivin.
\clearpage

\rhead{Tafel 13.}
\vspace*{\fill}
\begin{figure}[H]
\centering
\includegraphics[width=\textwidth,keepaspectratio]{figs/13-1.png}
\caption{\small Figur 1 --- Bronzitkügelchen, radialfaserig, aus dem Stein von Tipperary. Vergrößerung 70.}
\end{figure}
\vspace*{\fill}
\clearpage

\rhead{Tafel 13.}
\vspace*{\fill}
\begin{figure}[H]
\centering
\includegraphics[width=\textwidth,keepaspectratio]{figs/13-2.png}
\caption{\small Figur 2 --- Bronzitkügelchen, undeutlich radialfaserig aus dem Stein von Gnadenfrei. Vergrößerung 60.}
\end{figure}
\vspace*{\fill}
\clearpage

\rhead{Tafel 13.}
\vspace*{\fill}
\begin{figure}[H]
\centering
\includegraphics[width=\textwidth,keepaspectratio]{figs/13-3.png}
\caption{\small Figur 3 --- Durchschnitt eines aus vier Faserbündeln bestehenden Kügelchens im Stein von Mezö-Madaras. Vergrößerung 60.}
\end{figure}
\vspace*{\fill}
\clearpage

\rhead{Tafel 13.}
\vspace*{\fill}
\begin{figure}[H]
\centering
\includegraphics[width=\textwidth,keepaspectratio]{figs/13-4.png}
\caption{\small Figur 4 --- Dichtes Bronzitkügelchen quer gegen die Faserung geschnitten, im Stein von Knyahinya. Vergrößerung 80.}
\end{figure}
\vspace*{\fill}
\clearpage

\rhead{Tafel 14.}
\subsection{Erklärung der Tafel 14.}
\paragraph{}
Figur 1. Die im Steine von Dhurmsala vorkommenden feinfaserigen Kügelchen sind hier durch das Bild eines Schnittes, welcher den divergierenden Fasern parallel ist, repräsentiert. Feine graue fächerförmig angeordnete Linien bezeichnen die exzentrische Faserung. Am Rande oben und rechts gehen die trüben Fasern in größere durchsichtige Bronzitindividuen aus. Opake Einschlüsse von Magnetkies machen sich besonders im oberen Teile bemerklich, im übrigen sind sie spärlich vertreten und bilden zuweilen sehr kleine sternförmige Gruppen. Eine Rinde ist nicht unterscheidbar.

Figur 2. Auch hier ist die radialfaserige Textur ausgesprochen, doch weniger deutlich, als im vorigen Falle, weil dickere Stengelchen und abweichend orientierte Körner von Bronzit öfters auftreten. Einschlüsse von Magnetkies finden sich besonders in den äußeren Teilen, ein größerer kugeliger Einschluss dieser Art ist im unteren Teile bemerklich. An dem Umrisse, der elliptisch ist, hebt sich die etwas durchsichtige dünne Rinde von dem Inneren deutlich ab. Sie zeigt bei optischer Prüfung eine körnige Zusammensetzung. In der Nachbarschaft bemerkt man Olivin- und Bronzitkügelchen.

Figur 3. Das Auftreten jenes farblosen doppelbrechenden Silikates, welches eine Ähnlichkeit mit Monticellit zeigt, ist hier durch ein Bild aus dem Stein von Knyahinya charakterisiert. Die weißen Stellen entsprechen dem genannten Silikat. Sie scheinen von demselben Individuum herzurühren, da von untergeordneten Abweichungen abgesehen, alle gleichzeitig auslöschen. Im Ferneren bietet sich nichts Charakteristisches dar. Es zeigen sich nur krumme unregelmäßige Sprünge und als Einschlüsse sehr kleine doppelbrechende Körnchen sowie eine Gruppe von Magnetkieskörnchen. Links oben bemerkt man gekröseartigen Olivin, links unten parallelfaserigen Bronzit, welche beide mit dem farblosen Silikat innig verwachsen sind. Rechts ist ein Gemenge von Olivin und Magnetkies verbreitet.

Figur 4. Dieses Bild stellt dasselbe Silikat in einem Individuum vor, das wiederum eine einheitliche, etwas undulöse Auslöschung zeigt, hier jedoch an manchen Stellen feine Spaltlinien in drei Richtungen erkennen lässt. Die unregelmäßigen Sprünge und die aus doppelbrechenden Körnchen bestehenden Einschlüsse treten auch hier auf. Zur Rechten hat man ein Korn von Olivin mit einer Tendenz zur Lamellenbildung in enger Verwachsung mit dem vorigen Silikat. Die Umgebung wird von Magnetkies und Olivin, der in Folge einer Zersetzung des ersteren braun gefärbt ist, gebildet.
\clearpage

\rhead{Tafel 14.}
\vspace*{\fill}
\begin{figure}[H]
\centering
\includegraphics[width=\textwidth,keepaspectratio]{figs/14-1.png}
\caption{\small Figur 1 --- Bronzitkügelchen, in einem Längsschnitte die fächerartig divergierenden Fasern zeigend. Chondrit von Dhurmsala. Vergrößerung 60.}
\end{figure}
\vspace*{\fill}
\clearpage

\rhead{Tafel 14.}
\vspace*{\fill}
\begin{figure}[H]
\centering
\includegraphics[width=\textwidth,keepaspectratio]{figs/14-2.png}
\caption{\small Figur 2 --- Undeutlich faseriges Bronzitkügelchen mit dünner Rinde. Chondrit von Mezö-Madaras. Vergrößerung 60.}
\end{figure}
\vspace*{\fill}
\clearpage

\rhead{Tafel 14.}
\vspace*{\fill}
\begin{figure}[H]
\centering
\includegraphics[width=\textwidth,keepaspectratio]{figs/14-3.png}
\caption{\small Figur 3 --- Monticellitähnliches farbloses doppelbrechendes Silikat im Chondrit von Knyahinya. Vergrößerung 70.}
\end{figure}
\vspace*{\fill}
\clearpage

\rhead{Tafel 14.}
\vspace*{\fill}
\begin{figure}[H]
\centering
\includegraphics[width=\textwidth,keepaspectratio]{figs/14-4.png}
\caption{\small Figur 4 --- Dasselbe Silikat mit feinen Spaltrissen. Chondrit von Knyahinya. Vergrößerung 70.}
\end{figure}
\vspace*{\fill}
\clearpage

\rhead{Tafel 15.}
\subsection{Erklärung der Tafel 15.}
\paragraph{}
Figur 1. Der Gemengteil, welcher als Augit bestimmt wurde, erscheint hier in Körnern, welche in dem dunklen Chondrit von Renazzo polysomatische Kügelchen zusammensetzen. Das Bild gibt die Durchschnitte zweier solcher Chondren, die einander berühren, nach der Aufnahme im gewöhnlichen Lichte wieder. Die Augitkörnchen haben stellenweise Umrisse, die an Kristalle erinnern. Manche umschließen rundliche Olivinkörner, die wiederum Glaseinschlüsse und Magnetkies enthalten. Auch im Augit sind Kügelchen von Magnetkies verstreut. Zwischen den Augitkörnern ist eine braune Masse eingeklemmt, welche aus Glas und feinen Fasern besteht. Die Umgebung der Chondren ist eine schwärzliche Grundmasse.

Figur 2. Dieselben zwei Chondren bieten im polarisierten Lichte das Bild einer lamellaren Zwillingsverwachsung, welche dem Augit in den Chondriten allgemein zuzukommen scheint. Die Lamellierung ist weder so scharf noch so eben, wie in den Plagioklasen. Die deutlich gestreiften Schnitte zeigen wenig lebhafte Farben, die anderen aber öfters schöne helle Farbentöne. Der größte Unterschied in der Auslöschung benachbarter Lamellen beträgt etwa 35°. Die Olivinkörner heben sich von der Umgebung durch ihre Farben deutlich ab.

Figur 3. Ein Beispiel jener Augitchondren, in welchen die Individuen eine Durchwachsung zeigen, aus dem Stein von Mezö-Madaras. In der Mitte bemerkt man ein Individuum, welches in aufrechter Richtung Spaltlinien und gestreckte Einschlüsse zeigt, oben eine dachförmige Endigung erkennen lässt und unten von einem kleinen Individuum durchdrungen ist. Das große Individuum gibt im polarisierten Lichte ein System von öfters krummen und absätzigen Streifen parallel der aufrechten Richtung, entsprechend vielen Lamellen parallel 1 0 0, deren Auslöschungen mit dieser Ebene beiderseits ungefähr 20° bilden. Rechts hat man ein zweites Individuum, welches in eine schiefe Spitze ausgeht und keine Lamellierung zeigt. Die Auslöschungsschiefe ist hier kaum 9°. In der Lücke zwischen diesem und dem Hauptindividuum ist ein Gemenge von Glas und Augit, letzterer in Körnern und Nadeln, eingeklemmt. Das grünliche Glas erscheint auch in den schlauchförmigen Einschlüssen des großen Individuums. Links unten zeigen sich mehrere Individuen, deren größtes schief durchschnitten ist, so dass die schlauchförmigen Einschlüsse einen spindelförmigen Querschnitt ergeben. Die Umgebung ist zunächst eine dunkle Grundmasse, ferner treten in der Nachbarschaft Olivinchondren auf.

Figur 4. Der spreuförmige Augit mit Olivin, Magnetkies und Glasgrundmasse ein Gemenge darstellend, welches viele Chondren im Stein von Renazzo bildet und auch in andren Chondriten gefunden wird. Die meisten der stäbchenförmigen Augitschnitte zeigen schiefe Auslöschung, die breiten oft eine grobe Lamellierung. Bei starker Vergrößerung bieten die Augite, namentlich bezüglich der Glaseinschlüsse, denselben Charakter dar wie im vorigen Bilde. Die Olivinkörner heben sich durch ihre Form und ihr Verhalten im polarisierten Lichte von der Umgebung ab. Der Magnetkies ist in runden Körnchen verstreut. Die Grundmasse zeigt die Erscheinungen der Entglasung.
\clearpage

\rhead{Tafel 15.}
\vspace*{\fill}
\begin{figure}[H]
\centering
\includegraphics[width=\textwidth,keepaspectratio]{figs/15-1.png}
\caption{\small Figur 1 --- Zwei Kügelchen, zumeist aus Augit bestehend, in dem Stein von Renazzo. Vergrößerung 70.}
\end{figure}
\vspace*{\fill}
\clearpage

\rhead{Tafel 15.}
\vspace*{\fill}
\begin{figure}[H]
\centering
\includegraphics[width=\textwidth,keepaspectratio]{figs/15-2.png}
\caption{\small Figur 2 --- Dieselben im polarisierten Lichte. Nicolhauptschnitte horizontal und vertikal.}
\end{figure}
\vspace*{\fill}
\clearpage

\rhead{Tafel 15.}
\vspace*{\fill}
\begin{figure}[H]
\centering
\includegraphics[width=\textwidth,keepaspectratio]{figs/15-3.png}
\caption{\small Figur 3 --- Ein Augitkügelchen, die Durchwachsung mehrerer Individuen zeigend. Chondrit von Mezö-Madaras. Vergrößerung 160.}
\end{figure}
\vspace*{\fill}
\clearpage

\rhead{Tafel 15.}
\vspace*{\fill}
\begin{figure}[H]
\centering
\includegraphics[width=\textwidth,keepaspectratio]{figs/15-4.png}
\caption{\small Figur 4 --- Augit und Olivin, ein spreuartiges Gemenge darbietend. Chondrit von Renazzo. Vergrößerung 160.}
\end{figure}
\vspace*{\fill}
\clearpage

\rhead{Tafel 16.}
\subsection{Erklärung der Tafel 16.}
\paragraph{}
Figur 1. Der Gemengteil der Chondrite, welcher als Plagioklas bestimmt wurde, in der Form eines scharfen Splitters, umgeben von dunkler Grundmasse und Olivinkörnern im Stein von Murcia. Im polarisierten Lichte wird die Lamellierung, welche schon im gewöhnlichen Lichte bemerkbar ist, sehr deutlich und es zeigen sich abwechselnde breitere und sehr schmale Streifen. Die Nicolhauptschnitte sind hier und im folgenden Bilde horizontal und vertikal zu denken. Ich verdanke den Schliff Hrn. Oberbergrat M. Websky.

Figur 2. Ein Plagioklaskörnchen mit rundlichem Umriss, verwachsen mit Olivin und Magnetkies in dem Steine von Mocs. Während im gewöhnlichen Lichte keine Lamellierung zu sehen ist, erscheint dieselbe im polarisierten Lichte deutlich, indem verhältnismäßig breite Streifen hervortreten.

Figur 3. Das gewöhnliche Vorkommen der Plagioklaskörner in den Chondriten wird hier durch ein Bild aus dem Stein von Milena bei schwacher Vergrößerung charakterisiert. Alle weißen Stellen entsprechen dem Plagioklas, welcher mit Olivin, stellenweise auch mit Magnetkies innig verbunden ist und meistens als Zwischenklemmungsmasse vorkommt. Der Plagioklas umschließt häufig größere bis staubartig feine Körnchen von Olivin und auch von Magnetkies. Im polarisierten Lichte gibt er nur selten Streifen, meistens eine undulöse Auslöschung.

Figur 4. Eine zuweilen vorkommende, etwas regelmäßige Verwachsung von Plagioklas mit Olivin und Magnetkies, in der Form eines Kügelchens im Steine von Dhurmsala. Alle blassen Stellen entsprechen dem Plagioklas, der stark vorherrscht und von schmalen Streifen, welche sowohl von Lamellen als von Stäbchen herrühren dürften, durchsetzt wird. Dieselben tragen den Charakter des Olivins und bilden vorwiegend zwei Parallelsysteme, die jedes für sich einheitlich auslöschen. Dies erinnert lebhaft an die Olivinlamellen mit zwischengeklemmtem Glas in Fig. 2 und 4 auf Taf. 10. Hier aber ist der Plagioklas das Zwischenmittel, doch zeigt sich im polarisierten Lichte der letztere aus mehreren großen und einigen kleineren Körnern zusammengesetzt, welche letzteren zum Teile die feine Streifung zeigen. Die großen Körner erscheinen einfach, sie werden in der Erstreckung durch die Olivinlamellen gar nicht beschränkt, sie greifen vielmehr immer über mehrere desselben Systeme hinaus. Die schwarzen Stellen entsprechen dem Magnetkies, welchem jedoch auch ein wenig dunklen Glases anzuhängen scheint. Die stellenweise vorkommenden schwarzen feinen Netze und staubigen Einschlüsse können auf Chromit oder Magnetkies bezogen werden. Die Umgebung des Kügelchens, welches auf der rechten Seite abgeplattet erscheint, ist ein Gemenge von Olivin, Eisen, Magnetkies und dem Monticellitähnlichen Silikat.
\clearpage

\rhead{Tafel 16.}
\vspace*{\fill}
\begin{figure}[H]
\centering
\includegraphics[width=\textwidth,keepaspectratio]{figs/16-1.png}
\caption{\small Figur 1 --- Plagioklas mit scharfeckigem Umriss im polarisierten Lichte. Chondrit von Murcia [?]. Vergrößerung 360.}
\end{figure}
\vspace*{\fill}
\clearpage

\rhead{Tafel 16.}
\vspace*{\fill}
\begin{figure}[H]
\centering
\includegraphics[width=\textwidth,keepaspectratio]{figs/16-2.png}
\caption{\small Figur 2 --- Plagioklaskörnchen mit rundlichem Umriss. Pol. Licht. Chondrit von Mocs. Vergrößerung 360.}
\end{figure}
\vspace*{\fill}
\clearpage

\rhead{Tafel 16.}
\vspace*{\fill}
\begin{figure}[H]
\centering
\includegraphics[width=\textwidth,keepaspectratio]{figs/16-3.png}
\caption{\small Figur 3 --- Plagioklas in zahlreichen kleinen Körnchen mit Olivin verwachsen. Chondrit von Milena. Vergrößerung 70.}
\end{figure}
\vspace*{\fill}
\clearpage

\rhead{Tafel 16.}
\vspace*{\fill}
\begin{figure}[H]
\centering
\includegraphics[width=\textwidth,keepaspectratio]{figs/16-4.png}
\caption{\small Figur 4 --- Plagioklas mit Olivin und Magnetkies ein Kügelchen bildend. Chondrit von Dhurmsala. Vergrößerung 80.}
\end{figure}
\vspace*{\fill} 
\clearpage

\rhead{Tafel 17.}
\subsection{Erklärung der Tafel 17.}
\paragraph{}
Figur 1. Das farblose einfachbrechende Glas, als Maskelynit bezeichnet, welches in vielen Chondriten vorkommt, ist hier durch eine Probe aus dem Stein von Alfianello bei stärkerer Vergrößerung repräsentiert. Im oberen Teile des Bildes erblickt man parallele Linien, welche bloß von einer Verschiedenheit der Lichtbrechung herrühren. Rechts oben und links unten ist der Maskelynit von Olivin begrenzt, im übrigen von Magnetkies. Im Inneren zeigen sich an einigen Stellen Körner von Olivin. Zur Rechten hat der Maskelynit eine geradlinige Begrenzung.

Figur 2. Das gewöhnliche Auftreten des Maskelynits in unregelmäßigen meist lappigen Partikeln, welche häufig Olivinkörner einschließen und von Magnetkies begleitet werden, ist hier durch ein Bild aus dem vorgenannten Stein dargestellt. Die Grundmasse, in welcher der Maskelynit verstreut erscheint, besteht vorwiegend aus größeren und kleineren Körnchen von Olivin und Magnetkies. Links unten hat man auch ein scharfeckiges Partikelchen von Chromit.

Figur 3. Das Bild gibt den Durchschnitt einer der schwarzen, im Bruche glasglänzenden Kugeln wieder, welche in dem Stein von Chateau Renard stellenweise vorkommen. Die Grundmasse der Kugel ist Maskelynit, der an einigen Stellen wiederum die feinen parallelen Linien zeigt und an der Oberfläche der Kugel nur wenige schwarze Einschlüsse enthält. Im Inneren sind aber die opaken Körner in großer Menge vorhanden. Einige derselben lassen sich im auffallenden Lichte als Magnetkies erkennen. Die Umgebung der Kugel wird hauptsächlich von körnigem Olivin, zum Teil von Magnetkies gebildet.

Figur 4. Eine halbdurchsichtige blassblaue Kugel im Stein von Tipperary, welche die Erscheinungen vorgeschrittener Entglasung darbietet. Im polarisierten Lichte zeigt sich eine Zusammensetzung aus vielen kleinen doppelbrechenden Körnchen ohne scharfe Umrisse die miteinander und mit der glasigen Grundmasse verfließen. Dieselben dürften für Olivin zu halten sein. Die sehr dünne doppelbrechende Rinde hat den Charakter des Olivins. Die Umgebung der Kugel ist eine körnige Grundmasse, welche aus Olivin und Eisen besteht.
\clearpage

\rhead{Tafel 17.}
\vspace*{\fill}
\begin{figure}[H]
\centering
\includegraphics[width=\textwidth,keepaspectratio]{figs/17-1.png}
\caption{\small Figur 1 --- Maskelynit im Chondrit von Alfianello. Vergrößerung 300.}
\end{figure}
\vspace*{\fill}
\clearpage

\rhead{Tafel 17.}
\vspace*{\fill}
\begin{figure}[H]
\centering
\includegraphics[width=\textwidth,keepaspectratio]{figs/17-2.png}
\caption{\small Figur 2 --- Maskelynit in demselben Stein. Vergrößerung 60.}
\end{figure}
\vspace*{\fill}
\clearpage

\rhead{Tafel 17.}
\vspace*{\fill}
\begin{figure}[H]
\centering
\includegraphics[width=\textwidth,keepaspectratio]{figs/17-3.png}
\caption{\small Figur 3 --- Schwarze Kugel, zumeist aus Maskelynit bestehend, im Chondrit von Chateau Renard. Vergrößerung 160.}
\end{figure}
\vspace*{\fill}
\clearpage

\rhead{Tafel 17.}
\vspace*{\fill}
\begin{figure}[H]
\centering
\includegraphics[width=\textwidth,keepaspectratio]{figs/17-4.png}
\caption{\small Figur 4 --- Bläuliche halbglasige Kugel im Chondrit von Tipperary. Vergrößerung 160.}
\end{figure}
\vspace*{\fill}
\clearpage

\rhead{Tafel 18.}
\subsection{Erklärung der Tafel 18.}
\paragraph{}
Figur 1. Um den Charakter der vorzugsweise aus Glas bestehenden Chondren durch ein einziges Bild anzudeuten, wurde dieses Beispiel aus dem Chondrit von Mezö-Madaras gewählt, welches in der blass-bräunlichen Grundmasse mehrere lange so wie auch einige kurze Olivinkristall, ferner rechts unten farnkrautähnliche Bildungen und links oben eine netzförmige Kristallisation, aus rechtwinkelig angeordneten Nadeln bestehend, endlich auch einzelne feine Nadeln erkennen lässt. Die Olivine sind öfters am Ende gabelig. Die Umgebung der Glaskugel wird von einer Grundmasse, in der Olivinkristall hervortreten, gebildet.

Figur 2. Die Erscheinung der Entglasung, welche die Zwischenmasse der Chondren so häufig darbietet, ist hier durch einen Fall im Steine von Lancé dargestellt. Eine porphyrische Olivinkugel, in welcher die Kristalle teils enge aneinander liegen, teils durch eine halbglasige Zwischenmasse getrennt sind, zeigt in der letzteren an vielen Stellen teils einzelne feine doppelbrechende Nadeln, teils netzartige Kristallisationen von rechtwinkelig angeordneten Nädelchen, endlich wie im vorliegenden Falle Haufwerke von derlei Nadeln, so dass an diesen Stellen kein durchsichtiges Glas, sondern eine trübe Zwischenmasse die Olivinkristall verbindet. An einzelnen Stellen ist die glasige Zwischenmasse dunkelbraun gefärbt.

Figur 3. Eine Andeutung des ungemein seltenen Falles, in welchem der Glaseinschluss in den Kristallen eine Libelle zeigt, wird hier durch eine Darstellung aus dem Stein von Lancé gegeben. In der schwarzen Grundmasse liegt ein Splitter von Olivin mit einem Glasei. Dasselbe erscheint im Bilde links oben und ist von einer kleinen Libelle begleitet. Das Glas ist kaum merklich gefärbt.

Figur 4. In den festen körnigen Chondriten enthält der Olivin zahlreiche Glaseinschlüsse, deren Charakter hier durch eine Probe aus dem Stein von Stauropol bezeichnet ist. Die kleineren Einschlüsse haben oft scharfe Umrisse und sind parallel angeordnet, entsprechen also negativen Kristallen; auch manche der unregelmäßigen Einschlüsse haben noch diesen Charakter, so namentlich der große Einschluss links oben mit horizontaler Reifung. Im übrigen sind eiförmige und verschiedentlich geformte Glaspartikel vorhanden. Stellenweise zeigen die kleinen Einschlüsse eine Tendenz zu linearer Anordnung.
\clearpage

\rhead{Tafel 18.}
\vspace*{\fill}
\begin{figure}[H]
\centering
\includegraphics[width=\textwidth,keepaspectratio]{figs/18-1.png}
\caption{\small Figur 1 --- Glaskugel, Olivinkristall und Mikrolithe einschließend, aus dem Chondrit von Mezö-Madaras. Vergrößerung 90.}
\end{figure}
\vspace*{\fill}
\clearpage

\rhead{Tafel 18.}
\vspace*{\fill}
\begin{figure}[H]
\centering
\includegraphics[width=\textwidth,keepaspectratio]{figs/18-2.png}
\caption{\small Figur 2 --- Teilweise entglaste Zwischenmasse in einer Olivinkugel des Steines von Renazzo. Vergrößerung 300.}
\end{figure}
\vspace*{\fill}
\clearpage

\rhead{Tafel 18.}
\vspace*{\fill}
\begin{figure}[H]
\centering
\includegraphics[width=\textwidth,keepaspectratio]{figs/18-3.png}
\caption{\small Figur 3 --- Glaseinschluss mit Libelle im Olivin des Steines von Lancé. Vergrößerung 300.}
\end{figure}
\vspace*{\fill}
\clearpage

\rhead{Tafel 18.}
\vspace*{\fill}
\begin{figure}[H]
\centering
\includegraphics[width=\textwidth,keepaspectratio]{figs/18-4.png}
\caption{\small Figur 4 --- Glaseinschlüsse im Olivin des Steines von Stauropol. Vergrößerung 300.}
\end{figure}
\vspace*{\fill}
\clearpage

\rhead{Tafel 19.}
\subsection{Erklärung der Tafel 19.}
\paragraph{}
Figur 1. Zweierlei Entglasung in einer braunen Kugel des Meteoriten von Mezö-Madaras. Einerseits erkennt man eine zarte exzentrisch-radiale Faserung, deren Strahlungspunkt im Bilde an der unteren Grenze der Kugel liegt, anderseits bemerkt man viele Mikrolithe in radial angeordneten Flocken, welche im Durchschnitte blumenartige oder sternförmige Zeichnungen liefern. Die Fasern sind ungemein fein, ihre Farbe ist die des braunen Glases im selben Chondriten. Am oberen Rande zeigt die Kugel eine tiefe Einbuchtung. Die Umgebung ist zum Teile dunkle Grundmasse mit Partikeln von Magnetkies und Eisen, zum Teile sind es Olivinchondren oder Splitter von solchen.

Figur 2. Eine Doppelkugel im Stein von Borkut. Der kleinere Teil ist eine monosomatische gefächerte Olivinkugel mit trüber Zwischenmasse und blasser durchsichtiger Rinde; die große Kugel, welche die kleinere zur Hälfte umschließt, ist von derselben Beschaffenheit und löscht gleichzeitig mit dieser aus. Da der Stein von Borkut beim Schleifen leicht zermahlen wild, so ist auch die Doppelkugel zum Teil aus ihrer Verbindung mit den Nachbarn gebracht.

Figur 3. Um die häufig vorkommende Umhüllung der Chondren durch eine Eisenrinde in einem Beispiele darzustellen wurde ein Präparat des Steines aus Cabarras City bei gleichzeitiger Wirkung des auffallenden und des durchfallenden Lichtes fotografiert. Das Eisen, welches bei der Betrachtung im durchgehenden Lichte schwarz erscheinen würde, zeichnet sich hier grau mit den Merkmalen der Rauigkeit auf der geschliffenen Fläche. Die Eisenrinde, welche die porphyrische Olivinkugel umgibt, ist von ungleicher Dicke und stellenweise schwammig.

Figur 4. Der Tuffcharakter, welcher in vielen Chondriten erkennbar ist, wird hier durch eine Probe aus dem Chondrit von Mezö-Madaras illustriert. Das Bild zeigt durchweg Splitter von Chondren, verbunden durch eine spärliche dunkle Grundmasse. Oberhalb sieht man Teile von Olivinchondren, einen kleinen Splitter mit deutlichem Lamellenbau, in der Mitte das stumpfeckige Bruchstück einer porphyrischen Olivinkugel mit heller Glasmasse, rechts davon Olivinsplitter und Magnetkies, links ein Eisenkorn. Unterhalb hat man links die Splitter von einer radialfaserigen Bronzitkugel, rechts ein Stück von einer porphyrischen Olivinkugel mit stark entglaster Zwischenmasse.
\clearpage

\rhead{Tafel 19.}
\vspace*{\fill}
\begin{figure}[H]
\centering
\includegraphics[width=\textwidth,keepaspectratio]{figs/19-1.png}
\caption{\small Figur 1 --- Bronzitkugel mit sternförmigen Mikrolithen. Chrondrit von Mezö-Madaras. Vergrößerung 60.}
\end{figure}
\vspace*{\fill}
\clearpage

\rhead{Tafel 19.}
\vspace*{\fill}
\begin{figure}[H]
\centering
\includegraphics[width=\textwidth,keepaspectratio]{figs/19-2.png}
\caption{\small Figur 2 --- Olivin-Doppelkugel aus dem Chondrit von Borkut. Vergrößerung 160.}
\end{figure}
\vspace*{\fill}
\clearpage

\rhead{Tafel 19.}
\vspace*{\fill}
\begin{figure}[H]
\centering
\includegraphics[width=\textwidth,keepaspectratio]{figs/19-3.png}
\caption{\small Figur 3 --- Eisenhülle einer Olivinkugel im Chondrit von Cabarras City. Vergrößerung 60.}
\end{figure}
\vspace*{\fill}
\clearpage

\rhead{Tafel 19.}
\vspace*{\fill}
\begin{figure}[H]
\centering
\includegraphics[width=\textwidth,keepaspectratio]{figs/19-4.png}
\caption{\small Figur 4 --- Chondrensplitter im Chondrit von Mezö-Madaras. Vergrößerung 90.}
\end{figure}
\vspace*{\fill}
\clearpage

\rhead{Tafel 20.}
\subsection{Erklärung der Tafel 20.}
\paragraph{}
Figur 1. Um zu zeigen, dass die kohligen Chondrite wie jener vom Kapland nur durch die Imprägnation der Grundmasse von den übrigen verschieden sind, und Chondren derselben Art enthalten, wurde eine Stelle aus dem genannten Stein, welche eine porphyrische Olivinkugel mit dunkelbraunem Glase und ziemlich deutlichen Kristallen darbietet, zur Darstellung gebracht. Da die schwarze Grundmasse beim Schleifen leicht zerbröckelt, so ist diese im Präparat vielfach zerrissen. Man sieht in derselben außer den Chondren auch viele kleine durchsichtige Olivinsplitter.

Figur 2. Für die kohligen Chondrite sind die feinkörnigen lappigen Bildungen charakteristisch, welche hier durch eine Probe aus dem Stein von Grosnaja repräsentiert werden. Die Masse dieser Chondren erscheint im durchfallenden Lichte grau bis bräunlich, von filzartiger Textur mit vielen trüben Flocken und schwarzen Punkten. Im polarisierten Lichte erkennt man, von den trüben Stellen abgesehen, eine innige Verbindung feiner doppelbrechender Körnchen. Das Ganze hat Ähnlichkeit mit der trüben entglasten Zwischenmasse vieler Chondren.

Figur 3. Als ein Beispiel von löcheriger Struktur mancher Chondrite ist hier eine Stelle in einem Präparat aus dem Stein von Goalpara abgebildet. Die weißen Stellen in diesem Bilde sind nicht etwa durch Herausfallen einzelner Teilchen beim Schleifen bedingt, sondern rühren von ursprünglich vorhandenen Löchern her. Diese sind meist von schwarzer Grundmasse umgrenzt. Auf der rechten Seite des Bildes erkennt man zwei Kristallindividuen, deren eines rundliche Umrisse hat. Diese Körner, welche als Enstatit bestimmt wurden, bedingen das porphyrische Gefüge des Steines. In der Grundmasse heben sich viele kleine Körner hervor, die als Olivin zu deuten sind. Die schwarze Masse besteht aus Eisen und einem matten Körper, welcher als halbglasige kohligen Zwischenmasse die Körnchen umgibt und in deren Sprünge in feinen Verästelungen eindringt.

Figur 4. Die körnige Beschaffenheit der festen Chondrite ist hier durch ein dem Stein von Erxleben entnommenes Bild illustriert. In der Mitte oben zeichnet sich ein längliches Bronzitkorn mit kleinen durchsichtigen und großen länglichen dunklen Glaseinschlüssen. Darunter bemerkt man ein lichtes Körnchen ohne Zeichnung, welches einem Plagioklas angehört, da es im polarisierten Lichte eine deutliche Streifung zeigt. Im Übrigen hat man Olivinkörner, unten rechts ein solches mit horizontalen Streifen, welche von Lamellen und dem zwischenliegenden Glase herrühren. Die großen schwarzen Flecken sind die Schattenbilder von Eisen, teilweise von Magnetkies.
\clearpage

\rhead{Tafel 20.}
\vspace*{\fill}
\begin{figure}[H]
\centering
\includegraphics[width=\textwidth,keepaspectratio]{figs/20-1.png}
\caption{\small Figur 1 --- Olivinkugel im dem kohligen Chondrit von Cold-Bokkeveld. Vergrößerung 160.}
\end{figure}
\vspace*{\fill}
\clearpage

\rhead{Tafel 20.}
\vspace*{\fill}
\begin{figure}[H]
\centering
\includegraphics[width=\textwidth,keepaspectratio]{figs/20-2.png}
\caption{\small Figur 2 --- Feinkörnige lappige Chondren im Chondrit von Grosnaja. Vergrößerung 60.}
\end{figure}
\vspace*{\fill}
\clearpage

\rhead{Tafel 20.}
\vspace*{\fill}
\begin{figure}[H]
\centering
\includegraphics[width=\textwidth,keepaspectratio]{figs/20-3.png}
\caption{\small Figur 3 --- Löcheriger Chondrit von Goalpara. Vergrößerung 25.}
\end{figure}
\vspace*{\fill}
\clearpage

\rhead{Tafel 20.}
\vspace*{\fill}
\begin{figure}[H]
\centering
\includegraphics[width=\textwidth,keepaspectratio]{figs/20-4.png}
\caption{\small Figur 4 --- Körniger Chondrit von Erxleben. Vergrößerung 60.}
\end{figure}
\vspace*{\fill}
\clearpage

\rhead{Tafel 21.}
\subsection{Erklärung der Tafel 21.}
\paragraph{}
Figur 1. Die drei Lagen, aus welchen die Rinde der Chondrite besteht, sind hier in einem Bilde aus dem Stein von Chateau Renard ersichtlich, welches einen Vertikalschnitt darstellt. Da die Rinde als äußerster und dabei sprödester Teil ungemein leicht abbröckelt, so ist derselbe im Präparat nicht vollkommen erhalten. Links oberhalb erscheinen aber alle drei Zonen: zuerst eine dünne dunkle aus schwarzem bis braunem Glase bestehende äußere Schmelzrinde, darunter eine durchsichtige, hier aus Olivin und Maskelynit bestehende Saugzone, zu unterst die mächtige Imprägnationszone, welche dicker ist als die beiden vorigen zusammen. Letztere ist stellenweise von hellen Punkten durchsprenkelt. Unterhalb zeigt sich das unveränderte Gemenge, zumeist aus Olivinkörnern mit untergeordnetem farblosem Maskelynit, links ein Eisenkorn, unten eine radialfaserige Bronzitkugel.

Figur 2. Hier ist die Rinde vollkommen erhalten. Zu oberste erscheint die schwarze äußere Schmelzrinde, welche in den äußersten Teilen auch etwas von farblosem Glase (Maskelynit) erkennen lässt. Solche Stellen sind auf der äußeren Rinde im auffallenden Lichte glasig und sehen wie gefirnisst aus. Die zweite oder Saugzone ist wiederum hell, stellenweise von schwarzem Glase durchsetzt. Die dritte oder Imprägnationszone ist wiederum schwarz, stellenweise fein durchsprenkelt, rechts größere durchsichtige Olivinkörner enthaltend. Im unteren Teile des Bildes hat man die Darstellung des unveränderten Gemenges, welches hier aus Olivinkörnern, aus wenigen Bronzit- und Plagioklaskörnern und aus großen Partikeln von Magnetkies besteht.

Figur 3. An demselben Meteoriten von Mocs gelang ein Schnitt, welcher der Rinde ungefähr parallel geführt wurde. Da die Rinde krumm ist, so blieb derselbe nicht an allen Stellen in derselben Rindenschichte, sondern er geht gleichzeitig durch die Saugzone, welche durch die hellen Teile bezeichnet ist, und durch die Imprägnationszone, welche durchsprenkelt erscheint. In der Saugzone und in den hellen Partikeln der Imprägnationszone, lässt sich vorzugweise Olivin und nur hie und da ein Bronzitkorn erkennen. Die farblosen Körnchen und lappigen Häufchen, welche oft Olivinkörnchen einschließen und vollkommen dem Plagioklas gleichen, der in den übrigen Teilen des Steines den farblosen Gemengteil bildet, sind hier isotrop (Maskelynit).

Figur 4. Da ein Vertikalschnitt durch die Rinde eines Eukrits nicht gelang, so begnüge ich mich hier den Parallelschnitt durch die schwarze glasglänzenden Rinde des Steines von Juvinas darzustellen. Die Hauptmasse ist ein dunkelbraunes Glas mit vielen runden Vertiefungen und mit geschlossenen Blasen. Im Glase sind hie und da farblose Körner und Kristalle von Anorthit eingeschlossen, welche im polarisierten Lichte das gewöhnliche Verhalten zeigen. Ein Beispiel ist links am Rande des Bildes zu sehen, in der Mitte aber zeichnen sich feinkörnige doppelbrechende Partikel. Augit ist nicht zu bemerken.
\clearpage

\rhead{Tafel 21.}
\vspace*{\fill}
\begin{figure}[H]
\centering
\includegraphics[width=\textwidth,keepaspectratio]{figs/21-1.png}
\caption{\small Figur 1 --- Vertikalschnitt durch die Rinde des Chondrits von Chateau Renard. Vergrößerung 70.}
\end{figure}
\vspace*{\fill}
\clearpage

\rhead{Tafel 21.}
\vspace*{\fill}
\begin{figure}[H]
\centering
\includegraphics[width=\textwidth,keepaspectratio]{figs/21-2.png}
\caption{\small Figur 2 --- Rinde des Chondrits von Mocs im Vertikalschnitt. Vergrößerung 70.}
\end{figure}
\vspace*{\fill}
\clearpage

\rhead{Tafel 21.}
\vspace*{\fill}
\begin{figure}[H]
\centering
\includegraphics[width=\textwidth,keepaspectratio]{figs/21-3.png}
\caption{\small Figur 3 --- Rinde des Chondrits von Mocs, ungefähr parallel durchschnitten. Vergrößerung 70.}
\end{figure}
\vspace*{\fill}
\clearpage

\rhead{Tafel 21.}
\vspace*{\fill}
\begin{figure}[H]
\centering
\includegraphics[width=\textwidth,keepaspectratio]{figs/21-4.png}
\caption{\small Figur 4 --- Parallelschnitt durch die Rinde des Eukrits von Juvinas. Vergrößerung 160.}
\end{figure}
\vspace*{\fill}
\clearpage

\rhead{Tafel 22.}
\subsection{Erklärung der Tafel 22.}
\paragraph{}
Figur 1. Ein Beispiel für die in den Chondriten häufig vorkommenden Klüfte, welche sich im Querschnitte bald als feine bald als breite Adern darstellen. Die Grundmasse, welche hier vorwiegend Körner von Olivin, eine geringe Menge von Bronzitkörnern, Häufchen von Plagioklas und Körner von Magnetkies enthält, wird von zwei Systemen von Sprüngen durchsetzt, welche mit einer schwarzen glanzlosen Masse erfüllt sind. Die Sprünge schmiegen sich, wie dieser Fall erkennen lässt, gerne an die Magnetkieskörner an.

Figur 2. Von den breiten schwarzen gangähnlichen Massen, welche in Verbindung mit den vorbezeichneten Klüften in manchen Exemplaren des Meteoriten von Mocs beobachtet werden, ist hier eine gewählt, welche die Verhältnisse gut erkennen lässt. Der dunkle Streifen, welcher die Grundmasse durchzieht, ist ein Querschnitt des Ganges. Derselbe zeigt nur stellenweise eine schärfere Abgrenzung gegen die Grundmasse, meist jedoch einen allmähligen Übergang in dieselbe. In der schwarzen Masse liegen Eisentropfen und längliche Eisenklümpchen. Da die Aufnahme des Bildes zugleich im durchgehenden und im auffallenden Lichte ausgeführt wurde, so erscheint das Eisen im Bilde nicht schwarz sondern grau. In der Mitte der gangförmigen Masse bemerkt man nach der Länge derselben gestreckte Eisenfäden, ferner feine Quersprünge mit Eisen erfüllt. Besonders deutlich sind diese Eisenklüfte im unteren Teile des Bildes, wo dieselben mit einem länglichen Eisenklümpchen in Verbindung stehen. Eine solche nach rechts verlaufende Kluft hängt mit einem offenen Sprung zusammen. Die schwarze Masse ist von hellen Olivinpartikeln durchsprenkelt.

Figur 3. Der klastische Charakter vieler Stellen in dem Meteoriten von der Sierra de Chaco zeigt sich hier in einem Bilde, welches große und kleine Splitter von Plagioklas nebst Splittern von Bronzit und Olivin durch eine Eisengrundmasse verbunden darstellt. Der Plagioklas lässt schon im gewöhnlichen Lichte die Zwillingsbildung erkennen. Der Bronzit charakterisiert sich durch feine Spaltrisse, der Olivin ist mit dunklen Einschlüssen erfüllt.

Figur 4. Hier ist jener Plagioklas des Meteoriten von der Sierra de Chaco, welcher durch große durchsichtige kristallisierte Einschlüsse merkwürdig erscheint, dargestellt. Wie das Bild zeigt, sind die Einschlüsse vorzugsweise im Inneren der Kristalle angehäuft. Die Form derselben ist bald kurz-säulenförmig, bald ist der Umriss dreiseitig, fast quadratisch, fünfseitig oder rundlich. Manche sind von dreiseitigen Flächen eingeschlossen und haben ungefähr monokline Gestalt. Einige liegen mit ihren Längsaxen den Plagioklaslamellen parallel, andere nicht. Der Plagioklas zeigt außerdem einige schwarze opake Einschlüsse und ist von einem dunklen von Einschlüssen stark durch setzten Bronzit und von Augit umgeben.
\clearpage

\rhead{Tafel 22.}
\vspace*{\fill}
\begin{figure}[H]
\centering
\includegraphics[width=\textwidth,keepaspectratio]{figs/22-1.png}
\caption{\small Figur 1 --- Netzartig verbundene Klüfte im Chondrit von Mocs. Vergrößerung 60.}
\end{figure}
\vspace*{\fill}
\clearpage

\rhead{Tafel 22.}
\vspace*{\fill}
\begin{figure}[H]
\centering
\includegraphics[width=\textwidth,keepaspectratio]{figs/22-1.png}
\caption{\small Figur 2 --- Gangförmige Masse im Chondrit von Mocs. Vergrößerung 20.}
\end{figure}
\vspace*{\fill}
\clearpage

\rhead{Tafel 22.}
\vspace*{\fill}
\begin{figure}[H]
\centering
\includegraphics[width=\textwidth,keepaspectratio]{figs/22-1.png}
\caption{\small Figur 3 --- Grahamit von der Sierra de Chaco. Splitter von Plagioklas und Bronzit. Vergrößerung 30.}
\end{figure}
\vspace*{\fill}
\clearpage

\rhead{Tafel 22.}
\vspace*{\fill}
\begin{figure}[H]
\centering
\includegraphics[width=\textwidth,keepaspectratio]{figs/22-1.png}
\caption{\small Figur 4 --- Plagioklas mit Einschlüssen. Grahamit von der Sierra de Chaco. Vergrößerung 60.}
\end{figure}
\vspace*{\fill} 
\clearpage

\rhead{Tafel 23.}
\subsection{Erklärung der Tafel 23.}
\paragraph{}
Figur 1. Der Bronzit in der Masse von der Sierra de Chaco zeigt im Längsschnitte ein feinfaseriges Ansehen, wie es in diesem Bilde deutlich wird. Der Schnitt ist beiläufig parallel 0 1 0. Am oberen Ende sind Spuren einer Kristallendigung wahrzunehmen. Im Inneren bemerkt man einige feine Längslinien, welche Querschnitten dünner Blättchen entsprechen. Die Auslöschungsschiefe derselben von ungefähr 50° deutet auf einen Augit. Im Übrigen zeigen sich opake Einschlüsse von denen die größeren im auffallenden Lichte als Magnetkies erkannt werden. Einige haben ziemlich scharfe Umrisse wie negative Kristalle, andere sind unregelmäßig, zum Teil staubartig. Ein quer verlaufender Sprang ist mit Magnetkieskörnchen besetzt. Oberhalb ist der Bronzit von Eisen eingefasst, unten von Magnetkies, Eisen und Plagioklas begrenzt.

Figur 2. Der im selben Meteoriten stellenweise vorkommende Augit erscheint hier in einem Querschnitte der von vielen Spaltrissen durchzogen ist und dessen untere Grenze der Querfläche 1 0 0 entspricht. Der Schnitt liefert das Bild einer optischen Axe sehr schön. Einschlüsse sind sparsam. Von der faserigen Beschaffenheit des Bronzits ist hier nichts zu sehen. Oberhalb grenzt der Augit an feinkörnigen Plagioklas, welcher viele kleine kristallinische Einschlüsse enthält. Im übrigen besteht die Nachbarschaft aus Eisen, Magnetkies, Olivin, Bronzit.

Figur 3. Von den Olivinkörnern im Grahamit ist hier eines zur Darstellung gebracht, welches eine doppelte Rinde besitzt. Die innere ist eine schwarze Imprägnation, die äußere eine feinkörnige fast trübe Masse. Dies erinnert an manche Olivinchondren. Das innere Olivinkorn ist ein Individuum. Im Bilde zeigt dieses nur grobe Sprünge, doch lassen sich bei stärkerer Vergrößerung auch feine dunkle Striche wahrnehmen, die sich an den breiten Sprung horizontal ansetzen. Links vom Olivin hat man Splitter von Plagioklas, darunter einen, der von Einschlüssen ganz erfüllt ist.

Figur 4. Dieses Bild gibt eine Vorstellung von den feinen Linien, welche im Olivin der Grahamit und Mesosiderite öfters beobachtet werden. Von der Rinde und von den Sprüngen des Olivins her erstrecken sich zahlreiche feine Nadeln, welche teils grau teils braun erscheinen, in zwei aufeinander senkrechten Richtungen in das Innere des Kristalls. In den breiten Sprüngen erkennt man bei auffallendem Lichte stellenweise Magnetkies. Die feinen grauen Nadeln erweisen sich bei stärkerer Vergrößerung als Kanäle. Die braunen dürften ihre Füllung einer Oxydation des Magnetkieses verdanken. Im auffallenden Lichte zeigt das Olivinkorn einen bläulichen Schiller wie der Mondstein, besonders dort wo die grauen Nadeln häufiger sind. Auch dieses Korn ist von dichter Masse umgeben, welche gleichfalls Olivin zu sein scheint und mit Körnchen von Magnetkies gemengt ist. Früher habe ich diese Olivinkörner mit feinen Nadeln nicht als solche erkannt, sondern dieselben als ein anderes dem Cordierit ähnliches Silikat betrachtet (Sitzungsber. d. Wiener Akad. Bd. 88 Abt. 1. pag. 353).
\clearpage

\rhead{Tafel 23.}
\vspace*{\fill}
\begin{figure}[H]
\centering
\includegraphics[width=\textwidth,keepaspectratio]{figs/23-1.png}
\caption{\small Figur 1 --- Bronzit im Grahamit von der Sierra de Chaco. Vergrößerung 50.}
\end{figure}
\vspace*{\fill}
\clearpage

\rhead{Tafel 23.}
\vspace*{\fill}
\begin{figure}[H]
\centering
\includegraphics[width=\textwidth,keepaspectratio]{figs/23-2.png}
\caption{\small Figur 2 --- Augit und dichter Plagioklas in demselben Grahamit. Vergrößerung 70.}
\end{figure}
\vspace*{\fill}
\clearpage

\rhead{Tafel 23.}
\vspace*{\fill}
\begin{figure}[H]
\centering
\includegraphics[width=\textwidth,keepaspectratio]{figs/23-3.png}
\caption{\small Figur 3 --- Olivin mit feinkörniger Rinde neben Plagioklas in demselben Grahamit. Vergrößerung 50.}
\end{figure}
\vspace*{\fill}
\clearpage

\rhead{Tafel 23.}
\vspace*{\fill}
\begin{figure}[H]
\centering
\includegraphics[width=\textwidth,keepaspectratio]{figs/23-4.png}
\caption{\small Figur 4 --- Olivin mit Gitterzeichnung in demselben Grahamit. Vergrößerung 70.}
\end{figure}
\vspace*{\fill}
\clearpage

\rhead{Tafel 24.}
\subsection{Erklärung der Tafel 24.}
\paragraph{}
Figur 1. Das körnige Gemenge, welches die Silikatmasse des Mesosiderits von Estherville bildet, enthält an der hier abgebildeten Stelle mehrere größere Körner von Plagioklas, deren eines ebensolche kristallisierte Einschlüsse zeigt, wie der Plagioklas in der Masse von der Sierra de Chaco, während der langgestreckte Plagioklas im oberen Teile des Bildes durch staubartig feine Glaseier und Körnchen getrübt ist. Die Grundmasse zeigt außer Eisen und Magnetkies noch Bronzit mit feinen Spaltrissen und Olivin mit vielen dunkelbraunen Glaseinschlüssen.

Figur 2. Ansicht eines Schnittes durch ein Körnchen von Peckhamit parallel einer Spaltungsfläche. Das Blättchen ist graugrünlich, ziemlich trübe. Die größeren Einschlüsse erscheinen als untereinander parallele stäbchenförmige oder spindelförmige Hohlräume die öfter teilweise mit Glas gefüllt sind, ferner als schwarze Kugeln, welche Glasfüllungen zu enthalten scheinen, die kleinen Einschlüsse als rundliche oder kurzsäulenförmige braune Glaspartikel, endlich an die letzteren anschließend als feiner Staub, mit welchem die ganze Masse durchsäet ist.

Figur 3. Dieses Bild gibt einen Durchschnitt jenes Olivins in der Masse von Hainholz, der an Einschlüssen besonders reich ist, wieder. Der Olivin ist farblos, von Sprüngen durchzogen, welche mit rotbrauner, bei der Verwitterung entstandener Masse gefüllt sind. Die Einschlüsse sind rund oder eckig, dabei immer opak. Viele derselben zeigen im auffallenden Lichte die Farbe des Magnetkieses, andere sind schwarz und scheinen Glaseinschlüsse zu sein.

Figur 4. Eine Stelle in der Grundmasse des Mesosiderits von Hainholz, die frei von Eisen ist und wenig von den Oxydationsprodukten enthält, welche in dieser schon etwas verwitterten Masse verbreitet erscheinen. Dieselbe ist hier dargestellt um die feinkörnige Textur und das feste Gefüge zur Anschauung zu bringen. Die Kristalle und Körner sind Bronzit mit meistens deutlichen Spaltrissen, Augit von dem vorigen durch die schiefe Auslöschung unterscheidbar, Plagioklas in farblosen Körnern und Leistchen mit deutlicher Zwillingsbildung. Zwischen die Körner und Kristalle klemmt sich etwas von bräunlichem Glase ein, worin stellenweise feine Nadeln erkennbar sind.
\clearpage

\rhead{Tafel 24.}
\vspace*{\fill}
\begin{figure}[H]
\centering
\includegraphics[width=\textwidth,keepaspectratio]{figs/24-1.png}
\caption{\small Figur 1 --- Plagioklasreiche Stelle in dem Mesosiderit von Estherville. Vergrößerung 70.}
\end{figure}
\vspace*{\fill}
\clearpage

\rhead{Tafel 24.}
\vspace*{\fill}
\begin{figure}[H]
\centering
\includegraphics[width=\textwidth,keepaspectratio]{figs/24-2.png}
\caption{\small Figur 2 --- Pekhamit, bronzitähnlich, aus dem Mesosiderit von Estherville. Vergrößerung 120.}
\end{figure}
\vspace*{\fill}
\clearpage

\rhead{Tafel 24.}
\vspace*{\fill}
\begin{figure}[H]
\centering
\includegraphics[width=\textwidth,keepaspectratio]{figs/24-3.png}
\caption{\small Figur 3 --- Olivin mit Einschlüssen im Mesosiderit von Hainholz. Vergrößerung 120.}
\end{figure}
\vspace*{\fill}
\clearpage

\rhead{Tafel 24.}
\vspace*{\fill}
\begin{figure}[H]
\centering
\includegraphics[width=\textwidth,keepaspectratio]{figs/24-4.png}
\caption{\small Figur 4 --- Feinkörnige Grundmasse des Mesosiderits von Hainholz. Vergrößerung 120.}
\end{figure}
\vspace*{\fill}
\clearpage

\rhead{Tafel 25.}
\subsection{Erklärung der Tafel 25.}
\paragraph{}
Figur 1. Eines der körnigen Häufchen, welche der Tridymit in der Masse von Rittersgrün bildet, im Durchschnitte. Die Aufnahme geschah im polarisierten Lichte. Man erkennt an den hellgrauen Tönen die verschiedene Stellung der zwillingsgemäss verbundenen Individuen. Diese sind öfters ziemlich breit, zuweilen auch sehr schmal. Stellenweise sind winzige Individuen in Zwillingsstellang von größeren umschlossen. Der körnige Tridymit ist von breiten und schmäleren Sprüngen durchzogen, welche mit rotbrauner von einer Oxydation herrührender Masse erfüllt sind. Rechts grenzt Bronzit an.

Figur 2. Nach Entfernung des Eisens durch Salzsäure aus einer Probe der Masse von Rittersgrün wurde ein Blättchen von Tridymit isoliert, welches im polarisierten Lichte aufgenommen ist. Die Hauptschnitte des Nicols sind horizontal und vertikal zu denken. Das Blättchen zeigt oberhalb zwei unvollkommene Randkanten und ist von einem ziemlich ebenen Flächenpaar begrenzt. Als Hauptindividuum erscheint der mittlere helle Teil von sechsseitigem Umrisse. Derselbe hat rechts einen hakenförmigen Fortsatz, dessen Form durch dunkle Zwickel bedingt ist, welche einem Individuum von anderer Stellung angehören. Auch der linke Teil des Blättchens hat die letztere Stellung mit Ausnahme eines dreieckigen Feldes, welches sich an das Hauptindividuum anschließt und einen Mittelton zeigt.

Figur 3. Der Bronzit im Rittersgrüner Meteoriten ist hier durch einen Schnitt, welcher durch ein Korn mit stellenweise ebenflächiger Begrenzung hindurchgeht, repräsentiert. Die schelferige Schnittfläche, die gröberen und feineren Spaltlinien, die querlaufenden Sprünge treten im Bilde deutlich hervor. Die schwarzen Einschlüsse sind Magnetkies (Troilit), der braune rundliche Einschluss oberhalb ist doppelbrechend, ebenso der durchsichtige Teil des Einschlusses rechts unten. Es dürfte Olivin sein.

Figur 4. Die linke Hälfte des Bildes zeigt einige von G. Rose als Röhren bezeichnete Kanäle im Olivin des Brahiner Pallasits. Die größte Röhre erscheint am oberen Ende durch braunes Glas gefüllt. In der rechten Hälfte, welche dem Olivin aus der Masse von Krasnojarsk entspricht, hat man eine große vollständig mit braunem Glas gefüllte und eine kürzere farblos erscheinende Röhre, welche keine solche Füllung zeigt, jedoch ein fast farbloses Glas zu enthalten scheint.
\clearpage

\rhead{Tafel 25.}
\vspace*{\fill}
\begin{figure}[H]
\centering
\includegraphics[width=\textwidth,keepaspectratio]{figs/25-1.png}
\caption{\small Figur 1 --- Körniger Tridymit im Siderophyr von Rittersgrün im polaris. Lichte. Vergrößerung 60.}
\end{figure}
\vspace*{\fill}
\clearpage

\rhead{Tafel 25.}
\vspace*{\fill}
\begin{figure}[H]
\centering
\includegraphics[width=\textwidth,keepaspectratio]{figs/25-2.png}
\caption{\small Figur 2 --- Tridymitblättchen aus demselben Siderophyr im polaris. Lichte. Vergrößerung 100.}
\end{figure}
\vspace*{\fill}
\clearpage

\rhead{Tafel 25.}
\vspace*{\fill}
\begin{figure}[H]
\centering
\includegraphics[width=\textwidth,keepaspectratio]{figs/25-3.png}
\caption{\small Figur 3 --- Bronzit in demselben Siderophyr. Vergrößerung 60.}
\end{figure}
\vspace*{\fill}
\clearpage

\rhead{Tafel 25.}
\vspace*{\fill}
\begin{figure}[H]
\centering
\includegraphics[width=\textwidth,keepaspectratio]{figs/25-4.png}
\caption{\small Figur 4 --- Olivin mit feinen Röhren im Pallasit von Brahin (links) Vergrößerung 300. und im P. von Krasnojarsk (rechts) Vergrößerung 200.}
\end{figure}
\vspace*{\fill}
\clearpage
\end{document}
