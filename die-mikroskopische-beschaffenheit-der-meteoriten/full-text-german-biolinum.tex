\documentclass[a4paper, 11pt, oneside, polutonikogreek, german]{article}
\usepackage[sfdefault]{biolinum}
% Load encoding definitions (after font package)
\usepackage[LGR,T1]{fontenc}
\usepackage{textalpha}

\usepackage{listings}
\lstset{basicstyle=\ttfamily}

% Babel package:
\usepackage{babel}

% With XeTeX$\$LuaTeX, load fontspec after babel to use Unicode
% fonts for Latin script and LGR for Greek:
\ifdefined\luatexversion \usepackage{fontspec}\fi
\ifdefined\XeTeXrevision \usepackage{fontspec}\fi

% "Lipsiakos" italic font `cbleipzig`:
\newcommand*{\lishape}{\fontencoding{LGR}\fontfamily{cmr}%
		       \fontshape{li}\selectfont}
\DeclareTextFontCommand{\textli}{\lishape}

\usepackage{float}
\usepackage{graphicx}
\graphicspath{ {./figures/} }
\usepackage[figurename=]{caption}
\usepackage{booktabs}
\setlength{\emergencystretch}{15pt}
\usepackage{fancyhdr}
\usepackage{microtype}

\begin{document}
\begin{titlepage} % Suppresses headers and footers on the title page
	\centering % Centre everything on the title page
	\scshape % Use small caps for all text on the title page

	%------------------------------------------------
	%	Title
	%------------------------------------------------
	
	\rule{\textwidth}{1.6pt}\vspace*{-\baselineskip}\vspace*{2pt} % Thick horizontal rule
	\rule{\textwidth}{0.4pt} % Thin horizontal rule
	
	\vspace{0.75\baselineskip} % Whitespace above the title

        {\LARGE Die mikroskopische Beschaffenheit der Meteoriten erl"autert durch photographische Abbildungen} % Title
	
	\vspace{0.75\baselineskip} % Whitespace below the title
	
	\rule{\textwidth}{0.4pt}\vspace*{-\baselineskip}\vspace{3.2pt} % Thin horizontal rule
	\rule{\textwidth}{1.6pt} % Thick horizontal rule
	
	\vspace{1\baselineskip} % Whitespace after the title block
	
	%------------------------------------------------
	%	Subtitle
	%------------------------------------------------
	
	{herausgegeben von G. Tschermak.} % Subtitle or further description
	
	\vspace*{1\baselineskip} % Whitespace under the subtitle
	
	%------------------------------------------------
	%	Editor(s)
	%------------------------------------------------

 	{\small Die Aufnahmen von J. Grimm in Offenburg.\\ 25 Tafeln mit 100 mikrophotographischen Abbildungen.} % Subtitle or further description

	\vspace{1\baselineskip} % Whitespace before the editors

    %------------------------------------------------
	%	Cover photo
	%------------------------------------------------
	
	%\includegraphics[scale=1]{cover}
	
	%------------------------------------------------
	%	Publisher
	%------------------------------------------------
		
	\vspace*{\fill}% Whitespace under the publisher logo
	
	Stuttgart 1885. % Publication year
	
	{\small E. Schweizerbart'sche Verlagshandlung (E. Koch).} % Publisher

	\vspace{1\baselineskip} % Whitespace under the publisher logo

    Internet Archive Online Edition  % Publication year
	
	{\small Attribution NonCommercial ShareAlike 4.0 International } % Publisher
\end{titlepage}
\setlength{\parskip}{1mm plus1mm minus1mm}
\setcounter{tocdepth}{3}
\setcounter{secnumdepth}{3}
\tableofcontents
\clearpage
\section{Ank"undigung.}
\paragraph*{}
Da in dem jetzt abgeschlossenen Werk von Herrn Prof. Tschermak nur die Meteorsteine Ber"ucksichtigung finden, so haben die Herren A. Brezina und E. Cohen es "ubernommen, eine w"unschenswerthe Erg"anzung durch "ahnliche Behandlung der Meteoreisen zu liefern. Mit der Bearbeitung derselben ist schon begonnen worden, und die bisherigen Aufnahmen, welche nat"urlich im reflectirten Licht stattfinden m"ussen, haben durchaus befriedigende Resultate geliefert. Es wird das Haupfgewicht darauf gelegt werden, alle wichtigeren Structurformen zur Darstellung zu bringen, welche beim Aetzen polirter Platten hervortreten. Ferner werden die Reichenbach'schen Lamellen, die Art des Auftretens accessorischer Gemengteile, Ver"anderungen in der Structur in der N"ahe der Brandrinde u. s. w. zur Darstellung gelangen.

Unterzeichnete Verlagshandlung hofft, dass die erste Lieferung bis Ostern 1886 erscheinen kann. Auch f"ur dieses Werk sind zun"achst drei Lieferungen in Aussicht genommen.

Stuttgart, 15. Juli 1885.

\bigskip

E. Schweizerbart'sche Verlagshandlung (E. Koch).
\clearpage
\section{Vorwort.}
\paragraph*{}
Nur Wenigen ist es verg"onnt, die Meteoriten genauer kennen zu lernen und doch ist es der Wunsch Vieler sich mit den wichtigsten Eigenschaften dieser K"orper vertraut zu machen, welche aus fernen Himmelsr"aumen zur Erde gelangt, uns "uber einen Teil der Sternenwelt unmittelbare Nachricht bringen. Die Meteoriten, diese Splitter untergegangener oder vielleicht noch kreisender planetarischer Massen, geben uns eine fr"uher ungeahnte Gelegenheit, fremde kosmische K"orper mit den H"anden zu greifen, zu w"agen, zu messen, zu zerlegen, mit allen Mitteln der Mineralogie und Chemie zu pr"ufen. Was wir an ihnen wahrnehmen, erlaubt uns wichtige Kombinationen mit den Resultaten astronomischer Beobachtung sowie direkte Vergleiche mit der Steinrinde unseres Planeten. Diese bilden die Grundlage interessanter und bedeutungsvoller Schl"usse, die weit "uber den Bereich unserer Wohnst"atte hinausreichend unser Wissen und Vermuten "uber die fr"uheren Zust"ande sowohl der Erde als auch der ganzen Sternenwelt erheblich zu vervollkommnen geeignet sind.

Der Versuch, die Kenntnis von der Zusammensetzung dieser merkw"urdigen K"orper einem gr"o"seren Kreise zug"anglich zu machen, d"urfte daher von Vielen gebilligt werden. F"ur eine umfangreiche systematische Darstellung scheint es mir aber etwas zu fr"uhe, weil manche der einfacheren F"alle noch nicht gen"ugend untersucht sind. Dagegen halte ich es f"ur n"utzlich, jenen Weg einzuschlagen, welcher die Grundlage f"ur die fernere schriftliche Verst"andigung bietet, n"amlich die Publikation guter Abbildungen. Solche mangeln f"ur einige Abteilungen der Meteoriten g"anzlich, die vorhandenen aber sind in verschiedenen Werken und Zeitschriften verstreut. Schon die Sammlung der bisher ver"offentlichten Zeichnungen w"are vorteilhaft, besser aber ist es, wenn es gelingt, nicht blo"s das bisher dargestellte in getreuen Bildern wiederzugeben, sondern alles "uberhaupt Wichtige in gleicher Weise zu illustrieren.

Derlei Bilder k"onnen entweder die "au"sere Form und die Beschaffenheit der Oberfl"ache schildern oder das Gef"uge und die mineralogische Zusammensetzung der Meteoriten darstellen. Das erstere hat eine besondere Bedeutung f"ur die Naturgeschichte der Meteoriten, das letztere ein allgemeineres Interesse, weil nicht nur der Bestand jener kosmischen K"orper zur Anschauung gebracht, sondern auch der Vergleich mit der Textur der Gesteine und der Ausbildung der tellurischen Minerale erm"oglicht wird. Demnach mag der Versuch gerechtfertigt erscheinen, durch Auswahl guter Beispiele das Gef"uge und den Bestand jener Meteoriten, welche die Pr"ufung im durchfallenden Lichte gestatten, auf die bestm"ogliche Weise zur Anschauung zu bringen. Ich glaubte deshalb eine dahin gerichtete Anfrage des Herrn Prof. E. Cohen in Stra"sburg, welcher sich durch die Herausgabe der "Sammlung von Mikrophotographien zur Veranschaulichung der mikroskopischen Struktur von Mineralien und Gesteinen" so sehr verdient gemacht hat, in Betracht nehmen und der Aufforderung des Herrn J. Grimm, welcher jene Bilder in so vorz"uglicher Ausf"uhrung herstellte, sowie jener des Herrn Verlegers E. Koch entsprechen zu sollen, obgleich ich im Augenblicke schon durch die Herausgabe eines Lehrbuches der Mineralogie und andere Arbeiten in Anspruch genommen bin und obgleich mir die Schwierigkeit der Beschaffung des Materials keine geringe zu sein schien.

Da manche der in Betracht kommenden Meteoriten durch Kauf nicht zu erhalten sind, so war ich in mehreren F"allen auf die Bereitwilligkeit der Herren Museumsvorst"ande und Besitzer von Meteoritensammlungen hingewiesen. Es freut mich, sagen zu k"onnen, dass ich von mehreren Seiten durch leihweise "Uberlassung von Pr"aparaten (P), durch "Uberlassung von Splittern als Geschenk (G) oder im Tausche (T) in zuvorkommender Weise unterst"utzt wurde, und zwar von folgenden Herren, denen ich hiermit den geb"uhrenden Dank ausspreche:

Herr Direktor Dr. S. Aichhorn in Graz (T), Se. Excellenz Hr. Geheimrat Freiherr v. Braun in Wien (G), Hr. A. Fauser in Pest (G), Hr. Oberbergdirektor Dr. C. W. v. G"umbel in M"unchen (P), Hr. Intendant Dr. F. v. Hochstetter in Wien (P), durch freundliche Vermittlung des Hrn. Dr. A. Brezina, Hr. Professor Dr. J. A. Krenner in Pest (G), Hr. Professor A. v. Lasaulx in Bonn (P), Hr. Nevil-Story Maskelyne in Salthrop (P), Hr. Oberbergrat M. Websky in Berlin (P), Hr. Oberbergrat v. Zepharovich in Prag (G), Hr. Professor F. Zirkel in Leipzig (P). F"ur die eifrige Mithilfe bei der Bearbeitung des Materials bin ich Hrn. Dr. Max Schuster zu vielem Danke verpflichtet.

Die mir vorliegende Aufgabe habe ich so aufgefasst, dass es vor allem n"otig sei, Bilder zusammen zu stellen, welche das Gef"uge und die Gemengteile von Repr"asentanten aller Abteilungen der Meteorsteine darstellen, dass hingegen die Wiedergabe der mikroskopischen Verh"altnisse von weniger allgemeiner Bedeutung vorl"aufig wegbleiben k"onne. Die vorliegenden Tafeln sollen demnach dem Besitzer gleichsam eine systematische Pr"aparatensammlung ersetzen, ihm die Beurteilung der Meteorsteine nach ihrer inneren Beschaffenheit erm"oglichen und im vorkommenden Falle die Bestimmung der Gemengteile erleichtern.

F"ur die vorliegende Sammlung wurden drei Hefte zu acht Tafeln in Aussicht genommen, welche zugleich mit dem Texte alle Abteilungen der Meteoriten mit Ausnahme der silikatfreien Eisenmassen in systematischer Folge behandeln.

Wien im Mai 1883.

G. Tschermak.
\clearpage
\section{Allgemeines "uber die Beschaffenheit der Meteoriten.}
\subsection{"Au"sere Form.}
\paragraph*{}
Die Gestalt der Meteoriten ist keine regelm"a"sige, sie zeigt vielmehr immer nur zuf"allige Begrenzungen und zwar solche, welche beim Zersprengen eines Gesteins von richtungsloser Struktur entstehen. Die Meteoriten haben also die Form von Bruchst"ucken. Die Kanten sind jedoch h"aufig abgerundet, die Unebenheiten "ofters ausgeglichen. Fast alle Meteoriten sind mit einer dunklen Rinde "uberzogen, welche die Merkmale einer Schmelzung oft auch einer Trift an sich tr"agt and gew"ohnlich als Schmelzrinde bezeichnet wird. An den Meteoreisen hat die Rinde das Ansehen and die Beschaffenheit des Hammerschlages, sie ist zugleich Oxydationsrinde. Die Abrundung, Trift und Rindenbildung betrachtet man als Folgen des Widerstandes der Luft beim Eindringen der mit planetarischer Geschwindigkeit ankommenden Meteoriten und als das Resultat der Erhitzung, welche durch die Zusammendr"uckung der Luft entsteht.

\subsection{Gef"uge.}
\paragraph*{}
Das gr"obere Gef"uge oder die Struktur der Meteoriten ist verschieden, jedoch nicht sehr mannigfaltig. Viele Meteoreisen haben keine Struktur. Sie bestehen aus einem einzigen Individuum wie jenes von Braunau, oder sie stellen einen Kristallstock dar, wie die Eisen von Agram, Toluca. Andere sind jedoch k"ornig, wie die von Zacatecas, Rasgatà. Jene, welche den "Ubergang zu den Meteorsteinen bilden, indem sie K"orner und Kristalle von Silikaten eingelagert enthalten, haben oft ein porphyrartiges Ansehen, weil die Silikate in einer Grundmasse von Eisen zu schwimmen scheinen, z. B. die von Pallas entdeckte Masse von Krasnojarsk. Die n"achste Stufe bilden solche, die viele Silicatk"orner in einem zarteren Eisenschwamm zeigen, wie die St"ucke von Hainholz.

Manche Meteoreisen bieten den Charakter einer Breccie dar, indem sie Bruchst"ucke von Silikatgestein umschlie"sen, z. B. jene von Tula, Copiapo. Unter den Meteorsteinen kommen "ofters Breccien vor. Die grauen Bruchst"ucke sind durch eine dunklere Masse verkittet, wie in den Steinen von Dacca, St. Mesmin, Rutlam. Zuweilen ist diese Struktur undeutlich, wie in dem Stein von Juvinas, der blo"s den Wechsel grobk"orniger und feink"orniger Masse wahrnehmen l"asst. Die Grundmasse, welche die Tr"ummer verbindet, ist mitunter schw"arzlich, halbglasig und zeigt Spuren oder auch deutliche Kennzeichen einer Fluidaltextur, wie in den Durchschnitten der Steine von Orvinio und Chantonnay. Von der deutlichen Breccienstruktur bis zum gleichartigen Ansehen gibt es allerlei "Uberg"ange, die angeschliffen entweder blo"s feine dunkle wirr verlaufende Linien oder aber eine wolkige Zeichnung, endlich ein marmoriertes Ansehen darbieten. Haidinger hat auf diese Strukturverh"altnisse, besonders aber auf die Breccien- und Tuffstruktur aufmerksam gemacht.

Sehr h"aufig bestehen die Steine aus kleinen Bruchst"ucken, aus Splittern und aus Gesteinspulver. Solche Massen haben den Charakter eines vulkanischen Tuffs. Die Splitter sind bald ungef"ahr gleichartig (Shalka), bald auffallend ungleichartig (Loutolaks). Bisweilen unterscheidet sich die feste Grundmasse stark von den eingelagerten K"ornern, so dass ein porphyrartiges Aussehen entsteht, wie in dem Stein von Goalpara.

Manche von den steinartigen Meteoriten erscheinen kristallinisch-k"ornig, z. B. jene von Chassigny, Shergotty, Ibbenb"uhren, doch gibt es "Uberg"ange zur Tuffstruktur, so dass derselbe Stein von einem Beobachter f"ur kristallinisch von dem andern als klastisch bezeichnet wird, z. B. der von Stannern. Dichte halbglasige Massen sind selten (Tadjera). Sie haben "Ahnlichkeit mit der dunklen Grundmasse einiger fr"uher genannten Breccien.

Ein Gef"uge, welches an der Mehrzahl der Steine bald in auffallender Weise, bald weniger deutlich ausgesprochen vorkommt, ist das chondritische. K"ugelchen und "uberhaupt rundliche K"orper, welche bald aus einem einzigen Kristallindividuum, bald aus mehreren bestehen, "ofters auch aus verschiedenen Gemengteilen zusammengesetzt sind, bilden das Gestein fast allein (Borkut), oder sie lagern unverletzt, "ofters auch zersplittert in einer lockeren bis festen Tuffmasse (Ausson). Die kugeligen Gebilde, welche gew"ohnlich kleiner als erbsengro"s sind, werden von mir als Erstarrungsprodukte angesehen, oder, um es ungef"ahr anzudeuten, als erstarrte Tropfen. Die Grundmasse ist bisweilen schwarzgef"arbt, z. B. in Renazzo, Grosnaja. Einige Steine zeigen eine fast gleichartige schwarze durch Kohle gef"arbte Masse, wie jene von Cold Bokkeveld.

Das chondritische Gef"uge ist nur den Meteoriten eigent"umlich, den tellurischen Felsarten aber fremd. Die Sph"arulite in manchem Obsidian und Perlit sind zwar auch Silikatk"ugelchen, doch zeigen sie eine radiale Faserung um ihr Centrum, w"ahrend die Chondren, wenn sie "uberhaupt faserig sind, eine exzentrische Faserung besitzen, "uberdies ist hier und dort die Grundmasse von verschiedenem Gef"uge.

In manchen Steinen bemerkt man schwarze Kl"ufte, deren F"ullung wiederum der schwarzen Grundmasse in den fr"uher bezeichneten Breccien entspricht. Die W"ande an der Kluft erscheinen bisweilen gegeneinander verschoben. Beim Zerspringen erfolgt "ofters eine Trennung nach diesen Kl"uften und die entstehenden Fl"achen sind striemig und metallisch gl"anzend, sie zeigen Harnische (Pultusk, Mocs, Murcia).

Die tuffartigen Meteorsteine sind fein por"os. Fl"ussigkeiten werden begierig aufgesogen. Die Steine verhalten sich aber h"aufig so, als ob sie erhitzt oder gefrittet worden w"aren. Sie klingen beim Anschlagen wie die Backsteine. Selten sieht man gr"o"sere Poren, wie in dem Stein von Richmond. In jenem von Juvinas bemerkt man in den kleinen Hohlr"aumen Kristalle. In beiden F"allen erhalt man den Eindruck, als ob die Erscheinung von einer nachtr"aglichen Erhitzung des Steines herr"uhrte.

\subsection{Gemengteile.}
\paragraph*{}
Die homogenen Teile, welche die Meteoriten zusammensetzen, bieten keine gro"se Mannigfaltigkeit.

\emph{Eisen}. Reines Eisen und nickelhaltiges Eisen bilden die Hauptmasse des Meteoreisens, ferner die Grundmasse oder den Eisenschwamm der "Ubergangsglieder zu den Meteorsteinen. In den letzteren sind auch oft noch zusammenh"angende Eisenf"aden vorhanden, endlich ist das Eisen blo"s in Form getrennter Partikelchen verteilt.

\emph{Grafit} erscheint in knollenf"ormigen Einschl"ussen in manchen Meteoreisen und umgibt manchmal die Troilit.

\emph{Schreibersit} (Phosphornickeleisen, Rhabdit) ist oft in unregelm"a"sigen, scharfkantigen, tafelf"ormigen oder in nadelf"ormigen Einschl"ussen im Meteoreisen verteilt.

\emph{Troilit} und \emph{Magnetkies}. In den Eisenmassen bildet das Einfach-Schwefeleisen FeS, welches Haidinger Troilit nannte, kleinere oder gr"o"sere knollige, "ofters auch tafelf"ormige Einschl"usse. In den Steinen ist das Schwefeleisen etwas heller gef"arbt. Seine Zusammensetzung entspricht dem Magnetkies und hat auch dessen Form. Nach Brezina hat auch der Troilit die Magnetkiesform. Seltenere Schwefelverbindungen sind der dichte schwarze Daubreelit, welcher Chrom, Eisen und Schwefel enth"alt und der Oldhamit CaS, endlich der Osbornit.

\emph{Chromit}. In vielen Meteoriten in K"ornern oder oktaedrischen Kristallen enthalten.

Selten ist der \emph{Magnetit}, bisher blo"s im Stein von Shergotty, und der Tridymit (Asmanit) bisher nur in den Massen von Breitenbach und Rittersgr"un.

\emph{Olivin}. In ziemlich gro"sen Kristallen und K"ornern, in einigen porphyrartigen Meteoreisen und "Ubergangsgliedern, ferner in allen Chondriten und einigen anderen Meteoritenarten. F"ur sich allein bildet er den Stein von Chassigny.

\emph{Bronzit}. Sowohl die fast farblose oder wei"se eisenarme Verbindung Enstatit als auch der eisenhaltige Bronzit, endlich auch das von den Mineralogen als Hypersthen bezeichnete eisenreiche Glied ist in den Meteoriten repr"asentiert. Die Verbreitung ist die gleiche wie beim Olivin, auch bildet der Bronzit f"ur sich eine Meteoritenart.

\emph{Augit}. Kristalle von der Form des Augits, aber von geringerem Kalkgehalte als der tellurische Pyroxen, wurden in mehreren Steinen beobachtet. In der Form gr"o"serer und kleinerer K"ornchen ist er auch in anderen ziemlich verbreitet. Bronzit und Augit werde ich "ofters als "Pyroxen" zusammenfassen.

\emph{Anorthit}. In mehreren Steinen wurden Kristalle und K"orner von Anorthit als wesentlicher Gemengteil erkannt. In vielen Steinen finden sich K"orner von \emph{Plagioklas}, der wohl nicht immer die Zusammensetzung des Anorthits haben d"urfte. Merkw"urdig ist der

\emph{Maskelynit} von Plagioklas-Zusammensetzung, jedoch von einfacher Lichtbrechung.

\emph{Glas}. Au"ser den kristallisierten und kristallinischen Gemengteilen kommen "ofters auch glasartige vor. Die einen sind farblos, dem Maskelynit "ahnlich, die anderen farbig, meist braun, "ofters mit Anf"angen der Entglasung. Diese d"urften vorzugsweise Magnesiasilikat enthalten.

\emph{Kohle} und schmelzbare, in Alkohol aufl"osliche Kohlenwasserstoffe wurden in schwarzen Meteoriten nachgewiesen, als Seltenheit ein Carbonat von der Beschaffenheit des Breunnerits. Das in solchen Steinen gefundene Wasser halten Viele f"ur nachtr"aglich aufgenommen, weil die "ubrigen Meteoriten kein Wasser enthalten, ferner die gefundenen Sulfate f"ur sekund"are Bildungen, hervorgerufen durch die Verwitterung des enthaltenen Schwefeleisens.

Es bleibt noch zu bemerken, dass in manchen Schriften solche Bestandteile von Meteoriten angegeben wurden, welche nicht nachgewiesen sind, wie z. B. Blei, Eisenkies, Leucit, Schwefel, oder welche mit neuen Namen belegt wurden, sich sp"ater aber als etwas Bekanntes erwiesen, wie Shepardit, Piddingtonit.

\subsection{Einteilung.}
\paragraph*{}
G. Rose hat eine Einteilung der Meteoriten nach den herrschenden Gemengteilen vorgeschlagen, die mit den Meteoreisen beginnt und mit denjenigen Steinen schlie"st, welche die meiste "Ahnlichkeit mit tellurischen Felsarten haben.\footnote{Beschreibung und Einteilung der Meteoriten. Berlin, 1864.} Diese Folge verl"auft ungef"ahr im selben Sinne, wie die Abnahme des spezifischen Gewichtes und ist insofern eine geologische zu nennen, als nach der Parallele, welche von Daubrée zwischen der Zusammensetzung der Meteoriten und jener unseres Planeten gezogen wurde, zuerst der angenommene metallische Kern der Erde als die "alteste Bildung, hierauf die spezifisch leichteren Silikatmassen als die j"ungeren Bildungen in Betracht kommen.\footnote{Etudes recentes sur les Météorites. Journal de savants 1870.} Wenn die Meteoriten Splitter sind, welche von einem oder von mehreren kleinen planetarischen K"orpern herr"uhren, so wird man sich jedes solche kleine Gestirn "ahnlich wie die Erde gebaut denken, also in der Vorstellung eine metallische Kugel mit einer Silikatrinde konstruieren, welche letztere aus Erstarrungsprodukten und aus Tuffmassen bestand.\footnote{Tschermak. Die Bildung der Meteoriten und der Vulkanismus. Sitzungsber. d. Wiener Akad. Bd. 71. Abt. 2. (1875.)}

Die Rose'sche Einteilung lasst sich demnach als eine solche betrachten, welche den gegenw"artigen Vorstellungen von der Bildungsfolge Rechnung tr"agt und kann daher als eine nat"urliche bezeichnet werden. Trotzdem werde ich hier nicht diese, sondern die umgekehrte Anordnung befolgen, weil es mir f"ur den vorliegenden Fall zweckm"a"siger scheint, mit jenen Meteoriten zu beginnen, welche die meisten Ankn"upfungspunkte an die bekannten Felsarten darbieten.

Die Abteilungen werden folgende sein:
\begin{enumerate}
    \item Calciumreiche Steine, arm an gediegenem Eisen.  
    \item Magnesiumreiche Steine, arm an gediegenem Eisen.  
    \item Magnesiumreiche chondritische Steine mit gediegenem Eisen.  
    \item Eisen mit Silikaten.  
    \item Meteoreisen.
\end{enumerate}
\paragraph*{}
Innerhalb dieser Abteilungen werden den Grunds"atzen der Petrographie gem"a"s einzelne Meteoritenarten unterschieden, deren jede ein besonderes Gemenge oder eine eigent"umliche Struktur darbietet. Zu den von G. Rose aufgestellten Arten sind noch einige hinzugekommen, welche ich schon in dem Verzeichnisse von 1872 angegeben habe.\footnote{Mineralogische Mitteilungen 1872, p. 165.} Durch meine letzten Untersuchungen ist die Zusammensetzung einiger schon fr"uher unterschiedener Abteilungen genauer bestimmt, ferner sind neue bekannt geworden, so dass in einigen F"allen eine Umstellung und Neubenennung erforderlich wurde.

Die bis jetzt bekannten Meteoritenarten sind:
\begin{enumerate}
    \item Die wesentlichen Gemengteile sind Pyroxen und Plagioklase. Die Rinde ist gl"anzend.
    \begin{itemize}
        \item \emph{Eukrit} G. Rose. Augit und Anorthit, statt des letzteren auch Maskelynit.
        \item \emph{Howardit} G. Rose. Augit, Bronzit, Anorthit.
    \end{itemize}
    \item Pyroxen und Olivine bilden die wesentlichen Gemengteile. Die Rinde ist wenig gl"anzend bis matt, ebenso in den folgenden Abteilungen.
    \begin{itemize}
        \item \emph{Bustit} Autor. Diopsid und Enstatit.
        \item \emph{Chladnit} G. Rose. Enstatit mit wenig Anorthit.
        \item \emph{Diogenit} Aut. Bronzit.
        \item \emph{Amphoterit} Aut. Bronzit und Olivin.        
        \item \emph{Chassignit} G. Rose. Olivin.
    \end{itemize}
    \item Bronzit, Olivin, Eisen als wesentliche Gemengteile.
    \begin{itemize}
        \item \emph{Chondrit} G. Rose. Textur chondritisch.
    \end{itemize}
    \item Eisen, netzf"ormig, darin Silikate: Plagioklas, Olivin, Pyroxen, Troilit.
    \begin{itemize}
        \item \emph{Grahamit} Aut. Plagioklas, Bronzit, Augit im Eisen.
        \item \emph{Siderophyr} Aut. Bronzit im Eisen.
        \item \emph{Mesosiderit} G. Rose. Bronzit, Olivin im Eisen.
        \item \emph{Pallasit} G. Rose. Olivin im Eisen.
    \end{itemize}
    \item Eisen mit untergeordnetem Troilit, Schreibersit \emph{etc.}
    \begin{itemize}
        \item \emph{Meteoreisen}.
    \end{itemize}
\end{enumerate}
\paragraph*{}
Den Namen Shalkit, welchen G. Rose f"ur das Gemenge von Bronzit und Olivin vorgeschlagen hatte, habe ich aufgegeben, weil f"ur den Meteorstein von Shalka widersprechende Resultate bekannt wurden. Stattdessen will ich die Bezeichnung Amphoterit vorschlagen, ferner f"ur die aus Bronzit allein bestehenden Steine den Namen Diogenit.\footnote{Nach Diogenes von Apollonia, welcher zuerst klare Vorstellungen "uber den kosmischen Ursprung und die siderische Natur der Meteoriten aussprach.} Die neue Meteoritenart unter 4. musste auch durch eine Bezeichnung unterschieden werden, wof"ur ich Grahamit\footnote{Nach dem Chemiker Graham benannt, welcher den in Meteoreisen absorbiert enthaltenen Wasserstoff entdeckte.} w"ahlte, endlich waren auch die Massen von Breitenbach und Rittersgr"un als eine besondere Art (Siderophyr) hervorzuheben.
\clearpage
\section{Beschreibung der dargestellten Arten.}
\subsection{}
\subsubsection{Eukrit.}
\paragraph{}
Hierher geh"oren die Steine von Juvinas, Jonzac, Stannern, Petersborough, Konstantinopel. Sie sind wesentlich Gemenge von Augit und Anorthit. Am besten ist der Stein von Juvinas untersucht.

Das Gef"uge ist, wie schon G. Rose a. a. O. bemerkte, an verschiedenen Stellen sehr verschieden indem kristallinisch-kleink"ornige und undeutlich feink"ornige Gemenge wechseln. Die letzteren erscheinen unter dem Mikroskop kristallinisch bis tuffartig. Das Ganze hat also einen undeutlich breccienartigen Charakter. (Taf. 1, Fig. 1, 3.)

Der Anorthit ist deutlich auskristallisiert. Die eingeschlossenen Kristalle zeigen scharfe Umrisse, die in kleinen Drusenr"aumen sitzenden haben eine durch Vorherrschen von M = (0 1 0) tafelf"ormige Gestalt, an welcher noch die Flachen T, \emph{l}, P, \emph{x} zu bemerken sind.

Die Kristalle sind teils wasserhell, teils durch Einschl"usse getr"ubt und wei"s. Bei starker Vergr"o"serung erkennt man teils rundliche Glaseinschl"usse teils feine nadelf"ormige Gestalten, welche den Randzonen parallel angeordnet erscheinen. Das optische Verhalten ist im "ubrigen dasselbe, wie an dem Anorthit vom Vesuv.

Die eingeschlossenen Kristalle sind meist zwillingsartig gebaut oder sie erweisen sich als Zwillingsst"ocke von komplizierter Zusammensetzung. Ihre Gr"o"se betr"agt oft 2 mm. Im D"unnschliff erscheinen dieselben im auffallenden Lichte bl"aulich, im durchfallenden br"aunlich. Sie sind ungemein reich an sehr kleinen Einschl"ussen, die meistens farblos und nur selten braun gef"arbt befunden werden. Die meisten Einschl"usse sind rundlich, manche aber auch langgestreckt, alle zeigen sehr schmale Kontouren. Zuweilen haben sie eine Libelle, einige wenige enthalten ein schwarzes K"ornchen. Auf das polarisierte Licht "uben sie keine Wirkung. Ihre Anordnung entspricht immer der "au"seren Begrenzung, die l"anglichen sind meistens zur L"angsausdehnung der Kristalle parallel gestreckt. Man darf sonach die rundlichen wie die gestreckten als Glaseinschl"usse ansehen. Einschl"usse mit breiter Kontur, welche sich als Gasporen erkennen lassen, sind selten. (Taf. 1, Fig. 1; Taf. 2, Fig. 2, 3, 4.)

Die Augitkristalle, welche in den Drusenr"aumen auftreten, sind braunschwarz. Die Form ist dieselbe wie bei manchem Diopsid, indem gew"ohnlich \emph{c} = (0 0 1), \emph{u} = (1 1 1), \emph{o} = (2$^{\prime}$ 2 1), \emph{a} = (1 0 0), \emph{m} = (1 1 0) und \emph{b} = (0 1 0) auftreten.

Die Ausl"oschung auf der L"angsfl"ache gibt einen Winkel von 52° 10$^{\prime}$, w"ahrend derselbe Winkel f"ur Diopsid von Ala = 51° 6$^{\prime}$ n. Des Cl. Bl"attchen parallel \emph{a} = 100 geben das Bild einer optischen Axe in "ahnlicher Lage wie der Diopsid von Ala.

Die Schnitte parallel \emph{a} und \emph{b} lassen die Kristalle aus unz"ahligen d"unnen Lamellen parallel \emph{c} aufgebaut erscheinen. An manchen Stellen kann man fast mit Sicherheit erkennen, dass eine wiederholte Zwillingsbildung nach \emph{c} die Ursache ist, also genau so wie bei der entsprechenden schaligen Zusammensetzung der Diopside. Die schwarze Farbe des Augits r"uhrt von zahllosen Einschl"ussen her, welche meist schwarz seltener braun sind, teils nadelf"ormig teils rundlich geformt erscheinen. Die braunen rundlichen erweisen sich als Glaseinschl"usse, was die nadelf"ormigen sind, lie"s sich nicht bestimmen. Die rundlichen und staubartigen sind vorzugsweise der Endfl"ache \emph{c} = 0 0 1 parallel angeordnet, die nadelf"ormigen der aufrechten Axe parallel gelagert.

Die im Gestein eingeschlossenen Augite sind unvollkommene Kristalle oder K"orner ohne geradlinige Umrisse, voll von den eben genannten Einschl"ussen und h"aufig durch Querspr"unge gegliedert. An den W"anden der Spr"unge zeigen sich die Einschl"usse oft so sehr vermindert, dass man die blass br"aunliche Farbe des reinen Diopsids wahrnimmt. Dagegen sind die Spr"unge mit schwarzer Masse erf"ullt. Es sieht so aus, als ob die Substanz der Einschl"usse aus den W"anden in die Kl"ufte gewandert w"are. (Taf. 1, Fig. 1, 3; Taf. 3, Fig. 1.)

Ein fernerer Gemengteil ist jenes gelbe Silikat, welches schon G. Rose in der Form kleiner Bl"attchen wahrnahm. Dieselben finden sich hie und da in der Grundmasse, an manchen Punkten ragen sie in die Drusenr"aume oder setzen durch diese hindurch. Im D"unnschliffe zeigt sich, dass dieselben aus winzigen K"ornchen von blass br"aunlicher Farbe bestehen und dass sie an vielen Stellen dieselbe feine Lamellentextur wie der vorbeschriebene Augit besitzen. Die Schmelzbarkeit zu schwarzem Glase stimmt zu dieser "Ahnlichkeit, so dass man wohl kaum irre gehen wird, diese blassen K"ornchen f"ur Diopsid ohne schwarze Einschl"usse zu halten. Die Bl"attchen erscheinen demnach wie Pseudomorphosen nach einem nicht angebbaren Silikat. Im D"unnschliff zeigt sich aber, dass das gelbe Silikat nicht nur in dieser Form, sondern auch in verschiedenen k"ornigen Partikeln in der Grundmasse unregelm"a"sig verteilt sei und die Maschen zwischen den kleinen Anorthitkristallchen der Grundmasse ausf"ulle.

Obwohl man sich den Vorgang nicht leicht erkl"aren kann, so macht doch alles dies den Eindruck, als ob die fr"uher tuffartige Grundmasse umgeschmolzen w"are, wobei wieder Anorthit, andererseits aber gereinigter Diopsid auskristallisiert w"aren. Die gelben Bl"attchen w"aren dann Paramorphosen. Die Umschmelzung der por"osen Grundmasse w"urde auch die Entstehung der vielen kleinen Drusenr"aume begreiflich machen. Von opaken Gemengteilen, die aber nur in geringer Menge vorhanden sind, kennt man Magnetkies, Chromit und Nickeleisen.

Den Magnetkies hat G. Rose in Drusenr"aumen kristallisiert gefunden und die Form bestimmt, ferner auch das sparsame Vorkommen von Nickeleisen beobachtet. Ich fand au"serdem eisenschwarze K"ornchen, welche das Verhalten des Chromits zeigen.

Der Meteorit von Stannern ist dem vorigen sehr "ahnlich, doch zeigt er schon eine ausgesprochene Tuffstruktur.\footnote{G. Rose a. a. O. und Tschermak. Mineralogische Mitteilungen 1872, p. 83.} An demselben St"ucke sieht man deutlich k"ornige, kleine strahlige und fast dichte Splitter und Tr"ummer nebeneinander. Einzelne Steine sind k"ornig wie der von Juvinas, andere sind viel dunkler gef"arbt von feink"ornigem bis dichtem Gef"uge.

Der Anorthit und der Augit haben dieselben Eigenschaften wie im Stein von Juvinas, nur sieht man scharfe Kristallumrisse seltener, dagegen h"aufig eine Verwachsung von beiden Mineralen, wobei dieselben oft als abwechselnde Platten erscheinen. Das gelbe Silikat und der Magnetkies sind auch zu erkennen. (Taf. 2, Fig. 1; Taf. 3, Fig. 3.)

Die Meteoriten von Jonzac sind jenem von Juvinas ungemein "ahnlich, der von Petersborough in Tennessee n"ahert sich in seiner Beschaffenheit nach G. Rose dem Stein von Stannern.

Zu den Eukriten ist ferner noch der Meteorit von Shergotty zu z"ahlen, den ich vor Jahren beschrieb.\footnote{Sitzungsberichte der Wiener Akad. Bd. 65. Abt. 1, p. 122 and Tschermaks Mineralog. Mitteil. 1872, p. 87.} Derselbe ist ein deutlich k"orniges Gemenge, wesentlich bestehend aus gelblich grauen matten K"ornchen und Prismen, welche als ein Augit bestimmt wurden, ferner aus wasserhellen glasigen K"ornchen und S"aulchen, welche auf kein bisher bekanntes Mineral zu beziehen sind und von mir als Maskelynit bezeichnet wurden.

Der Augit verh"alt sich in durchfallendem Lichte wie der vulkanische Augit, er erscheint lichtbraun, fast ganz frei von Einschl"ussen. Eine Zwillingsbildung nach a ist ziemlich h"aufig, jedoch ohne Wiederholung. Die Tr"ubung, welche ihm ein fremdartiges Aussehen verleiht, bleibt ihm teilweise auch im durchfallenden Lichte. Dieselbe r"uhrt von ungemein feinen unregelm"a"sigen Spr"ungen her, welche schlie"sen lassen, dass der Augit eine mechanische Ver"anderung erfahren habe. Durch diese Beschaffenheit ist er von dem tellurischen Augit verschieden, ebenso durch die chemische Zusammensetzung, da er weniger Kalk enth"alt. Der Maskelynit ist vollkommen farblos und wasserhell. Er zeigt im D"unnschliffe meist langgestreckte Umrisse und parallel der L"ange feine Linien, so dass er im gew"ohnlichen Lichte ganz und gar den Eindruck von Plagioklas hervorruft. Die chemische Zusammensetzung entspricht gleichfalls einem Plagioklas aus der Labradoritreihe. Im polarisierten Lichte l"oscht er aber vollkommen aus, erweist sich also einfach brechend. Die Spr"unge in demselben entsprechen einem deutlich muscheligen Bruche wie bei einem Glas. Demnach verh"alt sich der Maskelynit wie ein durch Schmelzung oder "uberhaupt durch blo"se mechanische Ver"anderung in den amorphen Zustand "ubergef"uhrter Labradorit.

Einschl"usse sind in demselben "ofters zu bemerken. Sie sind ganz unregelm"a"sig geformt und bestehen aus Augit und aus Magnetit.

Stellenweise zeigen sich wei"se tr"ube Partikel, welche nur tr"uber Maskelynit sind, ferner sehr sparsam ein gelbes Silikat, welches ungemein kleine K"ornchen bildet und sich optisch zweiaxig verh"alt.

Von opaken Gemengteilen wurden zwei erkannt. Der eine ist Magnetit, welcher hier zum ersten male als Gemengteil eines Meteoriten gefunden wurde, der zweite, sehr sp"arlich verbreitete, Magnetkies.

\subsubsection{Howardit.}
\paragraph{}
Nach den bis jetzt bekannten Untersuchungen sind hierher zu rechnen die Meteoriten von Massing, Loutolaks, Białystok, Le Teilleul, Nobleborough, Francfort. Darunter sind die beiden zuerst angef"uhrten am besten bekannt.

Der Meteorit von Loutolaks hat ein tuffartiges Gef"uge. In einer erdigen lockeren grauen Grundmasse liegen Splitter und K"orner von gr"ungelber wei"ser und schwarzer Farbe, ferner auch kleine Bruchst"ucke eines Gemenges, welches leicht als Eukrit zu erkennen ist.

Das Ganze hat den Charakter eines vulkanischen Tuffs, indem Splitter von verschiedenen Mineralen, wie sie sonst nicht in demselben kristallinischen Gestein zusammenvorkommen, beisammen liegen and blo"s unregelm"a"sige Begrenzungen, selten aber Spuren von Kristallumrissen wahrzunehmen sind. Ich konnte unter den durchsichtigen Gemengteilen dreierlei Anorthite, viererlei Augite, ferner Bronzit unterscheiden.

Der Anorthit findet sich in den genannten kleinen Eukritbruchst"ucken mit denselben Eigenschaften wie in dem Stein von Stannern. Die kleinen runden Glaseinschl"usse sind in derselben Form und Verteilung vorhanden. Derjenige Anorthit oder "uberhaupt Plagioklas, welcher ohne Verwachsung mit Augit in Splittern verbreitet ist, erscheint entweder dem vorigen gleich oder er enth"alt gro"se dunkle Einschl"usse von Glas oder Grundmasse, oder aber er ist fast ganz frei von Einschl"ussen.

Der Augit, welcher in den Eukritbruchst"ucken enthalten ist, hat dieselben Eigenschaften, wie jener in dem Stein von Stannern. Man sieht braune K"orner, bisweilen mit schwarzen Linien und schwarz gef"ullten Spr"ungen, vorwiegend aber gelben k"ornigen Augit mit feinschaliger Zusammensetzung. Jene Augitsplitter und K"orner, welche in gro"ser Menge in der Grundmasse liegen, sind entweder den beiden vorigen gleich oder sie sind mehr gr"unlich gef"arbt und von ausgezeichnet feinschaliger Zusammensetzung nach 0 0 1. An manchen dieser K"orner l"asst sich auch die entsprechende Zwillingsbildung erkennen, an anderen beobachtet man viel schwarze nadelf"ormige parallel gelagerte Einschl"usse. Die vierte Form des Augits erscheint in gr"o"seren Splittern von sehr blass br"aunlicher Farbe ohne schalige Zusammensetzung.

Der Bronzit bildet sehr blass gr"unlich gef"arbte gr"o"sere Splitter fast ohne Einschl"usse. Die gerade Ausl"oschung und das faserige Wesen charakterisieren diesen Gemengteil hinreichend. Um aber vollst"andig sicher zu gehen, habe ich die gelbgr"unen K"orner, welche nach der mikroskopischen Pr"ufung als Bronzit bestimmt wurden, noch besonders gepr"uft, weil sie fr"uher f"ur Olivin angesehen worden waren. Ich erhielt aber die Spaltbarkeit des Bronzits. Bei der Behandlung des Pulvers mit konzentrierter Salzs"aure wurde dasselbe nur sehr wenig angegriffen.

F"ur Olivin halte ich einzelne kleine Splitter in der Grundmasse, ferner vermute ich denselben in jenen Gemengen, welche als kleink"ornige Gesteinsplitter vorkommen und oft reich an beigemengten schwarzen K"ornchen sind. Letztere bilden einen Teil der schon mit freiem Auge wahrnehmbaren dunklen K"orner und Splitter. Die anderen erwiesen sich als gleichartig mit jenen, welche auch im Stein von Stannern auftreten und feink"orniger bis dichter Eukrit sind.

Einige kleine pechschwarze K"orner sind wohl als Chromit anzusehen. Dass eine sehr geringe Menge von Magnetkies und von gediegen Eisen vorhanden sei, geht schon aus den Beobachtungen von Partsch und G. Rose hervor. (Taf. 4, Fig. 1, 2, 4.)

Die angef"uhrten Beobachtungen wurden an einem Exemplar gemacht, welches ich von Hrn. Prof. Wiik, also aus der besten Quelle erhielt, und welches mit dem Exemplar des Wiener Hofmuseums vollst"andig "ubereinstimmte. Beim Vergleich mit G. Roses Resultaten stellt sich heraus, dass die von diesem Forscher f"ur Olivin gehaltenen gelbgr"unen K"orner von mir als Augit und Bronzit bestimmt wurden. Ein Umstand, welcher fr"uher die richtige Beurteilung des Meteoriten erschwerte, ist eine von Berzelius ausgef"uhrte Analyse, nach welcher der Stein gr"o"stenteils aus Olivin best"unde. Die Analysen von Arppe (Rammelsberg, D. chem. Nat. d. Meteoriten 1870), welche der von mir angegebenen Zusammensetzung vollkommen entsprechen, haben jedoch jene irrt"umliche Bestimmung beseitigt.

Der Stein von Massing ist dem vorigen sehr "ahnlich. Ich konnte dies an dem kleinen Pr"aparate, welches mir von Hrn. Oberbergrat v. G"umbel "uberlassen wurde, gen"ugend sicher erkennen. Der Stein ist ebenfalls ein Tuff in dem sowohl Kristallsplitter, als auch kleine Bruchst"ucke dichten Gesteins durch eine erdige Grundmasse verbunden sind. Unter den Splittern sieht man Anorthit vom gleichen Aussehen und mit den gleichen Einschl"ussen, wie in dem Stein von Loutolaks. Der Augit ist in derselben Weise vertreten in braunen, gelben, sowie in den gr"unlichgrauen Splittern mit feinschaligem Baue.

Der Bronzit hat dasselbe Ansehen, doch kommen "ofters Kristalle mit gut erhaltener Form vor, wo von einer auf Taf. 4, Fig 3 dargestellt ist. Vereinzelt finden sich aber auch stengelige Splitter, "ahnlich jenen, welche in den Chondriten so gew"ohnlich sind.

Die kleinen Gesteinsbruchst"ucke sind auch von ungef"ahr gleicher Art und auch ungemein dicht, so dass hier die Gegenwart von Bronzit nur beil"aufig zu bestimmen ist. Chromit und Magnetkies erscheinen auch in derselben Weise, wie im vorigen Meteorit. Mit diesem Befunde stimmen die Beobachtungen G"umbels bis auf die Deutung der gr"unlichen Splitter als Olivin sehr gut "uberein und die Analyse Schwagers harmoniert ebenfalls mit demselben.\footnote{Sitzungsberichte der bayrischen Akademie. 1878. 1.}

Der Meteorit von Białystok ist nach G. Rose dem vorigen sehr "ahnlich, jener von Le Teilleul, welcher in dem Verzeichnis des Pariser Museums zu den Howarditen gerechnet wird, scheint mir, nach dem im Wiener Hofmuseum liegenden St"ucke zu urteilen, bestimmt dazu zu geh"oren.
\clearpage
\subsection{}
\subsubsection{Bustit.}
\paragraph{}
Das Gemenge von Diopsid und Enstatit ist bisher blo"s durch den Stein von Busti bei Goruckpur (gefallen am 2. Dezember 1852) repr"asentiert. Maskelyne hat denselben untersucht, die einzelnen Gemengteile gemessen und analysirt.\footnote{Proceedings of the Royal Society 18. 146.} Das Gef"uge ist beinahe kristallinisch, doch unterscheidet man Kristalle und gr"o"sere Splitter, welche in einer aus feinen Splittern bestehenden Grundmasse liegen. Flight hat eine Abbildung des ganzen Steines ver"offentlicht, welche die ungleichartige Mengung deutlich wahrnehmen lasst.\footnote{Geological Magazine, September 1875.} Der Stein hat keine Rinde.

Der Diopsid, welcher das herrschende Mineral ist, erscheint im auffallenden Lichte grau bis violett. Maskelyne konnte an K"ornern die Prismenzone bestimmen und auch eine Pyramidenfl"ache erkennen. Dieser Gemengteil ist meist auffallend durch seine feinschalige Zusammensetzung nach 1 0 0, welche oft mit wiederholter Zwillingsbildung nach dieser Fl"ache verbunden ist. Au"ser dieser Bl"atterung, welche dem Diallag entspricht, ist "ofters auch noch eine schalige Zusammensetzung nach 0 0 1 mit einer deutlichen Zwillingsbildung nach dieser Fl"ache wahrnehmbar. Die oft reichlichen Einschl"usse sind schwarz und bald nadelf"ormig, der ersten Lamellierung parallel gelagert, bald rundlich. Sie sind die Ursache der violetten F"arbung.

Der Enstatit l"asst "ofters scharfe Begrenzungen wahrnehmen. Maskelyne konnte nur die Prismenzone bestimmen. Derselbe unterscheidet dreierlei Enstatite, den grauen undurchsichtigen, den graulichwei"sen durchscheinenden und den farblosen wasserhellen Enstatit. Im D"unnschliff erkennt man ebenfalls verschiedene Arten. Der graue f"uhrt eine gro"se Anzahl von Glaseinschl"ussen mit sich, welche bisweilen eine fixe Libelle haben. Sie zeigen sehr oft einen polygonalen Umriss und erscheinen als negative Kristalle, die mit einem blass br"aunlichen Glase erf"ullt sind. Wenn viele solche Einschl"usse vorhanden sind, ist der Enstatit tr"ube. Es gibt aber auch v"ollig farblose Splitter, die ganz frei von Einschl"ussen sind.

Au"ser diesen beiden Gemengteilen fanden sich untergeordnet Plagioklas, Oldhamit, Nickeleisen, Osbornit. Der Plagioklas wird von Maskelyne nicht angef"uhrt, er ist auch nur sp"arlich vorhanden, doch konnte ich die farblosen Splitter, welche fast frei von Einschl"ussen sind und keine Zwillingslamellen erkennen lassen, mit gro"ser Wahrscheinlichkeit auf Plagioklas beziehen, weil dieser Gemengteil mit dem Plagioklas im Stein von Bishopville im "ubrigen v"ollig "ubereinstimmt. Der Oldhamit CaS ist nur in einem Teile des Steines in rundlichen K"ornern von tesseraler Spaltbarkeit vorhanden. Das Nickeleisen ist nur in geringer Menge, noch sp"arlicher der Osbornit enthalten. Der letztere zeigt Oktaeder und die Reaktionen auf Schwefel, Calcium und Titan oder Zirconium. Die Einschl"usse im Diopsid d"urften nach Maskelyne Osbornit sein.

Die Bilder Fig. 1 und 2 auf Taf. 5 sind aus einem Pr"aparat erhalten, welches mir Herr Nevil-Story Maskelyne bereitwilligst zur Ben"utzung "uberlie"s.

\subsubsection{Chladnit.}
\paragraph{}
Auch dieses Gemenge ist bisher nur in einem einzigen Meteoriten und zwar in jenem von Bishopville gefunden worden. Der Stein ist grobk"ornig und besteht zum gr"o"sten Teil aus schneewei"sem lockeren Enstatit. G. Rose bemerkte auch noch andere wei"se K"ornchen, vermochte sie jedoch nicht zu bestimmen. Nach meinen Beobachtungen geh"oren dieselben zum Plagioklas. Der dritte Gemengteil ist Magnetkies. Der Stein hatte eine marmorierte Rinde, teils farblos, teils schwarz, wei"s, bl"aulich und grau.

Der Enstatit bildet meist gro"se, aber auch kleine K"orner. An einem der letzteren konnte ich scharfe Umrisse wahrnehmen. Der Schnitt ging ungef"ahr parallel \emph{a} = 1 0 0. Die Endigung des Kristalles war dreifl"achig, eine Fl"ache entsprach der Zone \emph{p} \emph{a},  die beiden andern den Zonen \emph{u} \emph{b}.\footnote{Hier und im Folgenden sind f"ur Bronzit und Enstatit \emph{a} = \emph{b} bei v. Rath, \emph{u} = \emph{u} bei v. Rath. \emph{b} = \emph{a} bei v. Rath, \emph{p} = \emph{k} bei v. Rath.} Die K"orner sind von vielen feinen unregelm"a"sigen Spr"ungen durchsetzt, abgesehen von den Spaltrissen, welche beim Pr"aparieren entstehen. Einschl"usse sind nur in geringer Menge vorhanden und bestehen aus opaken K"ornchen, seltener aus schwarzen Nadeln.

Der Plagioklas ist meistens mit den kleinen Enstatitk"ornern verbunden. Niemals beobachtete ich eine regelm"a"sige Begrenzung. Die Umrisse sind rundlich lappig oder gestreckt. Im polarisierten Lichte sieht man bisweilen eine sehr deutliche Zwillingstextur, indem entweder breite Lamellen in Wechselstellung erscheinen oder aber manche K"orner aus ungemein schmalen Lamellen zusammengesetzt sind, so dass dieselben zwischen gekreuzten Nicols "au"serst fein liniiert erscheinen. Die "ubrigen K"orner haben eine einfache, gew"ohnlich aber eine undul"ose Ausl"oschung, manche sind aus mehreren kleinen K"ornchen zusammengesetzt. Auf das Verhalten im polarisierten Lichte gr"undet sich die Bestimmung als Plagioklas. Der Versuch, einzelne K"ornchen f"ur weitere Pr"ufung aus dem Gemenge zu sondern, misslang nicht nur wegen ihrer Kleinheit, sondern auch deshalb, weil dieselben weder durch die Farbe noch durch den Glanz Vom Enstatit unterschieden werden k"onnen.

Der Plagioklas zeigt stellenweise Schlieren und zarte Tr"ubung, in welchem Falle derselbe im durchfallenden Lichte br"aunlich erscheint. Kleine opake Einschl"usse sind selten, dagegen kommen gr"o"sere, oft spindelf"ormige Bronziteinschl"usse nicht selten vor. Der Magnetkies bildet gr"o"sere und kleinere K"orner, die in den vorliegenden St"ucken von einem braunen, durch Einwirkung der Luft entstandenen Hof umgeben sind.

Mit der angegebenen mikroskopischen Beschaffenheit stimmt die Analyse Rammelsbergs,\footnote{Monatsberichte der Berliner Akademie. 1861, p. 895.} welche au"ser den Bestandteilen des Enstatits auch Tonerde, Kalk and Alkalien in geringer Menge angibt, vollkommen "uberein.

G. Rose gibt auch noch geringe Mengen von Nickeleisen and ein schwarzes Mineral an, welches hie und da feine Kluftausf"ullungen bildet. Beim Zerbrechen erhielt ich auf solchen Kl"uften gl"anzende Harnische, "ahnlich wie in sp"ater anzuf"uhrenden Meteoriten, in welchen diese aus Eisen, Magnetkies and Silikatschmelze bestehen.

Da der Enstatit keine anderen Erscheinungen darbietet als der Bronzit in folgenden Meteoriten, so wurde in den Fig. 3 und 4 auf Tafel 5 vorzugsweise der Plagioklas zur Anschauung gebracht.

\subsubsection{Diogenit.}
\paragraph{}
Diese Abteilung ist von der vorigen mineralogisch wenig verschieden, da der wesentliche Gemengteil dem Bronzit oder Hypersthen entspricht, Gattungen, welche mit dem Enstatit durch "Uberg"ange verbunden sind. Die Trennung erfolgt also nur wegen des bedeutenden Gehaltes an Eisenoxydul. Hierher geh"oren die Steine von Manegaum, Ibbenb"uhren und wohl auch der von Shalka.

Das Innere der beiden ersteren ist hell graulichgelb mit gr"o"seren lichtgelbgr"unen K"ornern. Sowohl diese K"orner als auch die "ubrige Masse bestehen aus einem Bronzit mit 20.5 resp. 17.5 Eisenoxydul. In dem Stein von Manegaum fand Maskelyne auch geringe Mengen von Chromit und gediegen Eisen.\footnote{Philosophical Transactions 160. p. 189. (1870.)} Im Stein von Ibbenb"uhren konnte v. Rath nur Bronzit erkennen, abgesehen von wenigen Einschl"ussen.\footnote{Monatsberichte der Berliner Akademie 1872, p. 27. Pogg. Ann. 146. p. 474.}

Der Meteorit von Ibbenb"uhren ist ungemein gleichartig. (Taf. 6. Fig. 2.) Der Bronzit bildet gro"se und kleine K"orner fast ohne eine Spur von regelm"a"siger Form. Im polarisierten Lichte zeigen manche Individuen eine zarte Streifung, jedoch ohne ausgesprochenen Zwillingscharakter. Zuweilen zeigen sich auch einzelne sehr d"unne Lamellen von schiefer Ausl"oschung eingeschlossen. Dieselben d"urften einem Augit angeh"oren. Einschl"usse sind nur sparsam vorhanden, und zwar teils rotbraune Glaseinschl"usse, teils opake K"ornchen, welche Magnetkies und Chromit sein d"urften. Auch sieht man bisweilen sehr schmale schwarze Kl"ufte, deren F"ullung eine braunschwarze Masse ist. An einer Stelle beobachtete ich zwischen den Bronzitk"ornern ein farbloses Mineral, aus einem Aggregat kleiner K"ornchen bestehend, welche Zwillingsbildungen "ahnlich denen der Plagioklase wahrnehmen lie"sen, doch nicht so ausgesprochen, dass die Bestimmung sicher w"are. Es k"onnte auch Tridymit sein.

Der Stein von Shalka zeigt in einer hellgrauen etwas zerreiblichen Masse gr"o"sere gr"unlichgraue K"orner von Bronzit und schwarze K"orner von Chromit. Im D"unnschliff l"asst sich erkennen, dass alles Durchsichtige Bronzit ist und die gro"sen K"orner desselben, die bisweilen Kristallumrisse zeigen, in einer Grundmasse von Bronzitsplittern liegen. (Taf. 6, Fig. 1.) Der Bronzit enth"alt "ofters braune Glaseinschl"usse oder opake K"ornchen. Manche der letzteren sind nach den Spr"ungen im Bronzit angeordnet, d"urften also erst nachtr"aglich abgesetzt worden sein. Es ist mir wahrscheinlich, dass die letzteren aus Magnetkies bestehen. Beim Behandeln des Meteoriten mit S"aure wird in der Tat etwas Schwefelwasserstoff entwickelt. Gr"ungelbe K"ornchen, die man hie und da beobachtet, hielt G. Rose (a. a. O. p. 125) f"ur Olivin, daher dieser Forscher den Stein von Shalka als ein Olivingemenge definierte, wof"ur die Abteilung Shalkit aufgestellt wurde. Maskelyne vermochte jedoch keinen Olivin zu finden, auch mir gelang es nicht, solchen nachzuweisen. Ich isolierte einzelne der gelbgr"unen K"ornchen, fand jedoch die Spaltbarkeit des Bronzits und bei der Behandlung mit conc. Salzs"aure nur eine sehr geringe Zersetzung. Somit ist nur erwiesen, dass au"ser dem herrschenden gr"unlichgrauen Bronzit auch gelbgr"uner untergeordnet vorkommt, was bei der Tuffstruktur des Meteoriten begreiflich ist.

\subsubsection{Amphoterit.}
\paragraph{}
Von dieser Abteilung kennt man bisher nur einen Meteoriten, n"amlich den von Manbhoom in Bengalen (22. Dez. 1863). Derselbe ist ein gr"ungelbliches k"orniges Gemenge, in welchem der Bronzit und Olivin fast die gleiche Farbe zeigen. Au"ser diesen sind auch zahlreiche K"orner von Magnetkies und wenige K"orner von Eisen bemerkbar.

Bei einem Versuche, welchen ich vor Jahren ausf"uhrte, erhielt ich ungef"ahr 33\% in S"aure Unl"osliches, welches als Bronzit erkannt wurde. In einem D"unnschliffe, welchen ich damals herstellen lie"s und welchen ich aus dem k. k. Hofmuseum zur Ben"utzung erhielt, l"asst sich k"orniger Olivin, von vielen Spr"ungen durchzogen und arm an Einschl"ussen als Hauptgemengteil and Bronzit in l"anglichen bis rundlichen K"ornern von etwas faserigem Ansehen leicht erkennen. Beide sind blassgr"un. Au"ser diesen sind aber auch farblose K"ornchen hie und da eingestreut, welche sich optisch so verhalten, wie der sp"ater beim Chondrit zu beschreibende Plagioklas. Die rundlichen opaken K"orner sind Magnetkies, einige l"angliche Eisen. (Taf. 6, Fig. 3.)

\subsubsection{Chassignit.}
\paragraph{}
Auch diese Art ist bisher nur durch einen Meteoriten, den Stein von Chassigny repr"asentiert. Nach G. Rose bildet dieser eine kleink"ornige, fast gleichartige etwas zerreibliche Masse von gr"unlichgelber ins Graue ziehender Farbe. Vauquelin fand schon, dass das Pulver von Salzs"aure unter Gallertbildung zersetzt wird und fand bei der Analyse die Verh"altnisse des Olivins.

Im D"unnschliffe sieht man blass gelbgr"une, beil"aufig gleichgro"se K"orner, die fast "uberall enge aneinanderschlie"sen und jene f"ur den meteoritischen Olivin oft so charakteristischen gr"oberen und feineren Spr"unge zeigen. (Taf. 6, Fig. 4.) Sie enthalten nur wenige br"aunliche Glaseinschl"usse. Zwischen den Olivink"ornern bleiben hie und da kleine, oft dreiseitige Zwischenr"aume, die mit farblosem oder braunem Glase ausgef"ullt sind. Diese Glaspartikel erscheinen oft als das Zentrum von radial in den Olivin verlaufenden Spr"ungen. Bei st"arkerer Vergr"o"serung bemerkt man in dem Glas oft sehr viele farblose K"ornchen oder zierliche Nadeln, welche Doppelbrechung zeigen oder auch braune Krist"allchen. Es ist also in vielen derselben schon eine Entglasung eingetreten.

Chromit, oft in deutlichen Oktaedern, ist ungef"ahr gleichf"ormig eingestreut. Bisweilen liegt ein kleiner Chromitkristall mitten in einem Glaspartikel. Der Schliff ist Eigentum des k. k. Min. Hofmuseums.
\clearpage
\subsection{}
\subsubsection{Chondrit.}
\paragraph{}
Hierher geh"ort die gro"se Mehrzahl der steinartigen Meteoriten, womit gesagt ist, dass die Meteorsteine sowohl der Textur als dem Bestande nach meistens gleichartig sind.

In Bezug auf die Gemengteile wiederholen die Chondrite den Bestand der Amphoterite, da sie haupts"achlich aus Olivin und Bronzit bestehen, jedoch enthalten sie au"serdem auch Eisen und Magnetkies in erheblicher Menge, sowie untergeordnet Chromit. Akzessorisch kommen aber auch die Minerale der Eukrit, n"amlich Plagioklas und Augit vor, "ofters auch farbloses Glas (dem Maskelynit "ahnlich), braunes Glas, ein doppelbrechender noch nicht bestimmter Gemengteil und ein bisher nicht erw"ahnter, n"amlich Kohle.

Das feinere Gef"uge ist wesentlich durch das Vorkommen der Chondren charakterisiert. Dasselbe schwankt aber zwischen den folgenden Extremen

a. Vollkommen chondritisch, wenn von Grundmasse fast nichts zu bemerken ist, die Chondren fast allein herrschen.

b. Tuffartig, erdig, wofern die aus kleinen Splittern bestehende Grundmasse vorwiegt und zuweilen auch Gesteinbruchst"ucke vorkommen.

c. Halbglasig, wenn der Stein eine dichte schimmernde Masse darstellt.

d. Kristallinisch, wenn der Meteorit fast g"anzlich aus festgef"ugten K"ornern besteht und die Chondren sehr zur"ucktreten.

Die Grundmasse besteht aus Splittern oder K"ornern, von welchen die ersteren sich nicht immer genauer bestimmen lassen. Unter den Splittern sind h"aufig auch Bruchst"ucke von Chondren deutlich erkennbar. Au"ser den durchsichtigen Partikelchen kommen immer auch opake vor, welche sich als Eisen, Magnetkies, zuweilen auch als Chromit erkennen lassen. Wenn Kohle auftritt, so ist sie meistens in der Grundmasse gleichf"ormig verteilt, so dass dieselbe schwarz erscheint. Auch der Magnetkies verbreitet sich zuweilen als Impr"agnation stellenweise in der Grundmasse.

Au"ser den Splittern und K"ornern, welche sich auf Olivin und Bronzit, zuweilen auch auf Augit beziehen lassen, finden sich in der Grundmasse oft auch kleine rundliche farblose K"orner, die entweder einfach brechend sind, so dass man die Wahl h"atte, sie als Glas oder Maskelynit anzusprechen, oder doppelbrechend sind und gew"ohnlich eine undul"ose Ausl"oschung zeigen, so dass man mit einiger Wahrscheinlichkeit Plagioklas annehmen darf. Manche dieser K"ornchen zeigen aber im polarisierten Lichte Zwillingslamellen in Wechselstellung in genau derselben Weise wie die triklinen Feldspate, daher die Bestimmung als Plagioklas wohl eicher ist. Da letztere K"ornchen dieselbe Form, Gr"o"se und Verwachsung mit der Umgebung zeigen wie die vorigen, so ist es wahrscheinlich, dass alle drei Arten substantiell gleich, also die doppelbrechenden auf Plagioklas, die einfach brechenden auf Maskelynit zu beziehen sind. Alle diese K"ornchen und K"ornergruppen sind mit der Umgebung innig verwachsen und schlie"sen oft Partikelchen der anderen Silikate ein, sie sind niemals zersplittert. Daraus l"asst sich schlie"sen, dass sie sp"ater gebildet seien, und zwar nach der Ablagerung des Gesteintuffs.

Au"ser diesen K"ornern finden sich in der Grundmasse auch scharfkantige gr"o"sere K"orner, welche gleichfalls mit der Grundmasse innig verbunden erscheinen. Sie sind fast farblos, lassen Spuren einer unvollkommenen Spaltbarkeit, "ofters auch viele zarte, krumm verlaufende Spr"unge erkennen. Zwischen gekreuzten Nicols geben sie nur geringe Aufhellung, einen grauen Farbenton, jedoch niemals sch"onere Interferenzfarben. Durch dieses Verhalten, die Farblosigkeit und die zarten Spr"unge sind sie von allen "ubrigen durchsichtigen Gemengteilen leicht zu unterscheiden. Im konvergenten Lichte l"asst sich konstatieren, dass sie zweiaxig sind. Ich konnte diese K"orner bisher mit keinem bekannten Mineral identifizieren.

Die Chondren, welche schon von Reichenbach\footnote{Poggendorffs Annalen, Band 111, p. 353.} und G. Rose als merkw"urdige Bildungen hervorgehoben wurden, sind eine charakteristische Form, in welcher dieselben K"orper, die auch in der Grundmasse vorkommen, einzeln oder gemengt auftreten. Ihr Zusammenvorkommen ist aber ein anderes, als jenes der Gemengteile in echt kristallinischem Gestein, indem nicht Chondren von bestimmter Beschaffenheit nebeneinander liegen, sondern die verschiedenartigsten im selben Gemenge angetroffen werden, in der Art wie die verschiedenartigsten Minerale in einem Tuff beisammen liegen. Beispiele geben Taf. 7, Fig. 1 bis 4.

Die Gr"o"se der Chondren ist variabel, manchmal werden solche von Walnussgro"se beobachtet, zuweilen wiederum solche von staubartiger Kleinheit. Am h"aufigsten sind sie etwa hirsekorngro"s. Die Oberfl"ache ist meistens etwas rau bis h"ockerig, seltener glatt (an harten faserigen K"ugelchen).

Die "au"sere Form der Chondren ist verschieden und wechselt in einem und demselben Meteoriten. Von den vollkommen runden Chondren bis zu den unf"ormlichen St"ucken lassen sich alle "Uberg"ange wahrnehmen.

Manche Chondren erscheinen kugelrund. (Tafel 7 und 8.) Sind solche Chondren sehr fest und die Grundmasse locker, so k"onnen erstere leicht herausgenommen werden, worauf sie eine Runde H"ohlung hinterlassen. Die lockeren zerbrechen dagegen oft, wenn man sie zu isolieren versucht. Die festen runden Chondren zeigen oft eine merkw"urdige Deformation. Sie bieten dann runde Aush"ohlungen dar, so, als ob sich an dem noch weichen K"ugelchen ein anderes hartes abgeformt h"atte.\footnote{Zuerst in dem Stein von Tieschitz beobachtet. Denkschriften d. Kais. Ak. d. Wiss. z. Wien. Math. naturw. Kl. Bd. 39. p. 187.} Auf Tafel 7, Fig. 3 sind die Durchschnitte zweier solcher K"ugelchen dargestellt. Andere Chondren sind abgeplattet oder l"anglichrund, wieder andere zeigen Vorspr"unge und Einbuchtungen. An diese schlie"sen sich bez"uglich der Form jene an, welche lappig oder fetzenartig erschienen. Derlei Chondren lassen sich meistens nicht unverletzt aus der Grundmasse nehmen, daher sich die Gestalt meist nur aus den Umrissen des Durchschnittes ergibt. Auf Tafel 7 sind Beispiele solcher Formen gegeben. An jene Chondren, welche wie abgerundete Splitter aussehen, schlie"sen sich endlich jene, welche als gr"o"sere Gesteinst"ucke mit rundlichen Kanten erscheinen und so das Extrem der Chondrenbildung darstellen, wie in dem Stein von Alexinaé. Die Kontouren der Chondren sind nicht immer scharf, daher bei der mikroskopischen Beobachtung die Grenze gegen die Grundmasse ganz oder teilweise undeutlich erscheint. Bei fl"uchtiger Betrachtung werden viele Chondren leicht "ubersehen und mit Grundmasse verwechselt, scharfe Kristalle als der letzteren zugeh"orig betrachtet, w"ahrend sie einem porphyrischen Chondrum angeh"oren \emph{etc.}

Da die Gemengteile der Chondren im Allgemeinen dieselben sind, wie jene der Grundmasse, so zeigt sich auch "ofters wenig Unterschied in der Farbe der beiden, besonders bei den grauwei"sen, wie Milena, Alfianello, h"aufig aber sind die Chondren desselben Steines ungleich gef"arbt, die einen wei"s oder grau, die anderen braun oder schwarz und auch die Grundmasse unterscheidet sich "ofters durch die Farbe von jenen, besonders in den Steinen mit schwarzer Grundmasse, wie Renazzo.

Die Textur der Chondren ist mannigfaltig. Manche bestehen aus einem einzigen Kristallindividuum, sind monosomatisch, wofern man von den darin vorkommenden Einschl"ussen absieht. Viele bestehen aus mehreren Individuen derselben Art, sind polysomatisch und erscheinen k"ornig oder bl"atterig, st"angelig, faserig. Die gemischten, aus mehrerlei Gemengteilen bestehenden sind wiederum k"ornig, bl"atterig, faserig oder aber porphyrisch. Bisweilen zeigt sich in solchen auch zweierlei Textur, indem z. B. ein Teil des Chondrums bl"atterig oder k"ornig, der andere faserig erscheint. Die dichten Chondren sind wohl meistens zu den gemischten zu rechnen, doch m"ogen auch einfache darunter vorkommen.

Obwohl die Zusammensetzung der Chondren wenig mannigfaltig erscheint, so wird doch die mikroskopische Bestimmung "ofters schwierig und bei sehr kleinen Individuen zuweilen unsicher, da namentlich der Olivin der Chondrite dem Bronzit oft ungemein "ahnlich ist. In der Mehrzahl der F"alle kann man aber auch dann, wenn keine Formen erkennbar sind, aus dem Charakter der Spaltlinien und der Oberfl"ache des Schliffes einen sicheren Schluss ziehen.

Nach den Gemengteilen und der Mischung angeordnet, ergeben sich folgende Arten des Baues der Chondren. \emph{Olivinchondren}. Monosomatische K"ugelchen kommen "ofters vor. Nur wenige sind frei von Einschl"ussen, so dass sie ein wahres kugelf"ormiges Individuum darstellen. Andere sind in regelm"a"siger Weise von Einschl"ussen durchsetzt und erscheinen als gef"acherte Kugeln. Um den Charakter derselben richtig aufzufassen, geht man von der Beschaffenheit der Olivinkristall aus, wie solche in den porphyrischen Chondren h"aufig scharf ausgebildet vorkommen. An diesen erkennt man im Innern "ofters eine schalige oder "uberhaupt l"uckenhafte Bildung, indem sich dort eine Glasmasse ausbreitet, die in den Schnitten bald eine ungef"ahr sichelf"ormige Figur, bald mehrere solche in paralleler Stellung aufeinander folgende Figuren ergibt.

Beispiele solcher Kristalle sind in Fig. 1 auf Taf. 9 zu sehen, oberhalb ein Kristall mit einem gro"sen Glaseinschluss in der Mitte und zwei kleineren solchen Einschl"ussen, entsprechend dem schichtenartigen Baue, unterhalb ein gr"o"serer gef"acherter Kristall mit mehreren tafelf"ormigen im Bilde ungef"ahr horizontalen Glaseinschl"ussen. Das Glas erscheint hier und in vielen der folgenden Bilder dunkelgrau bis schwarz, w"ahrend es tats"achlich braun und durchsichtig ist. In den Steinen von Borkut und von Knyahinya wurden "ofter solche Kristalle beobachtet, welche aus einer vollkommen geschlossenen gleich dicken Rinde, im Inneren aber aus einem Fachwerk bestehen, welches mit der Rinde zusammenh"angt und aus mehreren parallel der Fl"ache \emph{b} = 0 1 0 gelagerten Lamellen besteht. Zwischen diesen Lamellen ist ein braunes Glas oder eine in Entglasung begriffene Masse eingelagert, welche letztere eine Neigung zur k"ornigen bis faserigen Ausbildung verr"at. Die Lamellenbildung im Inneren der Kristalle ist bisweilen wohl noch deutlich, jedoch nicht mehr so gleichartig, wie im letzteren Falle. Dies zeigt Fig. 3 auf derselben Tafel in einem Kristalldurchschnitte, welcher durch zwei gro"se Glaseinschl"usse unterbrochen und einerseits ge"offnet ist. Die Einschl"usse erscheinen geweihf"ormig, indem sie die R"aume zwischen zackenf"ormig vorspringenden Lamellen ausf"ullen.

Mit den Kristallen, welche aus einer Rinde und im Inneren aus gleichorientierten Lamellen bestehen, zwischen denen Glasmasse eingeschlossen ist, kommen die gef"acherten Olivinkugeln "uberein, wovon eine in Fig. 2 auf Taf. 10 dargestellt ist. Die Rinde ist einheitlich gebildet und l"oscht gleichzeitig mit s"amtlichen Olivinlamellen aus. Da in diesem Beispiele der Olivin gelbgr"un, das Glas hellbraun gef"arbt ist, so hebt sich im Bilde das Glas wenig ab. Wenn die Lamellen sich nicht ununterbrochen durch das Innere erstrecken, sondern immer nur auf kurze Strecken fortsetzen, so erscheint die zwischengelagerte Glasmasse netzartig wie in Fig. 4 auf Taf. 9. Die Glasmasse, welche in geeigneten Schnitten solcher Kugeln Streifen oder Netze bildet, ist seltener hellfarbig und durchsichtig, h"aufiger dunkelbraun oder tr"ube durchsichtig. Oft ist die Glasmasse sp"arlich und die Grenze gegen die Lamellen unscharf, so dass die Streifen und Netze blo"s als eine graue Zeichnung erscheinen. Sehr h"aufig ist aber die F"ullmasse teilweise oder fast ganz entglast. In letzterem Falle besteht dieselbe aus Glas und vielen feinen K"ornchen oder Fasern. Statt des braunen Glases ist die F"ullmasse in manchen F"allen ein k"orniger Plagioklas oder auch ein farbloses Glas (Maskelynit).

Makroskopisch sind jene Olivink"ugelchen, welche durchgehende Lamellen enthalten, sehr auffallend, weil sie beim Zerschlagen T"afelchen liefern und einen vollkommen spaltbaren Gemengteil vermuten lassen. Die Angaben von Feldspat in manchen "alteren Publikationen d"urften sich hierauf beziehen. Ohne genauere Untersuchung w"urde man auch jetzt noch derlei Kugeln eher f"ur Bronzit als f"ur Olivin halten. G. Rose hat die parallelen Streifen, welche viele Kugeln im durchfallenden Lichte zeigen, schon beobachtet und abgebildet. Die von ihm benutzten Pr"aparate waren jedoch, wie ich mich durch die G"ute des Herrn Oberbergrates Websky "uberzeugen konnte, viel zu dick, als dass er den Unterschied zwischen Olivin und Glasmasse h"atte wahrnehmen k"onnen.

Wenn der Schnitt, welcher Olivinkugeln von der angegebenen Beschaffenheit trifft, schief gegen die Ebene der Lamellen gerichtet ist, so wird die Regelm"a"sigkeit des Baues weniger deutlich hervortreten, besonders in den K"ugelchen mit netzartig verteilter Glasmasse. Hierher geh"ort das in Fig. 4 auf Taf. 8 gegebene Bild, in welchem eine Olivinkugel mit ungew"ohnlich dicker Rinde dargestellt ist und die F"arbung, welche das netzartig verteilte Glas dem Inneren erteilt, hervorgehoben wird. Wenn im Inneren keine parallelen Lamellen auftreten und demzufolge das Glasnetz unregelm"a"sig erscheint, so gibt der Durchschnitt oft eine gekr"oseartige Textur wie in Fig. 1 auf Taf. 11.

Sowohl an den Kristallen von Olivin, als auch an den Kugeln ist "ofters eine Einseitigkeit der Ausbildung bemerklich, wof"ur auf Taf. 9 die Figuren 3 und 4 Beispiele geben. Unter den monosomatischen Olivink"ugelchen zeigen sich bisweilen auch solche, die zwar auch aus einem l"uckenhaft gebildeten Kristall bestehen, aber eine ganz andere Ausbildung desselben zeigen, indem derselbe innen kompakt, nach au"sen aber skelettartig oder strauchartig geformt ist und hier in den L"ucken reichliche Glasmasse beherbergt. "Ofters werden auch solche Kugeln beobachtet, welche gleichf"ormig l"uckenhaft gebildete Individuen sind, indem sie nach Art der gestrickten Formen aus unz"ahligen St"abchen bestehen, welche scharenweise nach derselben Richtung gestreckt sind. Das ganze Skelett l"oscht gleichzeitig aus. Die L"ucken sind durch Glas erf"ullt. Ein Beispiel gibt Fig. 3 auf Taf. 10. Hier weicht nur ein kleiner Teil des K"ugelchens in seiner optischen Orientierung von der Hauptmasse ab. Im Stein von Mez"o-Madaras kommen monosomatische K"ugelchen vor, in welchen die feinen St"abchen der gestrickten Bildung deutlich drei auf einander senkrechte Richtungen verfolgen, welche gem"a"s der Ausl"oschung zugleich die Richtungen der Kristallaxen sind.

Die polysomatischen Olivinchondren sind mannigfaltig. Die einen reihen sich an die vorbeschriebenen gef"acherten Kugeln an, indem sie aus mehreren Systemen paralleler Tafeln mit zwischenliegender Glasmasse bestehen. Fig. 4 auf Taf. 10 gibt ein hierhergeh"origes Beispiel. Andere schlie"sen sich insofern an, als sie innen einheitlich gebildet sind und hier aus abwechselnden Lamellen von Olivin und Glas bestehen, nach Au"sen aber polysomatisch sind, indem die Rinde aus vielen Individuen zusammengesetzt ist. Fig. 2 auf Taf. 11. Derlei Kugeln haben oft eine dunkle Rinde, indem sich daselbst Einschl"usse von Eisen und Magnetkies massenhaft einstellen. Die Bedeckung der Olivink"ugelchen mit Eisen und Magnetkies, ferner die Durchtr"ankung der Rinde seitens dieser opaken Begleiter ist eine h"aufige Erscheinung, daher viele Olivink"ugelchen, wie schon G. Rose bemerkte, beim Zerschlagen eine dunkle Rinde zeigen.

Viele Olivinkugeln sind porphyrisch, indem sie deutliche Kristalle in glasiger seltener feink"orniger Grundmasse zeigen. Manche haben eine dicke Rinde und schlie"sen sich in dieser Hinsicht an die zuvor besprochenen Gebilde an, Fig. 3 auf Taf. 8 und 11, w"ahrend andere kaum eine Andeutung von Rinde wahrnehmen lassen. Fig. 4 auf Taf. 7. In letzter Figur bat man ein ausgezeichnetes Beispiel der porphyrischen Struktur. Die Kristalle sind bald kompakt, bald von Glaseinschl"ussen durchzogen, wie schon fr"uher bemerkt wurde. Nach den Durchschnitten zu urteilen, ist die Form eine einfache, meistens blo"s aus \emph{m} = (1 1 0), \emph{b} = (0 1 0), \emph{k} = (0 2 1) bestehende, wie an den Kristallen in den Olivinschlacken. Zwillinge wurden nicht beobachtet. Alle diese Kristalle sind von vielen feinen Rissen durchsetzt. Das optische Verhalten bietet nichts ungew"ohnliches. Zuweilen sieht man in der Glasgrundmasse der K"ugelchen unvollendete Olivinkristall in der Form von gabeligen Mikrolithen von zierlichen farnkraut"ahnlichen Gestalten, von netzartigen H"aufchen oder von gr"oberen skelettartigen Bildungen, aus T"afelchen und St"abchen bestehend, welche sich rechtwinkelig anordnen. Zu den letzteren geh"ort der in Fig. 1 Taf. 10 gegebene Durchschnitt. Die fadenf"ormigen oder nadelf"ormigen Glaseinschl"usse, welche sich hier zeigen, stimmen der Lage nach mit jenen "uberein, welche in manchen kompakten Kristallen bei st"arkerer Vergr"o"serung als St"abchen wahrgenommen werden. Obwohl Glaseinschl"usse im Olivin so au"serordentlich h"aufig sind, so finden sich doch Libellen, welche einem eingeschlossenen Dampfe entsprechen, sehr selten. Das Beispiel einer Libelle gibt Fig. 3 auf Taf. 18.

An die porphyrischen Olivinkugeln schlie"sen sich durch allm"ahlige "Uberg"ange verbunden die k"ornigen an. Wenn die Glasgrundmasse abnimmt, schlie"sen sich die Kristalle enge aneinander an, doch zeigen sich noch scharfe Umrisse, wie in Fig. 2 auf Taf. 7, bei verschwindender Grundmasse bemerkt man an den K"ornern selten mehr eine bestimmte Form: Fig. 2 auf Taf. 8. Diese Figur gibt auch ein Beispiel daf"ur, dass die k"ornigen Kugeln oft nach au"sen zu reich an Eisen und Magnetkies erscheinen und demnach eine dunkle Rinde darbieten. H"aufig sind, besonders in den kohligen Chondriten h"ochst feink"ornige Chondren von rundlicher oder lappiger Form, welche nach der "Ahnlichkeit des Gef"uges mit den fr"uher bezeichneten auch als Olivinchondren anzusehen sind. Kleink"ornige bis feink"ornige Chondren sind in Fig. 1 und 3 Taf. 7 sowie in Fig. 2 auf Taf. 20 dargestellt.

Unter den porphyrischen bis k"ornigen Olivinchondren sind jene merkw"urdig, welche neben den Kristallen und K"ornern auch eine monosomatische Olivinkugel mit Glasnetz enthalten. Solche wurden in den Steinen von Dhurmsala und Mez"o-Madaras beobachtet. Fig. 1 auf Taf. 8 gibt ein Beispiel. Von der Anschauung ausgehend, dass die Chondren erstarrte Tropfen sind, wird man die Erscheinung dadurch erkl"aren k"onnen, dass man sich vorstellt, ein kleiner schon erstarrter Tropfen sei durch einen noch fl"ussigen gro"sen Tropfen umh"ullt und eingeschlossen worden. Die Umh"ullung einer kleinen Kugel durch eine gr"o"sere wurde makroskopisch von G. Rose in dem Stein von Krasnoj-Ugol und von mir im Stein von Mocs beobachtet.

Manche der k"ornigen Chondren sind im auffallenden Lichte dunkel bis schwarz gef"arbt und zeigen sich bei der mikroskopischen Pr"ufung so reich an K"ornern von Eisen und Magnetkies, dass nur wenige durchsichtige Stellen "ubrig bleiben. Viele der schwarzen Punkte d"urften auch auf Chromit zu beziehen sein.

Opake Einschl"usse sind in Olivinchondren jeder Art h"aufig und die Menge derselben nimmt in den k"ornigen gegen die Oberfl"ache gew"ohnlich zu. Diese Einschl"usse sind vorwiegend Eisen und Magnetkies. Das Eisen erscheint oft in kleinen K"ugelchen, w"ahrend der Magnetkies gew"ohnlich K"ornchen von unbestimmter Gestalt bildet. In geringerer Menge ist Chromit verbreitet, welcher kleine schwarze K"ornchen oder staubartige H"aufchen darstellt.

\emph{Bronzitchondren}. Die Mannigfaltigkeit der Ausbildung ist hier geringer als bei den vorigen, die Chondren sind meist stengelig bis faserig. Monosomatische Chondren wurden nicht konstatiert. Die gro"sen Bronzitindividuen, welche in den Chondriten beobachtet werden, sind immer mit k"orniger Masse verbunden, sind nur Teile einer Kugel oder geh"oren keiner deutlich erkennbaren Kugel an. Solche Individuen haben zuweilen eben erkennbare Kristallumrisse wie in Fig. 1 Taf. 12. Die Einschl"usse haben nichts charakteristisches. Teils sind es opake K"ornchen oder K"ugelchen, teils Glaseinschl"usse, welche eif"ormig, fadenf"ormig gestaltet sind, zuweilen auch negative Kristalle ausf"ullen wie die dunkelbraunen Einschl"usse, welche in Fig. 2 auf derselben Tafel dargestellt sind. Gro"se Glaseinschl"usse, wie solche im Olivin vorkommen, fehlen im Bronzit g"anzlich, Libellen wurden niemals beobachtet.

Die Bronzitchondren bestehen zuweilen aus wenigen gro"sen Kristallen oder K"ornern, zwischen welchen eine geringe Menge von Glas liegt. Nur selten ist die Glasmasse im Inneren solcher Chondren betr"achtlich, wie in dem Fig. 3 auf Taf. 12 dargestellten Falle. Die gro"sen Kristalle solcher aus wenigen Individuen bestehenden Kugeln l"oschen bisweilen nicht einheitlich aus, indem einzelne langgestreckte schmale Teile in der Ausl"oschung von dem Hauptindividuum etwas abweichen. Zuweilen treten kreuzf"ormige Durchwachsungszwillinge auf.

Die meisten Bronzitchondren sind stengelig bis faserig wie die Beispiele auf Taf. 13, 14, ferner Fig. 4 auf Taf. 7 zeigen. Die Anordnung der Stengel oder Fasern ist eine exzentrische. Im D"unnschliffe kommen freilich auch Schnitte vor, welche eine konzentrische Anordnung zeigen, doch tritt dieser Fall nur ein, wenn die l"angsten Stengel oder Fasern senkrecht getroffen werden. "Ofters finden sich solche Chondren, welche zwei Systeme von exzentrischer Faserung zeigen, in den knolligen Chondren sieht man bisweilen auch mehrere solche Teile. Manche Bronzitchondren erscheinen wirrfaserig, besonders jene von lappigem Durchschnitte. Unter den wirrfaserigen sind jene auffallend, welche im Durchschnitte eine gitterartige Zeichnung darbieten, weil auf einem faserigen Grunde gr"obere Stengel in verschiedenen Richtungen sich kreuzend hervortreten. Fig. 4 auf Taf. 12 gibt ein Beispiel. Die meisten Chondren haben eine Rinde. Je feiner die Fasern der Chondren sind, desto deutlicher tritt im Allgemeinen die Rinde hervor, welche immer aus vielen Individuen besteht. Es ist vielleicht nicht "uberfl"ussig, zu bemerken, dass die Olivinchondren, welche aus abwechselnden Lamellen von Olivin und Glas bestehen, Durchschnitte liefern, zu T"auschungen Veranlassung geben und f"ur parallelstengelige Aggregate von Bronzit gehalten werden k"onnen.

H"aufig sind die runden harten braunen Chondren von feinfaseriger Textur, welche eine glatte Oberfl"ache haben und sich leicht aus der Grundmasse herausl"osen lassen, wofern diese nicht sehr fest ist. Viele derselben erscheinen fast tr"ube und lassen nur in sehr d"unnen Pr"aparaten bei st"arkerer Vergr"o"serung die faserige Textur erkennen. Oft bestehen sie noch zum Teil aus braunem Glase und gew"ahren den Eindruck einer unvollst"andigen Entglasung. Alle haben eine Rinde, welche heller gef"arbt und zuweilen auffallend dick ist. Fig. 4 auf Taf. 13 gibt ein Beispiel, in welchem die Faserung noch deutlich erkennbar, die Rinde d"unn ist. In Fig. 3 auf Taf. 7 hat man das Bild einer fast dicht erscheinenden eingedr"uckten Kugel mit dicker Rinde, in Fig. 2 auf derselben Tafel das einer fast dicht erscheinenden braunen Kugel mit sehr d"unner Rinde.

Von durchsichtigen Einschl"ussen findet sich in den Bronzitchondren blo"s ein dunkelbraunes Glas, von opaken Eisen in K"ugelchen, K"ornchen und Flittern, Magnetkies in K"ornern oder staubartig verteilt. Von den schwarzen Einschl"ussen k"onnte einiges auf Chromit zu beziehen sein.

\emph{Olivin-Bronzitchondren}. Viele Chondren sind nicht einfach, sondern bestehen aus den beiden Hauptgemengteilen der Chondrite, aus Olivin und Bronzit. Die einen sind k"ornige Mischungen mit gleicher Ausbildung der beiden Gemengteile, in welchem Falle die Bronzitk"orner kaum durch das mehr faserige Ansehen von den anderen unterschieden werden. Zuweilen mischt sich noch ein wenig Augit in d"unnen Prismen hinzu, wie in dem Stein von Renazzo. Andere Chondren bestehen zum Teil aus einer k"ornigen Masse von Olivin, zum Teil aber aus dem exzentrisch faserigen Bronzit oder die Kugel ist einerseits porphyrisch und enth"alt daselbst Olivinkristall in glasiger bis feink"orniger Grundmasse, anderseits wird sie von strahlig-faserigem Bronzit gebildet. Dieser nimmt zuweilen "uberhand und dann bestehen die Chondren vorwiegend aus faserigem Bronzit, in welchem Olivink"orner eingestreut liegen. Unter den porphyrischen Chondren kommen oft solche vor, in welchen der Olivin gr"o"sere Kristalle bildet, die Grundmasse aber von kleineren Bronzitkristallen und Glas gebildet wird. Da der Bronzit die Zwischenr"aume einnimmt, so muss man schlie"sen, der Olivin sei fr"uher auskristallisiert, sei das "altere Mineral, w"ahrend der Bronzit sp"ater gebildet wurde. Ein Beispiel gibt Fig. 4 auf Taf. 11. Der Bronzit, welcher von Glasmasse umgeben ist, bietet oft kreuzf"ormige Durchwachsungszwillinge dar, wie solche von Becke in den mineralog.-petrogr. Mitt. Bd. 7 pag. 95 beschrieben wurden. Nach sp"ater anzuf"uhrenden Beobachtungen erscheinen in dem Glase, welches in vielen Olivinchondren die Zwischenmasse bildet, infolge beginnender Entglasung viele feine Nadeln, welche wohl richtig f"ur Bronzit zu halten sind. Solche Chondren werden demnach zu den eben angef"uhrten geh"oren und deren fr"uheren Zustand darstellen.

\emph{Augitchondren}. In manchen Chondriten finden sich Kugeln mit Kristallen oder K"ornern von gr"unlichgrauer F"arbung, welche dem Bronzit "ahnlich sind, jedoch im polarisierten Lichte Erscheinungen zeigen, die auf Augit schlie"sen lassen. In den Steinen von Renazzo und Knyahinya sind solche Kugeln h"aufiger als in der Mehrzahl der Chondrite. S. Taf. 15. Die Durchschnitte der Kristalle und K"orner, welche g"unstig liegen, erscheinen im polarisierten Lichte aus Lamellen von abwechselnder Stellung und schiefer Ausl"oschung zusammengesetzt, was einer wiederholten Zwillingsbildung entspricht. Die Lamellen sind weder so scharf begrenzt noch so gleichf"ormig dick wie in den Plagioklasen, vielmehr oft abs"atzig oder auch etwas gekr"ummt. In dem Stein von Knyahinya bemerkt man den Parallelismus der Lamellen mit den feinen Spaltlinien, daher die Zwillingsebene parallel 1 1 0 oder 1 0 0 anzunehmen ist, wovon die letztere Lage als die beim tellurischen Augit gew"ohnliche die wahrscheinlichere ist. Aus der Ausl"oschungsschiefe lie"s sich kein bestimmtes Resultat entnehmen, da nicht so viele Durchschnitte beobachtet wurden, um einerseits das Zwillingsgesetz zu bestimmen, anderseits wenn das letztere als herrschend angenommen wird, die Ausl"oschungsschiefe auf 0 1 0 aus den extremen Zahlen herzuleiten. In dem Stein von Knyahinya sind die Augitb"undel in den Chondren h"aufig durcheinander gewachsen, w"ahrend sie in dem Steine von Renazzo einzeln nebeneinander liegen und oft durch Glas getrennt sind, so dass eine k"ornige bis porphyrische Struktur entsteht. Manche der Chondren enthalten spreuf"ormigen Augit, an dem keine bestimmten Kristallumrisse zu bemerken sind, wohl aber wiederum die Zusammensetzung aus Zwillingslamellen wahrgenommen wird. Im Stein von Renazzo ist der Augit von K"ornern begleitet und schlie"st auch K"orner ein, welche nach dem Verhalten im polarisierten Lichte und dem Mangel an deutlichen feinen Spaltrissen als Olivin bestimmt wurde.

\emph{Plagioklas-Chondren}. Es ist eine seltene Erscheinung, dass die Hauptmasse einer Kugel aus jenem Gemengteil besteht, welcher nach seinem optischen Verhalten als Plagioklas bestimmt wurde. Ein hierher geh"origer Fall ist auf Taf. 16 in Fig. 4 dargestellt. Hier bildet k"orniger Plagioklas die Zwischenmasse, welche die R"aume zwischen parallelen Lamellen und St"abchen von Olivin ausf"ullt. Dieser gibt dem Gef"uge den Charakter, obwohl der Plagioklas bei weitem "uberwiegt. W"ahrend in den meisten F"allen die haupts"achlich aus Olivin bestehenden F"acherkugeln als Zwischenmasse Glas oder feink"ornige Grundmasse, in wenigen F"allen Plagioklas enthalten, ist hier das Verh"altnis v"ollig umgekehrt; der Plagioklas tritt vor und das F"acherwerk des Olivins bildet ein zartes Gerippe. Die Figur erinnert an den auf Taf. 10 in Fig. 2 abgebildeten Durchschnitt, jedoch ist daselbst die Zwischenmasse ein Glas, welches an Menge hinter dem Olivin zur"uckbleibt.

\emph{Glas-Chondren}. Zuweilen kommen Kugeln vor, welche blo"s aus Glas bestehen, h"aufiger solche, in welchen Glas vorherrscht und die Kristallbildungen zur"ucktreten. Beide F"alle wurden in dem Stein von Mez"o-Madaras beobachtet, der letztere ist auf Taf. 18 in Fig. 1 dargestellt. Das br"aunliche Glas bildet die Hauptmasse, in welcher lange unvollkommen ausgebildete Olivinkristall verstreut sind. Diese verraten durch ihre Gabelung die Tendenz zur Bildung gef"acherter Individuen. Au"serdem erkennt man als Entglasungsprodukte farnkraut"ahnliche Mikrolithe, einzelne feine doppelbrechende Nadeln und netzf"ormige H"aufchen von rechtwinkelig angeordneten Nadeln derselben Art. Letztere d"urften auf Bronzit zu beziehen sein. Halbglasige Chondren werden in vielen Steinen angetroffen. Oft ist die Entglasung ziemlich gleichf"ormig vorgeschritten, wof"ur Fig. 4 auf Taf. 17 ein Beispiel gibt. Die im durchgehenden Lichte blassblaue Kugel besteht nur zum Teile aus Glas, im "ubrigen aus doppelbrechenden K"ornchen und Flittern ohne scharfe Umrisse, die gleichsam mit der Grundmasse verflie"sen. Das Entglasungsprodukt und die d"unne doppelbrechende Rinde scheinen aus Olivin zu bestehen, die Grundmasse d"urfte ein Feldspatglas sein. Eine fast vollst"andige Entglasung bieten auch die "ofters vorkommenden braunen Bronzitk"ugelchen, welche eine ungemein zarte exzentrische Faserung besitzen, die erst im polarisierten Lichte deutlicher wird und welche bisweilen auch noch sternf"ormige Flocken von Mikrolithen enthalten, wie solche in Fig. 1 auf Taf. 19 erscheinen.

Sowie in den Glaskugeln, zeigt sich auch in der Zwischenmasse der Olivinchondren die Entglasung durch Entstehung feiner Nadeln sehr h"aufig. Einen hierher geh"origen Fall gibt Fig. 2 auf Taf. 18. Da sich von der Bildung der Nadeln bis zur Einlagerung deutlicher Bronzitkristalle die "Uberg"ange beobachten lassen, so hat man die Nadeln als die Anf"ange von Bronzitkristallen anzusehen, das braune Glas vorwiegend als Bronzitglas zu betrachten. Gew"ohnlich bilden sich aber in der Zwischenmasse keine gr"o"seren Kristalle, vielmehr f"uhrt die Entglasung blo"s zur Bildung einer tr"uben aus unz"ahligen Mikrolithen bestehenden Masse. Die Zwischenmasse ist also in den meisten Chondren tr"ube durchscheinend. "Ofters besteht die Zwischenmasse aus einem dunkelbraunen bis beinahe schwarzen Glase, welches fast undurchsichtig ist. Nicht selten ist ein Teil der Zwischenmasse lichtbraun, ziemlich durchsichtig und zum Teil entglast, im "ubrigen tiefbraun, fast undurchsichtig.

\emph{Schwarze Chondren}. In den Chondriten finden sich nicht selten schwarze, im Bruche matte Kugeln, welche Olivin und au"ser diesem "ofters auch Bronzit enthalten und durch eine gro"se Menge von Einsprenglingen, die im durchfallenden Lichte schwarz erscheinen, so dunkel gef"arbt sind. Die Einsprenglinge sind Eisen, Chromit, Magnetkies oder schwarzes Glas. Derlei schwarze Kugeln kommen zugleich mit anderen vor, welche im auffallenden Lichte ziemlich dunkel gef"arbt erscheinen und fallen daher nicht besonders auf. Beispiele daf"ur bieten die Steine von Knyahinya, Mez"o-Madaras, Lancé, Renazzo. Verschieden von diesen sind aber jene tiefschwarzen Kugeln, welche im Bruche Glasglanz bis Fettglanz haben, in den wei"slichen Chondriten einzeln vorkommen und f"ur diese charakteristisch sind. Die Steine von Alfianello, Chateau Renard, Milena, Mocs liefern gute Beispiele. Diese schwarzen Kugeln bestehen haupts"achlich aus Maskelynit oder aus k"ornigem Plagioklas. Sie sind gegen die Oberfl"ache zu durchsichtig und farblos und enthalten hier nur wenige schwarze Einsprenglinge. Im Inneren aber sind sie voll von eckigen bis rundlichen, im durchfallenden Lichte schwarzen K"ornchen. Letztere geh"oren mindestens zu einem Teile dem Magnetkies an, da bei auffallendem Lichte mehrere Stellen den Glanz und die braune Farbe zeigen, welche dem Magnetkies zukommen. Ein hierher geh"origes Beispiel ist auf Taf. 17 in Fig. 3 dargestellt.

\emph{Eisenchondren}. Kugeln, die wesentlich aus Eisen bestehen, sind in den Chondriten nicht h"aufig. Vollkommen runde Chondren dieser Art werden in den Steinen von Renazzo, Mez"o-Madaras, Borkut, Dhurmsala beobachtet. Die Eisenkugeln in Renazzo haben "ofters eine schwache unvollst"andige Rinde, welche von braunem Glase oder Bronzit gebildet wird. Sie stellen sich, da viele Silikatkugeln desselben Steines einen runden Eisenkern einschlie"sen und dieser bald klein ist, bald an Menge die Silikate "ubertrifft, als der Endpunkt einer Reihe von eisenhaltigen Chondren dar. Rundliche Eisenkl"umpchen, welche oft mit Magnetkies verbunden erscheinen, sind in vielen Chondriten, z. B. in jenen von Mocs, Barbotan, Lucé, Klein-Wenden enthalten. Sie bestehen, wie G. Rose zeigte (p. 87), bald aus einem, bald aus mehreren Individuen, was nach dem "Atzen der Durchschnitte leicht erkannt wird. In vielen Chondriten kommen Olivinkugeln mit Rinden vor, in welchen letzteren das Eisen in Gestalt feiner Tr"opfchen verteilt ist und der Rinde ein dunkles Ansehen gibt. Ein Beispiel hat man auf Taf. 8 in Fig. 2. Dies f"uhrt zu der nicht seltenen Erscheinung einer g"anzlichen Einh"ullung von Olivinchondren durch eine Eisenschale, wie dies in Fig. 3 auf Taf. 19 zu sehen ist. Die Eisenh"ulle ist in solchen F"allen teils kompakt, teils schwammig.

Chondren von Magnetkies habe ich in den Chondriten nicht beobachtet, obwohl "ofters gr"o"sere Kl"umpchen vorkommen. W"ahrend also von Olivin als dem schwerste schmelzbaren, folglich am fr"uhesten erstarrenden Gemengteil am h"aufigsten Chondren gebildet werden und der Bronzit sich anschlie"st, sind die Chondren des bei der Abk"uhlung sp"ater erstarrenden Nickeleisens seltener und von dem erst bei verh"altnism"a"sig niederen Temperaturen erstarrenden Magnetkies wurden gar keine Chondren gebildet.

Gruppen von Chondren wurden bisher in keinem Meteoriten wahrgenommen, doch zeigen sich als eine seltene Erscheinung Doppelchondren, wovon Fig. 2 auf Taf. 19 ein Beispiel gibt. Hier sind zwei monosomatische Olivinkugeln in paralleler Stellung so verwachsen, dass die gr"o"sere die Einbuchtung enth"alt, in welcher die kleinere liegt. Dieses Vorkommen erinnert an jene meist aus feinfaserigem Bronzit bestehenden Kugeln, die eine Einbuchtung zeigen, welche wie die Abformung einer zweiten Kugel aussieht (vergl. Fig. 3 auf Taf. 7).

Wie schon fr"uher bemerkt wurde, sieht man au"ser den vollst"andigen Chondren fast immer auch Bruchst"ucke derselben und zwar am deutlichsten in solchen Steinen, deren Chondren scharfe Umrisse zeigen, w"ahrend in jenen, deren Chondren sich wenig von der Grundmasse abheben, auch das Vorkommen von Bruchst"ucken schwer zu konstatieren ist. Der unge"ubte Beobachter kommt oft in Versuchung, die Bruchst"ucke f"ur urspr"ungliche Bildungen zu halten und ihre Formen unrichtig zu deuten. Dies kann bei den bl"atterigen Olivinchondren eintreten, deren Bruchst"ucke als bl"atterige Tafeln erscheinen, auch bei den exzentrisch-radialfaserigen Bronzitchondren, deren Bruchst"ucke nicht selten spitz-pyramidale Formen erkennen lassen und eine "Ahnlichkeit mit Hagelk"ornern darbieten. Beispiele liefern die Steine von Mez"o-Madaras, Barbotan, Knyahinya. Auch die Splitter der porphyrischen, k"ornigen und dichten Chondren k"onnen zu T"auschungen Veranlassung geben, da sie wie Einschl"usse einer fremden Gesteinsart aussehen. Die zusammengeh"origen Bruchst"ucke derselben Kugel findet man fast niemals neben einander, woraus zu schlie"sen ist, dass die Zersplitterung schon vor der Ablagerung des Meteoritentuffs stattfand. Zerschlagene Chondren sind in den Steinen von Barbotan, Cabarras City, Chateau Renard, Knyahinya, Pultusk, Tipperary und vielen anderen h"aufig. Ein Beispiel von Chondrensplittern gibt Fig. 4 auf Taf. 19.

Die \emph{Grundmasse} der Chondrite oder jene Masse, welche au"ser den deutlich erkennbaren Chondren und Chondrensplittern vorhanden ist, bedingt n"achst diesen durch ihre wechselnde Menge und ihre Beschaffenheit das "au"sere Ansehen der Chondrite.

Viele Steine, deren Chondren scharf ausgebildet sind und eine bedeutende Festigkeit besitzen, enthalten fast gar keine Grundmasse. Sie bestehen fast nur aus Chondren und aus deren Bruchst"ucken, ihr Ansehen ist demnach ein vollkommen chondritisches. Das sp"arliche Bindemittel besteht vorzugsweise aus Flittern und K"ornchen von Eisen und Magnetkies und was au"serdem von staubf"ormiger Silikatmasse hinzukommt, ist bei mehreren dieser Steine durch die opake Beimengung dunkel gef"arbt, so dass die Chondren und Splitter umso deutlicher hervortreten. Ein ausgezeichnetes Beispiel ist der Stein von Borkut, in welchem die Chondren meist vollkommen rund und sehr fest sind, w"ahrend das in geringer Menge vorhandene Bindemittel wenig zusammenh"alt, demnach die Chondren leicht auseinanderfallen, ferner der leicht zerreibliche Stein von Ornans, dessen Chondren von staubartiger Kleinheit und nur lose verbunden sind. In anderen Steinen wie in dem von Mez"o-Madaras ist das sp"arliche Bindemittel fester und ziemlich dunkel gef"arbt (s. Taf. 7, Fig. 2 und Taf. 19, Fig. 1 und 4). Auch in dem Stein von Tieschitz erscheint die Grundmasse dunkel und noch reichlicher vorhanden (s. Taf. 7, Fig. 3). Wenn das Bindemittel nicht so dunkel und dabei sp"arlich entwickelt ist, so erscheinen die Chondren und Splitter enge aneinandergepresst, wie im Stein von Knyahinya. An vielen Punkten zeigt sich dann kein Bindemittel. Wo aber solches vorhanden ist, bemerkt man "ofter die Anzeichen eines sp"ater entstandenen Kittes in der Form von kleinen farblosen doppelbrechenden P"unktchen und H"aufchen, welche undul"ose Ausl"oschung und im polarisierten Lichte dasselbe Aussehen darbieten wie der Plagioklas in den fr"uher bezeichneten Plagioklaskugeln. Diese P"unktchen und H"aufchen erf"ullen L"ucken in der Bindemasse, sind mit dieser innig verschmolzen, schlie"sen K"ornchen von Olivin ein und sind h"aufig mit Magnetkies verbunden, ganz ebenso wie in den sp"ater zu bezeichnenden wei"sgrauen tuffartigen Chondriten, nur sind sie sparsamer entwickelt. Au"serdem sieht man in der Bindemasse zuweilen jenes farblose Silikat, welches nicht mit dem Plagioklas "ubereinstimmt und welches in der Tafelerkl"arung als ein dem Monticellit "ahnlicher Gemengteil bezeichnet ist. Auf Taf. 14 ist in den Fig. 3 und 4 das Auftreten desselben in dem Stein von Knyahinya charakterisiert. Hier schmiegt sich dasselbe an die Chondrensplitter an und umgibt dieselben zum Teile. Eine genauere Bestimmung der qualitativen Zusammensetzung dieses sparsam vertretenen Gemengteilen ist mir, wie gesagt, bis jetzt nicht gelungen. Zu den Steinen mit hellem festem aber nicht reichlich vorhandenem Bindemittel geh"oren au"ser jenem von Knyahinya auch die von Dhurmsala, Chateau Renard, Cabarras City, Tipperary u. a. m. In manchen, wie beispielsweise in dem von Ensisheim, wechselt helles Bindemittel mit dem dunklen, daher die Masse auf dem polierten Durchschnitte marmoriert aussieht.

Einige Chondrite haben vollst"andig das Aussehen eines klastischen Gesteines mit Tuffgrundmasse, z. B. der Stein von Alexinaé (Sokobanja). Eine aschgraue bis gelblichgraue Masse von erdigem Bruche, die aber nicht locker, sondern ziemlich fest ist und beim Anschlagen fast wie ein Backstein klingt, schlie"st nicht nur Chondren und deren Splitter, sondern auch kleinere und gr"o"sere bis 15 cm lange eckige Bruchst"ucke eines dunkleren gefrittet aussehenden chondrischen Gesteines, ferner scharfkantige Bruchst"ucke von k"ornigem Magnetkies ein. Bei der mikroskopischen Beobachtung erscheint die Hauptmasse vorzugsweise aus Chondrensplittern zusammengesetzt und die Grundmasse aus dem entsprechenden Staube. Auch hier stellen sich die farblosen doppelbrechenden Punkte und H"aufchen ein, von welchen aber manche zwischen gekreuzten Nicols zwickelf"ormige, zwillingsartig verbundene Individuen erkennen lassen, "ahnlich wie der Tridymit im Rittersgr"uner Meteoriten, so dass man in diesem Falle an der Bestimmung als Plagioklas irre wird. Die gefrittet aussehenden Gesteinsbruchst"ucke zeigen unter dem Mikroskope Chondren, welche mit der undeutlich k"ornigen Grundmasse verflie"sen und in den enthaltenen Kristallindividuen reichliche, wahrscheinlich sekund"ar gebildete Glaseinschl"usse, wie z. B. in dem auf Taf. 12 in Fig. 2 abgebildeten Bronzit. Auch der Stein von Siena zeigt an vielen Stellen ein deutlich klastisches Gef"uge, da sowohl Chondren und deren Splitter als auch Bruchst"ucke von gefrittetem und solche von schwarz impr"agniertem Gestein vorkommen. An die deutlich klastischen Chondrite mit heller Grundmasse schlie"sen sich die wei"sgrauen Chondrite, welche eine ziemliche Reihe, wie z. B. die Steine von Alfianello, Girgenti, Mauerkirchen, Milena, Mocs, Tourinnes la Gro"se umfassen. In der hellen matten tuffartigen, aber ziemlich festen Masse sieht man mit freiem Auge bald h"aufiger, bald seltener deutliche Kugeln, ferner kleine bis gr"o"sere K"orner von Magnetkies und Flitter bis Kl"umpchen von Eisen. Mikroskopisch zeigt die Masse wenige deutliche Chondren und fast gar keine deutlichen Chondrensplitter, im "Ubrigen ein undeutlich k"orniges Haufwerk, in welchem alle K"ornchen von Spr"ungen durchzogen sind und in welchem die einzelnen Partikel durch ihr Gef"uge doch wieder an Teile von Chondren erinnern. Die Bilder in Fig. 3 auf Taf. 16, Fig. 2 auf Taf. 17, Fig. 1 und 2 auf Taf. 21 geben eine Vorstellung von der Undeutlichkeit dieses Gemenges, in welchen blo"s stellenweise die Chondrentextur erkennbar ist. Magnetkies und Eisen sind allenthalben zerstreut. Wenn der Magnetkies eine Kugel umh"ullt, wie in Fig. 1 auf Taf. 11 oder einen Kristall umgibt wie in Fig. 1 auf Taf. 12, so treten dieselben ausnahmsweise scharf aus der Umgebung hervor. In der Grundmasse erblickt man nicht selten jene farblosen doppelbrechenden Punkte und H"aufchen, welche meist nur eine undul"ose Ausl"oschung und blo"s hie und da im polarisierten Lichte die abwechselnden Streifen zeigen, welche die Plagioklase charakterisieren. Fig. 2 auf Taf. 16 stellt eines jener K"ornchen im Stein von Mocs dar, in welchem ich zuerst die Zwillingsstreifung an diesem Gemengteil der Chondrite bemerkte. Die einzelnen K"ornchen und die H"aufchen sind mit der Grundmasse innig verbunden, sie verzweigen sich meistens in derselben, f"ullen L"ucken aus, schlie"sen Grundmasse in der Form kleiner K"ornchen ein (Fig. 3 auf Taf. 16), verhalten sich also in dem Gemenge wie eine zuletzt entstandene Impr"agnation. Sie siedeln sich besonders h"aufig in der Umgebung der Chondren an und verbinden dort die feinste Grundmasse. Wahrscheinlich ist der Plagioklas auch unmerklich zwischen den K"ornern und Splittern der Grundmasse verteilt und bedingt zum Teile die Festigkeit des Ganzen.

Am h"aufigsten beobachtet man den Plagioklas in den wei"sgrauen Chondriten, wie in jenen von Alfianello, Girgenti, Mauerkirchen, Milena, Mocs, Tourinnes la Gro"se, aber auch in den grauen Chondriten, wie in jenen von Aigle, Ausson, Chantonnay, Dhurmsala, Ensisheim, Gro"s-Divina, Knyahinya, Lissa, Mez"o-Madaras, New-Concord, Pultusk, in den klastischen wie Alexinaé, Siena und in den kristallinisch aussehenden wie Erxleben, Murcia ist er vertreten. In dem letzteren fand ich den Plagioklas sparsam in scharfkantigen Splittern (s. Taf. 16, Fig. 1). Von den farblosen K"ornern und H"aufchen, welche das zuvor beschriebene Vorkommen zeigen, sind jedoch manche einfachbrechend. Da dieser isotrope Gemengteil genau dieselbe Form und Verteilung zeigt wie der Plagioklas und im gew"ohnlichen Lichte denselben Eindruck macht wie dieser, so glaubte ich aus dieser auffallenden Gleichheit der "au"seren Form auf eine Gleichheit der chemischen Zusammensetzung schlie"sen und denselben f"ur Maskelynit halten zu d"urfen. In dem Chondrit von Alfianello kommen sowohl doppelbrechende als auch isotrope K"orner vor, endlich auch solche, die nur teilweise aufhellen, so dass hier ein "Ubergang vom Plagioklas zum Maskelynit vorzuliegen scheint. Dies f"uhrt dazu, die isotropen K"orner als umgeschmolzenen, also durch Erhitzung isotrop gewordenen Plagioklas anzusehen, wovon schon fr"uher (pag. 7) gesprochen wurde. Dies wird noch dadurch bekr"aftigt, dass die farblosen K"orner in der Schmelzrinde des plagioklashaltigen Chondrits von Mocs isotrop sind. In dem Chondrit von Alfianello, der die Bilder Fig. 1 und 2 auf Taf. 17 geliefert hat, zeigt der Maskelynit im gew"ohnlichen Lichte "ofters feine untereinander parallele Striche, welche durch das ganze Korn laufen, in gr"o"seren K"ornern blo"s eine Strecke weit anhalten und Kanten parallel sind, welche an dem Korn wahrgenommen werden. Diese Striche sind blo"s durch eine Verschiedenheit der Lichtbrechung hervorgerufen, da sie nicht bei jeder Richtung des durchgehenden Lichtes auftreten. Sie erinnern an die Lamellierung im Plagioklas. Der Maskelynit ist viel seltener als der Plagioklas. In dem Chondrit von Chateau Renard ist blo"s Maskelynit und kein Plagioklas zu erkennen. So wie der Plagioklas und der Maskelynit sind auch das Eisen und der Magnetkies in den meisten Chondriten mit der Grundmasse innig verbunden. Sie schmiegen sich an die K"ornchen derselben, f"ullen L"ucken aus, umh"ullen die Chondren, verhalten sich also wie eine sp"ater gebildete Impr"agnation.

In den grauwei"sen Chondriten beobachtet man auch am h"aufigsten die schwarzen Kl"ufte, welche im Querschnitte als Adern erscheinen, sowie auch jene breiteren gangf"ormigen Massen, welche beiden Bildungen jedoch in den grauen Chondriten ebenfalls oft vorkommen. Im D"unnschliffe sind die Adern im durchfallenden Lichte schwarz. Sie durchsetzen das Gestein bald in gerader Richtung, bald sind sie unregelm"a"sig gekr"ummt, bald einzeln, bald ver"astelt und netzartig. Sie weichen meist den harten Chondren aus und vereinigen sich gern mit den K"ornern von Magnetkies (Fig. 1 auf Taf. 22). Eine Verschiebung der W"ande habe ich daran selten beobachtet. Im auffallenden Lichte sieht man im Inneren der Adern "au"serst zarte Eisenbl"atter, welche der Richtung der Adern parallel liegen, im Querschnitte also wie ungemein d"unne F"aden erscheinen, "uberdies auch zuweilen Tr"opfchen von Eisen und Magnetkies. Die Hauptmasse ist schwarz, fast matt, spr"ode. Sie scheint eine halbglasige Masse zu sein. "Ofters schlie"st sie Splitter des Gesteines ein. Beim Zerschlagen trennt sich das Gestein "ofters nach den Kl"uften und auf diesen erblickt man sodann einen ziemlich deutlich gl"anzenden striemigen Harnisch, welcher durch die genannten Eisenbl"atter hervorgerufen ist. In dem Stein von St"alldalen sind Adern, welche Harnische liefern, sehr h"aufig. In einem St"ucke des Steines von Murcia fand ich Kl"ufte, die vollst"andig mit Eisen gef"ullt sind, welches im Querbruche die tesserale Spaltbarkeit erkennen l"asst. In manchen Exemplaren der Chondrite findet sich ein enges Netz von Adern oder es erscheint eine breitere gangf"ormige schw"arzliche Masse, wie ich eine solche mit scharfer Begrenzung in dem Stein von Orvinio beobachtete.\footnote{Die Tr"ummerstruktur der Meteoriten von Orvinio und Chantonnay. Sitzungsber. d. Wiener Ak. Bd. 70 Abt. 1. November 1874.} Hier sind hellere Bruchst"ucke der chondritischen Masse von einer dunklen dichten Masse, welche Eisenbl"attchen enth"alt and Fluidalstruktur zeigt, umgeben und durch diese verbunden. Auch in dem Stein von Chantonnay sind Bruchst"ucke des Chondrits durch eine schwarze Bindemasse vereinigt und durch dieselbe zum Teil impr"agniert. Die schwarze Masse zeigt hier ein feines unregelm"a"siges Netz von Eisen. Die benachbarte Silikatmasse hat viel Maskelynit und an einigen Stellen sind Splitter von Olivin, welche von der Masse umschlossen erscheinen, teilweise oder ganz verglast und verhalten sich isotrop. Unter den Steinen von Mocs sind solche, die von einer schwarzen Masse gangartig durchsetzt werden, nicht selten.\footnote{Sitzungsber. d. Wiener Ak. Bd. 85 Abt. 1. M"arz 1882.} Die M"achtigkeit der letzteren betr"agt bis 19 mm. Die schwarze Masse ist dicht, halbglasig, spr"ode. Im durchfallenden Lichte erscheinen darin viele Splitter des Nebengesteines. Die Grenze gegen das Nebengestein ist zuweilen scharf, doch zeigt sich oft ein allm"ahliger "Ubergang durch Impr"agnation des letzteren (Fig. 2 auf Taf. 22). Wo die schwarze Masse kompakt erscheint, bemerkt man im auffallenden Lichte viele K"ugelchen von Eisen, auch rundliche langgestreckte Eisenkl"umpchen, endlich feine, der L"angsrichtung des Ganges entsprechend gestreckte Eisenf"aden, welche die Querschnitte d"unner Eisenbl"atter sind. Diese Eisenf"aden bringen den Eindruck einer Fluidalstruktur hervor. Von den Eisenkl"umpchen gehen "ofter feine Eisenadern aus, welche die Gangmasse quer durchsetzen. Diese Adern endigen bisweilen in leere Querkl"ufte. Solche leere Spr"unge zeigen sich auch in den entsprechenden Massen der Chondrite von Chantonnay und Orvinio. Statt des Eisens bildet auch der Magnetkies K"ugelchen in der schwarzen Masse. Diese besteht nebst Eisen und Magnetkies auch aus einem schwarzen Glase. Weil dieselbe sehr viele Splitter des Nebengesteins enth"alt, so zeigt sie im Bruche nur geringen Glanz. Die Gangmasse ist demnach zum geringen Teil eine Injektion, die aus Eisen, Magnetkies und Glas besteht, zum gr"o"seren Teil eine Impr"agnation der Grundmasse des Chondrits. Die gangartigen Massen stehen mit den Adern in Verbindung, welche sich als Apophysen der vorigen darstellen. Das Ganze macht den Eindruck, als ob sich die chondritische Masse an ihrer Ablagerungsst"atte durch rasche Erhitzung zerkl"uftet und als ob sie eine Schmelze bis in die feinsten Kapillarspalten aufgesogen h"atte. Unter den vielen Exemplaren der Steinregen von Pultusk\footnote{Zeitschr. d. niederrh. Ges. f. Natur- und Heilkunde zu Bonn 1868.} und von Mocs sind einzelne gefunden worden, welche sich im Bruche durch schwarze Farbe und gr"o"sere H"arte von den "ubrigen unterscheiden, so dass es den Anschein hatte, als ob diese St"ucke aus einer fremden Masse best"unden. Dieselbe stimmt jedoch vollst"andig mit der eben geschilderten gangartigen Masse "uberein.

Mit dieser nahe verwandt ist auch die schwarze dichte Grundmasse einiger Meteorite, z. B. jene in dem merkw"urdigen Stein von Goalpara, der im Durchschnitte ein porphyrisches Aussehen zeigt, weil Enstatitkristalle in einer unvollkommen chondritischen Olivinmasse liegen, die zugleich l"ocherig ist (s. Taf. 20, Fig. 3). Eine schwarze, fast halbglasige Masse impr"agniert die feine Grundmasse, bildet die W"ande der L"ocher und Spalten, umgibt die H"aufchen der Olivink"orner, dringt zwischen diese ein und ver"astelt sich daselbst in den feinsten Ausl"aufern. Die schwarze Masse enth"alt nach meinen Beobachtungen ein feines Netz von kristallinischem Eisen, ferner Magnetkies, Kohle und ein durch S"aure zersetzbares Glas. Der Stein von Richmond enth"alt ebenfalls zwischen den Chondren und Splittern eine schwarze fast halbglasige Grundmasse, die sich in feinen Ver"astelungen bis in die feinsten Kl"ufte zwischen den K"ornchen der Silikate verbreitet. Auch dieser Stein hat kleine L"ocher, doch sind deren W"ande etwas drusig. Der Stein von Tadjera von dichtem bis halbglasigem Bruche scheint eine "ahnliche Grundmasse zu besitzen. Ich konnte denselben nicht n"aher untersuchen. Die eben besprochenen schwarzen Impr"agnationen weisen auf eine Ver"anderung der chondritischen Masse durch Erhitzung, wobei Kohlenwasserstoffe eine Rolle gespielt haben d"urften. In dem Stein von Goalpara betr"agt der Gehalt an Kohlenwasserstoffen 0.85 pz.

An die Steine mit schwarzer Impr"agnation reihen sich diejenigen mit matter schwarzer kohligen Grundmasse, die kein Eisen enth"alt. Die einen derselben, wie die Steine von Renazzo, Grosnaja bestehen vorzugsweise aus Chondren, au"serdem aus harter Grundmasse und sind fester, w"ahrend die anderen, welche zumeist aus einer weicheren Masse bestehen, locker erscheinen. Von den letzteren habe ich nur den Stein von Cold Bokkeveld untersuchen k"onnen. In allen diesen schwarzen kohligen Meteoriten sind die Chondren glasreich und ziemlich mannigfaltig, indem au"ser porphyrischen Olivinkugeln auch Bronzit-Olivinkugeln von verschiedenartiger Bildung und auch Augitkugeln vorkommen, charakteristisch f"ur mehrere dieser Steine ist das Auftreten von feink"ornigen lappigen Chondren, welche im Durchschnitte tr"ube und filz"ahnlich erscheinen. Sie bestehen wahrscheinlich aus Olivin (s. Taf. 20, Fig. 1 und 2).

Die k"ornigen Chondrite, welche noch zu besprechen sind und zu denen die Steine von Cleguerec, Erxleben, Klein-Wenden, Pilistfer, Stauropol u. a. geh"oren, sind in ihrer mikroskopischen Beschaffenheit den wei"sgrauen Chondriten insofern "ahnlich, als sie nicht viele deutliche Chondren enthalten. Diese sind aber makroskopisch kaum zu erkennen, da sie mit der Grundmasse verschmolzen erscheinen. Die Grundmasse erscheint u. d. M. verschmolzen k"ornig, die einzelnen K"orner scheinen oft miteinander und mit den unscharf begrenzten Chondren zu verflie"sen (s. Taf. 20, Fig. 4). Das makroskopisch-k"ornige Aussehen der Bruchfl"ache wird nur zum Teile durch die Spaltfl"achen der K"orner, zum Teile aber auch durch die zerrissenen Eisenpartikel und Magnetkiesk"orner hervorgebracht. Die K"orner und Kristalle der Grundmasse und der Chondren und zwar die Olivine wie die Bronzite sind ungemein reich an Glaseinschl"ussen (s. Taf. 18, Fig. 4). Der Charakter dieser Einschl"usse und das verschmolzene Aussehen der ganzen Masse berechtigen zu der Vermutung, dass die hier beobachteten Glaseinschl"usse nicht urspr"ungliche, sondern durch eine nachtr"agliche Erhitzung entstandene seien. Die k"ornig aussehenden Chondrite w"aren demnach als gefrittete Gesteine anzusehen. Fr"uher wurde schon bemerkt, dass in den eminent klastischen Chondriten wie in jenen von Alexinaé und Siena gefrittet aussehende Bruchst"ucke vorkommen. Das mikroskopische Bild derselben ist demjenigen fast gleich, welches die eben genannten k"ornigen Chondrite liefern. Dasselbe Verflie"sen der Chondren mit der Grundmasse und der K"orner in der letzteren, dieselbe H"aufigkeit der Glaseinschl"usse, welche ungemein oft negativen Kristallen entsprechen. Es fehlen blo"s die Eisenpartikelchen, welche in den k"ornigen Chondriten h"aufig sind.

Zum Schlusse ist noch eine Bemerkung "uber die Rinde der Chondrite anzuf"ugen. Diese ist "au"serlich schwarz bis braun oder grau, fast matt und zeigt nur selten fettgl"anzende wie gefirnisst aussehende Punkte dort, wo Plagioklas, Maskelynit oder Augit angeschmolzen sind. Einzelne Punkte haben auch das Aussehen des Hammerschlages, wenn Eisenpartikel an die Oberfl"ache treten und manche erscheinen mit einem braunen Pulver "uberzogen, wo freiliegender Magnetkies abbrannte. Auf die Formen der Rinde, gem"a"s welcher an manchen Exemplaren eine Brust- und R"uckenseite, sowie ein Schlackensaum unterschieden werden k"onnen, gehe ich hier nicht ein, da nur Makroskopisches und Bekanntes zu wiederholen w"are. Der Bau der Rinde, welcher zuerst von Brezina beschrieben wurde, ist merkw"urdig. Die Rinde besteht oft aus drei wohl unterscheidbaren Gliedern, welche im Durchschnitte des Steines Zonen bilden (s. Taf. 21, Fig. 1 und 2). Die "au"serste Rinde oder eigentliche Schmelzrinde ist glasig. Sie erscheint zum Teile schwarz, undurchsichtig, zum Teil aus einem braunen, selten einem farblosen Glase zusammengesetzt. Das braune Glas wird man von Olivin und Bronzit, das farblose von Plagioklas oder Maskelynit ableiten, da selbes nur in der Rinde solcher Chondrite vorkommt, welche die letzteren enthalten. Die schwarze Farbe ist den Resten von Magnetkies und dem Eisenoxyduloxyd zuzuschreiben. Die zweite Zone oder Saugzone besteht aus den Gemengteilen des Gesteins und stellenweise aus einer geringen Menge zwischen den feinen Kl"uften eingeklemmter schwarzer, brauner bis farbloser Masse. Diese Zone ist also durchsichtig. In dem Chondrit von Mocs enth"alt sie Maskelynit, w"ahrend der "ubrige Stein Plagioklas und keinen Maskelynit enth"alt. Die dritte, innerste oder Impr"agnationszone ist am breitesten. Sie zeigt wiederum die unver"anderten Gemengteile des Meteoriten, jedoch sind die Silikate mit einer gro"sen Menge schwarzer Masse impr"agniert. Demnach erscheint hier die schwarze Masse von durchsichtigen K"ornchen durchsprenkelt. Die schwarze Masse der Impr"agnationszone zeigt im auffallenden Lichte immer viele sehr kleine gelbe Flitter, welche auf Magnetkies bezogen werden k"onnen, dagegen selten feine Adern von metallischem Eisen. Durch diese Beschaffenheit unterscheidet sich die schwarze Masse der Rinde von jener in den Adern und gangartigen Injektionen der Chondrite. An Steinen, welche wenig por"os sind, wie der von Knyahinya, fehlt "ofters die zweite und dritte Zone und es ist blo"s die glasige Schmelzrinde zu bemerken. Dass auf die Schmelzrinde eine durchsichtige, nur wenig impr"agnierte Zone folgt, ist dadurch zu erkl"aren, dass hier die Schmelze d"unnfl"ussig war, folglich durch die por"ose hei"se Masse rasch aufgesogen und weitergef"uhrt wurde. In der dritten Zone hat sich sodann die Schmelze in dem k"uhleren Teile der Kruste angesammelt und ist hier erstarrt. Schnitte, welche ungef"ahr parallel der Schmelzrinde durch die Kruste gef"uhrt wurden, ergaben Resultate, welche den vorigen entsprechen. Man sieht wiederum die durchsichtige unmerklich impr"agnierte Saugzone und die dicke dunkle Impr"agnationszone aufeinanderfolgen (s. Taf. 21, Fig. 3). Zum Vergleiche mit den Chondriten wurden auch Pr"aparate aus der Kruste des Eukrits von Juvinas versucht, jedoch gelangen blo"s Parallelschliffe, die ein blasiges braunes Glas mit Splittern und Kristallen von Plagioklas darboten (s. Taf. 21, Fig. 4).\footnote{Von den Schriften, welche die mikroskopische Beschaffenheit der Chondrite behandeln, m"ogen hier noch folgende angef"uhrt werden: Alfianello, Foullon, Sitzungsber. d. Wiener Ak. Bd. 88. 1. 433. Chondren, Sorby, Nature Bd. 15. p. 495. G"umbel, Sitzungsber. bayr. Ak. 1875 p. 313 und 1878 p. 14. Chondrite und Meteoriten "uberhaupt, Tschermak, Sitzungsber. Wien. Ak. 88. 1. 347. Wadsworth. Mem. Mus. Comp. Zoology. 11. Part. 1. Glas, Lasaulx, Sitzungsber. Niederrhein. Ges. 1882. Juli 3. Gopalpur, Tschermak, Sitzungsber. Wiener Ak. Bd. 65. 1. Februar 1872. Goalpara, T., ebend. 62. 2. Dez. 1870. Grosnaja, T., Mineralog. petrogr. Mitt. 1. p. 153. Knyahinya, Kenngott, Sitzungsber. Wiener Ak. Bd. 59. 2. Mai 1869. Lancé, Drasche, Tschermaks Min. Mitt. 1875. p. 1. Mocs, Tschermak, Sitzungsber. Wiener Ak. 85. 1. 195; Rinde, Brezina, ebendas. p. 335. St. Denis Westrem, Prinz, Leg Météorites tombées en Belgique Bruxelles 1885. Tieschitz, Makowsky und Tschermak, Denkschr. Wien. Ak. Bd. 39, p. 187. Zsadany, Cohen, Verh. d. naturhist. med. Vereines Heidelberg. 2. 2.}
\clearpage
\subsection{}
\subsubsection{Grahamit.}
\paragraph{}
Dieses Gemenge, welches sich fast wie eine Mischung von Eisen und Howardit verh"alt, ist bis jetzt nur durch die Massen von der Sierra de Chaco und von Mejillones vertreten. In manchen Sammlungen erscheint erstere mit der spezielleren Fundortangabe Vaca Muerta. Makroskopisch bemerkt man ein Netz von Eisen, das nach G. Rose aus vielen Individuen besteht, welche Widmannst"adten'sche Figuren zeigen. Dasselbe ist von dunklen K"ornern von Magnetkies (Troilit) begleitet. Mit den F"aden des Netzes verschmolzen, treten hie und da Eisenkugeln auf, die bis 5 mm im Durchmesser haben. Sie bestehen auch aus mehreren Individuen. Eingebettet in dem Netze sind K"orner und Splitter, selten Kugeln von Silikaten, welche dunkelgr"un bis braun, zuweilen auch wei"s erscheinen. Die Silikatmasse erscheint k"ornig. Unter d. M. ist das Gef"uge der durchsichtigen Teile wohl auch meistens k"ornig, doch zeigt es sich an manchen Stellen deutlich klastisch, indem Splitter von Plagioklas und anderen Silikaten von Eisen umgeben sind (s. Taf. 22, Fig. 3), selten chondritisch, da einzelne undeutliche Chondren vorkommen. Ein ziemlich gro"ser Teil der Silikate ist Plagioklas, welcher meist K"orner mit undeutlichen Kristallumrissen bildet. Im polarisierten Lichte gibt er pr"achtige Farben und erscheint aus ziemlich breiten Zwillingslamellen zusammengesetzt, deren Ausl"oschungsrichtungen auf ein dem Anorthit nahestehendes Glied hinweisen. Sehr viele dieser Plagioklase zeichnen sich durch reichlich eingestreute br"aunliche kristallisierte bis rundliche Einschl"usse aus, welche durch ihre Gr"o"se und Form auffallen (s. Taf. 22, Fig. 4). Dieselben haben gew"ohnlich ungef"ahr 0,007 mm L"ange und 0,003 mm Breite, doch kommen auch solche von 0,013 mm Durchmesser vor. Eine bestimmbare Kristallform wurde daran nicht erkannt. Alle sind doppelbrechend. Die einen haben das Ansehen von Prismen und sind oft mit der l"angsten Axe den Plagioklaslamellen parallel gelagert, geben gerade Ausl"oschung, in den ungef"ahr quadratischen Querschnitten diagonale Ausl"oschung, was zugleich mit den Spaltrissen auf Bronzit hindeutet. Andere zeigen eine ungef"ahr monokline Form wie Titanit, au"serdem gibt es viele von rundlicher Form. Die Plagioklase, welche als Bruchst"ucke auftreten, sind h"aufig frei von diesen Einschl"ussen. Dieselben zeigen auch schm"alere Zwillingslamellen als die k"ornigen. Der Plagioklas erscheint auch zuweilen dicht und enth"alt in diesem Falle viele sehr kleine rundliche Glaseinschl"usse (Taf. 23, Fig. 2).

Der pyroxenische Gemengteil ist zumeist Bronzit von gr"unlichgrauer F"arbung, welcher im L"angsschnitte ein faseriges Ansehen hat. Dieser bildet K"orner, die zuweilen mit freiem Auge sichtbar und isolierbar sind, ferner undeutliche Kristalle, die mit Eisen umgeben oder mit Olivin oder Plagioklas verwachsen sind. Die Einschl"usse im Bronzit sind teils K"orner von Magnetkies, teils wenig deutliche negative Kristalle, die von schwarzer Masse erf"ullt sind oder rundliche kleine bis staubartige opake K"ornchen. Feine Bl"attchen von schiefer Ausl"oschung, welche parallel 1 0 0 eingeschaltet vorkommen, sind wohl auf Augit zu beziehen (s. Taf. 23, Fig. 1). Seltener als der Bronzit ist brauner Augit, welcher vollkommen klar erscheint und in der Form von K"ornern ohne deutliche Kristallfl"achen auftritt. Er hat das Ansehen des Augits mancher Basalte, zeigt aber ebenso wenig wie die anderen Gemengteile eine Zuwachsschichtung (Taf. 23, Fig. 2).

Der Olivin ist ebenso stark oder st"arker vertreten als der Bronzit. Meistens sind die K"orner desselben ziemlich klein, innig miteinander verwachsen, tr"ube und voll staubartiger Einschl"usse; zuweilen aber sind die K"orner klar und bilden gro"se Kugeln und mit freiem Auge erkennbare Individuen. Einige derselben haben einen chondritischen Bau, indem ein rundliches Kristallindividuum an der Oberfl"ache schwarz impr"agniert, au"serdem aber von einer tr"uben k"ornigen Olivinrinde umgeben ist, welche voll staubartiger Einschl"usse erscheint. Die Silikate der Umgebung solcher den Chondren entsprechenden Gebilde sind gew"ohnlich Bruchst"ucke (s. Taf. 23, Fig. 3). Derselbe Olivin zeigt bisweilen eine merkw"urdige Beschaffenheit, welche mich anf"anglich dazu verleitete, darin ein anderes Silikat zu vermuten. Parallel zu zwei aufeinander senkrechten, den Ausl"oschungen parallelen Richtungen liegen ungemein feine graue oder braune Nadeln, welche in scharfe Spitzen endigen. Dieselben gehen vom Rande oder von den Spr"ungen des Olivinkornes aus, welche mit Magnetkies und rotbrauner Masse, die ein Oxydationsprodukt ist, erf"ullt sind. Da die Nadeln in gro"ser Zahl vorhanden sind, so bilden sie feine Parallelgitter und Kreuzgitter am Rande und neben den Kl"uften. Die grauen Nadeln scheinen Kan"ale zu sein, welche mit einem hellfarbigen Glase gef"ullt sind, w"ahrend in den braunen entschieden jenes Oxydationsprodukt, welches in den Kl"uften vorhanden, enthalten ist (s. Taf. 23, Fig. 4).

An einigen wenigen Stellen des Gemenges finden sich auch farblose Partikel, welche im polarisierten Lichte dieselben zwickelartigen Individuen und im Ganzen dasselbe Ansehen darbieten, wie der Tridymit des Rittersgr"uner Meteoriten. Da die Masse an mehreren Punkten eine klastische Beschaffenheit hat, so d"urfte das Vorkommen von Tridymit neben Olivin nicht f"ur ein urspr"ungliches zu halten sein. Endlich findet sich in der Silikatmasse stellenweise auch br"aunliches Glas in geringer Menge, worin feine gr"une Nadeln von rhombischer Form auftreten. Diese d"urften f"ur Bronzit zu halten sein.

\subsubsection{Siderophyr.}
\paragraph{}
Die Masse von Rittersgr"un enth"alt in einem Schwamm von Eisen ein k"orniges Gemenge, in welchem nach den Untersuchungen von Maskelyne und v. Lang Bronzit and Asmanit enthalten sind. Der Bronzit bildet nicht selten deutliche Kristalle mit vielen gl"atten Fl"achen, sonst aber K"orner, an welchen "ubrigens auch "ofters einzelne deutliche Fl"achen auftreten. Er hat eine gr"une Farbe und deutliche Spaltbarkeit nach dem Prisma. Er zeigt keinen deutlichen Pleochroismus und wenige Einschl"usse. Diese sind K"orner von Troilit und rundliche durchsichtige doppelbrechende K"ornchen, welche ich nicht genauer bestimmen konnte (s. Taf. 25, Fig. 3). Mit dem Bronzit verwachsen zeigen sich k"ornige H"aufchen jenes farblosen Gemengteilen, welcher Asmanit genannt worden, der aber nach den Untersuchungen von Winkler and Weisbach and nach dem optischen Verhalten auf Tridymit zu beziehen ist. Derselbe l"asst "ofter die Form sechsseitiger T"afelchen erkennen, zeigt im polarisierten Lichte in d"unnen Schichten keine deutlichen Farben, dagegen eine Zusammensetzung aus Individuen in mindestens drei verschiedenen Stellungen. Die einzelnen Individuen erscheinen h"aufig zwickelf"ormig und hakenf"ormig, seltener breit-tafelf"ormig, sonst leistenf"ormig, geben eine zu den Seitenkanten schiefe Ausl"oschung und verhalten sich optisch zweiaxig bei schiefer Stellung der Mittellinie gegen die gr"o"ste Fl"ache der Bl"attchen (s. Taf. 25, Fig. 1 and 2). Das optische Verhalten ist demnach dasselbe wie jenes beim Tridymit, welcher auch im spez. Gewicht mit diesem Gemengteil "ubereinstimmt.

\subsubsection{Mesosiderit.}
\paragraph{}
Das Gemenge von Eisen mit Olivin and Bronzit erscheint in einfachster Form in dem Meteoriten von Lodran, welchen ich vor l"angerer Zeit beschrieb.\footnote{Sitzungsber. d. Wiener Ak. Bd. 61. Abt. 2. April 1870.} Das Netz von Eisen ist in demselben so fein, dass es in dieser Beziehung den "Ubergang zu den k"ornigen Chondriten herstellt; jedoch ist von einer Chondrenbildung nichts darin zu bemerken. Der Olivin bildet K"orner oder deutliche Kristalle, die "ofters ebene Fl"achen darbieten; dieselben sind oberfl"achlich blaugrau bis berlinerblau gef"arbt and "au"serlich mit Chromitstaub "uberzogen, innen aber von hellgr"uner Farbe. Spr"unge im Innern sind mit K"ornchen eines opaken Gemengteilen besetzt, was ich f"ur eine sekund"are Erscheinung halte. Der Bronzit hat eine spargelgr"une bis gelbgr"une Farbe, deutliche prismatische Spaltbarkeit und enth"alt eif"ormige Einschl"usse von Plagioklas, haarf"ormige opake Nadeln parallel der Prismenzone und opake rundliche K"orner, wahrscheinlich von Chromit. Au"ser Eisen, Olivin und Bronzit sind noch untergeordnet K"orner von Troilit und oktaedrische Kristalle von Chromit in dem Gemenge enthalten.

Der Mesosiderit von Hainholz zeigt bald ein feineres, bald ein gr"oberes Eisennetz von k"orniger Textur, darin ein k"orniges Silikatgemenge mit Troilit, stellenweise aber auch gro"se K"orner und Kristalle von Olivin liegen. Reichenbach gibt einen Kristall von 4,5 cm L"ange an, ferner auch ziemlich gro"se Kugeln. Die Olivink"orner sind am Rande mit der Grundmasse verwachsen und erscheinen klar, doch enthalten manche derselben auch rundliche Einschl"usse von Troilit (s. Taf. 24, Fig. 3). Stellenweise kommen K"orner von jener Beschaffenheit und mit denselben braunen Nadeln vor, wie solche in der Masse von der Sierra de Chaco beobachtet wurden and deren eines auf Taf. 23 in Fig. 4 abgebildet ist. Der Bronzit bildet kleinere K"orner als der Olivin und zeigt keine deutlichen Kristallumrisse. Einschl"usse sind h"aufig. Sie bestehen aus opaken K"ornern und braunen durchsichtigen Glaseiern. Stellenweise zeigen sich in dem Gemenge K"orner von Plagioklas mit breiten Zwillingslamellen, bald frei von Einschl"ussen, bald reichlich erf"ullt. Augit ist nur hie and da vertreten. Er bildet K"orner von feinschaliger Zusammensetzung und grauer Farbe, welche letztere durch viele staubartige Einschl"usse hervorgebracht wird. Diese sind teils braune Glaseier, teils opake K"orner. Alle diese in gr"o"seren K"ornern vorkommenden Silikate sind mit einer Grundmasse umgeben, welche zum Teile aus gr"o"seren rundlichen von Staub erf"ullten Olivink"ornern, so wie aus den "ubrigen schon genannten Gemengteilen in bunter Verwachsung und aus zwischengeklemmtem braunem Glase besteht (s. Taf. 24, Fig. 4). Die Masse von Hainholz weist demnach dieselben Gemengteile in "ahnlicher Ausbildung auf, wie der Meteorit von der Sierra de Chaco doch mit dem Unterschiede, dass in der Masse von Hainholz der Plagioklas zur"ucktritt.

Der Meteoritenfall von Estherville, welcher einen Schwarm von vielen kleinen und einigen gro"sen Exemplaren zur Erde brachte, ist wohl auch hierher zu rechnen. Viele der kleinen St"ucke bestehen blo"s aus Eisen, andere nur aus Silikatmasse, die "ubrigen aus beiden zugleich. In den gro"sen Exemplaren sieht man auch beide vereinigt. Denkt man sich alle St"ucke des Schwarmes zu einer Gesteinsmasse vereinigt, so g"abe dies ein grobes unregelm"a"siges Gemenge von Eisen und k"orniger Silikatmasse. Das Eisen kommt nach L. Smith auch in der Form von Knollen innerhalb der Silikatmasse vor. Die Analyse gab diesem Beobachter au"ser Eisen einen durch S"aure zersetzbaren Anteil von der Zusammensetzung des Olivins, einen unzersetzbaren Anteil von der Zusammensetzung des Bronzits und in geringer Menge die Bestandteile des Troilits und Chromits.\footnote{Comptes rend. Bd. 90. pag. 960.} G. vom Rath beobachtete in der Silikatmasse gro"se K"orner von Olivin eingeschlossen, ferner kleine Drusenr"aume, worin die kristallisierten Erhabenheiten messbare Kanten bildeten. In der Masse fand er auch farblose durchsichtige K"orner, stellenweise mit Kristallfl"achen. Ob dieselben einem Plagioklas zugeh"oren, l"asst er dahingestellt.\footnote{Sitzungsber. d. Niederrhein. Ges. zu Bonn. Ber. v. 8. Nov. 1880.} Der D"unnschliff der Silikatmasse l"asst, abgesehen von den gro"sen Individuen des Olivins eine gr"une k"ornige Masse wahrnehmen, in welcher als Grundlage ein kleink"orniger von vielen Einschl"ussen staubiger Olivin und in diesem schwebende Kristalle und K"orner von Bronzit zu unterscheiden sind. Der Bronzit hat teils das gew"ohnliche Ansehen und enth"alt wenige Einschl"usse, teils aber ist er durch einen feinen Staub getr"ubt und zeigt au"serdem noch gr"o"sere Glaseinschl"usse. Diese tr"uben K"orner haben makroskopisch ein ungew"ohnliches Ansehen. Sie sind fettgl"anzend und erscheinen durch die Tr"ubung heller gef"arbt als die "ubrigen Gemengteile. Smith hat solche K"orner besonders untersucht und eine Zusammensetzung gefunden, nach welcher dieselben zu zwei Dritteln aus Bronzit-, zu einem Drittel aus Olivinsubstanz bestehen. Er hielt sie demnach f"ur einen besonderen Gemengteil, den er als Peckhamit bezeichnete. Durch die G"ute des Herrn N. H. Winchell in Minneapolis habe ich sowohl eine Probe des Silicatgemenges mit einigen fettgl"anzenden K"ornern, als auch ein gr"o"seres Korn von Peckhamit erhalten. Letzteres zeigte die prismatische Spaltbarkeit des Bronzits, gab aber auch Spaltflachen, die auf Krystallflachen des Olivins bezogen werden konnten. Das optische Verhalten war fast dasselbe wie das des Bronzits. Ein Schliff parallel einer prismatischen Spaltflache gab das Bild in Fig. 2 auf Taf. XXIV. Der ganze D"unnschliff ist durch einen feinen Staub getr"ubt und enthalt au"serdem gr"o"sere Einschl"usse von zweierlei Art. Die einen sind dunkelbraune bis schwarze Kugeln, die anderen stabf"ormige oder spindelf"ormige lichtgef"arbte Glaseinschl"usse, welche negativen Krystallen entsprechen und gleichgef"arbte runde Glaseinschl"usse. Ein Blick auf das Bild gen"ugt zu erkennen, dass ein Gemenge vorliegt, welches bei der Analyse kein Resultat gibt, welches einem einfachen Gemengteil entspricht. Da nun die getr"ubten Bronzite in dem Silicatgemenge denselben Charakter zeigen wie der oben geschilderte Peckhamit und da alle "Ubergange vom reinen Bronzit zum Peckhamit vorkommen, so mochte ich diesen f"ur einen Bronzit halten, welcher durch die gro"se Menge von Einschl"ussen getr"ubt und fettgl"anzend erscheint. An manchen Stellen des Gemenges erblickt man farblose durchsichtige Krystalle und Gruppen von Plagioklas, welche breite Zwillingslamellen darbieten, bald frei von Einschl"ussen sind, bald wieder solche kristallisierte Einschl"usse wie die Masse von der Sierra de Chaco enthalten, bald durch viele sehr kleine runde Glaseinschl"usse staubig erscheinen. Taf. 24, Fig. 1 gibt das Bild einer Stelle, wo der Plagioklas mit Olivin und Bronzit verwachsen ist. Troilit und Chromit kommen in K"ornern allenthalben in der Silikatmasse vor.

\subsubsection{Pallasit.}
\paragraph{}
In den Massen von Krasnojarsk, Brahin, Bitburg, Atacama bildet Meteoreisen die Grundmasse in der Form eines groben Netzes, worin Olivinkristall eingeschlossen sind. G. Rose und v. Kokscharow haben die Kristalle von Krasnojarsk sorgf"altigen Messungen unterzogen. Als Nebengemengteile treten "uberall Troilit und Chromit, zumeist in Verbindung mit dem Eisen auf. In der Masse von Brahin beobachtet man stellenweise auch Splitter von Olivinkristallen in der Eisengrundmasse. Der Olivin ist klar und durchsichtig, mit Ausnahme jenes in der Masse von Atacama, worin der Olivin von unz"ahligen feinen Spr"ungen durchsetzt wird und nach diesen krummfl"achige fettig gl"anzende Abl"osungen bildet. In derselben Masse bemerkt man auch viele netzartig verbreitete schwarze Kl"ufte, welche durch den Olivin, die breiteren auch durch das Eisen hindurchsetzen und mit einem schwarzen Glase gef"ullt sind. In sehr d"unnen Schichten erscheint dieses braun. In mikroskopischer Beziehung bieten die Olivine nichts Auffallendes, au"ser den von G. Rose im Olivin der Pallasmasse wahrgenommenen R"ohren, welche ich auch in dem Olivin der Brahiner Masse bemerkte. Wo diese R"ohren in gr"o"serer Anzahl vorkommen, sind sie alle einander parallel und bringen bei der Beobachtung mit freiem Auge einen wei"slichen Schiller oder einen bl"aulichen Lichtschein hervor. Es sind nach der aufrechten Axe gestreckte Kan"ale von rundlichem bis vierseitigem Querschnitte, welche nach meinem Daf"urhalten negativen Kristallen entsprechen und bald mit einem farblos erscheinenden, bald mit einem tiefbraunen Glase gef"ullt sind (s. Taf. 25, Fig. 4).
\clearpage
\section{Schlussbemerkung.}
\paragraph*{}
Die bisher bekannten Meteoritenarten, von welchen alle mit Ausnahme des Meteoreisens kurz beschrieben wurden, bieten bestimmte Eigent"umlichkeiten der Struktur und der mikroskopischen Beschaffenheit dar, welche hier nochmals "ubersichtlich hervorgehoben werden m"ogen.

Bez"uglich der Eigenschaften der Kristalle und Individuen "uberhaupt, ist die H"aufigkeit der Glaseinschl"usse zu bemerken. Der Olivin mit seinen oft enormen Glasmassen steht obenan und auch die zuweilen vorkommende staubartige Verteilung des Glases, wie im Olivin des Grahamits und Mesosiderits ist eine besondere Erscheinung. Zun"achst steht der Plagioklas, der oft solche Einschl"usse zeigt und im Eukrit selbe in so feiner Verteilung enth"alt, dass sie auch bei starker Vergr"o"serung nicht mehr einzeln erkannt werden, jedoch eine zarte Tr"ubung veranlassen, welche im auffallenden Lichte eine bl"auliche, im durchfallenden eine gelbliche Farbe hervorrufen. Bronzit und Augit sind "armer an Glas gegen"uber dem Olivin, jedoch sind dieselben auch bisweilen von einem Glasstaub durchsetzt und der Augit im Eukrit beherbergt merkw"urdige linear angeordnete Einschl"usse von dunkelbraunem Glase. Obwohl aber Glaseinschl"usse allenthalben zu sehen sind, so erscheinen doch Dampfporen selten und gerade im Olivin, der die gr"o"sten Glaseinschl"usse darbietet, findet sich nur sehr selten eine fixe Libelle. Flussigkeitseinschl"usse sind, wie schon Sorby anf"uhrte, nirgends zu beobachten. Dieses vollst"andige Fehlen gibt den ersten Hinweis darauf, dass bei der Bildung der meisten Meteoriten eine Mitwirkung des Wassers ausgeschlossen war. Dem entspricht auch die vollst"andige Abwesenheit wasserhaltiger Silikate. In dem kohligen Meteoriten von Orgueil sind allerdings wasserhaltige Salze gefunden worden. Wenn hier der Wassergehalt ein urspr"unglicher ist, so sind derlei kohligen Meteorite von anderer Bildung als die "ubrigen and geh"oren im geologischen Sinne einer sp"ateren Bildungsepoche an.

An den Kristallen habe ich niemals Zuwachsschichten unterscheiden k"onnen, wie solche in den vulkanischen Felsarten am Augit and Plagioklas h"aufig wahrgenommen werden. Die einzige Verschiedenheit im Inneren ist die "ofters beobachtete Abnahme der Glaseinschl"usse in der Rinde, welche beim Olivin and Plagioklas konstatiert wurde. Bemerkenswert ist anderseits die H"aufigkeit der schaligen und der wiederholt zwillingsartigen Zusammensetzung beim Augit der Meteorite and die lagenf"ormige Anordnung der dunklen Einschl"usse im Augit der Eukrit, welche die schwarzen Streifen hervorruft.

Eine Eigent"umlichkeit der gew"ohnlichen Meteorsteine bilden die Chondren, welche durch ihre Textur von allen "ahnlichen tellurischen Bildungen abweichen. Nicht nur der Olivin und Bronzit, sondern auch die "ubrigen in gr"o"serer Menge vorkommenden Gemengteile au"ser dem Magnetkies bilden Chondren, unter denen die aus Glas bestehenden besonders hervorzuheben sind. Die bunte Zusammensetzung, die Glaseinschl"usse und Glaskugeln, das Vorkommen von Kugeln mit Einbuchtungen, die Vereinigung von Chondren und deren Splitter beweisen, dass die Chondren sich nicht in der kompakten Gesteinsmasse als eine den Magnesiasilikaten eigent"umliche Erstarrungsform gebildet haben und sprechen f"ur die schon eingangs erw"ahnte Ansicht, nach welcher die Chondren rasch erstarrte Tropfen sind, deren viele infolge der gro"sen Spr"odigkeit zerbrachen.

Zu den bemerkenswerten Erscheinungen geh"ort die oft vorkommende, bis ins Feinste gehende Durchkl"uftung der Silikate. Die Kristallindividuen sind meistens von unz"ahligen feinen Spr"ungen durchzogen, am auffallendsten jene der tuffartigen Chondrite, am wenigsten die kristallinisch aussehenden und die vorwiegend aus Eisen bestehenden Massen mit eingesprengten Silikaten, aber auch unter diesen zeigt eine und zwar jene von Atacama die Durchkl"uftung des Olivins. Demnach bieten alle diese Meteorite bei der mikroskopischen Untersuchung den Anblick von Massen, welche durch rasche Temperatur"anderungen bis zu den kleinsten Splittern zersprengt and zerkl"uftet worden sind.

Aus den Beschreibungen geht hervor, dass bei den Meteoriten die Tr"ummerstruktur h"aufig sei, sehr viele bald deutlich, bald undeutlich klastisch sind und dass eine Anzahl der Meteorsteine ein vollst"andig tuffartiges Ansehen haben. Auch in den eisenreichen Massen, wie in jenen von Brahin, Atacama, der Sierra de Chaco sind Bruchst"ucke von Kristallen verbreitet. Diese Erscheinungen stimmen mit der Ansicht von einer allgemein vulkanischen Bildung der Meteorite und entsprechen der zuvor gedachten Herkunft der Chondren.

In der Grundmasse der Meteorsteine macht sich "ofters ein Bindemittel bemerkbar, welches Plagioklas oder Maskelynit, in den kristallinisch oder gefrittet aussehenden auch braunes Glas ist. Auch Eisen und Magnetkies erscheinen als letzte Bildungen und als Impr"agnation der Grundmasse. In den feinsten Kl"uften der Kristalle sieht man "ofters Ansiedelungen von opaken K"ornern and "Astchen, von denen manche als Magnetkies zu erkennen sind. In den schwarzen Kluftf"ullungen und den gangartigen schwarzen Massen treten wiederum Eisen und Magnetkies in Flasern und K"ugelchen, umgeben von schwarzem Glase auf. In der an die schwarzen gangartigen Bildungen grenzenden Silikatmasse wurde die Umwandlung von Plagioklas und von Olivin in isotrope K"orner, also eine Verglasung beobachtet. Diese Erscheinungen umfassen die Merkmale von Impr"agnationen, Frittungen und Injektionen, welche eine nachtr"agliche Ver"anderung der Silikatmassen durch Erhitzung bedeuten.

Eine Besonderheit der Meteoriten ist die dunkle Rinde, deren "au"sere schlackige Beschaffenheit und deren innere Gliederung samt den Verglasungserscheinungen eine oberfl"achliche Erhitzung der einzelnen Exemplare beweist.

Die aufgez"ahlten Eigenschaften bedingen, abgesehen von der Rinde, einen Habitus der Meteoriten, durch welchen sie von den tellurischen Felsarten verschieden und in den meisten F"allen leicht unterscheidbar sind. Es gibt kein tellurisches Gestein, welches mit einer Meteoritenmasse verwechselt werden k"onnte, selbst wenn die mineralogische Zusammensetzung beider quantitativ dieselbe w"are. Die Gemengteile der Meteoriten sind zwar gr"o"stenteils der Gattung nach den Gemengteilen tellurischer Gesteine gleich, doch sind sie der Art nach von denselben verschieden. Selbst das Eisen von Ovifak, welches nach den Beobachtungen von Steenstrup als ein tellurisches anzusehen ist, unterscheidet sich durch Textur und Zusammensetzung von den bekannten Meteoreisen und auch die im Basalt bei Ovifak beobachteten, mit Eisen verbundenen Silikatgemenge lassen sich nicht mit dem Eukrit der Meteorite identifizieren, da von der eigent"umlichen halbklastischen Struktur der letzteren abgesehen auch die Hauptgemengteile, der Plagioklas und der Augit, sich durch die Einschl"usse und deren Anordnung als Bildungen eigener Art charakterisieren.

Mai 1885.
\clearpage
\pagestyle{fancy}
\fancyhf{}
\rhead{Tafel 1.}
\section{Tafeln 1 bis 25. Zu jeder derselben eine Tafelerkl"arung.}
\subsection{Erkl"arung der Tafel 1.}
\paragraph{}
Figur 1. Gibt ein Bild von der Zusammensetzung des Meteoriten von Juvinas. Die gestreckten Kristalle, welche bei Anwendung von polarisiertem Lichte aus farbigen Lamellen zusammengesetzt erscheinen, sind Anorthit, die braunen oft schwarz gestreiften K"orner aber Augit. Braunschwarze bis rabenschwarze K"ornchen werden auf Chromit bezogen.

Figur 2. Zum Vergleiche mit den Meteoriten ist hier auch ein Bild von jenem Gestein aufgenommen, welches als Einschluss im Basalt von Ovifak in Gr"onland gefunden wurde, und aus Anorthit, Augit, Nickeleisen und einem dem Hisingerit "ahnlichen Silikat besteht. Zuerst wurden diese eukritischen Einschl"usse, sowie die im Basalt und lose gefundenen Eisenklumpen von Nordenski"old f"ur Meteoriten gehalten, gegenw"artig werden dieselben den Beobachtungen Steenstrups zufolge von den meisten Forschern f"ur tellurisch erkl"art. Im Bilde treten die langgestreckten Anorthitkristall deutlich hervor, die dunkleren K"orner sind Augit, die vollst"andig schwarzen Teile Nickeleisen mit Rinde von einem braunschwarzen hisingeritartigen Mineral. Weder der Anorthit noch der Augit zeigen solche Einschl"usse wie die entsprechenden Kristalle der Meteoriten.

Figur 3. Gibt den Charakter vieler Stellen im Eukrit von Juvinas wieder. Zuerst fallen die gro"sen K"orner von Augit auf, da sie durch schwarze Einschl"usse wie liniiert aussehen. Sie zeigen hellfarbig begrenzte Querspr"unge, die oft mit einer schwarzen Masse gef"ullt sind. Auf der linken Seite bemerkt man ein Bruchst"uck von farblosem Anorthit mit feinen punktartigen Einschl"ussen. Diese gr"oberen St"ucke von Augit und Anorthit erscheinen durch eine feinkristallinische Masse verbunden, deren Beschaffenheit oben am Rande des Bildes wahrzunehmen ist. L"angliche Anorthitkrist"allchen bilden Maschen, in welchen ein gelbes Silikat liegt. Dieses hat auch die Eigenschaften des Augits.

Figur 4. Stellt eine f"ur die Bildungsgeschichte des Eukrits von Juvinas wichtige Stelle dar. Breite Kristalle von Anorthit und K"orner von dunklem Augit bilden ein gr"oberes Gemenge. Der gelbe feink"ornige Augit erscheint in zwei Streifen, welche Durchschnitte von Lamellen sind. Letztere k"onnen als Pseudomorphosen gedeutet werden. Auch in der Grundmasse ist gelber kleink"orniger Augit verteilt.
\clearpage

\cfoot{\thepage}
\vspace*{\fill}
\begin{figure}[H]
\centering
\includegraphics[width=\textwidth,keepaspectratio]{figs/1-1.png}
\caption{\small Figur 1 --- Eukrit von Juvinas. Eine Stelle von gr"oberer Textur im polaris. Lichte. Vergr"o"serung 75.}
\end{figure}
\vspace*{\fill}
\clearpage

\rhead{Tafel 1.}
\vspace*{\fill}
\begin{figure}[H]
\centering
\includegraphics[width=\textwidth,keepaspectratio]{figs/1-2.png}
\caption{\small Figur 2 --- Eisenhaltiger tellurischer Eukrit von Ovifak im polaris. Lichte. Vergr"o"serung 75.}
\end{figure}
\vspace*{\fill}
\clearpage

\rhead{Tafel 1.}
\vspace*{\fill}
\begin{figure}[H]
\centering
\includegraphics[width=\textwidth,keepaspectratio]{figs/1-3.png}
\caption{\small Figur 3 --- Eukrit von Juvinas. Eine Stelle mit zweierlei Textur. Vergr"o"serung 75.}
\end{figure}
\vspace*{\fill}
\clearpage

\rhead{Tafel 1.}
\vspace*{\fill}
\begin{figure}[H]
\centering
\includegraphics[width=\textwidth,keepaspectratio]{figs/1-4.png}
\caption{\small Figur 4 --- Eukrit von Juvinas. Mit Durchschnitten von Lamellen eines gelben Silikates. Vergr"o"serung 75.}
\end{figure}
\vspace*{\fill}
\clearpage
\rhead{Tafel 2.}
\subsection{Erkl"arung der Tafel 2.}
\paragraph{}
Figur 1. Ist das Bild eines Anorthitindividuums, welches in der tuffartigen Masse des Eukrits von Stannern liegt. Vom fr"uheren Kristallumriss ist nur an der linken Seite etwas zu erkennen. Die langen Linien liegen parallel M = (0 1 0) und zeigen die Grenzen der Zwillingslamellen an, welche hier ungew"ohnlich zahlreich sind. Diese Grenzen sind "ofters mit kleinen rundlichen Glaseinschl"ussen und l"anglichen Gasporen besetzt. Die kurzen, zu den vorgenannten beil"aufig senkrechten Linien sind linear angeordnete l"angliche Glaseinschl"usse oder auch Spr"unge. Die beiden anderen Systeme kurzer Linien, welche den Prismenfl"achen T = (1 1$^{\prime}$ 0) und \emph{l} = (1 1 0) parallel sind, haben denselben Charakter wie die vorigen.

Figur 2. Liefert ein Bild von der gew"ohnlichen Beschaffenheit des Innern der Anorthitkristalle im Eukrit von Juvinas. Es ist das Ende einer geh"auften Zwillingsgruppe dargestellt Man bemerkt unz"ahlige kleine rundliche Glaseinschl"usse, welche zuweilen perlschnurartig aneinandergereiht sind und "uberhaupt eine Tendenz zu reihenf"ormiger Anordnung zeigen. Die Mehrzahl derselben sind submikroskopisch und bedingen den bl"aulichen Farbenton der Anorthitbl"attchen im auffallenden, den blass braunen im durchfallenden Lichte.

Figur 3. Zeigt die innere Beschaffenheit mancher Anorthitkristalle im Stein von Juvinas, vieler im Eukrit von Stannern an. Die rundlichen Glaseinschl"usse sind in geringer Zahl vorhanden, dagegen treten viele langgestreckte Glaseinschl"usse auf, die im Bilde verschiedene Richtungen zeigen, da die Anorthitpartie aus mehreren Individuen zusammengesetzt ist. Die Grundmasse besteht aus kleink"ornigem Augit und opaken K"ornern von Magnetkies.

Figur 4. Um den f"ur Anorthit charakteristischen Zwillingsbau darzustellen, wurde auch ein Bild aufgenommen, welches das abgebrochene Ende eines Kristalls im polarisierten Lichte gesehen darstellt. Die Zwillingslamellen sind von sehr ungleicher Dicke. Einschl"usse parallel M und auch ungef"ahr senkrecht dazu gelagert sind bemerklich, ebenso zwei Spr"unge in der letzteren Richtung. Das Kristallbruchst"uck ist von k"ornigem Augit und Magnetkies umgeben.
\clearpage

\rhead{Tafel 2.}
\vspace*{\fill}
\begin{figure}[H]
\centering
\includegraphics[width=\textwidth,keepaspectratio]{figs/2-1.png}
\caption{\small Figur 1 --- Anorthit in dem Eukrit von Stannern. Vergr"o"serung 160.}
\end{figure}
\vspace*{\fill}
\clearpage

\rhead{Tafel 2.}
\vspace*{\fill}
\begin{figure}[H]
\centering
\includegraphics[width=\textwidth,keepaspectratio]{figs/2-2.png}
\caption{\small Figur 2 --- Anorthit mit vielen rundlichen Glaseinschl"ussen. Eukrit von Juvinas. Vergr"o"serung 160.}
\end{figure}
\vspace*{\fill}
\clearpage

\rhead{Tafel 2.}
\vspace*{\fill}
\begin{figure}[H]
\centering
\includegraphics[width=\textwidth,keepaspectratio]{figs/2-3.png}
\caption{\small Figur 3 --- Anorthit mit langgezogenen Glaseinschl"ussen. Eukrit von Juvinas. Vergr"o"serung 160.}
\end{figure}
\vspace*{\fill}
\clearpage

\rhead{Tafel 2.}
\vspace*{\fill}
\begin{figure}[H]
\centering
\includegraphics[width=\textwidth,keepaspectratio]{figs/2-4.png}
\caption{\small Figur 4 --- Kristallbruchst"uck von Anorthit. Eukrit von Juvinas. Polaris. Licht. Vergr"o"serung 75.}
\end{figure}
\vspace*{\fill}
\clearpage

\rhead{Tafel 3.}
\subsection{Erkl"arung der Tafel 3.}
\paragraph{}
Figur 1. Ein Durchschnitt des braunen Augits, welcher einen Hauptgemengteil der Eukrit von Juvinas und Stannern bildet. Die zahlreichen dunklen Streifen, welche im Bilde nach rechts geneigt sind, bestehen aus braunen bis schwarzen K"ornern und aus braunen Glaseinschl"ussen. Sie lagern der Endfl"ache (0 0 1) parallel. Untergeordnet erscheint ein zweites Streifensystem, welches gegen links geneigt ist. "Uberdies machen sich Spr"unge bemerkbar, welche nach zwei Richtungen verlaufen und der prismatischen Spaltbarkeit entsprechen.

Figur 2. Der gelbe k"ornige Augit, welcher schon auf Taf. 1, Fig. 4 in streifenf"ormigen Durchschnitten dargestellt wurde, welcher sich aber auch "ofters zwischen den Krystallen von Augit und Anorthit mit unbestimmten Umrissen ausbreitet oder zwischenklemmt, bei st"arkerer Vergr"o"serung. An mehreren Stellen ist die feinschalige Zusammensetzung der Korner parallel (0 0 1) bemerklich, welche in den Mineralen der Diopsidreihe so h"aufig beobachtet wird. Kleine opake Korner sind in der k"ornigen Masse oft schwarmweise verteilt. Die Umgebung bilden Krystalle von Anorthit und ein Korn von Magnetkies.

Figur 3. Um die charakteristische Textur der Meteoriten von Stannern, welche aber auch in jenen von Juvinas stellenweise zu beobachten ist anzudeuten, wurde ein Bild aufgenommen, in dem die Korner und Splitter des farblosen Anorthits von der gleichf"ormig aussehenden aus Augitsplittern bestehenden Grundmasse umgeben erscheinen. In dem gro"sen Individuum von Anorthit bemerkt man als Beispiel der hier gew"ohnlichen Erscheinung einen Einschluss von Augit. Nirgendsfindensich Krystallumrisse. Die Augitk"orner lassen h"aufig die feinen schwarzen Streifen erkennen, welche in Fig. 1 vergr"o"sert dargestellt sind.

Figur 4. Gibt ein Bild des gleichf"ormigen Gemenges von Augit und Maskelynit in dem Stein von Shergotty. Der Augit zeigt keine schalige Zusammensetzung, blo"s hie und da Zwillinge nach (1 0 0), ferner auch keine Krystallformen , sondern f"ullt den Raum zwischen den Krystallen des Maskelynits. Der letztere ist farblos, zeigt einfache Lichtbrechung, hat jedoch meistens gestreckte Formen "ahnlich denen der Feldspate. Die Endigung der Krystalle, welche fr"uher von mir als tesseral gedeutet wurden, erscheint im Schnitte oft rechtwinkelig. Alle die hellen Stellen im Bilde bis an dessen Grenze beziehen sich auf Maskelynit. Man bemerkt ferner einige schwarze dem Magnetit entsprechende Punkte und Flecke.
\clearpage

\rhead{Tafel 3.}
\vspace*{\fill}
\begin{figure}[H]
\centering
\includegraphics[width=\textwidth,keepaspectratio]{figs/3-1.png}
\caption{\small Figur 1 --- Brauner Augit im Eukrit von Juvinas. Vergr"o"serung 250.}
\end{figure}
\vspace*{\fill}
\clearpage

\rhead{Tafel 3.}
\vspace*{\fill}
\begin{figure}[H]
\centering
\includegraphics[width=\textwidth,keepaspectratio]{figs/3-2.png}
\caption{\small Figur 2 --- Gelber k"orniger Augit im Eukrit von Juvinas. Vergr"o"serung 250.}
\end{figure}
\vspace*{\fill}
\clearpage

\rhead{Tafel 3.}
\vspace*{\fill}
\begin{figure}[H]
\centering
\includegraphics[width=\textwidth,keepaspectratio]{figs/3-3.png}
\caption{\small Figur 3 --- Eukrit von Stannern Tuffcharakter. Vergr"o"serung 75.}
\end{figure}
\vspace*{\fill}
\clearpage

\rhead{Tafel 3.}
\vspace*{\fill}
\begin{figure}[H]
\centering
\includegraphics[width=\textwidth,keepaspectratio]{figs/3-4.png}
\caption{\small Figur 4 --- Eukrit von Shergotty. Gleichf"ormig kristallinisch. Vergr"o"serung 65.}
\end{figure}
\vspace*{\fill}
\clearpage

\rhead{Tafel 4.}
\subsection{Erkl"arung der Tafel 4.}
\paragraph{}
Figur 1. Um die klastische Beschaffenheit der Howardit zur Anschauung zu bringen, ist hier eine Stelle aus dem Stein von Loutolaks bei schwacher Vergr"o"serung dargestellt. Splitter von Kristallen und Bruchst"ucke von kleink"ornigem Gestein lagern in einer pulverigen Grundmasse. Der Splitter, dessen Bild in der Mitte erscheint, geh"ort dem farblosen Anorthit an. Oberhalb desselben zeigen sich gr"o"sere Splitter von Augit; an dem einen rechts die geradlinige Spur einer Krystallfl"ache, an dem anderen die feinen Nadeln, welche in dem Pl"attchen schief gegen die Oberfl"ache aufsteigen. Unterhalb des Anorthits macht sich ein feink"orniges Gesteinsbruchst"uck bemerklich, worin zufolge der Beobachtung bei st"arkerer Vergr"o"serung Bronzit der herrschende Bestandteil zu sein scheint. Nebenan lagern Splitter mit schwacher Linienzeichnung, den Bronzit zugeh"orig, rechts ein kleines Augitk"orn mit Gitterzeichnung. Die schwarzen K"orper geh"oren dem Magnetkies, gediegenem Eisen und wohl auch dem Magnetit an.

Figur 2. Als charakteristisch f"ur Howardit ist hier ein aus Anorthit und Augit bestehendes Bruchst"uck abgebildet. Der Anorthit erscheint v"ollig gleich demjenigen in den Eukriten von Juvinas und Stannern, der Augit ist wiederum von zweierlei Beschaffenheit. Der in den Maschen des Anorthits liegende ist gelblich und zeigt stellenweise eine feinschalige Zusammensetzung, der andere Augit, z. B. der im Bilde oberhalb sichtbare, ist braun und enth"alt viele schwarze staubartige K"ornchen, welche sich "ofters linear anordnen. In den Spr"ungen dieses Augits bemerkt man eine schwarze F"ullung ferner im Anorthit und Augit, sowie zwischen denselben "ofters schwarze gr"o"sere K"orner. Demnach wiederholen sich in diesen Bruchst"ucken die im Eukrit gew"ohnlichen Erscheinungen.

Figur 3. Manche der Bronzite im Howardit von Loutolaks und Massing zeigen geradlinige Umrisse, wovon dieses Bild ein Beispiel gibt. Der Schnitt geht nahezu parallel der prismatischen Spaltfl"ache 1 1$^{\prime}$ 0. Die parallelen Risse sind der aufrechten Axe parallel. Die Umrisse zur Linken lassen auf die L"angsfl"ache (0 1 0), ferner auf das am Hypersthen bekannte horizontale Prisma \emph{d} = (0 2 1) schlie"sen. Zarte Querspr"unge sind mit P"unktchen besetzt, gr"o"sere schwarze runde oder eckige Einschl"usse unregelm"a"sig verteilt. Um den Bronzit lagern Augitpartikel und ein gro"ses schwarzes mattes Korn.

Figur 4. Unter den Augitsplittern des Howardits von Loutolaks haben viele das Ansehen des vulkanischen Augits unserer Gesteine. Sie erscheinen br"aunlich und gelblich. Andere aber, die eine blasse ins Gr"unliche fallende Farbe zeigen, sind durch eine feinschalige Zusammensetzung nach (0 0 1) ausgezeichnet. Ein Beispiel gibt diese Figur, die einen sehr unregelm"a"sig geformten Splitter von solchem Augit von Spr"ungen durchzogen darstellt. Derselbe ist von kleinen Splittern des braunen und des gelblichen Augits umgeben.
\clearpage

\rhead{Tafel 4.}
\vspace*{\fill}
\begin{figure}[H]
\centering
\includegraphics[width=\textwidth,keepaspectratio]{figs/4-1.png}
\caption{\small Figur 1 --- Howardit von Loutolaks mit sehr ausgesprochenem Tuffcharakter. Vergr"o"serung 65.}
\end{figure}
\vspace*{\fill}
\clearpage

\rhead{Tafel 4.}
\vspace*{\fill}
\begin{figure}[H]
\centering
\includegraphics[width=\textwidth,keepaspectratio]{figs/4-2.png}
\caption{\small Figur 2 --- Eukritbruchst"uck im Howardit von Loutolaks. Vergr"o"serung 160.}
\end{figure}
\vspace*{\fill}
\clearpage

\rhead{Tafel 4.}
\vspace*{\fill}
\begin{figure}[H]
\centering
\includegraphics[width=\textwidth,keepaspectratio]{figs/4-3.png}
\caption{\small Figur 3 --- Bronzit im Howardit von Massing, Schnitt beil"aufig parallel 1 1' 0. Vergr"o"serung 160.}
\end{figure}
\vspace*{\fill}
\clearpage

\rhead{Tafel 4.}
\vspace*{\fill}
\begin{figure}[H]
\centering
\includegraphics[width=\textwidth,keepaspectratio]{figs/4-4.png}
\caption{\small Figur 4 --- Bruchst"uck von Augit im Howardit von Loutolaks. Vergr"o"serung 200.}
\end{figure}
\vspace*{\fill}
\clearpage

\rhead{Tafel 5.}
\subsection{Erkl"arung der Tafel 5.}
\paragraph{}
Figur 1. Zeigt eine Stelle im Stein von Busti, wo der Enstatit einen deutlichen Umriss darbietet. Man sieht im unteren Teile des Bildes einen Kristall mit einer dachf"ormigen Endigung, welche wahrscheinlich dem Doma \emph{p} = (1 0 2) entspricht. Der Schnitt ist aber der optischen Pr"ufung zufolge zur aufrechten Axe schief nach aufw"arts geneigt, daher der innere Winkel sch"arfer ist als wenn der Schnitt parallel zu \emph{b} = 0 1 0 w"are. Oberhalb und links erscheinen ebenfalls Splitter von Enstatit und zwar mit gro"sen Glaseinschl"ussen, rechts ein Splitter von Plagioklas, etwas schlierig, fast ohne Einschl"usse.

Figur 2. Ist von einer Stelle genommen, wo der Diopsid vorwiegt. Die kleineren Splitter in der Mitte des Feldes zeigen eine "au"sert feine Lamellentextur. Sie enthalten ungemein kleine Einschl"usse. Der gro"se Diopsid im unteren Teile des Bildes zeigt einen Rest der urspr"unglichen Kristallausbildung. Er ist reich an feinen schwarzen staubartigen Einschl"ussen, enth"alt aber auch kleine Glaseinschl"usse. Im durchfallenden Lichte zeigen beide Diopside einen grauvioletten Farbenton, welcher aber dort fehlt, wo die Einschl"usse zur"ucktreten. Rechts ist wieder ein Splitter von Plagioklas zu bemerken.

Figur 3. Stellt eine kleink"ornige Partie des Steines von Bishopville dar. Im unteren Teile des Bildes ist ein gr"o"seres Korn von Enstatit erkennbar, links ein Durchschnitt durch eine farblose Plagioklastafel, an welche sich ein Korn von Magnetkies anf"ugt. Das "ubrige ist ein Gemenge von Enstatit- und Plagioklask"ornern.

Figur 4. Gibt die Erscheinung wieder, welche eines von den gr"o"seren Plagioklask"ornern desselben Steines zwischen gekreuzten Nicols darbietet. Die Zwillingslamellen sind ziemlich breit, die Spaltbarkeit ist nicht deutlich ausgesprochen. Links hat man Enstatit von kleink"ornigem Gef"uge.
\clearpage

\rhead{Tafel 5.}
\vspace*{\fill}
\begin{figure}[H]
\centering
\includegraphics[width=\textwidth,keepaspectratio]{figs/5-1.png}
\caption{\small Figur 1 --- Bustit von Busti bei Gorukpur. Enstatit und Plagioklas. Vergr"o"serung 65.}
\end{figure}
\vspace*{\fill}
\clearpage

\rhead{Tafel 5.}
\vspace*{\fill}
\begin{figure}[H]
\centering
\includegraphics[width=\textwidth,keepaspectratio]{figs/5-2.png}
\caption{\small Figur 2 --- Diopsid und Plagioklas im Stein von Busti. Vergr"o"serung 65.}
\end{figure}
\vspace*{\fill}
\clearpage

\rhead{Tafel 5.}
\vspace*{\fill}
\begin{figure}[H]
\centering
\includegraphics[width=\textwidth,keepaspectratio]{figs/5-3.png}
\caption{\small Figur 3 --- Chladnit von Bishopville. Enstatit, Plagioklas, Magnetkies. Vergr"o"serung 75.}
\end{figure}
\vspace*{\fill}
\clearpage

\rhead{Tafel 5.}
\vspace*{\fill}
\begin{figure}[H]
\centering
\includegraphics[width=\textwidth,keepaspectratio]{figs/5-4.png}
\caption{\small Figur 4 --- Plagioklas im Chladnit von Bishopville im polaris. Lichte. Vergr"o"serung 75.}
\end{figure}
\vspace*{\fill}
\clearpage

\rhead{Tafel 6.}
\subsection{Erkl"arung der Tafel 6.}
\paragraph{}
Figur 1. Gibt den Charakter des Meteoriten von Shalka wieder. Kin gr"o"serer Bronzitkristall, durch den Schnitt in einer zu den Kristallaxen schiefen Richtung getroffen, ist von einer k"ornigen Masse von Bronzit umgeben. In dem gro"sen Bronzit beobachtet man L"angsrisse der prismatischen Spaltbarkeit entsprechend, sowie einzelne grobe Spr"unge, welche gegen die feinen Risse beil"aufig senkrecht oder schief gerichtet sind. Kleine rundliche braune Glaseinschl"usse machen sich am unteren Ende des Kristallschnittes bemerklich.

Figur 2. Der Diogenit von Ibbenb"uhren, welcher meist aus gr"o"seren K"ornern von Bronzit zusammengesetzt ist, enth"alt zwischen diesen auch kleink"ornige Teile. Aus einem solchen ist die Figur entnommen. Sie zeigt den Bronzit rechts in L"angsschnitten, links aber in Querschnitten, welche durch das Schleifen nach der prismatischen Spaltbarkeit zersprungen sind, teils wohl auch schon vordem zersprungen waren.

Figur 3. Eine Stelle aus dem Amphoterit von Manbhoom, welche das k"ornige Gemenge von Olivin, Bronzit und Magnetkies erkennen l"asst. Der Olivin bildet unregelm"a"sige K"orner mit krummen Spr"ungen. Der Bronzit im Bilde rechts hat eine zarte Faserung nach der Spaltbarkeit. Die gr"o"seren und kleineren opaken K"orner entsprechen dem Magnetkies.

Figur 4. Eignet sich sehr gut, den Charakter des gleichf"ormig-k"ornigen Olivins vieler Meteorsteine darzustellen. Die K"orner haben keine regelm"a"sige Begrenzung und sind von vielen gr"oberen Spr"ungen, welche ganz unregelm"a"sig verlaufen, ferner aber auch von unz"ahligen feinen, oft schwach gekr"ummten Spr"ungen, die beil"aufig parallel verlaufen, durchzogen. Die letztere Anordnung der feinen Spr"unge macht "ofters die Unterscheidung von Bronzit und Olivin recht schwierig. Zwischen den Olivink"ornern sieht man hie und da kleine tr"ube Partikel, zuweilen auch farblose durchsichtige K"ornchen, z. B. im Bilde rechts oben, eingeklemmt. H"aufig sind K"ornchen von Chromit mit Andeutungen tesseraler Formen.
\clearpage

\rhead{Tafel 6.}
\vspace*{\fill}
\begin{figure}[H]
\centering
\includegraphics[width=\textwidth,keepaspectratio]{figs/6-1.png}
\caption{\small Figur 1 --- Bronzit in dem Stein von Shalka. Vergr"o"serung 160.}
\end{figure}
\vspace*{\fill}
\clearpage

\rhead{Tafel 6.}
\vspace*{\fill}
\begin{figure}[H]
\centering
\includegraphics[width=\textwidth,keepaspectratio]{figs/6-2.png}
\caption{\small Figur 2 --- Diogenit von Ibbenb"uhren. Bronzit. Vergr"o"serung 80.}
\end{figure}
\vspace*{\fill}
\clearpage

\rhead{Tafel 6.}
\vspace*{\fill}
\begin{figure}[H]
\centering
\includegraphics[width=\textwidth,keepaspectratio]{figs/6-3.png}
\caption{\small Figur 3 --- Amphoterit von Manbhoom. Olivin, Bronzit, Magnetkies. Vergr"o"serung 200.}
\end{figure}
\vspace*{\fill}
\clearpage

\rhead{Tafel 6.}
\vspace*{\fill}
\begin{figure}[H]
\centering
\includegraphics[width=\textwidth,keepaspectratio]{figs/6-4.png}
\caption{\small Figur 4 --- Chassignit von Chassigny. Olivin, Chromit. Vergr"o"serung 80.}
\end{figure}
\vspace*{\fill}
\clearpage

\rhead{Tafel 7.}
\subsection{Erkl"arung der Tafel 7.}
\paragraph{}
Figur 1. Der Chondrit von Lancé eignet sich wegen der Dunkelheit der Grundmasse und der Kleinheit der Einschl"usse zur Darstellung des chondritischen Charakters. Das Bild zeigt links unterhalb den Durchschnitt eines rundlichen oberhalb abgeplatteten Chondrums von einem Sprung durchsetzt. Es ist monosomatisch. Rechts davon hat man Durchschnitte von zwei zusammengesetzten (polysomatischen) K"ugelchen. Im "Ubrigen sieht man sowohl kleine k"ornige Kn"ollchen mit vielen opaken Einschl"ussen, als auch l"angliche Fetzen, endlich kleine Splitter, welche bald aus einem bald aus mehreren Individuen bestehen.

Figur 2. Hier ist das Zusammenvorkommen verschiedenartiger Chondren im Stein von Mez"o-Madaras dargestellt. In der Mitte zeigt sich der Durchschnitt eines porphyrischen K"ugelchens. Die Rinde ist mit K"ornern von Magnetkies gemengt, das Innere besteht aus Olivink"ornern in halbglasiger Grundmasse. Rechts unterhalb erscheint der Durchschnitt eines dichten Chondrums (Bronzit), in welchem die Fasertextur eben erkennbar ist. Die "ubrigen Chondren sind meist k"ornig. Oberhalb des zentralen K"ugelchens ist auch ein zwischengeklemmter k"orniger Splitter wahrzunehmen.

Figur 3. Die hier wiedergegebene Stelle aus dem Stein von Tieschitz l"asst in dunkler Grundmasse rechts einen Teil von einer undeutlich faserigen Kugel (Bronzit) erkennen, welche eine blasse Rinde besitzt und an der gegen die Mitte des Bildes gerichteten Seite ausgeh"ohlt erscheint. Im oberen Teile ist ein schaliges K"ugelchen (Bronzit) zu bemerken, welches gleichfalls eine Konkavit"at besitzt. Im "Ubrigen sind einige k"ornige K"ugelchen, sowie lappige Durchschnitte (Olivin) sichtbar, wovon die letzteren unregelm"a"sig geformten Chondren entsprechen. Fast in der Mitte des Bildes hat man den scharf gezeichneten Splitter einer strahligen Kugel (Bronzit).

Figur 4. In diesem Bilde, welches dem Stein von Homestead, Iowa City (2. Febr. 1875) entnommen ist, hat man rechts den Durchschnitt eines strahligen K"ugelchens (Bronzit), welches die exzentrische Lage des Radiationspunktes deutlich zeigt, links aber ein porphyrisches K"ugelchen, in dem scharf gezeichnete Olivinkristall von einer tr"uben halbglasigen Grundmasse getragen werden. Oberhalb schmiegt sich an beide ein k"orniges Olivinchondrum, unterhalb sind es Splitter und ein unregelm"a"sig gestaltetes Chondrum. Die schwarz erscheinenden Stellen werden von Eisen und Magnetkies eingenommen.
\clearpage

\rhead{Tafel 7.}
\vspace*{\fill}
\begin{figure}[H]
\centering
\includegraphics[width=\textwidth,keepaspectratio]{figs/7-1.png}
\caption{\small Figur 1 --- Chondrit von Lancé. K"ugelchen, Kn"ollchen, Fetzen und Splitter in einer schwarzen Grundmasse. Vergr"o"serung 65.}
\end{figure}
\vspace*{\fill}
\clearpage

\rhead{Tafel 7.}
\vspace*{\fill}
\begin{figure}[H]
\centering
\includegraphics[width=\textwidth,keepaspectratio]{figs/7-2.png}
\caption{\small Figur 2 --- Chondrit von Mez"o-Madaras. Ein porphyrisches, ein dichtes und mehrere k"ornige K"ugelchen. Vergr"o"serung 75.}
\end{figure}
\vspace*{\fill}
\clearpage

\rhead{Tafel 7.}
\vspace*{\fill}
\begin{figure}[H]
\centering
\includegraphics[width=\textwidth,keepaspectratio]{figs/7-3.png}
\caption{\small Figur 3 --- Chondrit von Tieschitz. K"ugelchen, davon zwei mit Eindr"ucken, Kn"ollchen, Splitter. Vergr"o"serung 75.}
\end{figure}
\vspace*{\fill}
\clearpage

\rhead{Tafel 7.}
\vspace*{\fill}
\begin{figure}[H]
\centering
\includegraphics[width=\textwidth,keepaspectratio]{figs/7-4.png}
\caption{\small Figur 4 --- Chondrit von Homestead. Ein porphyrisches und ein strahliges K"ugelchen. Vergr"o"serung 65.}
\end{figure}
\vspace*{\fill}
\clearpage

\rhead{Tafel 8.}
\subsection{Erkl"arung der Tafel 8.}
\paragraph{}
Figur 1. Der Stein von Dhurmsala, dem das Bild entnommen ist, enth"alt viele gr"o"sere Chondren von l"anglichrunder Form. Ein solches ist hier dargestellt um zu zeigen, dass bisweilen, wenn auch selten, in einem gro"sen Chondrum ein kleines eingeschlossen ist. Das gro"se Chondrum ist unvollkommen porphyrisch. Kristalle, K"orner und das kleine Chondrum, alle aus Olivin bestehend, sind von einer dichten, bei st"arkerer Vergr"o"serung feink"ornig erscheinenden Grundmasse umgeben, zum Teile aber sitzen die K"orner hart aneinander. Das kleine Chondrum ist monosomatisch.

Figur 2. Die Figur stellt ein kleink"orniges Chondrum aus dem Stein von Seres dar. Dasselbe ist von l"anglich runder, etwas abgeplatteter Form und besteht haupts"achlich aus gelblichgr"unem Olivin, dessen K"orner oft Glask"ugelchen und opake Einschl"usse f"uhren. Zwischen denselben ist nur eine geringe Menge amorpher Grundmasse bemerkbar. Selten ist ein br"aunliches Bronzitk"ornchen zu sehen. Die Rinde ist voll von Magnetkies und Eisen, auch im Innern des Chondrums treten zwei gr"o"sere K"orner von Magnetkies auf.

Figur 3. Die in den Chondren enthaltene Glasmasse ist bisweilen in auffallender Menge vorhanden. Hier ist ein solcher Fall aus dem Stein von Lancé abgebildet. Die Rinde des K"ugelchens ist geschlossen, aus Olivink"ornern zusammengesetzt. Das Innere besteht aus braunem Glas, welches wiederum Olivink"orner einschlie"st. Der Olivin enth"alt K"ugelchen von braunem Glas und Magnetkies eingeschlossen.

Figur 4. Das hier dargestellte Olivink"ugelchen im Stein von Tieschitz zeigt an allen Punkten gleichzeitige Ausl"oschung, entspricht also einem einzigen Kristall und hat "Ahnlichkeit mit einem Kernkristall. Die blass gr"unliche H"ulle enth"alt viele braune Glask"ugelchen, auch opake K"ornchen. Der braune Kern besteht aus Olivin und aus braunem Glase. Der Olivin ist eine Fortsetzung der H"ulle, doch ist er hier l"uckenhaft gebildet, aus krummen und lappigen, im Bilde beil"aufig horizontalen W"anden bestehend, welche nur geringe Zwischenr"aume lassen. Diese sind mit braunem Glase gef"ullt.
\clearpage

\rhead{Tafel 8.}
\vspace*{\fill}
\begin{figure}[H]
\centering
\includegraphics[width=\textwidth,keepaspectratio]{figs/8-1.png}
\caption{\small Figur 1 --- Ein porphyrisches Chondrum, K"orner und ein kleineres Chondrum einschlie"send. Stein von Dhurmsala. Vergr"o"serung 8.}
\end{figure}
\vspace*{\fill}
\clearpage

\rhead{Tafel 8.}
\vspace*{\fill}
\begin{figure}[H]
\centering
\includegraphics[width=\textwidth,keepaspectratio]{figs/8-2.png}
\caption{\small Figur 2 --- K"orniges Chondrum in dem Stein von Seres. Vergr"o"serung 65.}
\end{figure}
\vspace*{\fill}
\clearpage

\rhead{Tafel 8.}
\vspace*{\fill}
\begin{figure}[H]
\centering
\includegraphics[width=\textwidth,keepaspectratio]{figs/8-3.png}
\caption{\small Figur 3 --- K"ugelchen mit Glaseinschluss, aus dem Chondrit von Lancé. Vergr"o"serung 160.}
\end{figure}
\vspace*{\fill}
\clearpage

\rhead{Tafel 8.}
\vspace*{\fill}
\begin{figure}[H]
\centering
\includegraphics[width=\textwidth,keepaspectratio]{figs/8-4.png}
\caption{\small Figur 4 --- Monosomatisches K"ugelchen mit dunklerem Kern und blasser H"ulle. Chondrit von Tieschitz. Vergr"o"serung 160.}
\end{figure}
\vspace*{\fill}
\clearpage

\rhead{Tafel 9.}
\subsection{Erkl"arung der Tafel 9.}
\paragraph{}
Figur 1. In einem porphyrischen K"ugelchen des ausgezeichnet chondritischen Steines von Borkut sind mehrere Olivinkristall umgeben von einer Matrix, welche aus braunem Glase besteht. Die Olivine schlie"sen viel von der Grundmasse ein. In dem oberen Kristall, welcher beil"aufig parallel 1 0 0 durchschnitten erscheint und die scharfe Kante 0 2 1 : 0 2$^{\prime}$ 1 darbietet, sind au"ser dem gro"sen Einschluss in der Mitte, welcher sich wie ein negativer Kristall verh"alt, noch andere Glaseinschl"usse zu sehen, welche, wenn auch unvollkommen, einen schaligen Bau hervorrufen. Unter diesem Kristall ist ein gro"ses Individuum zu sehen, welches auch noch deutliche Kristallumrisse erkennen l"asst. Dasselbe ist von mehreren tafelf"ormigen parallel gelagerten Glaseinschl"ussen durchzogen, so dass der Kristall gef"achert erscheint. Links ist ein Olivinkristall mit sehr wenig Einschl"ussen wiedergegeben.

Figur 2. Ein Olivinkristall, welcher beil"aufig parallel 0 1 0 durchschnitten ist und weniger scharfe Umrisse zeigt. Im Inneren sieht man radialfaserige Aggregate, welche eine andere Ausl"oschung zeigen als die Umgebung und welche Bronzit zu sein scheinen, au"serdem feine gestreckte Glaseinschl"usse. Das Ganze erscheint im Innern als eine unregelm"a"sige Verwachsung von Olivin mit Bronzit, nach au"sen zu aber als ein gleichartiger Kristall.

Figur 3. Schiefer Durchschnitt durch einen Olivinkristall. Die Richtung des Schnittes ist der Zone 1 0 0 : 0 1 0 gen"ahert. Auf der rechten Seite ist der Kristall ge"offnet und von der Grundmasse aus dringen zwei geweihf"ormige Glaseinschl"usse herein, die bez"uglich einer horizontalen Linie beil"aufig symmetrisch angeordnet sind. Der Kristall erscheint demnach hier in mehrere W"ande geteilt, welche gegen eine dickere Mittelwand symmetrisch gestellt sind. Links ist der Kristall einheitlich gebaut. Die vertikale Kante scheint der aufrechten Axe parallel zu sein. Die Umgebung besteht gr"o"stenteils aus Olivink"ornern.

Figur 4. Ein merkw"urdiges Individuum von Olivin, welches die Form eines Pilzes nachahmt. Der Stiel zur Linken wird von einem undeutlich ausgebildeten Kristall dargestellt, welcher ungef"ahr dieselbe Stellung hat, wie der oberste Kristall in Fig. 1. Der Hut wird von einem monosomatischen K"ugelchen gebildet, welches im Innern von einem braunen Glasnetz durchzogen ist. Der Olivin erscheint hier in viele Platten und St"abchen zerteilt, w"ahrend die Rinde vollkommen zusammenh"angend ist. Wie schon aus der Lage der Spaltrisse zu entnehmen ist, bilden Hut und Stiel ein Individuum. In der Tat geben alle Teile der ganzen Pilzform gleichzeitige Ausl"oschung. Der Doppelk"orper ist umgeben von braunem Glas und vielen aneinander gedr"angten Olivinkristallen. Das Bild stellt, wie der Vergleich mit Fig. 1 auf voriger Tafel zeigt, einen Teil des dort abgebildeten porphyrischen Olivinknollens dar.
\clearpage

\rhead{Tafel 9.}
\vspace*{\fill}
\begin{figure}[H]
\centering
\includegraphics[width=\textwidth,keepaspectratio]{figs/9-1.png}
\caption{\small Figur 1 --- Olivinkristall in einem glasreichen K"ugelchen des Steines von Borkut. Vergr"o"serung 160.}
\end{figure}
\vspace*{\fill}
\clearpage

\rhead{Tafel 9.}
\vspace*{\fill}
\begin{figure}[H]
\centering
\includegraphics[width=\textwidth,keepaspectratio]{figs/9-2.png}
\caption{\small Figur 2 --- Olivinkristall mit Bronzit- und Glaseinschl"ussen im Chondrit von Aigle. Vergr"o"serung 70.}
\end{figure}
\vspace*{\fill}
\clearpage

\rhead{Tafel 9.}
\vspace*{\fill}
\begin{figure}[H]
\centering
\includegraphics[width=\textwidth,keepaspectratio]{figs/9-3.png}
\caption{\small Figur 3 --- Olivinkristall mit einem geweihf"ormigen Glaseinschluss im Chondrit von Seres. Vergr"o"serung 160.}
\end{figure}
\vspace*{\fill}
\clearpage

\rhead{Tafel 9.}
\vspace*{\fill}
\begin{figure}[H]
\centering
\includegraphics[width=\textwidth,keepaspectratio]{figs/9-4.png}
\caption{\small Figur 4 --- Monosomatisches Olivink"ugelchen links mit einem gleichorientierten Fortsatz. Chondrit von Dhurmsala. Vergr"o"serung 35.}
\end{figure}
\vspace*{\fill}
\clearpage

\rhead{Tafel 10.}
\subsection{Erkl"arung der Tafel 10.}
\paragraph{}
Figur 1. In einem gemischten K"ugelchen, welches aus Olivin, Bronzit und braunem Glas besteht und in dem Stein von Tieschitz beobachtet wurde, erscheint der Olivin entweder in geschlossenen Kristallen mit vielen langgezogenen Glaseinschl"ussen oder in Kristallskeletten. Das Bild zeigt den gr"o"seren Teil eines solchen skelettartigen Individuums aus m"aanderf"ormig angeordneten Leisten bestehend, welche fadenf"ormige, an den Enden "ofters k"opfige braune Glaseinschl"usse enthalten. Die Leisten, welche nach den Kristallaxen gestreckt sind, umschlie"sen eine tiefbraune Glasmasse und sind von Glas, Olivinkristallen und radialfaserigem Bronzit umgeben. Fr"uher, als mir noch weniger Erfahrung zur Seite stand, hatte ich diese Leisten f"ur Bronzit gehalten (Denkschr. d. Wiener Akad. Bd. 39. pag. 197. Fig. 7).

Figur 2. Das Vorkommen von bl"atterigen Olivink"ugelchen, welche man in den Chondriten "ofters findet, wird hier durch ein deutliches Beispiel aus dem Stein von Mez"o-Madaras dargestellt. Das K"ugelchen besteht im Inneren aus parallel gestellten und optisch gleich orientierten blassgelbgr"unen Tafeln von Olivin und aus hellbraunem Glas, welches die Zwischenr"aume ausf"ullt. Der Schnitt ist ungef"ahr senkrecht gegen die Ebene der Platten gef"uhrt. Nach Au"sen ist das K"ugelchen von einer Olivinrinde geschlossen, welche mit den Platten des Inneren gleichzeitig ausl"oscht, so dass das Ganze, vom Glas abgesehen ein einziges von F"achern durchzogenes Individuum darstellt, einem netzartig ausgebildeten Kristall entsprechend.

Figur 3. W"ahrend in dem vorigen Falle das K"ugelchen aus Platten zusammengesetzt ist, sind es hier vorzugsweise St"abchen von Olivin, welche das Innere bilden. Dieselben scheinen nach den drei Kristallaxen gestreckt zu sein, da sie alle gleichzeitig ausl"oschen und an denselben drei Richtungen ausgesprochen sind. Nur der linke untere Teil der K"ugelchen zeigt eine andere Ausl"oschung als das "ubrige und dem entsprechend eine andere, vierte Richtung der St"abchen. Zwischen den St"abchen ist wiederum ein hellbraunes Glas verbreitet, auch ist an vielen Stellen des Umrisses eine d"unne Rinde bemerkbar, welche mit den St"abchen zugleich ausl"oscht. Am unteren Rande rechts, wo keine Rinde wahrnehmbar ist, erscheint das K"ugelchen unvollst"andig und so als ob ein St"uck abgebrochen w"are. In diesem K"ugelchen erscheint demnach der Olivin gr"o"stenteils als ein einziges netzartig ausgebildetes Individuum, entsprechend den gestrickten Formen. Die schwarzen Unterbrechungen im Bilde des K"ugelchens sind durch sekund"ar entstandene Kl"ufte mit brauner F"ullung hervorgebracht. Die Umgebung wird von Olivink"ornern und von Magnetkies gebildet.

Figur 4. Ein gro"ses polysomatisches Olivink"ugelchen, aus mehreren Systemen von ann"ahernd parallelen Tafeln von Olivin und von Glas zusammengesetzt. Jedes der Systeme zeigt eine andere einheitliche Ausl"oschung und erscheint bei st"arkerer Vergr"o"serung "ahnlich dem inneren Teile des in Fig. 2 dargestellten K"ugelchens, doch finden sich im Glase hie und da doppelbrechende Nadeln, welche f"ur Bronzit zu halten sind. Vier Systeme herrschen vor. Die geraden Begrenzungslinien derselben lassen auf ebene Zusammensetzungsfl"achen schlie"sen. Am Rande wird an vielen Stellen eine d"unne Olivinrinde bemerkbar, welche mit dem zugeh"origen Plattensystem gleichzeitig ausl"oscht. Die Umgebung besteht aus K"ornchen und K"ugelchen, die vorwiegend Olivin sind und aus Magnetkies.
\clearpage

\rhead{Tafel 10.}
\vspace*{\fill}
\begin{figure}[H]
\centering
\includegraphics[width=\textwidth,keepaspectratio]{figs/10-1.png}
\caption{\small Figur 1 --- Skeletartiger Olivin mit fadenf"ormigen Glaseinschl"ussen in dem Chondrit von Tieschitz. Vergr"o"serung 160.}
\end{figure}
\vspace*{\fill}
\clearpage

\rhead{Tafel 10.}
\vspace*{\fill}
\begin{figure}[H]
\centering
\includegraphics[width=\textwidth,keepaspectratio]{figs/10-2.png}
\caption{\small Figur 2 --- Gef"achertes K"ugelchen aus abwechselnden Lamellen von Olivin und Glas bestehend. Chondrit von Mez"o-Madaras. Vergr"o"serung 200.}
\end{figure}
\vspace*{\fill}
\clearpage

\rhead{Tafel 10.}
\vspace*{\fill}
\begin{figure}[H]
\centering
\includegraphics[width=\textwidth,keepaspectratio]{figs/10-3.png}
\caption{\small Figur 3 --- Olivin von gestrickter Form in einem K"ugelchen des Steines von Homestead. Vergr"o"serung 200.}
\end{figure}
\vspace*{\fill}
\clearpage

\rhead{Tafel 10.}
\vspace*{\fill}
\begin{figure}[H]
\centering
\includegraphics[width=\textwidth,keepaspectratio]{figs/10-4.png}
\caption{\small Figur 4 --- Olivink"ugelchen aus mehreren Systemen abwechselnder Olivin- und Glaslamellen bestehend. Chondrit von Knyahinya. Vergr"o"serung 25.}
\end{figure}
\vspace*{\fill}
\clearpage

\rhead{Tafel 11.}
\subsection{Erkl"arung der Tafel 11.}
\paragraph{}
Figur 1. Durchschnitt eines Olivink"ugelchens mit dicker, durchsichtiger Rinde, welche, von kleinen Abweichungen abgesehen, gleichzeitig mit dem Olivin des Inneren ausl"oscht. Der letztere bietet in diesem Schnitte eine gekr"oseartige Zeichnung, da das braune Glas ziemlich unregelm"a"sig verteilt ist. "Au"serlich erscheint das K"ugelchen mit einer zusammenh"angenden undurchsichtigen Schichte bedeckt, welche aus K"ornchen von Magnetkies und auch von Eisen besteht. Im Wesentlichen herrscht "Ahnlichkeit mit dem auf Tafel 8 in Fig. 4 dargestellten Objekte.

Figur 2. Auch dieses Olivink"ugelchen ist mit Ausnahme der "au"sersten Rinde monosomatisch. Im Inneren zeigen sich viele flache St"abchen von Olivin, die zum gr"o"seren Teile nach einer im Bilde vertikalen Richtung ausgedehnt und einander parallel gelagert sind und nur zum kleineren Teile einer anderen Richtung folgen, welche die vorige unter einem schiefen Winkel schneidet. Die St"abchen liegen in einer Grundmasse von braunem Glase. Die Rinde des K"ugelchens ist von zweierlei Beschaffenheit. Der innere durchsichtigere Teil gibt gleichzeitig mit den St"abchen im Inneren einheitliche Ausl"oschung. Er enth"alt viele kleine runde Einschl"usse von Magnetkies. Der "au"sere Teil erscheint aus vielen sehr verschieden orientierten K"ornchen zusammengesetzt, welche meistens aus Olivin, im "ubrigen aus Magnetkies bestehen.

Figur 3. Die Rinde dieses Olivink"ugelchens wird von mehreren kurzen, dicken, verschieden orientierten Individuen gebildet und umschlie"st eine Masse braunen Glases, welches mehrere Olivinkristall verbindet. Diese bieten z. T. langgestreckte z. T. kurze Durchschnitte dar, welche letztere die kranzf"ormige Anordnung der Rinde wiederholen. Die Umgebung bilden K"orner und K"ugelchen von Olivin, solche von Bronzit, ferner braunes Glas und Magnetkies.

Figur 4. Diese Figur zeigt die gew"ohnliche Anordnung der Olivin- und Bronzitkristalle in den glasf"uhrenden Chondren. Der Olivin bildet meist breite, der Bronzit schmale gestreckte Individuen. An manchen Stellen sind vierseitige Querschnitte der letzteren zu bemerken und im unteren Teile der Figur erscheint auch ein Durchdringungszwilling von Bronzit in der Gestalt eines schiefen Kreuzes abgebildet. Die Bronzite schmiegen sich "ofters an die Olivine an und dr"angen sich in den Kan"alen zwischen den Olivinkristallen zusammen. Alle diese Kristalle sind in einem braunen Glase eingebettet.
\clearpage

\rhead{Tafel 11.}
\vspace*{\fill}
\begin{figure}[H]
\centering
\includegraphics[width=\textwidth,keepaspectratio]{figs/11-1.png}
\caption{\small Figur 1 --- Ein monosomatisches Olivink"ugelchen mit dicker Rinde in dem Chondrit von Alfianello. Vergr"o"serung 60.}
\end{figure}
\vspace*{\fill}
\clearpage

\rhead{Tafel 11.}
\vspace*{\fill}
\begin{figure}[H]
\centering
\includegraphics[width=\textwidth,keepaspectratio]{figs/11-2.png}
\caption{\small Figur 2 --- Olivink"ugelchen, im Innern monosomatisch, mit dicker Rinde in dem Stein von Mez"o-Madaras. Vergr"o"serung 70.}
\end{figure}
\vspace*{\fill}
\clearpage

\rhead{Tafel 11.}
\vspace*{\fill}
\begin{figure}[H]
\centering
\includegraphics[width=\textwidth,keepaspectratio]{figs/11-3.png}
\caption{\small Figur 3 --- Polysomatisches Olivink"ugelchen mit vielem Glas und dicker Rinde in dem Chondrit von Seres. Vergr"o"serung 160.}
\end{figure}
\vspace*{\fill}
\clearpage

\rhead{Tafel 11.}
\vspace*{\fill}
\begin{figure}[H]
\centering
\includegraphics[width=\textwidth,keepaspectratio]{figs/11-4.png}
\caption{\small Figur 4 --- Olivin in gr"o"seren, Bronzit in kleineren Kristallen und Glasgrundmasse in einem gro"sen K"ugelchen des Steines von Knyahinya. Vergr"o"serung 70.}
\end{figure}
\vspace*{\fill}
\clearpage

\rhead{Tafel 12.}
\subsection{Erkl"arung der Tafel 12.}
\paragraph{}
Figur 1. Ein Beispiel des Vorkommens einzelner Bronzitkristalle in einer feink"ornig aussehenden Partie des Mocser Steines. Durch die Begrenzung von Magnetkies werden die Enden des Kristalls deutlicher hervorgehoben. Der Schnitt ist beil"aufig parallel 1 0 0. Die obere dachf"ormige Grenze entspricht der gew"ohnlich durch die Fl"achen \emph{e} = (1 2 2) hervorgebrachten stumpfen Endigung. Au"ser den feinen Spaltlinien gem"a"s der Spaltbarkeit nach dem aufrechten Prisma geben sich auch noch andere Trennungen zu erkennen. Die Einschl"usse sind teils durchsichtige, teils opake K"ornchen, selten Glask"ugelchen. Um die Spaltlinien st"arker hervortreten zu lassen, wurde das Bild bei starker Blendung aufgenommen Der Bronzitkristall hat links einen Fortsatz und grenzt dort an Bronzitk"orner, rechts aber zum Teil an k"ornigen Olivin mit Plagioklas.

Figur 2. Ein Bronzitindividuum in einem der harten braunen halbglasigen Steinsplitter, welche in dem tuffartigen Chondrit von Alexinaé so h"aufig sind. Der Schnitt ist der optischen Beobachtung zufolge beil"aufig parallel 0 1 0 gef"uhrt, also ungef"ahr senkrecht zur negativen Mittellinie. Die zahlreichen gro"sen Einschl"usse sind sehr auffallend. Sie erf"ullen negative Kristalle und bestehen aus Magnetkies, braunem Glase und einem Gemenge dieser beiden. Untergeordnet finden sich doppelbrechende K"ornchen verschiedener Orientierung eingeschlossen, welche wohl auf Olivin zu beziehen sind. Der Bronzit ist beiderseits von Magnetkies eingefasst.

Figur 3. Von den Bronzitchondren mit gro"sen unregelm"a"sig verbundenen Individuen gibt dieser Durchschnitt einen "ofter vorkommenden Fall an. Das l"angliche K"orperchen hat eine etwas zackige Begrenzung, die einer rauen Oberfl"ache entspricht. Die Rinde ist vollkommen kompakt, im Inneren erscheint eine ziemlich gro"se Menge braunen Glases eingeschlossen, welches auch kleine K"ornchen von Magnetkies enth"alt.

Figur 4. F"ur die Darstellung einer in manchen Chondren wiederkehrenden eigent"umlichen Textur des Bronzits wurde ein Beispiel aus dem Stein von Knyahinya benutzt. Dickere St"abchen von Bronzit bilden ein unregelm"a"siges Gitter, dessen "Offnungen mit parallelfaserigem bis radialfaserigem Bronzit erf"ullt sind. Im unteren Teile des Bildes hat man eine Grenze des K"ugelchens, auf der rechten Seite einen birnf"ormigen Einschluss von Magnetkies, welcher im Schliffe etwas ausgebrochen ist.
\clearpage

\rhead{Tafel 12.}
\vspace*{\fill}
\begin{figure}[H]
\centering
\includegraphics[width=\textwidth,keepaspectratio]{figs/12-1.png}
\caption{\small Figur 1 --- L"angsschnitt eines Bronzitkristalls in dem Chondrit von Mocs. Vergr"o"serung 160.}
\end{figure}
\vspace*{\fill}
\clearpage

\rhead{Tafel 12.}
\vspace*{\fill}
\begin{figure}[H]
\centering
\includegraphics[width=\textwidth,keepaspectratio]{figs/12-2.png}
\caption{\small Figur 2 --- Teil eines Bronzitl"angsschnittes mit vielen Einschl"ussen. Stein von Alexinaé. Vergr"o"serung 160.}
\end{figure}
\vspace*{\fill}
\clearpage

\rhead{Tafel 12.}
\vspace*{\fill}
\begin{figure}[H]
\centering
\includegraphics[width=\textwidth,keepaspectratio]{figs/12-3.png}
\caption{\small Figur 3 --- Polysomatisches Bronzitk"ugelchen mit Glaseinschluss im Chondrit von Pultusk. Vergr"o"serung 70.}
\end{figure}
\vspace*{\fill}
\clearpage

\rhead{Tafel 12.}
\vspace*{\fill}
\begin{figure}[H]
\centering
\includegraphics[width=\textwidth,keepaspectratio]{figs/12-4.png}
\caption{\small Figur 4 --- Teil eines Bronzitk"ugelchens von gitterartig stengeliger Textur im Stein von Knyahinya. Vergr"o"serung 70.}
\end{figure}
\vspace*{\fill}
\clearpage

\rhead{Tafel 13.}
\subsection{Erkl"arung der Tafel 13.}
\paragraph{}
Figur 1. Die radialstengelige Textur vieler Bronzitchondren wird hier durch ein Beispiel aus dem Stein von Tipperary illustriert. Das Gef"uge ist bei der Betrachtung mit freiem Auge radialfaserig und zwar ist die Faserung exzentrisch. Viele der im Bilde sichtbaren S"aulchen sind gerade, andere aber etwas gebogen. An den breiteren sind die feinen Spaltlinien entsprechend dem aufrechten Prisma und die welligen querverlaufenden Trennungen deutlich. An der Grenze des K"ugelchens erscheint ein Ansatz von k"ornigem Bronzit.

Figur 2. Durchschnitt eines harten braunen Bronzitk"ugelchens aus dem Chondrit von Gnadenfrei, welchen ich Hrn. Prof. v. Lasaulx verdanke. Ein Teil der Stengelchen ist vom Zentrum des K"ugelchens aus radial gerichtet, die Mehrzahl folgt aber anderen Richtungen. Die Spaltlinien und querverlaufenden Trennungen treten auch hier allenthalben hervor. Stellenweise zeigen sich unregelm"a"sig geformte Beimengungen von Magnetkies und von braunem Glase.

Figur 3. Dieser anf"anglich wirrstengelig erscheinende Durchschnitt besteht nach der optischen Pr"ufung aus vier parallelstengeligen Bronzitb"undeln, deren jedes einheitlich ausl"oscht. Der eine Teil zur Linken mit einer deutlichen Gitterzeichnung erscheint als ein querdurchschnittenes B"undel parallel verwachsener Individuen, der Teil rechts oben entspricht nahezu einem L"angsschnitt durch ein solches B"undel, w"ahrend die beiden "ubrigen Teile als schiefe Durchschnitte solcher Verwachsungen anzusehen sind. Das Bild vereinigt mehrere nicht selten vorkommende Arten von Durchschnitten des Bronzits in den Chondriten. Der Umriss des K"ugelchens erscheint oberhalb stark ausgr"undet. Die Umgebung wird von dunkler Grundmasse und verschiedenen Chondren gebildet.

Figur 4. Durchschnitt eines dunkelbraunen festen Bronzitk"ugelchens, welche in den Chondriten h"aufig sind. Die Ebene des Durchschnittes liegt quer gegen die feine Faserung. An vielen Stellen des Umrisses bemerkt man eine sehr d"unne Rinde von k"orniger Textur. Die n"achstgelegene innere Schichte ist grau, etwas durchsichtig und bietet an vielen Stellen eine sehr feine Gitterzeichnung dar. Ganze Partien haben einheitliche Ausl"oschung. Der innere Teil ist fast undurchsichtig, wahrscheinlich in Folge einer Beimischung opaker K"ornchen. Der Umriss ist ebenfalls ausgr"undet. In der Umgebung zeigen sich vorherrschend K"orner und Kristalle von Olivin.
\clearpage

\rhead{Tafel 13.}
\vspace*{\fill}
\begin{figure}[H]
\centering
\includegraphics[width=\textwidth,keepaspectratio]{figs/13-1.png}
\caption{\small Figur 1 --- Bronzitk"ugelchen, radialfaserig, aus dem Stein von Tipperary. Vergr"o"serung 70.}
\end{figure}
\vspace*{\fill}
\clearpage

\rhead{Tafel 13.}
\vspace*{\fill}
\begin{figure}[H]
\centering
\includegraphics[width=\textwidth,keepaspectratio]{figs/13-2.png}
\caption{\small Figur 2 --- Bronzitk"ugelchen, undeutlich radialfaserig aus dem Stein von Gnadenfrei. Vergr"o"serung 60.}
\end{figure}
\vspace*{\fill}
\clearpage

\rhead{Tafel 13.}
\vspace*{\fill}
\begin{figure}[H]
\centering
\includegraphics[width=\textwidth,keepaspectratio]{figs/13-3.png}
\caption{\small Figur 3 --- Durchschnitt eines aus vier Faserb"undeln bestehenden K"ugelchens im Stein von Mez"o-Madaras. Vergr"o"serung 60.}
\end{figure}
\vspace*{\fill}
\clearpage

\rhead{Tafel 13.}
\vspace*{\fill}
\begin{figure}[H]
\centering
\includegraphics[width=\textwidth,keepaspectratio]{figs/13-4.png}
\caption{\small Figur 4 --- Dichtes Bronzitk"ugelchen quer gegen die Faserung geschnitten, im Stein von Knyahinya. Vergr"o"serung 80.}
\end{figure}
\vspace*{\fill}
\clearpage

\rhead{Tafel 14.}
\subsection{Erkl"arung der Tafel 14.}
\paragraph{}
Figur 1. Die im Steine von Dhurmsala vorkommenden feinfaserigen K"ugelchen sind hier durch das Bild eines Schnittes, welcher den divergierenden Fasern parallel ist, repr"asentiert. Feine graue f"acherf"ormig angeordnete Linien bezeichnen die exzentrische Faserung. Am Rande oben und rechts gehen die tr"uben Fasern in gr"o"sere durchsichtige Bronzitindividuen aus. Opake Einschl"usse von Magnetkies machen sich besonders im oberen Teile bemerklich, im "ubrigen sind sie sp"arlich vertreten und bilden zuweilen sehr kleine sternf"ormige Gruppen. Eine Rinde ist nicht unterscheidbar.

Figur 2. Auch hier ist die radialfaserige Textur ausgesprochen, doch weniger deutlich, als im vorigen Falle, weil dickere Stengelchen und abweichend orientierte K"orner von Bronzit "ofters auftreten. Einschl"usse von Magnetkies finden sich besonders in den "au"seren Teilen, ein gr"o"serer kugeliger Einschluss dieser Art ist im unteren Teile bemerklich. An dem Umrisse, der elliptisch ist, hebt sich die etwas durchsichtige d"unne Rinde von dem Inneren deutlich ab. Sie zeigt bei optischer Pr"ufung eine k"ornige Zusammensetzung. In der Nachbarschaft bemerkt man Olivin- und Bronzitk"ugelchen.

Figur 3. Das Auftreten jenes farblosen doppelbrechenden Silikates, welches eine "Ahnlichkeit mit Monticellit zeigt, ist hier durch ein Bild aus dem Stein von Knyahinya charakterisiert. Die wei"sen Stellen entsprechen dem genannten Silikat. Sie scheinen von demselben Individuum herzur"uhren, da von untergeordneten Abweichungen abgesehen, alle gleichzeitig ausl"oschen. Im Ferneren bietet sich nichts Charakteristisches dar. Es zeigen sich nur krumme unregelm"a"sige Spr"unge und als Einschl"usse sehr kleine doppelbrechende K"ornchen sowie eine Gruppe von Magnetkiesk"ornchen. Links oben bemerkt man gekr"oseartigen Olivin, links unten parallelfaserigen Bronzit, welche beide mit dem farblosen Silikat innig verwachsen sind. Rechts ist ein Gemenge von Olivin und Magnetkies verbreitet.

Figur 4. Dieses Bild stellt dasselbe Silikat in einem Individuum vor, das wiederum eine einheitliche, etwas undul"ose Ausl"oschung zeigt, hier jedoch an manchen Stellen feine Spaltlinien in drei Richtungen erkennen l"asst. Die unregelm"a"sigen Spr"unge und die aus doppelbrechenden K"ornchen bestehenden Einschl"usse treten auch hier auf. Zur Rechten hat man ein Korn von Olivin mit einer Tendenz zur Lamellenbildung in enger Verwachsung mit dem vorigen Silikat. Die Umgebung wird von Magnetkies und Olivin, der in Folge einer Zersetzung des ersteren braun gef"arbt ist, gebildet.
\clearpage

\rhead{Tafel 14.}
\vspace*{\fill}
\begin{figure}[H]
\centering
\includegraphics[width=\textwidth,keepaspectratio]{figs/14-1.png}
\caption{\small Figur 1 --- Bronzitk"ugelchen, in einem L"angsschnitte die f"acherartig divergierenden Fasern zeigend. Chondrit von Dhurmsala. Vergr"o"serung 60.}
\end{figure}
\vspace*{\fill}
\clearpage

\rhead{Tafel 14.}
\vspace*{\fill}
\begin{figure}[H]
\centering
\includegraphics[width=\textwidth,keepaspectratio]{figs/14-2.png}
\caption{\small Figur 2 --- Undeutlich faseriges Bronzitk"ugelchen mit d"unner Rinde. Chondrit von Mez"o-Madaras. Vergr"o"serung 60.}
\end{figure}
\vspace*{\fill}
\clearpage

\rhead{Tafel 14.}
\vspace*{\fill}
\begin{figure}[H]
\centering
\includegraphics[width=\textwidth,keepaspectratio]{figs/14-3.png}
\caption{\small Figur 3 --- Monticellit"ahnliches farbloses doppelbrechendes Silikat im Chondrit von Knyahinya. Vergr"o"serung 70.}
\end{figure}
\vspace*{\fill}
\clearpage

\rhead{Tafel 14.}
\vspace*{\fill}
\begin{figure}[H]
\centering
\includegraphics[width=\textwidth,keepaspectratio]{figs/14-4.png}
\caption{\small Figur 4 --- Dasselbe Silikat mit feinen Spaltrissen. Chondrit von Knyahinya. Vergr"o"serung 70.}
\end{figure}
\vspace*{\fill}
\clearpage

\rhead{Tafel 15.}
\subsection{Erkl"arung der Tafel 15.}
\paragraph{}
Figur 1. Der Gemengteil, welcher als Augit bestimmt wurde, erscheint hier in K"ornern, welche in dem dunklen Chondrit von Renazzo polysomatische K"ugelchen zusammensetzen. Das Bild gibt die Durchschnitte zweier solcher Chondren, die einander ber"uhren, nach der Aufnahme im gew"ohnlichen Lichte wieder. Die Augitk"ornchen haben stellenweise Umrisse, die an Kristalle erinnern. Manche umschlie"sen rundliche Olivink"orner, die wiederum Glaseinschl"usse und Magnetkies enthalten. Auch im Augit sind K"ugelchen von Magnetkies verstreut. Zwischen den Augitk"ornern ist eine braune Masse eingeklemmt, welche aus Glas und feinen Fasern besteht. Die Umgebung der Chondren ist eine schw"arzliche Grundmasse.

Figur 2. Dieselben zwei Chondren bieten im polarisierten Lichte das Bild einer lamellaren Zwillingsverwachsung, welche dem Augit in den Chondriten allgemein zuzukommen scheint. Die Lamellierung ist weder so scharf noch so eben, wie in den Plagioklasen. Die deutlich gestreiften Schnitte zeigen wenig lebhafte Farben, die anderen aber "ofters sch"one helle Farbent"one. Der gr"o"ste Unterschied in der Ausl"oschung benachbarter Lamellen betr"agt etwa 35°. Die Olivink"orner heben sich von der Umgebung durch ihre Farben deutlich ab.

Figur 3. Ein Beispiel jener Augitchondren, in welchen die Individuen eine Durchwachsung zeigen, aus dem Stein von Mez"o-Madaras. In der Mitte bemerkt man ein Individuum, welches in aufrechter Richtung Spaltlinien und gestreckte Einschl"usse zeigt, oben eine dachf"ormige Endigung erkennen l"asst und unten von einem kleinen Individuum durchdrungen ist. Das gro"se Individuum gibt im polarisierten Lichte ein System von "ofters krummen und abs"atzigen Streifen parallel der aufrechten Richtung, entsprechend vielen Lamellen parallel 1 0 0, deren Ausl"oschungen mit dieser Ebene beiderseits ungef"ahr 20° bilden. Rechts hat man ein zweites Individuum, welches in eine schiefe Spitze ausgeht und keine Lamellierung zeigt. Die Ausl"oschungsschiefe ist hier kaum 9°. In der L"ucke zwischen diesem und dem Hauptindividuum ist ein Gemenge von Glas und Augit, letzterer in K"ornern und Nadeln, eingeklemmt. Das gr"unliche Glas erscheint auch in den schlauchf"ormigen Einschl"ussen des gro"sen Individuums. Links unten zeigen sich mehrere Individuen, deren gr"o"stes schief durchschnitten ist, so dass die schlauchf"ormigen Einschl"usse einen spindelf"ormigen Querschnitt ergeben. Die Umgebung ist zun"achst eine dunkle Grundmasse, ferner treten in der Nachbarschaft Olivinchondren auf.

Figur 4. Der spreuf"ormige Augit mit Olivin, Magnetkies und Glasgrundmasse ein Gemenge darstellend, welches viele Chondren im Stein von Renazzo bildet und auch in andren Chondriten gefunden wird. Die meisten der st"abchenf"ormigen Augitschnitte zeigen schiefe Ausl"oschung, die breiten oft eine grobe Lamellierung. Bei starker Vergr"o"serung bieten die Augite, namentlich bez"uglich der Glaseinschl"usse, denselben Charakter dar wie im vorigen Bilde. Die Olivink"orner heben sich durch ihre Form und ihr Verhalten im polarisierten Lichte von der Umgebung ab. Der Magnetkies ist in runden K"ornchen verstreut. Die Grundmasse zeigt die Erscheinungen der Entglasung.
\clearpage

\rhead{Tafel 15.}
\vspace*{\fill}
\begin{figure}[H]
\centering
\includegraphics[width=\textwidth,keepaspectratio]{figs/15-1.png}
\caption{\small Figur 1 --- Zwei K"ugelchen, zumeist aus Augit bestehend, in dem Stein von Renazzo. Vergr"o"serung 70.}
\end{figure}
\vspace*{\fill}
\clearpage

\rhead{Tafel 15.}
\vspace*{\fill}
\begin{figure}[H]
\centering
\includegraphics[width=\textwidth,keepaspectratio]{figs/15-2.png}
\caption{\small Figur 2 --- Dieselben im polarisierten Lichte. Nicolhauptschnitte horizontal und vertikal.}
\end{figure}
\vspace*{\fill}
\clearpage

\rhead{Tafel 15.}
\vspace*{\fill}
\begin{figure}[H]
\centering
\includegraphics[width=\textwidth,keepaspectratio]{figs/15-3.png}
\caption{\small Figur 3 --- Ein Augitk"ugelchen, die Durchwachsung mehrerer Individuen zeigend. Chondrit von Mez"o-Madaras. Vergr"o"serung 160.}
\end{figure}
\vspace*{\fill}
\clearpage

\rhead{Tafel 15.}
\vspace*{\fill}
\begin{figure}[H]
\centering
\includegraphics[width=\textwidth,keepaspectratio]{figs/15-4.png}
\caption{\small Figur 4 --- Augit und Olivin, ein spreuartiges Gemenge darbietend. Chondrit von Renazzo. Vergr"o"serung 160.}
\end{figure}
\vspace*{\fill}
\clearpage

\rhead{Tafel 16.}
\subsection{Erkl"arung der Tafel 16.}
\paragraph{}
Figur 1. Der Gemengteil der Chondrite, welcher als Plagioklas bestimmt wurde, in der Form eines scharfen Splitters, umgeben von dunkler Grundmasse und Olivink"ornern im Stein von Murcia. Im polarisierten Lichte wird die Lamellierung, welche schon im gew"ohnlichen Lichte bemerkbar ist, sehr deutlich und es zeigen sich abwechselnde breitere und sehr schmale Streifen. Die Nicolhauptschnitte sind hier und im folgenden Bilde horizontal und vertikal zu denken. Ich verdanke den Schliff Hrn. Oberbergrat M. Websky.

Figur 2. Ein Plagioklask"ornchen mit rundlichem Umriss, verwachsen mit Olivin und Magnetkies in dem Steine von Mocs. W"ahrend im gew"ohnlichen Lichte keine Lamellierung zu sehen ist, erscheint dieselbe im polarisierten Lichte deutlich, indem verh"altnism"a"sig breite Streifen hervortreten.

Figur 3. Das gew"ohnliche Vorkommen der Plagioklask"orner in den Chondriten wird hier durch ein Bild aus dem Stein von Milena bei schwacher Vergr"o"serung charakterisiert. Alle wei"sen Stellen entsprechen dem Plagioklas, welcher mit Olivin, stellenweise auch mit Magnetkies innig verbunden ist und meistens als Zwischenklemmungsmasse vorkommt. Der Plagioklas umschlie"st h"aufig gr"o"sere bis staubartig feine K"ornchen von Olivin und auch von Magnetkies. Im polarisierten Lichte gibt er nur selten Streifen, meistens eine undul"ose Ausl"oschung.

Figur 4. Eine zuweilen vorkommende, etwas regelm"a"sige Verwachsung von Plagioklas mit Olivin und Magnetkies, in der Form eines K"ugelchens im Steine von Dhurmsala. Alle blassen Stellen entsprechen dem Plagioklas, der stark vorherrscht und von schmalen Streifen, welche sowohl von Lamellen als von St"abchen herr"uhren d"urften, durchsetzt wird. Dieselben tragen den Charakter des Olivins und bilden vorwiegend zwei Parallelsysteme, die jedes f"ur sich einheitlich ausl"oschen. Dies erinnert lebhaft an die Olivinlamellen mit zwischengeklemmtem Glas in Fig. 2 und 4 auf Taf. 10. Hier aber ist der Plagioklas das Zwischenmittel, doch zeigt sich im polarisierten Lichte der letztere aus mehreren gro"sen und einigen kleineren K"ornern zusammengesetzt, welche letzteren zum Teile die feine Streifung zeigen. Die gro"sen K"orner erscheinen einfach, sie werden in der Erstreckung durch die Olivinlamellen gar nicht beschr"ankt, sie greifen vielmehr immer "uber mehrere desselben Systeme hinaus. Die schwarzen Stellen entsprechen dem Magnetkies, welchem jedoch auch ein wenig dunklen Glases anzuh"angen scheint. Die stellenweise vorkommenden schwarzen feinen Netze und staubigen Einschl"usse k"onnen auf Chromit oder Magnetkies bezogen werden. Die Umgebung des K"ugelchens, welches auf der rechten Seite abgeplattet erscheint, ist ein Gemenge von Olivin, Eisen, Magnetkies und dem Monticellit"ahnlichen Silikat.
\clearpage

\rhead{Tafel 16.}
\vspace*{\fill}
\begin{figure}[H]
\centering
\includegraphics[width=\textwidth,keepaspectratio]{figs/16-1.png}
\caption{\small Figur 1 --- Plagioklas mit scharfeckigem Umriss im polarisierten Lichte. Chondrit von Murcia [?]. Vergr"o"serung 360.}
\end{figure}
\vspace*{\fill}
\clearpage

\rhead{Tafel 16.}
\vspace*{\fill}
\begin{figure}[H]
\centering
\includegraphics[width=\textwidth,keepaspectratio]{figs/16-2.png}
\caption{\small Figur 2 --- Plagioklask"ornchen mit rundlichem Umriss. Pol. Licht. Chondrit von Mocs. Vergr"o"serung 360.}
\end{figure}
\vspace*{\fill}
\clearpage

\rhead{Tafel 16.}
\vspace*{\fill}
\begin{figure}[H]
\centering
\includegraphics[width=\textwidth,keepaspectratio]{figs/16-3.png}
\caption{\small Figur 3 --- Plagioklas in zahlreichen kleinen K"ornchen mit Olivin verwachsen. Chondrit von Milena. Vergr"o"serung 70.}
\end{figure}
\vspace*{\fill}
\clearpage

\rhead{Tafel 16.}
\vspace*{\fill}
\begin{figure}[H]
\centering
\includegraphics[width=\textwidth,keepaspectratio]{figs/16-4.png}
\caption{\small Figur 4 --- Plagioklas mit Olivin und Magnetkies ein K"ugelchen bildend. Chondrit von Dhurmsala. Vergr"o"serung 80.}
\end{figure}
\vspace*{\fill} 
\clearpage

\rhead{Tafel 17.}
\subsection{Erkl"arung der Tafel 17.}
\paragraph{}
Figur 1. Das farblose einfachbrechende Glas, als Maskelynit bezeichnet, welches in vielen Chondriten vorkommt, ist hier durch eine Probe aus dem Stein von Alfianello bei st"arkerer Vergr"o"serung repr"asentiert. Im oberen Teile des Bildes erblickt man parallele Linien, welche blo"s von einer Verschiedenheit der Lichtbrechung herr"uhren. Rechts oben und links unten ist der Maskelynit von Olivin begrenzt, im "ubrigen von Magnetkies. Im Inneren zeigen sich an einigen Stellen K"orner von Olivin. Zur Rechten hat der Maskelynit eine geradlinige Begrenzung.

Figur 2. Das gew"ohnliche Auftreten des Maskelynits in unregelm"a"sigen meist lappigen Partikeln, welche h"aufig Olivink"orner einschlie"sen und von Magnetkies begleitet werden, ist hier durch ein Bild aus dem vorgenannten Stein dargestellt. Die Grundmasse, in welcher der Maskelynit verstreut erscheint, besteht vorwiegend aus gr"o"seren und kleineren K"ornchen von Olivin und Magnetkies. Links unten hat man auch ein scharfeckiges Partikelchen von Chromit.

Figur 3. Das Bild gibt den Durchschnitt einer der schwarzen, im Bruche glasgl"anzenden Kugeln wieder, welche in dem Stein von Chateau Renard stellenweise vorkommen. Die Grundmasse der Kugel ist Maskelynit, der an einigen Stellen wiederum die feinen parallelen Linien zeigt und an der Oberfl"ache der Kugel nur wenige schwarze Einschl"usse enth"alt. Im Inneren sind aber die opaken K"orner in gro"ser Menge vorhanden. Einige derselben lassen sich im auffallenden Lichte als Magnetkies erkennen. Die Umgebung der Kugel wird haupts"achlich von k"ornigem Olivin, zum Teil von Magnetkies gebildet.

Figur 4. Eine halbdurchsichtige blassblaue Kugel im Stein von Tipperary, welche die Erscheinungen vorgeschrittener Entglasung darbietet. Im polarisierten Lichte zeigt sich eine Zusammensetzung aus vielen kleinen doppelbrechenden K"ornchen ohne scharfe Umrisse die miteinander und mit der glasigen Grundmasse verflie"sen. Dieselben d"urften f"ur Olivin zu halten sein. Die sehr d"unne doppelbrechende Rinde hat den Charakter des Olivins. Die Umgebung der Kugel ist eine k"ornige Grundmasse, welche aus Olivin und Eisen besteht.
\clearpage

\rhead{Tafel 17.}
\vspace*{\fill}
\begin{figure}[H]
\centering
\includegraphics[width=\textwidth,keepaspectratio]{figs/17-1.png}
\caption{\small Figur 1 --- Maskelynit im Chondrit von Alfianello. Vergr"o"serung 300.}
\end{figure}
\vspace*{\fill}
\clearpage

\rhead{Tafel 17.}
\vspace*{\fill}
\begin{figure}[H]
\centering
\includegraphics[width=\textwidth,keepaspectratio]{figs/17-2.png}
\caption{\small Figur 2 --- Maskelynit in demselben Stein. Vergr"o"serung 60.}
\end{figure}
\vspace*{\fill}
\clearpage

\rhead{Tafel 17.}
\vspace*{\fill}
\begin{figure}[H]
\centering
\includegraphics[width=\textwidth,keepaspectratio]{figs/17-3.png}
\caption{\small Figur 3 --- Schwarze Kugel, zumeist aus Maskelynit bestehend, im Chondrit von Chateau Renard. Vergr"o"serung 160.}
\end{figure}
\vspace*{\fill}
\clearpage

\rhead{Tafel 17.}
\vspace*{\fill}
\begin{figure}[H]
\centering
\includegraphics[width=\textwidth,keepaspectratio]{figs/17-4.png}
\caption{\small Figur 4 --- Bl"auliche halbglasige Kugel im Chondrit von Tipperary. Vergr"o"serung 160.}
\end{figure}
\vspace*{\fill}
\clearpage

\rhead{Tafel 18.}
\subsection{Erkl"arung der Tafel 18.}
\paragraph{}
Figur 1. Um den Charakter der vorzugsweise aus Glas bestehenden Chondren durch ein einziges Bild anzudeuten, wurde dieses Beispiel aus dem Chondrit von Mez"o-Madaras gew"ahlt, welches in der blass-br"aunlichen Grundmasse mehrere lange so wie auch einige kurze Olivinkristall, ferner rechts unten farnkraut"ahnliche Bildungen und links oben eine netzf"ormige Kristallisation, aus rechtwinkelig angeordneten Nadeln bestehend, endlich auch einzelne feine Nadeln erkennen l"asst. Die Olivine sind "ofters am Ende gabelig. Die Umgebung der Glaskugel wird von einer Grundmasse, in der Olivinkristall hervortreten, gebildet.

Figur 2. Die Erscheinung der Entglasung, welche die Zwischenmasse der Chondren so h"aufig darbietet, ist hier durch einen Fall im Steine von Lancé dargestellt. Eine porphyrische Olivinkugel, in welcher die Kristalle teils enge aneinander liegen, teils durch eine halbglasige Zwischenmasse getrennt sind, zeigt in der letzteren an vielen Stellen teils einzelne feine doppelbrechende Nadeln, teils netzartige Kristallisationen von rechtwinkelig angeordneten N"adelchen, endlich wie im vorliegenden Falle Haufwerke von derlei Nadeln, so dass an diesen Stellen kein durchsichtiges Glas, sondern eine tr"ube Zwischenmasse die Olivinkristall verbindet. An einzelnen Stellen ist die glasige Zwischenmasse dunkelbraun gef"arbt.

Figur 3. Eine Andeutung des ungemein seltenen Falles, in welchem der Glaseinschluss in den Kristallen eine Libelle zeigt, wird hier durch eine Darstellung aus dem Stein von Lancé gegeben. In der schwarzen Grundmasse liegt ein Splitter von Olivin mit einem Glasei. Dasselbe erscheint im Bilde links oben und ist von einer kleinen Libelle begleitet. Das Glas ist kaum merklich gef"arbt.

Figur 4. In den festen k"ornigen Chondriten enth"alt der Olivin zahlreiche Glaseinschl"usse, deren Charakter hier durch eine Probe aus dem Stein von Stauropol bezeichnet ist. Die kleineren Einschl"usse haben oft scharfe Umrisse und sind parallel angeordnet, entsprechen also negativen Kristallen; auch manche der unregelm"a"sigen Einschl"usse haben noch diesen Charakter, so namentlich der gro"se Einschluss links oben mit horizontaler Reifung. Im "ubrigen sind eif"ormige und verschiedentlich geformte Glaspartikel vorhanden. Stellenweise zeigen die kleinen Einschl"usse eine Tendenz zu linearer Anordnung.
\clearpage

\rhead{Tafel 18.}
\vspace*{\fill}
\begin{figure}[H]
\centering
\includegraphics[width=\textwidth,keepaspectratio]{figs/18-1.png}
\caption{\small Figur 1 --- Glaskugel, Olivinkristall und Mikrolithe einschlie"send, aus dem Chondrit von Mez"o-Madaras. Vergr"o"serung 90.}
\end{figure}
\vspace*{\fill}
\clearpage

\rhead{Tafel 18.}
\vspace*{\fill}
\begin{figure}[H]
\centering
\includegraphics[width=\textwidth,keepaspectratio]{figs/18-2.png}
\caption{\small Figur 2 --- Teilweise entglaste Zwischenmasse in einer Olivinkugel des Steines von Renazzo. Vergr"o"serung 300.}
\end{figure}
\vspace*{\fill}
\clearpage

\rhead{Tafel 18.}
\vspace*{\fill}
\begin{figure}[H]
\centering
\includegraphics[width=\textwidth,keepaspectratio]{figs/18-3.png}
\caption{\small Figur 3 --- Glaseinschluss mit Libelle im Olivin des Steines von Lancé. Vergr"o"serung 300.}
\end{figure}
\vspace*{\fill}
\clearpage

\rhead{Tafel 18.}
\vspace*{\fill}
\begin{figure}[H]
\centering
\includegraphics[width=\textwidth,keepaspectratio]{figs/18-4.png}
\caption{\small Figur 4 --- Glaseinschl"usse im Olivin des Steines von Stauropol. Vergr"o"serung 300.}
\end{figure}
\vspace*{\fill}
\clearpage

\rhead{Tafel 19.}
\subsection{Erkl"arung der Tafel 19.}
\paragraph{}
Figur 1. Zweierlei Entglasung in einer braunen Kugel des Meteoriten von Mez"o-Madaras. Einerseits erkennt man eine zarte exzentrisch-radiale Faserung, deren Strahlungspunkt im Bilde an der unteren Grenze der Kugel liegt, anderseits bemerkt man viele Mikrolithe in radial angeordneten Flocken, welche im Durchschnitte blumenartige oder sternf"ormige Zeichnungen liefern. Die Fasern sind ungemein fein, ihre Farbe ist die des braunen Glases im selben Chondriten. Am oberen Rande zeigt die Kugel eine tiefe Einbuchtung. Die Umgebung ist zum Teile dunkle Grundmasse mit Partikeln von Magnetkies und Eisen, zum Teile sind es Olivinchondren oder Splitter von solchen.

Figur 2. Eine Doppelkugel im Stein von Borkut. Der kleinere Teil ist eine monosomatische gef"acherte Olivinkugel mit tr"uber Zwischenmasse und blasser durchsichtiger Rinde; die gro"se Kugel, welche die kleinere zur H"alfte umschlie"st, ist von derselben Beschaffenheit und l"oscht gleichzeitig mit dieser aus. Da der Stein von Borkut beim Schleifen leicht zermahlen wild, so ist auch die Doppelkugel zum Teil aus ihrer Verbindung mit den Nachbarn gebracht.

Figur 3. Um die h"aufig vorkommende Umh"ullung der Chondren durch eine Eisenrinde in einem Beispiele darzustellen wurde ein Pr"aparat des Steines aus Cabarras City bei gleichzeitiger Wirkung des auffallenden und des durchfallenden Lichtes fotografiert. Das Eisen, welches bei der Betrachtung im durchgehenden Lichte schwarz erscheinen w"urde, zeichnet sich hier grau mit den Merkmalen der Rauigkeit auf der geschliffenen Fl"ache. Die Eisenrinde, welche die porphyrische Olivinkugel umgibt, ist von ungleicher Dicke und stellenweise schwammig.

Figur 4. Der Tuffcharakter, welcher in vielen Chondriten erkennbar ist, wird hier durch eine Probe aus dem Chondrit von Mez"o-Madaras illustriert. Das Bild zeigt durchweg Splitter von Chondren, verbunden durch eine sp"arliche dunkle Grundmasse. Oberhalb sieht man Teile von Olivinchondren, einen kleinen Splitter mit deutlichem Lamellenbau, in der Mitte das stumpfeckige Bruchst"uck einer porphyrischen Olivinkugel mit heller Glasmasse, rechts davon Olivinsplitter und Magnetkies, links ein Eisenkorn. Unterhalb hat man links die Splitter von einer radialfaserigen Bronzitkugel, rechts ein St"uck von einer porphyrischen Olivinkugel mit stark entglaster Zwischenmasse.
\clearpage

\rhead{Tafel 19.}
\vspace*{\fill}
\begin{figure}[H]
\centering
\includegraphics[width=\textwidth,keepaspectratio]{figs/19-1.png}
\caption{\small Figur 1 --- Bronzitkugel mit sternf"ormigen Mikrolithen. Chrondrit von Mez"o-Madaras. Vergr"o"serung 60.}
\end{figure}
\vspace*{\fill}
\clearpage

\rhead{Tafel 19.}
\vspace*{\fill}
\begin{figure}[H]
\centering
\includegraphics[width=\textwidth,keepaspectratio]{figs/19-2.png}
\caption{\small Figur 2 --- Olivin-Doppelkugel aus dem Chondrit von Borkut. Vergr"o"serung 160.}
\end{figure}
\vspace*{\fill}
\clearpage

\rhead{Tafel 19.}
\vspace*{\fill}
\begin{figure}[H]
\centering
\includegraphics[width=\textwidth,keepaspectratio]{figs/19-3.png}
\caption{\small Figur 3 --- Eisenh"ulle einer Olivinkugel im Chondrit von Cabarras City. Vergr"o"serung 60.}
\end{figure}
\vspace*{\fill}
\clearpage

\rhead{Tafel 19.}
\vspace*{\fill}
\begin{figure}[H]
\centering
\includegraphics[width=\textwidth,keepaspectratio]{figs/19-4.png}
\caption{\small Figur 4 --- Chondrensplitter im Chondrit von Mez"o-Madaras. Vergr"o"serung 90.}
\end{figure}
\vspace*{\fill}
\clearpage

\rhead{Tafel 20.}
\subsection{Erkl"arung der Tafel 20.}
\paragraph{}
Figur 1. Um zu zeigen, dass die kohligen Chondrite wie jener vom Kapland nur durch die Impr"agnation der Grundmasse von den "ubrigen verschieden sind, und Chondren derselben Art enthalten, wurde eine Stelle aus dem genannten Stein, welche eine porphyrische Olivinkugel mit dunkelbraunem Glase und ziemlich deutlichen Kristallen darbietet, zur Darstellung gebracht. Da die schwarze Grundmasse beim Schleifen leicht zerbr"ockelt, so ist diese im Pr"aparat vielfach zerrissen. Man sieht in derselben au"ser den Chondren auch viele kleine durchsichtige Olivinsplitter.

Figur 2. F"ur die kohligen Chondrite sind die feink"ornigen lappigen Bildungen charakteristisch, welche hier durch eine Probe aus dem Stein von Grosnaja repr"asentiert werden. Die Masse dieser Chondren erscheint im durchfallenden Lichte grau bis br"aunlich, von filzartiger Textur mit vielen tr"uben Flocken und schwarzen Punkten. Im polarisierten Lichte erkennt man, von den tr"uben Stellen abgesehen, eine innige Verbindung feiner doppelbrechender K"ornchen. Das Ganze hat "Ahnlichkeit mit der tr"uben entglasten Zwischenmasse vieler Chondren.

Figur 3. Als ein Beispiel von l"ocheriger Struktur mancher Chondrite ist hier eine Stelle in einem Pr"aparat aus dem Stein von Goalpara abgebildet. Die wei"sen Stellen in diesem Bilde sind nicht etwa durch Herausfallen einzelner Teilchen beim Schleifen bedingt, sondern r"uhren von urspr"unglich vorhandenen L"ochern her. Diese sind meist von schwarzer Grundmasse umgrenzt. Auf der rechten Seite des Bildes erkennt man zwei Kristallindividuen, deren eines rundliche Umrisse hat. Diese K"orner, welche als Enstatit bestimmt wurden, bedingen das porphyrische Gef"uge des Steines. In der Grundmasse heben sich viele kleine K"orner hervor, die als Olivin zu deuten sind. Die schwarze Masse besteht aus Eisen und einem matten K"orper, welcher als halbglasige kohligen Zwischenmasse die K"ornchen umgibt und in deren Spr"unge in feinen Ver"astelungen eindringt.

Figur 4. Die k"ornige Beschaffenheit der festen Chondrite ist hier durch ein dem Stein von Erxleben entnommenes Bild illustriert. In der Mitte oben zeichnet sich ein l"angliches Bronzitkorn mit kleinen durchsichtigen und gro"sen l"anglichen dunklen Glaseinschl"ussen. Darunter bemerkt man ein lichtes K"ornchen ohne Zeichnung, welches einem Plagioklas angeh"ort, da es im polarisierten Lichte eine deutliche Streifung zeigt. Im "Ubrigen hat man Olivink"orner, unten rechts ein solches mit horizontalen Streifen, welche von Lamellen und dem zwischenliegenden Glase herr"uhren. Die gro"sen schwarzen Flecken sind die Schattenbilder von Eisen, teilweise von Magnetkies.
\clearpage

\rhead{Tafel 20.}
\vspace*{\fill}
\begin{figure}[H]
\centering
\includegraphics[width=\textwidth,keepaspectratio]{figs/20-1.png}
\caption{\small Figur 1 --- Olivinkugel im dem kohligen Chondrit von Cold-Bokkeveld. Vergr"o"serung 160.}
\end{figure}
\vspace*{\fill}
\clearpage

\rhead{Tafel 20.}
\vspace*{\fill}
\begin{figure}[H]
\centering
\includegraphics[width=\textwidth,keepaspectratio]{figs/20-2.png}
\caption{\small Figur 2 --- Feink"ornige lappige Chondren im Chondrit von Grosnaja. Vergr"o"serung 60.}
\end{figure}
\vspace*{\fill}
\clearpage

\rhead{Tafel 20.}
\vspace*{\fill}
\begin{figure}[H]
\centering
\includegraphics[width=\textwidth,keepaspectratio]{figs/20-3.png}
\caption{\small Figur 3 --- L"ocheriger Chondrit von Goalpara. Vergr"o"serung 25.}
\end{figure}
\vspace*{\fill}
\clearpage

\rhead{Tafel 20.}
\vspace*{\fill}
\begin{figure}[H]
\centering
\includegraphics[width=\textwidth,keepaspectratio]{figs/20-4.png}
\caption{\small Figur 4 --- K"orniger Chondrit von Erxleben. Vergr"o"serung 60.}
\end{figure}
\vspace*{\fill}
\clearpage

\rhead{Tafel 21.}
\subsection{Erkl"arung der Tafel 21.}
\paragraph{}
Figur 1. Die drei Lagen, aus welchen die Rinde der Chondrite besteht, sind hier in einem Bilde aus dem Stein von Chateau Renard ersichtlich, welches einen Vertikalschnitt darstellt. Da die Rinde als "au"serster und dabei spr"odester Teil ungemein leicht abbr"ockelt, so ist derselbe im Pr"aparat nicht vollkommen erhalten. Links oberhalb erscheinen aber alle drei Zonen: zuerst eine d"unne dunkle aus schwarzem bis braunem Glase bestehende "au"sere Schmelzrinde, darunter eine durchsichtige, hier aus Olivin und Maskelynit bestehende Saugzone, zu unterst die m"achtige Impr"agnationszone, welche dicker ist als die beiden vorigen zusammen. Letztere ist stellenweise von hellen Punkten durchsprenkelt. Unterhalb zeigt sich das unver"anderte Gemenge, zumeist aus Olivink"ornern mit untergeordnetem farblosem Maskelynit, links ein Eisenkorn, unten eine radialfaserige Bronzitkugel.

Figur 2. Hier ist die Rinde vollkommen erhalten. Zu oberste erscheint die schwarze "au"sere Schmelzrinde, welche in den "au"sersten Teilen auch etwas von farblosem Glase (Maskelynit) erkennen l"asst. Solche Stellen sind auf der "au"seren Rinde im auffallenden Lichte glasig und sehen wie gefirnisst aus. Die zweite oder Saugzone ist wiederum hell, stellenweise von schwarzem Glase durchsetzt. Die dritte oder Impr"agnationszone ist wiederum schwarz, stellenweise fein durchsprenkelt, rechts gr"o"sere durchsichtige Olivink"orner enthaltend. Im unteren Teile des Bildes hat man die Darstellung des unver"anderten Gemenges, welches hier aus Olivink"ornern, aus wenigen Bronzit- und Plagioklask"ornern und aus gro"sen Partikeln von Magnetkies besteht.

Figur 3. An demselben Meteoriten von Mocs gelang ein Schnitt, welcher der Rinde ungef"ahr parallel gef"uhrt wurde. Da die Rinde krumm ist, so blieb derselbe nicht an allen Stellen in derselben Rindenschichte, sondern er geht gleichzeitig durch die Saugzone, welche durch die hellen Teile bezeichnet ist, und durch die Impr"agnationszone, welche durchsprenkelt erscheint. In der Saugzone und in den hellen Partikeln der Impr"agnationszone, l"asst sich vorzugweise Olivin und nur hie und da ein Bronzitkorn erkennen. Die farblosen K"ornchen und lappigen H"aufchen, welche oft Olivink"ornchen einschlie"sen und vollkommen dem Plagioklas gleichen, der in den "ubrigen Teilen des Steines den farblosen Gemengteil bildet, sind hier isotrop (Maskelynit).

Figur 4. Da ein Vertikalschnitt durch die Rinde eines Eukrits nicht gelang, so begn"uge ich mich hier den Parallelschnitt durch die schwarze glasgl"anzenden Rinde des Steines von Juvinas darzustellen. Die Hauptmasse ist ein dunkelbraunes Glas mit vielen runden Vertiefungen und mit geschlossenen Blasen. Im Glase sind hie und da farblose K"orner und Kristalle von Anorthit eingeschlossen, welche im polarisierten Lichte das gew"ohnliche Verhalten zeigen. Ein Beispiel ist links am Rande des Bildes zu sehen, in der Mitte aber zeichnen sich feink"ornige doppelbrechende Partikel. Augit ist nicht zu bemerken.
\clearpage

\rhead{Tafel 21.}
\vspace*{\fill}
\begin{figure}[H]
\centering
\includegraphics[width=\textwidth,keepaspectratio]{figs/21-1.png}
\caption{\small Figur 1 --- Vertikalschnitt durch die Rinde des Chondrits von Chateau Renard. Vergr"o"serung 70.}
\end{figure}
\vspace*{\fill}
\clearpage

\rhead{Tafel 21.}
\vspace*{\fill}
\begin{figure}[H]
\centering
\includegraphics[width=\textwidth,keepaspectratio]{figs/21-2.png}
\caption{\small Figur 2 --- Rinde des Chondrits von Mocs im Vertikalschnitt. Vergr"o"serung 70.}
\end{figure}
\vspace*{\fill}
\clearpage

\rhead{Tafel 21.}
\vspace*{\fill}
\begin{figure}[H]
\centering
\includegraphics[width=\textwidth,keepaspectratio]{figs/21-3.png}
\caption{\small Figur 3 --- Rinde des Chondrits von Mocs, ungef"ahr parallel durchschnitten. Vergr"o"serung 70.}
\end{figure}
\vspace*{\fill}
\clearpage

\rhead{Tafel 21.}
\vspace*{\fill}
\begin{figure}[H]
\centering
\includegraphics[width=\textwidth,keepaspectratio]{figs/21-4.png}
\caption{\small Figur 4 --- Parallelschnitt durch die Rinde des Eukrits von Juvinas. Vergr"o"serung 160.}
\end{figure}
\vspace*{\fill}
\clearpage

\rhead{Tafel 22.}
\subsection{Erkl"arung der Tafel 22.}
\paragraph{}
Figur 1. Ein Beispiel f"ur die in den Chondriten h"aufig vorkommenden Kl"ufte, welche sich im Querschnitte bald als feine bald als breite Adern darstellen. Die Grundmasse, welche hier vorwiegend K"orner von Olivin, eine geringe Menge von Bronzitk"ornern, H"aufchen von Plagioklas und K"orner von Magnetkies enth"alt, wird von zwei Systemen von Spr"ungen durchsetzt, welche mit einer schwarzen glanzlosen Masse erf"ullt sind. Die Spr"unge schmiegen sich, wie dieser Fall erkennen l"asst, gerne an die Magnetkiesk"orner an.

Figur 2. Von den breiten schwarzen gang"ahnlichen Massen, welche in Verbindung mit den vorbezeichneten Kl"uften in manchen Exemplaren des Meteoriten von Mocs beobachtet werden, ist hier eine gew"ahlt, welche die Verh"altnisse gut erkennen l"asst. Der dunkle Streifen, welcher die Grundmasse durchzieht, ist ein Querschnitt des Ganges. Derselbe zeigt nur stellenweise eine sch"arfere Abgrenzung gegen die Grundmasse, meist jedoch einen allm"ahligen "Ubergang in dieselbe. In der schwarzen Masse liegen Eisentropfen und l"angliche Eisenkl"umpchen. Da die Aufnahme des Bildes zugleich im durchgehenden und im auffallenden Lichte ausgef"uhrt wurde, so erscheint das Eisen im Bilde nicht schwarz sondern grau. In der Mitte der gangf"ormigen Masse bemerkt man nach der L"ange derselben gestreckte Eisenf"aden, ferner feine Querspr"unge mit Eisen erf"ullt. Besonders deutlich sind diese Eisenkl"ufte im unteren Teile des Bildes, wo dieselben mit einem l"anglichen Eisenkl"umpchen in Verbindung stehen. Eine solche nach rechts verlaufende Kluft h"angt mit einem offenen Sprung zusammen. Die schwarze Masse ist von hellen Olivinpartikeln durchsprenkelt.

Figur 3. Der klastische Charakter vieler Stellen in dem Meteoriten von der Sierra de Chaco zeigt sich hier in einem Bilde, welches gro"se und kleine Splitter von Plagioklas nebst Splittern von Bronzit und Olivin durch eine Eisengrundmasse verbunden darstellt. Der Plagioklas l"asst schon im gew"ohnlichen Lichte die Zwillingsbildung erkennen. Der Bronzit charakterisiert sich durch feine Spaltrisse, der Olivin ist mit dunklen Einschl"ussen erf"ullt.

Figur 4. Hier ist jener Plagioklas des Meteoriten von der Sierra de Chaco, welcher durch gro"se durchsichtige kristallisierte Einschl"usse merkw"urdig erscheint, dargestellt. Wie das Bild zeigt, sind die Einschl"usse vorzugsweise im Inneren der Kristalle angeh"auft. Die Form derselben ist bald kurz-s"aulenf"ormig, bald ist der Umriss dreiseitig, fast quadratisch, f"unfseitig oder rundlich. Manche sind von dreiseitigen Fl"achen eingeschlossen und haben ungef"ahr monokline Gestalt. Einige liegen mit ihren L"angsaxen den Plagioklaslamellen parallel, andere nicht. Der Plagioklas zeigt au"serdem einige schwarze opake Einschl"usse und ist von einem dunklen von Einschl"ussen stark durch setzten Bronzit und von Augit umgeben.
\clearpage

\rhead{Tafel 22.}
\vspace*{\fill}
\begin{figure}[H]
\centering
\includegraphics[width=\textwidth,keepaspectratio]{figs/22-1.png}
\caption{\small Figur 1 --- Netzartig verbundene Kl"ufte im Chondrit von Mocs. Vergr"o"serung 60.}
\end{figure}
\vspace*{\fill}
\clearpage

\rhead{Tafel 22.}
\vspace*{\fill}
\begin{figure}[H]
\centering
\includegraphics[width=\textwidth,keepaspectratio]{figs/22-2.png}
\caption{\small Figur 2 --- Gangf"ormige Masse im Chondrit von Mocs. Vergr"o"serung 20.}
\end{figure}
\vspace*{\fill}
\clearpage

\rhead{Tafel 22.}
\vspace*{\fill}
\begin{figure}[H]
\centering
\includegraphics[width=\textwidth,keepaspectratio]{figs/22-3.png}
\caption{\small Figur 3 --- Grahamit von der Sierra de Chaco. Splitter von Plagioklas und Bronzit. Vergr"o"serung 30.}
\end{figure}
\vspace*{\fill}
\clearpage

\rhead{Tafel 22.}
\vspace*{\fill}
\begin{figure}[H]
\centering
\includegraphics[width=\textwidth,keepaspectratio]{figs/22-4.png}
\caption{\small Figur 4 --- Plagioklas mit Einschl"ussen. Grahamit von der Sierra de Chaco. Vergr"o"serung 60.}
\end{figure}
\vspace*{\fill} 
\clearpage

\rhead{Tafel 23.}
\subsection{Erkl"arung der Tafel 23.}
\paragraph{}
Figur 1. Der Bronzit in der Masse von der Sierra de Chaco zeigt im L"angsschnitte ein feinfaseriges Ansehen, wie es in diesem Bilde deutlich wird. Der Schnitt ist beil"aufig parallel 0 1 0. Am oberen Ende sind Spuren einer Kristallendigung wahrzunehmen. Im Inneren bemerkt man einige feine L"angslinien, welche Querschnitten d"unner Bl"attchen entsprechen. Die Ausl"oschungsschiefe derselben von ungef"ahr 50° deutet auf einen Augit. Im "Ubrigen zeigen sich opake Einschl"usse von denen die gr"o"seren im auffallenden Lichte als Magnetkies erkannt werden. Einige haben ziemlich scharfe Umrisse wie negative Kristalle, andere sind unregelm"a"sig, zum Teil staubartig. Ein quer verlaufender Sprang ist mit Magnetkiesk"ornchen besetzt. Oberhalb ist der Bronzit von Eisen eingefasst, unten von Magnetkies, Eisen und Plagioklas begrenzt.

Figur 2. Der im selben Meteoriten stellenweise vorkommende Augit erscheint hier in einem Querschnitte der von vielen Spaltrissen durchzogen ist und dessen untere Grenze der Querfl"ache 1 0 0 entspricht. Der Schnitt liefert das Bild einer optischen Axe sehr sch"on. Einschl"usse sind sparsam. Von der faserigen Beschaffenheit des Bronzits ist hier nichts zu sehen. Oberhalb grenzt der Augit an feink"ornigen Plagioklas, welcher viele kleine kristallinische Einschl"usse enth"alt. Im "ubrigen besteht die Nachbarschaft aus Eisen, Magnetkies, Olivin, Bronzit.

Figur 3. Von den Olivink"ornern im Grahamit ist hier eines zur Darstellung gebracht, welches eine doppelte Rinde besitzt. Die innere ist eine schwarze Impr"agnation, die "au"sere eine feink"ornige fast tr"ube Masse. Dies erinnert an manche Olivinchondren. Das innere Olivinkorn ist ein Individuum. Im Bilde zeigt dieses nur grobe Spr"unge, doch lassen sich bei st"arkerer Vergr"o"serung auch feine dunkle Striche wahrnehmen, die sich an den breiten Sprung horizontal ansetzen. Links vom Olivin hat man Splitter von Plagioklas, darunter einen, der von Einschl"ussen ganz erf"ullt ist.

Figur 4. Dieses Bild gibt eine Vorstellung von den feinen Linien, welche im Olivin der Grahamit und Mesosiderite "ofters beobachtet werden. Von der Rinde und von den Spr"ungen des Olivins her erstrecken sich zahlreiche feine Nadeln, welche teils grau teils braun erscheinen, in zwei aufeinander senkrechten Richtungen in das Innere des Kristalls. In den breiten Spr"ungen erkennt man bei auffallendem Lichte stellenweise Magnetkies. Die feinen grauen Nadeln erweisen sich bei st"arkerer Vergr"o"serung als Kan"ale. Die braunen d"urften ihre F"ullung einer Oxydation des Magnetkieses verdanken. Im auffallenden Lichte zeigt das Olivinkorn einen bl"aulichen Schiller wie der Mondstein, besonders dort wo die grauen Nadeln h"aufiger sind. Auch dieses Korn ist von dichter Masse umgeben, welche gleichfalls Olivin zu sein scheint und mit K"ornchen von Magnetkies gemengt ist. Fr"uher habe ich diese Olivink"orner mit feinen Nadeln nicht als solche erkannt, sondern dieselben als ein anderes dem Cordierit "ahnliches Silikat betrachtet (Sitzungsber. d. Wiener Akad. Bd. 88 Abt. 1. pag. 353).
\clearpage

\rhead{Tafel 23.}
\vspace*{\fill}
\begin{figure}[H]
\centering
\includegraphics[width=\textwidth,keepaspectratio]{figs/23-1.png}
\caption{\small Figur 1 --- Bronzit im Grahamit von der Sierra de Chaco. Vergr"o"serung 50.}
\end{figure}
\vspace*{\fill}
\clearpage

\rhead{Tafel 23.}
\vspace*{\fill}
\begin{figure}[H]
\centering
\includegraphics[width=\textwidth,keepaspectratio]{figs/23-2.png}
\caption{\small Figur 2 --- Augit und dichter Plagioklas in demselben Grahamit. Vergr"o"serung 70.}
\end{figure}
\vspace*{\fill}
\clearpage

\rhead{Tafel 23.}
\vspace*{\fill}
\begin{figure}[H]
\centering
\includegraphics[width=\textwidth,keepaspectratio]{figs/23-3.png}
\caption{\small Figur 3 --- Olivin mit feink"orniger Rinde neben Plagioklas in demselben Grahamit. Vergr"o"serung 50.}
\end{figure}
\vspace*{\fill}
\clearpage

\rhead{Tafel 23.}
\vspace*{\fill}
\begin{figure}[H]
\centering
\includegraphics[width=\textwidth,keepaspectratio]{figs/23-4.png}
\caption{\small Figur 4 --- Olivin mit Gitterzeichnung in demselben Grahamit. Vergr"o"serung 70.}
\end{figure}
\vspace*{\fill}
\clearpage

\rhead{Tafel 24.}
\subsection{Erkl"arung der Tafel 24.}
\paragraph{}
Figur 1. Das k"ornige Gemenge, welches die Silikatmasse des Mesosiderits von Estherville bildet, enth"alt an der hier abgebildeten Stelle mehrere gr"o"sere K"orner von Plagioklas, deren eines ebensolche kristallisierte Einschl"usse zeigt, wie der Plagioklas in der Masse von der Sierra de Chaco, w"ahrend der langgestreckte Plagioklas im oberen Teile des Bildes durch staubartig feine Glaseier und K"ornchen getr"ubt ist. Die Grundmasse zeigt au"ser Eisen und Magnetkies noch Bronzit mit feinen Spaltrissen und Olivin mit vielen dunkelbraunen Glaseinschl"ussen.

Figur 2. Ansicht eines Schnittes durch ein K"ornchen von Peckhamit parallel einer Spaltungsfl"ache. Das Bl"attchen ist graugr"unlich, ziemlich tr"ube. Die gr"o"seren Einschl"usse erscheinen als untereinander parallele st"abchenf"ormige oder spindelf"ormige Hohlr"aume die "ofter teilweise mit Glas gef"ullt sind, ferner als schwarze Kugeln, welche Glasf"ullungen zu enthalten scheinen, die kleinen Einschl"usse als rundliche oder kurzs"aulenf"ormige braune Glaspartikel, endlich an die letzteren anschlie"send als feiner Staub, mit welchem die ganze Masse durchs"aet ist.

Figur 3. Dieses Bild gibt einen Durchschnitt jenes Olivins in der Masse von Hainholz, der an Einschl"ussen besonders reich ist, wieder. Der Olivin ist farblos, von Spr"ungen durchzogen, welche mit rotbrauner, bei der Verwitterung entstandener Masse gef"ullt sind. Die Einschl"usse sind rund oder eckig, dabei immer opak. Viele derselben zeigen im auffallenden Lichte die Farbe des Magnetkieses, andere sind schwarz und scheinen Glaseinschl"usse zu sein.

Figur 4. Eine Stelle in der Grundmasse des Mesosiderits von Hainholz, die frei von Eisen ist und wenig von den Oxydationsprodukten enth"alt, welche in dieser schon etwas verwitterten Masse verbreitet erscheinen. Dieselbe ist hier dargestellt um die feink"ornige Textur und das feste Gef"uge zur Anschauung zu bringen. Die Kristalle und K"orner sind Bronzit mit meistens deutlichen Spaltrissen, Augit von dem vorigen durch die schiefe Ausl"oschung unterscheidbar, Plagioklas in farblosen K"ornern und Leistchen mit deutlicher Zwillingsbildung. Zwischen die K"orner und Kristalle klemmt sich etwas von br"aunlichem Glase ein, worin stellenweise feine Nadeln erkennbar sind.
\clearpage

\rhead{Tafel 24.}
\vspace*{\fill}
\begin{figure}[H]
\centering
\includegraphics[width=\textwidth,keepaspectratio]{figs/24-1.png}
\caption{\small Figur 1 --- Plagioklasreiche Stelle in dem Mesosiderit von Estherville. Vergr"o"serung 70.}
\end{figure}
\vspace*{\fill}
\clearpage

\rhead{Tafel 24.}
\vspace*{\fill}
\begin{figure}[H]
\centering
\includegraphics[width=\textwidth,keepaspectratio]{figs/24-2.png}
\caption{\small Figur 2 --- Pekhamit, bronzit"ahnlich, aus dem Mesosiderit von Estherville. Vergr"o"serung 120.}
\end{figure}
\vspace*{\fill}
\clearpage

\rhead{Tafel 24.}
\vspace*{\fill}
\begin{figure}[H]
\centering
\includegraphics[width=\textwidth,keepaspectratio]{figs/24-3.png}
\caption{\small Figur 3 --- Olivin mit Einschl"ussen im Mesosiderit von Hainholz. Vergr"o"serung 120.}
\end{figure}
\vspace*{\fill}
\clearpage

\rhead{Tafel 24.}
\vspace*{\fill}
\begin{figure}[H]
\centering
\includegraphics[width=\textwidth,keepaspectratio]{figs/24-4.png}
\caption{\small Figur 4 --- Feink"ornige Grundmasse des Mesosiderits von Hainholz. Vergr"o"serung 120.}
\end{figure}
\vspace*{\fill}
\clearpage

\rhead{Tafel 25.}
\subsection{Erkl"arung der Tafel 25.}
\paragraph{}
Figur 1. Eines der k"ornigen H"aufchen, welche der Tridymit in der Masse von Rittersgr"un bildet, im Durchschnitte. Die Aufnahme geschah im polarisierten Lichte. Man erkennt an den hellgrauen T"onen die verschiedene Stellung der zwillingsgem"ass verbundenen Individuen. Diese sind "ofters ziemlich breit, zuweilen auch sehr schmal. Stellenweise sind winzige Individuen in Zwillingsstellang von gr"o"seren umschlossen. Der k"ornige Tridymit ist von breiten und schm"aleren Spr"ungen durchzogen, welche mit rotbrauner von einer Oxydation herr"uhrender Masse erf"ullt sind. Rechts grenzt Bronzit an.

Figur 2. Nach Entfernung des Eisens durch Salzs"aure aus einer Probe der Masse von Rittersgr"un wurde ein Bl"attchen von Tridymit isoliert, welches im polarisierten Lichte aufgenommen ist. Die Hauptschnitte des Nicols sind horizontal und vertikal zu denken. Das Bl"attchen zeigt oberhalb zwei unvollkommene Randkanten und ist von einem ziemlich ebenen Fl"achenpaar begrenzt. Als Hauptindividuum erscheint der mittlere helle Teil von sechsseitigem Umrisse. Derselbe hat rechts einen hakenf"ormigen Fortsatz, dessen Form durch dunkle Zwickel bedingt ist, welche einem Individuum von anderer Stellung angeh"oren. Auch der linke Teil des Bl"attchens hat die letztere Stellung mit Ausnahme eines dreieckigen Feldes, welches sich an das Hauptindividuum anschlie"st und einen Mittelton zeigt.

Figur 3. Der Bronzit im Rittersgr"uner Meteoriten ist hier durch einen Schnitt, welcher durch ein Korn mit stellenweise ebenfl"achiger Begrenzung hindurchgeht, repr"asentiert. Die schelferige Schnittfl"ache, die gr"oberen und feineren Spaltlinien, die querlaufenden Spr"unge treten im Bilde deutlich hervor. Die schwarzen Einschl"usse sind Magnetkies (Troilit), der braune rundliche Einschluss oberhalb ist doppelbrechend, ebenso der durchsichtige Teil des Einschlusses rechts unten. Es d"urfte Olivin sein.

Figur 4. Die linke H"alfte des Bildes zeigt einige von G. Rose als R"ohren bezeichnete Kan"ale im Olivin des Brahiner Pallasits. Die gr"o"ste R"ohre erscheint am oberen Ende durch braunes Glas gef"ullt. In der rechten H"alfte, welche dem Olivin aus der Masse von Krasnojarsk entspricht, hat man eine gro"se vollst"andig mit braunem Glas gef"ullte und eine k"urzere farblos erscheinende R"ohre, welche keine solche F"ullung zeigt, jedoch ein fast farbloses Glas zu enthalten scheint.
\clearpage

\rhead{Tafel 25.}
\vspace*{\fill}
\begin{figure}[H]
\centering
\includegraphics[width=\textwidth,keepaspectratio]{figs/25-1.png}
\caption{\small Figur 1 --- K"orniger Tridymit im Siderophyr von Rittersgr"un im polaris. Lichte. Vergr"o"serung 60.}
\end{figure}
\vspace*{\fill}
\clearpage

\rhead{Tafel 25.}
\vspace*{\fill}
\begin{figure}[H]
\centering
\includegraphics[width=\textwidth,keepaspectratio]{figs/25-2.png}
\caption{\small Figur 2 --- Tridymitbl"attchen aus demselben Siderophyr im polaris. Lichte. Vergr"o"serung 100.}
\end{figure}
\vspace*{\fill}
\clearpage

\rhead{Tafel 25.}
\vspace*{\fill}
\begin{figure}[H]
\centering
\includegraphics[width=\textwidth,keepaspectratio]{figs/25-3.png}
\caption{\small Figur 3 --- Bronzit in demselben Siderophyr. Vergr"o"serung 60.}
\end{figure}
\vspace*{\fill}
\clearpage

\rhead{Tafel 25.}
\vspace*{\fill}
\begin{figure}[H]
\centering
\includegraphics[width=\textwidth,keepaspectratio]{figs/25-4.png}
\caption{\small Figur 4 --- Olivin mit feinen R"ohren im Pallasit von Brahin (links) Vergr"o"serung 300. und im P. von Krasnojarsk (rechts) Vergr"o"serung 200.}
\end{figure}
\vspace*{\fill}
\clearpage
\end{document}
