\documentclass[a4paper, 11pt, oneside]{article}
\usepackage[utf8]{inputenc}
\usepackage[T1]{fontenc}
\usepackage[ngerman]{babel}
\usepackage{fbb} %Derived from Cardo, provides a Bembo-like font family in otf and pfb format plus LaTeX font support files
\usepackage{booktabs}
\setlength{\emergencystretch}{15pt}
\usepackage{fancyhdr}
\usepackage{graphicx}
\usepackage{microtype}
\graphicspath{ {./} }
\begin{document}
\begin{titlepage} % Suppresses headers and footers on the title page
	\centering % Centre everything on the title page
	%\scshape % Use small caps for all text on the title page

	%------------------------------------------------
	%	Title
	%------------------------------------------------
	
	\rule{\textwidth}{1.6pt}\vspace*{-\baselineskip}\vspace*{2pt} % Thick horizontal rule
	\rule{\textwidth}{0.4pt} % Thin horizontal rule
	
	\vspace{1.5\baselineskip} % Whitespace above the title
	
	{\scshape\LARGE Die Tr"ummerstruktur der Meteoriten}
	
	\vspace{1\baselineskip} % Whitespace after the title block

	{\scshape\LARGE von Orvinio und Chantonnay.}

	\vspace{1.5\baselineskip} % Whitespace above the title

	\rule{\textwidth}{0.4pt}\vspace*{-\baselineskip}\vspace{3.2pt} % Thin horizontal rule
	\rule{\textwidth}{1.6pt} % Thick horizontal rule
	
	\vspace{1\baselineskip} % Whitespace after the title block
	
	%------------------------------------------------
	%	Subtitle
	%------------------------------------------------
	
	{\scshape Von G. Tschermak, korrespondierendem Mitglied der kaiserlichen Akademie der Wissenschaften.} % Subtitle or further description
	
	\vspace*{1\baselineskip} % Whitespace under the subtitle
	
    {\scshape\small Vorgelegt in der Sitzung am 19. November 1874.\\Mit 2 Tafeln.} % Subtitle or further description
    
	%------------------------------------------------
	%	Editor(s)
	%------------------------------------------------
    \vspace*{\fill}

	\vspace{1\baselineskip}

	{\small\scshape 1874.}
	
	\vspace{0.5\baselineskip} % Whitespace after the title block

    \scshape Internet Archive Online Edition  % Publication year
	
	{\scshape\small Namensnennung Nicht-kommerziell Weitergabe unter gleichen Bedingungen 4.0 International} % Publisher
\end{titlepage}
\setlength{\parskip}{1mm plus1mm minus1mm}
\clearpage
\tableofcontents
\clearpage
\section{Orvinio.}
\paragraph{}
Am 31. August 1872 ereignete sich um 5 1/4, Uhr Morgens bei Orvinio in der r"omischen Provinz ein Meteoritenfall, welcher mehrere Steine lieferte. "uber die n"aheren Umst"ande und die beobachteten Erscheinungen berichtete Ph. Keller.\footnote{Poggendorff's Ann. Bd. 250, pag. 171 und ein nachtr"aglicher Bericht in den Mineralog. Mitteilungen. 1874, pag. 258.} "uber die Bahn der Feuerkugel existiert eine Mitteilung von G. S. Ferrari,\footnote{Ricerche fisico-astronomiche intorno all' uranolito caduto nell agro Romano il 31. di Agosto 1872. Roma 1873.} sowie von M. S. Rossi,\footnote{Studj sull uranolito, caduto nell’ agro Romano ecc. Roma 1973. Abdruck aus den Atti dell’ Accademia pontif. de’nuovi Lincei 1873.} welcher auch seine an den Steinen gemachten Wahrnehmungen beschrieb und darauf bez"ugliche Abbildungen ver"offentlichte.

Aus der Abhandlung Kellers wiederhole ich hier blo"s, dass im Ganzen sechs Steine gefunden wurden, welche zusammen "uber 3 Kilogramm wogen und deren schwerster ein Gewicht von 1.242 Kil. besa"s. Alle zeigten eine schwarze Kruste und im Inneren eine ungleichf"ormige von Spr"ungen durchzogene Masse.

W"ahrend meiner Anwesenheit in Rom im Fr"uhling des l. J. erhielt ich durch die G"ute des Herrn Ph. Keller einen vollst"andigen Stein von Orvinio, der nunmehr in der Sammlung des k. k. Hof-Museums aufbewahrt wird und der mir die folgenden Beobachtungen erm"oglichte. Es gereicht mir zum gr"o"sten Vergn"ugen, hier die Gelegenheit zu finden, Herrn Keller f"ur dieses wertvolle Geschenk meinen innigen Dank aussprechen zu k"onnen.

Der genannte Stein besitzt eine schwarze, d"unne, runzelige Rinde, welche an manchen Stellen fehlt, teils urspr"unglich, teils deshalb, weil sich beim Niederfallen Partikelchen von der spr"oden Masse abl"osten. Die Gestalt des Steines ist knollenf"ormig. Auf Taf. I Fig. 1 ist dieselbe in der halben Gr"o"se wiedergegeben und so gestellt, dass die allerdings etwas schwierig erkennbare Brustseite links und die R"uckenseite rechts zu liegen kommt. An der Begrenzung dieser beiden Teile des Steines bemerkt man eine schwache Randbildung derselben Art, wie sie bei jenen Steinen, welche aus schwerschmelzbaren Silicaten zusammengesetzt sind, "ofters beobachtet wird. Die Oberfl"ache tr"agt stellenweise tiefe Gruben und nirgends scharfe Kanten.

Durch den Stein wurde in der, in der Figur durch eine punktierte Linie angedeuteten Richtung ein Schnitt gef"uhrt. Die Struktur, welche dadurch enth"ullt wurde, ist eine ungew"ohnliche und merkw"urdige. Man erkennt n"amlich, dass der Stein aus hellgef"arbten Bruchst"ucken besteht, welche von einer dichten dunklen Bindemasse umgeben sind. Die Fig. 2 auf Taf. I ist ein Bild des Schnittes in nat"urlicher Gr"o"se.

Die Bruchst"ucke sind gelblichgrau, enthalten K"ugelchen und Partikelchen von Eisen und Magnetkies, sie sind also normaler Chondrit und besitzen in ihrem Gef"uge "ahnlichkeit mit der Masse des Steines von Seres in Makedonien.

Die Bindemasse ist schw"arzlich, dicht und splittrig. Sie enth"alt kleine Partikelchen vor Eisen und Magnetkies, welche meist gleichf"ormig eingestreut sind, an der Grenze gegen die Bruchst"ucke aber so angeordnet erscheinen, dass im Durchschnitte eine sehr deutliche Fluidaltextur* sichtbar wird. (Fig. 2.) Die Wahrnehmung macht es wohl im hohen Grade wahrscheinlich, dass die Bindemasse sich im einst plastischen Zustande und in Bewegung befand.

Die spr"ode Bindemasse hat hie und da feine Spr"unge, welche sich zuweilen durch die eingeschlossenen Bruchst"ucke fortsetzen. An den Grenzen der Bruchst"ucke und der Bindemasse erscheinen zuweilen schmale, offene Spr"unge, in denen das Nickeleisen in zarten gestrickt-blechf"ormigen Gestalten frei auskristallisiert erscheint. Die Bruchst"ucke sind an der Rinde, also an der Ber"uhrungsstelle mit der Bindemasse dunkler, h"arter und spr"oder als in der Mitte. Die letzteren Beobachtungen sprechen daf"ur, dass der plastische Zustand der Bindemasse von einer sehr hohen Temperatur begleitet war.

Die beiden Bestandteile, die Bindemasse und die Bruchst"ucke haben, wie sp"ater gezeigt wird, fast dieselbe chemische Zusammensetzung, fast das gleiche Volumgewicht, und so viel sich ermitteln l"asst, auch denselben mineralogischen Bestand. Demnach l"asst sich der Meteorit von Orvinio mit einer bestimmten Art tellurischer Gesteine vergleichen und zwar mit einer Breccie vulkanischen Gesteins, welche aus einer dichten Grundmasse und aus k"ornigen Tr"ummern desselben Gesteins zusammengesetzt ist. Bekanntlich sieht man derlei Breccien an Vulkanen und "uberhaupt im Bereiche der eruptiven Felsarten h"aufig. Sie bilden sich dadurch, dass "altere kristallinische Laven von einer j"ungeren dichteren durchbrochen werden.

Ich gehe nun zu einer genaueren Beschreibung der Bestandteile "uber.

Die hellen Bruchst"ucke in dem Meteoriten von Orvinio bestehen aus Chondrit. Die Chondrit sind mehr oder weniger tuf"ahnliche Massen, bestehend aus Gesteinsk"ugelchen und einer pulvrigen oder dichten gleich zusammengesetzten Grundmasse. So ist es auch hier. Ein D"unnschliff, welcher aus einen solchen Bruchst"ucke gewonnen wurde, zeigt K"ugelchen, welche meist aus einem, seltener aus mehreren Mineralen bestehen, und welche in einer aus Splittern derselben Minerale bestehenden Masse liegen, die auch dunkle Partikelchen von Nickeleisen und Magnetkies enth"alt. Fig. 3 auf Taf. I.

Unter den durchsichtigen Mineralen unterscheidet man eines, das nur unvollkommene Spaltbarkeit zeigt und in K"ornchen vorkommt, ziemlich leicht von den anderen. Nach den genannten Kennzeichen und den Daten der Analyse ist es f"ur Olivin zu halten. Das andere Mineral, welches in S"aulchen von deutlich erkennbarer Spaltbarkeit nach einem Prisma von fast quadratischem Querschnitte, ferner nach der Quer- und der L"angsfl"ache vorkommt, ist als Bronzit zu erkl"aren. Ein drittes, welches in feinbl"atterigen oder feinfaserigen Partikeln auftritt, k"onnte mit dem vorigen identisch sein, d"urfte aber, da die Analyse auf einen feldspatartigen Gemengteil hinweist, f"ur diesen zu halten sein. Eine Erscheinung, die an manchen Chondriten, z.B. Pultusk, Alessandria, Chateau Renard, auftritt, findet sich auch hier: an manchen Bruchst"ucken sind schwarze Spiegelfl"achen mit parallelen Streifen zu beobachten.

Manches, was hier bez"uglich der chondritischen Masse ferner zu sagen w"are, habe ich schon bei einer fr"uheren Gelegenheit, als ich den Meteorit von Gopalpur beschrieb,\footnote{Diese Berichte Band LXV. Abt. I. pag. 122. Die beigegebene Tafel enth"alt Abbildungen eines D"unnschliffes und verschiedener K"ugelchen. Ein Auszug der Abhandlung in den Mineralog. Mitteil. 1872, pag. 95.} ausgesprochen. Ich wiederhole hier nur das Eine, dass ich die Chondrit f"ur Zierreibungs-Tuffe, und die K"ugelchen derselben f"ur solche Gesteinspartikelchen halte, welche wegen ihrer Z"ahigkeit bei dem Zerreiben des Gesteines nicht in Splitter aufgel"ost, sondern abgerundet wurden.

Die Bruchst"ucke in dem hier behandelten Meteoriten haben eine dunklere, h"artere Rinde. Die mikroskopische Untersuchung zeigt, dass das Gestein hier von einer schwarzen Masse impr"agniert ist, welche mit der sogleich zu besprechenden Bindemasse zusammenh"angt. Diese schwarze Masse dringt in alle feinen Fugen zwischen den Mineralpartikelchen und auch in die Spaltungsfugen ein, sodass die Rinde der Bruchst"ucke an Durchsichtigkeit sehr einb"u"st. (Fig. 4 auf Tafel I.) Da die schwarze Masse halbglasig und hart ist, wird die Ver"anderung der Rinde erkl"arlich. Eine solche Impr"agnation, wie sie hier beobachtet wurde, zeigt auch der Chondrit von Tadjera, welcher "au"serlich schw"arzlich und halbglasig erscheint, und "ahnliches Aussehen zeigen im D"unnschliffe solche Meteoriten, welche stark erhitzt wurden, wobei der Magnetkies fl"ussig gemacht, in die feinen Fugen eingedrungen ist.*

Die schw"arzliche Bindemasse besteht aus zwei Teilen, n"amlich aus einem auch im D"unnschliffe undurchsichtigen halbglasigen Teile und aus Partikeln, welche genau so aussehen wie Teilchen der dunklen Rinde der Bruchst"ucke. Da in der N"ahe der gro"sen Bruchst"ucke "ofter derlei Partikel wahrnehmbar sind, welche genau an die Kontur der Bruchst"ucke passen, so kann man alle diese Partikel kaum f"ur etwas anderes als f"ur Splitter halten, die von den gro"sen Bruchst"ucken sich abgel"ost und mit der Bindemasse vermischt haben. Viele der Splitter sind noch als Olivin und Bronzit zu erkennen. Die Menge der eigentlichen Bindemasse ist sonach bedeutend geringer, als es f"ur den ersten Anblick scheint. Da sie beinahe opak ist, war mir eine mikroskopische Unterscheidung der enthaltenen Silikate nicht m"oglich, dagegen lassen sich die metallischen Beimengungen im auffallenden Lichte erkennen. Die Partikel des Nickeleisens und des Magnetkieses sind hier durchschnittlich viel kleiner als in den Bruchst"ucken. In der homogenen schwarzen Masse erscheinen diese Partikel rundlich, gegen die Bruchst"ucke zu aber flaserig angeordnet, daher die Fluidaltextur. Bei der Impr"agnation der gro"sen Bruchst"ucke und der kleinen Splitter treten diese beiden Gemengteile h"aufig als feine Adern hinein. Das Nickeleisen, welches in der Bindemasse vorkommt, zeigt nach dem "atzen unter dem Mikroskop ebenso wenig Widmannst"adten'sche Figuren wie die Eisenpartikelchen der chondritischen Bruchst"ucke, beide Eisenteilehen sind aber individualisierte K"orperehen und zeigen nach den "atzen Linien wie das Braunauer Meteoreisen.

Die chemische Zusammensetzung der beiden Steinarten wurde von dem Herrn L. Sip"ocz im Laboratorium des Herrn Prof. E. Ludwig bestimmt. Derselbe erhielt f"ur die chondritischen Bruchst"ucke die Zahlen unter I und f"ur die schwarze Bindemasse jene unter II.
\begin{center}
\begin{tabular}{ l r r }
     & I. & II.\\
    Kiesels"aure & 38.01 & 36.82\\
    Tonerde & 2.22 & 2.31\\
    Eisenoxydul & 6.55 & 9.41\\
    Magnesia & 24.11 & 21.69\\
    Kalkerde & 2.33 & 2.31\\
    Natron & 1.46 & 0.96\\
    Kali & 0.31 & 0.26\\
    Schwefel & 1.94 & 2.04\\
    Eisen & 22.34 & 22.11\\
    Nickel & 2.15 & 3.04\\
     & 101.42 & 100.95\\
\end{tabular}
\end{center}
\paragraph{}
Die beiden Bestandteile, die Bindemasse und die Bruchst"ucke haben, wie sp"ater gezeigt wird, fast dieselbe chemische Zusammensetzung, fast das gleiche Volumgewicht, und so viel sich ermitteln l"asst, auch denselben mineralogischen Bestand. Demnach l"asst sich der Meteorit von Orvinio mit einer bestimmten Art tellurischer Gesteine vergleichen und zwar mit einer Breccie vulkanischen Gesteins, welche aus einer dichten Grundmasse und aus k"ornigen Tr"ummern desselben Gesteins zusammengesetzt ist. Bekanntlich sieht man derlei Breccien an Vulkanen und "uberhaupt im Bereiche der eruptiven Felsarten h"aufig. Sie bilden sich dadurch, dass "altere kristallinische Laven von einer j"ungeren dichteren durchbrochen werden.

Ich gehe nun zu einer genaueren Beschreibung der Bestandteile "uber.

Die hellen Bruchst"ucke in dem Meteoriten von Orvinio bestehen aus Chondrit. Die Chondrit sind mehr oder weniger tuf"ahnliche Massen, bestehend aus Gesteinsk"ugelchen und einer pulvrigen oder dichten gleich zusammengesetzten Grundmasse. So ist es auch hier. Ein D"unnschliff, welcher aus einen solchen Bruchst"ucke gewonnen wurde, zeigt K"ugelchen, welche meist aus einem, seltener aus mehreren Mineralen bestehen, und welche in einer aus Splittern derselben Minerale bestehenden Masse liegen, die auch dunkle Partikelchen von Nickeleisen und Magnetkies enth"alt. Fig. 3 auf Taf. I.

Unter den durchsichtigen Mineralen unterscheidet man eines, das nur unvollkommene Spaltbarkeit zeigt und in K"ornchen vorkommt, ziemlich leicht von den anderen. Nach den genannten Kennzeichen und den Daten der Analyse ist es f"ur Olivin zu halten. Das andere Mineral, welches in S"aulchen von deutlich erkennbarer Spaltbarkeit nach einem Prisma von fast quadratischem Querschnitte, ferner nach der Quer- und der L"angsfl"ache vorkommt, ist als Bronzit zu erkl"aren. Ein drittes, welches in feinbl"atterigen oder feinfaserigen Partikeln auftritt, k"onnte mit dem vorigen identisch sein, d"urfte aber, da die Analyse auf einen feldspatartigen Gemengteil hinweist, f"ur diesen zu halten sein. Eine Erscheinung, die an manchen Chondriten, z.B. Pultusk, Alessandria, Chateau Renard, auftritt, findet sich auch hier: an manchen Bruchst"ucken sind schwarze Spiegelfl"achen mit parallelen Streifen zu beobachten.

Manches, was hier bez"uglich der chondritischen Masse ferner zu sagen w"are, habe ich schon bei einer fr"uheren Gelegenheit, als ich den Meteorit von Gopalpur beschrieb,\footnote{Diese Berichte Band LXV. Abt. I. pag. 122. Die beigegebene Tafel enth"alt Abbildungen eines D"unnschliffes und verschiedener K"ugelchen.} ausgesprochen. Ich wiederhole hier nur das Eine, dass ich die Chondrit f"ur Zierreibungs-Tuffe, und die K"ugelchen derselben f"ur solche Gesteinspartikelchen halte, welche wegen ihrer Z"ahigkeit bei dem Zerreiben des Gesteines nicht in Splitter aufgel"ost, sondern abgerundet wurden.

Die beiden Massen haben demnach fast gleiche Zusammensetzung. In Betracht des Umstandes, dass beide Gemenge sind, erscheinen die Differenzen ganz unerheblich au"ser bei Magnesia und Eisenoxydul. Wenn aber das atomistische Verh"altnis des Silicium zu der Summe von Magnesium und Eisenoxydul berechnet wird, ergibt sich f"ur die erstere Analyse 1:1.096 und f"ur die zweite 1:1.098. Es zeigt sich also, dass in der schwarzen Bindemasse zwar etwas weniger Magnesia vorhanden sei, dass aber daf"ur eine "aquivalente Menge Eisenoxydul eintrete.\footnote{Es existiert auch eine Analyse von G, Bellucci mit 16.84 Proc. Tonerde und 8.97 Proc. Magnesia. Die Zahlen sind ganz unrichtig und erinnern an die Analysen Holger's an dem Stein von Wessely, f"ur welchen dieser 39 Pct. Tonsilicat, 22.66 Pct. Schwefel etc. angab. Es w"are zu w"unschen, dass derlei Zahlen nicht durch kompilatorischen Eifer verewigt w"urden.}

Aus den Daten der Analyse l"asst sich entsprechend den, an dem Meteoriten von Gopalpur gemachten Erfahrungen schlie"sen, dass in den Silikaten au"ser dem Bronzit und Olivin, f"ur welche sich wenig verschiedene prozentische Mengen berechnen, auch noch ein Gemengteil vorhanden sein m"oge, dem die Tonerde und die Alkalien zukommen, also ein Feldspat "ahnlicher Gemengteil, der bisher noch nicht mechanisch gesondert werden konnte.

Das Volumgewicht eines chondritischen Bruchst"uckes fand ich 3:675, das der schwarzen Bindemasse 3.600.

Die geringere Zahl f"ur die halbglasige Bindemasse, welche gleichwohl einen etwas gr"o"seren Eisengehalt besitzt, harmoniert mit dem Umstande, dass diese Masse das Ansehen eines halbgeschmolzenen K"orpers hat, indem die Silicate im glasigem Zustande immer ein geringeres Volumgewicht zeigen.

Das mikroskopische Bild der schwarzen Bindemasse wird nun leichter verst"andlich. Sie erscheint als ein umgeschmolzener Chondrit derselben Art wie die Bruchst"ucke. Die sehr schwer schmelzbaren Silikate Olivin und Bronzit sind, wofern sie gr"o"sere K"ornchen bildeten, erhalten geblieben, die feineren Teilchen aber und s"amtliches Eisen und aller Magnetkies sind vollst"andig umgeschmolzen. Die Schmelze besteht vorwiegend aus Eisen und Magnetkies. Ersteres hat sich beim Erstarren in gr"o"seren Partikelchen ausgeschieden, der Magnetkies hingegen blieb feiner verteilt und wurde die Hauptursache der auch im D"unnschliffe beobachteten Undurchsichtigkeit der halbglasigen Schmelze. Die letztere muss d"unnfl"ussig gewesen sein, da sie in die feinsten Kl"ufte eindringt. Darnach w"are zu schlie"sen, dass die schwarze fl"ussige Masse mindestens die Temperatur des schmelzenden Eisens besa"s, aber keine h"ohere Temperatur hatte als die des schmelzenden Bronzits oder Olivins.
\section{Chantonnay.}
\paragraph{}
"uber diesen Meteoritenfall besitzen wir "altere Nachrichten,\footnote{Chladni. Gilbert's Annalen. Bd. 60, pag. 247. Cavoleau, Journal de Physique. Bd. 85, pag. 311.} ferner eine Analyse von Berzelius,\footnote{Poggend. Ann. Bd. 33, p. 28. Zeitschrift d. deutsch. geol. Ges. Bd. 22, pag. 889.} die sich blo"s auf den Silicatbestandteil bezieht, und eine Totalanalyse von Rammelsberg.* Die merkw"urdige breccienartige Struktur des Steines wird von mehren Autoren wie Daubrée, Reichenbach, Meunier erw"ahnt. Sie gewinnt aber neuerdings Interesse, wenn sie mit jener des zuvor genannten Meteoriten verglichen wird.

Der Stein von Chantonnay, von welchem das Wiener Museum ein gro"ses und mehre kleinere Exemplare besitzt, zeigt so wie jener eine sp"arliche schwarze runzelige matte Rinde. Die Schnittfl"ache, welche durch denselben gelegt ist, zeigt chondritische Bruchst"ucke, welche eine dunkle Rinde besitzen und durch eine reichliche schwarze, zum Teil halbglasige Bindemasse zusammengef"ugt sind. Fig. 5 auf Taf. I. Durch die Masse des ganzen Steines ziehen auch hier Spr"unge, welche darauf schlie"sen lassen, dass dieselbe erhitzt worden und beim Erkalten in Folge der ungleichartigen Beschaffenheit sich ungleichf"ormig zusammengezogen habe.

Die Bruchst"ucke sind ein Chondrit, welcher nicht sehr reich an K"ugelchen ist, jedoch deren hie und da gr"o"sere enth"alt. Er zeigt "ahnlichkeit mit dem Chondrit des zuvor beschriebenen Steines von Orvinio, enth"alt aber weniger Eisen. Die Figur 6 gibt das Bild einer Partie aus einem D"unnschliff. Man kann wiederum Olivin, Bronzit, ein feinfaseriges durchscheinendes Mineral, sowie Nickeleisen und Magnetkies erkennen. Ob Chromit vorhanden sei, konnte ich nicht entscheiden. Die Unterscheidung von Bronzit und Olivin gelang mir nicht an allen hierhergeh"origen Teilchen, obgleich die Studien an dem Stein von Lodran\footnote{Diese Berichte Bd. LXI. Abt. II, pag. 465. Dieser Meteorit gestattete die mechanische Trennung, die Messung der Winkel, die mikroskopische Untersuchung und chemische Analyse der Kristalle von Olivin, Bronzit und Chromit.} vorz"ugliche Kennzeichen liefern. Man sieht jedoch auch hier die deutliche Spaltbarkeit der Bronzitk"ornchen h"aufig.

Die Rinde der Bruchst"ucke ist sehr ungleich dick. Sie ist wiederum h"arter als das Innere und zeigt bei der mikroskopischen Pr"ufung eine Impr"agnation durch eine schwarze, in die feinsten Kl"ufte eingedrungene Masse.

Zuweilen zeigen sich in den Bruchst"ucken feine schwarze Adern oder G"ange, welche mit der schwarzen Bindemasse kommunizieren; sie sind Apophysen der Bindemasse, welche eben so gut im Stande war, gr"obere Kl"ufte auszuf"ullen, als sie die feinen impr"agnierte. Ganz gleich aussehende schwarze Adern sieht man bekanntlich an ziemlich vielen Meteoriten, wie Lissa, Kakowa, Chateau Renard, Alessandria, Pultusk. Bei manchen derselben "uberzeugt man sich, dass die schwarzen Linien nichts anderes sind, als die Querschnitte von Rutschfl"achen, wie an den Steinen von Chateau Renard, Pultusk, Alessandria. Bei anderen Meteoriten wie an denen von Lissa, Kakowa hingegen haben die Adern ganz den Charakter der zuvor genannten Apophysen. Ich glaube daher, dass die letzteren Meteoriten auf ihrer urspr"unglichen Lagerst"atte mit einer hei"sfl"ussigen Masse in Ber"uhrung gekommen und in solcher Weise injiziert worden sind. Reichenbach war der Ansicht, dass die schwarzen Adern mit der Schmelzrinde in Verbindung stehen, also w"ahrend des Fluges durch «die Atmosph"are gebildet wurden.* Dem widerspricht aber der Umstand, dass nach Beobachtung und Rechnung das Innere der Meteoriten bei ihrer Ankunft auf der Erde eine sehr niedere Temperatur besitzt, welche das Eindringen einer Schmelzmasse in kapillare R"aume verhindern muss. Einen Beleg daf"ur liefert das Folgende.

Zwischen den Bruchst"ucken und der schwarzen Bindemasse des Steines von Chantonnay zeigen sich zuweilen kapillare offene Kl"ufte. Eine solche Kluft m"undet an einer Stelle an der Oberfl"ache des Meteoriten. Hier sieht man die Schmelzrindenmasse in der Tat eingedrungen, aber nur auf eine Tiefe von 6 Mm., obgleich die Kluft teilweise offen war. Die Schmelze endet in der Kluft mit einigen in die L"ange gezogenen Tropfen.

Die schwarze Bindemasse besteht aus chondritischen schwarz impr"agnierten Partikeln und aus einem undurchsichtigen spr"oden halbglasigen Magma. Die Fig. 5, welche die Ansicht eines Schnittes gibt, zeigt, dass die Partikel in der Bindemasse beinahe verschwinden, doch erkennt man sie noch an den enthaltenen gr"o"seren Eisenteilchen. Die Menge des halbglasigen Magma ist also geringer. als man beim ersten Anblick zu glauben geneigt ist.

Eine Fluidaltextur zeigt sich dem freien Auge nicht, doch erkennt man eine solche Textur, welche auch hier von der Anordnung der Eisenp"unktchen in dem Magma herr"uhrt, mit der Lupe an mehren Stellen, wo sich die Bruchst"ucke und das Magma ber"uhren. Dass diese Textur hier weniger auffallend ausgesprochen ist, m"ochte wohl dem geringeren Gehalt von Nickeleisen zuzuschreiben sein, da er blo"s 79 Pct. betr"agt, w"ahrend er sich in dem Stein von Orvinio auf 25 Pct. bel"auft.

Das schwarze halbglasige Magma besteht aus einer vollst"andig undurchsichtigen Masse, worin Splitter der auch in den Bruchst"ucken enthaltenen Silicate, zuweilen auch einzelne K"ugelchen liegen. Im auffallenden Lichte sieht man feine P"unktchen von Nickeleisen und Magnetkies. Wo die Fluidaltextur erkannt wurde, sind diese P"unktchen perlschnurartig angeordnet. Man sieht auch sehr feine Adern der letztgenannten Minerale, welche zugleich mit der impr"agnierenden Masse in die chondritischen Partikel und Bruchst"ucke eindringen.

Die Menge des eigentlichen schwarzen Magma ist gering, denn die Hauptmasse alles dessen, was schwarz erscheint, ist nur impr"agnierter Chondrit.

Eine gesonderte chemische Untersuchung der Bruchst"ucke und der Bindemasse ist bisher noch nicht ausgef"uhrt worden. Berzelius gab blo"s die Analyse der Silicate der schwarzen Bindemasse, ohne die Menge des Eisens und des Magnetkieses zu bestimmen. Rammelsberg f"uhrt nicht an, welcher Art sein Material gewesen, wahrscheinlich waren beide Teile des Steines darin vertreten. Nach den Erfahrungen an dem Stein von Orvinio d"urfte auch hier die Zusammensetzung der Bindemasse von der der Bruchst"ucke nur unbedeutend differieren. Ich vergleiche nun hier die von Rammelsberg erhaltenen Zahlen mit den fr"uher angef"uhrten in der Weise, dass in der ersteren Analyse die Daten f"ur den in S"aure auf l"oslichen und den unaufl"oslichen Teil vereinigt werden.
\begin{center}
\begin{tabular}{ l r r r }
    Kiesels"aure & 37.38 & 38.01 & 36.82\\
    Tonerde & 2.53 & 2.22 & 2.31\\
    Eisenoxydul & 14.67 & 6.55 & 9.41\\
    Manganoxydul & 0.27 & -,- & -,-\\
    Magnesia & 25.37 & 24.11 & 21.69\\
    Kalkerde & 1.41 & 2.33 & 2.31\\
    Natron & 1.14 & 1.46 & 0.96\\
    Kali & 1.14 & 0.31 & 0.26\\
    Chromoxyd & 0.60 & -,- & -,-\\
    Eisenoxydul & 0.37 & -,- & -,-\\
    Schwefel & 2.24 & 1.94 & 2.04\\
    Eisen & 10.65 & 22.34 & 22.11\\
    Nickel & 1.16 & 2.15 & 3.04\\
     & 97.79 & 101.42 & 100.95\\
\end{tabular}
\end{center}
\paragraph{}
Der Unterschied ist gr"o"stenteils gering, nur im Eisengehalte differieren die beiden Meteoriten erheblicher. Rechnet man alles Eisen als metallisches Eisen, so geben die drei Analysen die Zahlen 22.63, 27.43 und 29.43.

Die Erscheinungen an den Meteoriten von Orvinio und von Chantonnay f"uhren zu dem Schlusse, dass diese Massen urspr"unglich nicht die gegenw"artige Beschaffenheit hatten, sondern dass sie durch Zertr"ummerung fester Gesteine und nachherige Zusammenf"ugung derselben mittelst eines halbglasigen Magma, in ihren gegenw"artigen Zustand gelangt sei. Ich habe daf"ur gleich eingangs eine Parallele mit den eruptiven Breccien unserer Erde gezogen, doch k"onnte es nunmehr scheinen als ob dieser Vergleich nicht vollkommen zutreffe. Die schwarze Bindemasse ist n"amlich nicht so homogen wie eine verkittende Lava, sondern enth"alt viele Gesteinsplitter in der halbglasigen Grundmasse. Dieser Umstand h"angt aber mit der "au"serst schwierigen Schmelzbarkeit der Silicate zusammen, welche die Hauptmasse Jener Meteoriten bilden. Wir besitzen auf unserer Erde keine Olivinfels- oder Bronzitfelslaven, daher werden wir auch etwas der schwarzen Rindemasse v"ollig Gleiches unter unseren vulkanischen Produkten nicht auffinden.

Wollte man aber trotzdem jene meteorischen Tr"ummergesteine mit anderen, nicht vulkanischen Bildungen unserer Erde vergleichen, so k"onnte man sie vielleicht mit den Dislokations-Breccien in eine Linie stellen, d. h. mit jenen Breccien, welche durch eine Zertr"ummerung und eine an derselben Stelle erfolgte Verkittung der Gesteintr"ummer durch den Absatz einer w"asserigen L"osung gebildet wurden. Man k"onnte sie vielleicht auch mit den im Durchschnitte marmoriert aussehenden Kalksteinen etc. vergleichen, deren Aderung durch w"asserige Einfl"usse entstanden ist. In der Tat besitzt der Stein von Chantonnay eine feine Textur, die einigerma"sen einer solchen metamorphischen Breccie entspricht.

Es k"onnte also scheinen, dass man sich die schwarze verkittende Masse der Meteoriten - Breccie auch durch allm"alig und bei m"a"siger Temperatur wirkende Ursachen gebildet vorstellen k"onnte. Dem ist aber entgegenzuhalten, dass die Spr"unge und Kl"ufte in dem ganzen Steine, der halbglasige Zustand der Bindemasse, der Eisen- und Magnetkies-Teilchen, die Fluidaltextur durchwegs auf die Mitwirkung einer hohen Temperatur hinweisen, ferner dass eine allm"alige Entstehung durch die vorliegenden Beobachtungen wohl nicht g"anzlich ausgeschlossen, aber doch nicht wahrscheinlich gemacht sei, weil in diesem Falle eine kristallinische Ausbildung des schwarzen Magma zu erwarten w"are.

Man mag "ubrigens den Tatsachen diese oder jene Auslegung geben, in jedem Falle ist durch dieselben bewiesen, dass die beiden Steine fr"uher Zeugen von Vorg"angen waren, die nur auf einem solchen Himmelsk"orper m"oglich sind, welcher an der Oberfl"ache und im Inneren verschiedene Zust"ande aufweist. Die beiden Steine geben uns also Nachricht von Ver"anderungen auf der starren Oberfl"ache eines oder mehrerer Planeten, welche sp"ater in Tr"ummer aufgel"ost wurden.
\clearpage
\section{Erkl"arung der Tafeln.}
\paragraph{}
Tafel 1: Fig. 1 --- Ansicht eines Meteorsteines von Orvinio in 1/2 der nat"url. Gr"o"se (linear). Links Brustseite, rechts R"uckenseite. Die punktierte Linie gibt die Richtung des durch den Meteoriten gef"uhrten Schnittes an.

Tafel 1: Fig. 2 --- Ansicht des Schnittes in nat"urlicher Gr"o"se im auffallenden Lichte. Die Tr"ummerstruktur, die Spr"unge sind deutlich. Ein Bruchst"uck links zeigt den Unterschied der helleren F"arbung im Inneren und der dunklen F"arbung gegen die Rinde zu, die kleineren Bruchst"ucke sind durchaus dunkel. Die dunkle Bindemasse zeigt eine Fluidaltextur, welche von h"ochst feinen Eisenflasern herr"uhrt und eine unnachahmliche Zartheit der Zeichnung besitzt.

Tafel 1: Fig. 3 --- Partie eines D"unnschliffes aus einem Bruchst"uck in dem Meteoriten von Orvinio. Durchfallendes Licht. Vergr"o"serung 20fach. Die dunklen Partikel sind Eisen und Magnetkies, letzterer ist feiner verteilt. Die Unterscheidung beider erfolgt nat"urlich nur im auffallenden Lichte.

Tafel 1: Fig. 4 --- Teil eines D"unnschliffes durch ein Bruchst"uck und die angrenzende Bindemasse. Vergr"o"serung 20. Das chondritische Bruchst"uck erscheint hier im Kontakte mit der Bindemasse von einem schwarzen Magma durchdrungen. An der Grenze beider endigt ein Sprung. Die Bindemasse ist von feinen Eisenadern durchzogen. Diese sind durch ein helleres Grau bezeichnet.

Tafel 1: Fig. 5 --- Teile eines D"unnschliffes durch die Bindemasse. Vergr"o"serung 20. Ein Teil der Bindemasse ist reich an chondritischen Splittern und rundlichen Eisenpartikeln, die andere ist dicht, die Eisenteilchen sind sehr klein.

Tafel 2: Fig. 6 --- Ansicht eines polierten Durchschnittes durch den Meteorstein von Chantonnay in nat"urlicher Gr"o"se. Auffallendes Licht. Die Tr"ummerstruktur wird hervorgebracht durch viele Bruchst"ucke, die von einem schwarzen Magma umh"ullt sind. Die Bruchst"ucke sind durch gr"o"sere Eisenpartikel kenntlich. Das Magma zeigt keine erkennbaren Eisenteilchen. Die drei gr"o"seren Bruchst"ucke sind im Inneren lichter gef"arbt. Sie zeigen eine an verschiedenen Stellen ungleich dicke dunkle Rinde. Die vielen kleinen Bruchst"ucke sind ganz und gar schwarz impr"agniert und heben sich nur durch die geringere Politur und die Eisenpartikel von dem umgebenden Magma ab. In der Masse des Steines sind unregelm"a"sige offene Spr"unge bemerkbar.

Tafel 2: Fig. 7 --- Eine Partie eines D"unnschliffes durch das helle Innere eines gro"sen Bruchst"uckes. Durchfallendes Licht. Vergr"o"serung 15. Die dunklen Partikel sind Eisen und Magnetkies.

Tafel 2: Fig. 8 --- Teil eines D"unnschliffes durch zwei impr"agnierte kleine Bruchst"ucke und die zwischenliegende schwarze Masse. Die letztere enth"alt chondritische Splitter. Vergr"o"serung 15.
\clearpage
\setlength\intextsep{0pt}
\pagestyle{fancy}
\fancyhf{}
\rhead{Tafel 1.}
\cfoot{\thepage}
\begin{figure}[p]
\includegraphics[scale=0.4,keepaspectratio]{Figures/Table1/table1-fig1.png}\tiny 1
\includegraphics[scale=0.5,keepaspectratio]{Figures/Table1/table1-fig2.png}\tiny 2
\includegraphics[scale=0.8,keepaspectratio]{Figures/Table1/table1-fig3.png}\tiny 3
\includegraphics[scale=0.6,keepaspectratio]{Figures/Table1/table1-fig4.png}\tiny 4
\includegraphics[scale=0.8,keepaspectratio]{Figures/Table1/table1-fig5.png}\tiny 5
\end{figure}
\clearpage
\rhead{Tafel 2.}
\cfoot{\thepage}
\begin{figure}[p]
\includegraphics[scale=0.2,keepaspectratio]{Figures/Table2/table2-fig6.png}\tiny 6
\includegraphics[scale=0.3,keepaspectratio]{Figures/Table2/table2-fig7.png}\tiny 7
\includegraphics[scale=0.3,keepaspectratio]{Figures/Table2/table2-fig8.png}\tiny 8
\end{figure}
\clearpage
\end{document}
