\documentclass[a4paper, 11pt, oneside]{article}
\usepackage[utf8]{inputenc}
\usepackage[T1]{fontenc}
\usepackage[ngerman]{babel}
\usepackage{yfonts}
%\usepackage{fbb} %Derived from Cardo, provides a Bembo-like font family in otf and pfb format plus LaTeX font support files
\usepackage{booktabs}
\setlength{\emergencystretch}{15pt}
\usepackage{fancyhdr}
\usepackage{graphicx}
\usepackage{microtype}
\graphicspath{ {./} }
\begin{document}
\frakfamily
\begin{titlepage} % Suppresses headers and footers on the title page
	\centering % Centre everything on the title page
	%\scshape % Use small caps for all text on the title page

	%------------------------------------------------
	%	Title
	%------------------------------------------------
	
	\rule{\textwidth}{1.6pt}\vspace*{-\baselineskip}\vspace*{2pt} % Thick horizontal rule
	\rule{\textwidth}{0.4pt} % Thin horizontal rule
	
	\vspace{1.5\baselineskip} % Whitespace above the title
	
	{\scshape\LARGE Der Meteoritenfall}
	
	\vspace{1\baselineskip} % Whitespace after the title block

	{\scshape\LARGE bei Tieschitz in Mähren.}

	\vspace{1.5\baselineskip} % Whitespace above the title

	\rule{\textwidth}{0.4pt}\vspace*{-\baselineskip}\vspace{3.2pt} % Thin horizontal rule
	\rule{\textwidth}{1.6pt} % Thick horizontal rule
	
	\vspace{1\baselineskip} % Whitespace after the title block
	
	%------------------------------------------------
	%	Subtitle
	%------------------------------------------------
	
	{\scshape Von dem w. M. G. Tschermak.} % Subtitle or further description
	
	\vspace*{1\baselineskip} % Whitespace under the subtitle
	
	%------------------------------------------------
	%	Editor(s)
	%------------------------------------------------
    \vspace*{\fill}

	\vspace{1\baselineskip}

	{\small\scshape 1875.}
	
	\vspace{0.5\baselineskip} % Whitespace after the title block

    \scshape Internet Archive Online Edition  % Publication year
	
	{\scshape\small Namensnennung Nicht-kommerziell Weitergabe unter gleichen Bedingungen 4.0 International} % Publisher
\end{titlepage}
\setlength{\parskip}{1mm plus1mm minus1mm}
\clearpage
\tableofcontents
\clearpage
\section{\frakfamily{Erster Bericht.}}
\paragraph{}
In der Sitzung am 10. Juli l. J. legte mir Herr Direktor J. Hann die Nachrieht über einen Meteoritenfall vor, welche Tags zuvor an die k. k. Zentralanstalt für Meteorologie in Wien gelangt war. Diese Nachricht bestand aus zwei Telegrammen des Herrn Postmeisters Franz Tillich in Nezamislitz, welcher am 17. und am 18. Juli an die Telegraphen-Hauptstation in Brünn dasjenige berichtete, was er über das im benachbarten Dorfe Tieschitz stattgefundene Ereignis in Erfahrung gebracht hatte. Diese Telegramme waren die Ursache, dass in Brünn sowohl als auch in Wien der Meteoritenfall rasch bekannt wurde.

Als ich am 20. Juli am Orte anlangte, erfuhr ich durch den eben gegenwärtigen Bezirkshauptmann aus Prerau, Herrn Marschowsky, ferner durch Herrn Postmeister Tillich, Herrn Ökonomieverwalter Strohschneider, Herrn Stationschef Krejci und den Herrn Müllermeister von Nezamislitz die näheren Umstände. Am nächsten Tage geleiteten mich die letztgenannten Herren an den Fallort, wo mir die Augenzeugen vorgeführt wurden, welche ich um die Einzelheiten selbst befragen konnte.

Der niedergefallene Meteorstein war aber eben vor meiner Ankunft an den mittlerweile herbeigeeilten Herrn Professor Makowsky aus Brünn zur Aufbewahrung übergeben worden, um fernere Beschädigungen des Objektes hintanzuhalten.

Das Dörfchen Tieschitz (in slavischer Schreibweise Tesic) liegt von Brünn in der Richtung Ost-Nordost 5 1/2 Meilen entfernt. Nach dem benachbarten Dorfe Nezamislitz ist die Gabelungsstation der Mährisch-schlesischen Nordbahn benannt, welche letztere Brünn einerseits mit Prerau, anderseits mit Olmütz und Sternberg verbindet.

Am 15. Juli Nachmittags war der Himmel zum Teil von Wolken bedeckt, als um 2 Uhr Nachmittags einige wenige Leute, die bei Tieschitz auf dem Felde arbeiteten, durch ein heftiges Getöse auf eine ungewöhnliche Erscheinung aufmerksam wurden, während Andere, welche den Lärm hörten, der Sache keine Aufmerksamkeit schenkten, weil sie gewohnt waren, von dem benachbarten Bahnhofe her öfters Lärm und Getöse zu vernehmen. Daher wurde auch auf dem Bahnhofe selbst nichts von dem Vorfalle beobachtet.

Die Bauern, welche südlich von Tieschitz auf dem Acker beschäftigt waren, hörten ein so starkes Getöse, dass sie dadurch erschreckt wurden. Dieselben vergleichen es mit dem Rollen eines schweren Lastwagens auf steiniger Chausée, doch war der Schall bedeutend stärker, als ihn ein solches Rollen hervorbringt.

Einer der Beobachter gab an, dass er nach dem Rollen auch ein sehr starkes Zischen wahrgenommen. Merkwürdigerweise fehlt jede Angabe über einen intensiven Knall, wie er sonst beim Niederfallen von Meteoriten häufig beobachtet wird, und der zuweilen so stark ist, dass die Leute in der Umgebung die Besinnung zu verlieren glauben. Als die Leute emporsahen, glaubte einer davon, ein graues Wölkchen wahrzunehmen, von dem der Lärm ausging, aber kaum blickten sie Alle zum Himmel, als etwas mit einem dumpfen Schlage auf den frisch gepflügten Acker vor den Augen der Leute und in geringer Entfernung vor ihnen niederfiel. Der Lärm hörte auf, sobald der Meteorit niedergefallen war.

Über die Richtung desselben im Azimut erhielt ich von den Leuten, welche im Augenblicke der Erscheinung sehr beunruhigt waren, keine übereinstimmenden Angaben. Nach den einen hätte sich der Meteorit in westlicher Richtung bewegt, doch sah ihn der Beobachter erst kurz vor der Berührung mit dem Boden, nach der anderen Angabe wäre die Richtung eine östliche gewesen. Brauchbare Angaben sind von anderen Beobachtern, die vom Fallorte entfernter waren, zu erwarten.

Die Zeit des Falles ergab sich aus den Angaben jener Landleute mit Bezug auf das Eintreffen eines Bahnzuges in der Station Nezamislitz. Darnach wäre dieselbe etwas vor 2 Uhr Lokalzeit. Als die Leute sahen, wie der schwarze Klumpen in den Boden einschlug und Staub aufwirbelte, fürchteten sie sich näher zu treten, bis ein Weib aus der Gesellschaft Muth fasste und Lei genauerer Besichtigung fand, dass es nur ein Stein sei, was mit so gewaltigem: Rollen ein hergezogen war. Die Männer, welche nun eine Bombe vermuteten, wagten es jedoch nicht näher zu kommen. Das Weib holte daher einen Bewohner des Dorfes herbei, damit er den Stein ausgrabe. Im Beisein aller Beobachter wurde nun der Stein gehoben und noch warn befunden. Die Leute merkten nicht darauf, in welcher Weise der Stein im Boden schlug war. Aus der Stellung der Punkte, welche beim Ausgraben verletzt wurden, schloss ich später bei der Besichtigung des Steines, dass derselbe auf die Brustseite gefallen war. Das Loch, welches der Stein in den frischgepflügten Boden schlug, war bloß einen halben Meter tief. Der Punkt, wo er niederfiel, liegt südlich vom Dorfe, 500 Schritte von letzteren entfernt. Der Stein wurde vom Demjenigen, welcher ihn ausgegraben hatte, ins Dorf gebracht und bei dem Gemeindewirtshause aufbewahrt. Leider wurden Stücke davon abgeschlagen und zerteilt. Die Partikel sind in der Umgegend verschleppt, später aber zum Teil von Herren Dr. Brezina für das Hof-Mineralienkabinet eingesammelt worden.

Als sich die Nachricht von dem Ereignis verbreitete, ließ der Pfarrer von Nezamislitz den Stein in die Ortscapelle bringen und daselbst zur Schau ausstellen. Bald wurden Reklamationen bezüglich des Eigentumsrechtes laut und man rief den Prerauer Bezirkshauptmann herbei, welcher, wie schonerwähnt, das Objekt an Herrn Professor Makowsky zur Aufbewahrung im Museum der technischen Hochschule in Brünn übergab.

In Brünn konnte ich durch die Freundlichkeit des Herrn Makowsky den Stein besichtigen. Derselbe ist zum allergrößten Teile von einer schwarzen Rinde bedeckt, welche durch feine radiale Erhabenheiten die Brustseite und durch den reicheren Schmelz und runzelige Oberfläche die Rückenseite deutlich erkennen lässt. Von dieser und von jener Seite gesehen, hat der Stein einen ungefähr dreiseitigen Umriss. Er besitzt nämlich beiläufig die Form einer schiefen vierseitigen Pyramide, deren größte Flächen die Brust- und die Rückenseite sind. Er ist in dieser Beziehung ähnlich dem Stein von Ohaba und dem größten Stein von Tabor. Die Höhe der Pyramide beträgt 30 Cm., die Breite 26 Cm. Das Gewicht war ursprünglich 28 Kilogramm, die Verletzungen haben dasselbe um etwas vermindert.

Die Oberfläche des Steines zeigt namentlich auf den Randflächen häufig die charakteristischen Gruben, welche wie Fingereindrücke aussehen, die Brustseite hat keine solchen Gruben.

Das Innere (es Meteorsteines ist aschgrau, im Bruche matt und uneben durch viele kleine Kügelchen und auch durch Splitter. Diese und jene zeigen eine tiefgraue bis weiße Farbe. Die Grundmasse hat einen erdigen Bruch, enthält außer dein Steinpulver auch zweierlei metallisch aussehende Körnchen. Die Kügelchen zeigen im durchfallenden Lichte die für Bronzit und für Olivin charakteristischen Texturen, die weißlichen Kügelchen und Splitter sind auf den eisenarmen Bronzit (Ensatit) zu beziehen. Diese Minerale haben viele Einschlüsse, sowohl solche von glasiger Beschaffenheit, als auch solche von metallischen Aussehen. Die Grundmasse besteht aus denselben Mineralien im Zustande feiner Zerteilung, ferner aus Partikeln von gediegenem Eisen und von Magnetkies. Demnach gehört dieser Meteorstein zu den Chondriten, und zwar zu denjenigen, welche in der von mir gegebene Einteilung\footnote{\frakfamily{Mineralog. Mitt. 1872, pag. 165.}} durch viele braune feinfaserige Kügelchen charakterisiert sind.

Nach einer Verabredung, weiche ich mit Herrn Professor Makowsky getroffen, soll der ausführliche Bericht über den Meteoritenfall von Tieschitz und die genaue Beschreibung des Steines von uns Beiden gemeinschaftlich in den Schriften der k. Akademie veröffentlicht werden. Die Beschreibung wird von mehreren Tafeln begleitet sein, für welche die photographischen Aufnahmen bereits in Brünn ausgeführt wurden. Die chemische Analyse hat gütigst Herr Professor Habermann übernommen, während Herr Professor v. Niessl sich der Mühe unterzog, die eingelaufenen Berichte über die an verschiedenen Punkten gehörte Detonation zusammenzustellen und zu einer beiläufigen Bahnbestimmung zu verwerten.
\section{\frakfamily{Zweiter Bericht.}}
\paragraph{}
Über diesen Meteoritenfall, welcher am 15. Juli 1878 stattfand, sind seit der Vorlage des ersten Berichtes noch fernere Nachrichten eingelaufen. Durch die Bemühungen der Herren Prot. A. Makowsky und Prof. G. v. Niessl in Brünn wurden die Aussagen vieler Zeugen gesammelt, welche die Detonation des Meteors gehört hatten. Herr Prof. v. Niessl musste sich mit den Angaben über die Schallwahrnehmung begnügen, da der Meteoritenfall am hellen Tage eintrat. Dennoch vermochte er aus diesen Daten eine beiläufige Bahnbestimmung in Bezug auf die Erdoberfläche auszuführen, wonach die Richtung eine östliche war und die Bahnlage durch Azimut 108° Höhe 40° bestimmt erscheint. Für die Zeit des Falles wurde 2 Uhr 45 Minuten Nachmittags als annähernde Bestimmung erhalten, wonach sich auch die siderische Bahn beiläufig erschließen lässt.

Der Meteorit wurde mittlerweile bezüglich der äußeren Form von Herrn Prof. Makowsky und bezüglich der chemischen Zusammensetzung von Herrn Prof. J. Habermann in Brünn untersucht. Es wurde schon im ersten Berichte erwähnt, dass der Stein ungefähr die Form einer schiefen vierseitigen Pyramide besitze. Die Oberfläche ist von einer schwarzen matten Schmelzrinde bedeckt, welche jene Anordnung feiner Runzeln darbietet, aus der die Orientierung des Steines gegen die Richtung seines Fluges durch die Atmosphäre bestimmt werden kann. Außerdem bemerkt man häufig kleine Erhabenheiten, welche von Kügelchen herrühren, die schwieriger schmelzbar sind als ihre Umgebung und daher langsamer als diese aufgezehrt wurden. Stellenweise hat die Rinde kleine rauhe Unterbrechungen welche darauf deuten, dass während des Fluges durch die Luft kleine Splitter abgesprungen seien.

Die chemische Zusammensetzung entspricht vollkommen der eines Chondrits. Die Analyse ergab:
\begin{center}
\begin{tabular}{ l r }
    Kieselsäure & 40.23\\
    Tonerde & 1.93\\
    Eisenoxydul & 19.48\\
    Manganoxydul & 0.32\\
    Magnesia & 20.55\\
    Kalkerde & 1.54\\
    Natron & 1.53\\
    Phosphorsäure & 0.22\\
    Schwefel & 1.65\\
    Eisen & 10.26\\
    Nickel & 1.31\\
     & 99.02\\
\end{tabular}
\end{center}
\paragraph{}
Das Volumgewicht des Steines ist 3.59.

Die Untersuchung der Textur und mineralogischen Beschaffenheit wurde von mir ausgeführt, wobei sich mehrere wichtige Tatsachen ergaben.

Der Stein gehört, wie schon früher bemerkt wurde, zu den Chondriten mit vielen braunen, harten, feinfaserigen Kügelchen. Bisher hatte ich in Meteoriten immer nur solche Kügelchen gefunden, welche kugelrund oder länglichrund waren und eine glatte oder rauhe Oberfläche darboten, die keine Unterbrechung der gleichförmigen Krümmung erkennen ließ. Da ich ferner in Übereinstimmung mit G. Rose den Mangel einer konzentrischen Anordnung als für die Kügelchen der Chondrit charakteristisch Übereinstimmung mit G. Rose den Mangel einer konzentrischen Anordnung als für die Kügelchen der Chondrit charakteristisch erkannte, so leitete mich die Form und Textur der Kügelchen zu der Vorstellung, dass die Kügelchen durch die bei vulkanischen Vorgängen eintretende Zerreibung zu dieser Gestalt gelangt seien.\footnote{\frakfamily{Diese Berichte, Bd. LXXI. 2. Abt. April 1875.}}

In dem Tieschitzer Stein finden sich aber Kügelchen mit runden Eindrücken, welche darauf hinweisen, dass manche Kügelchen plastisch und andere zu gleicher Zeit starr gewesen seien.

Ferner kommen an denselben Kügelchen kleine Auswüchse vor, welche die Rundung der Oberfläche unterbrechen. Im Inneren zeigt sich endlich bei manchen eine konzentrische Anordnung.

Diese Tatsachen veranlassen mich, die früher ausgesprochene Ansicht aufzugeben, da die beobachteten Erscheinungen derselben mit großer Bestimmtheit widersprechen. Obwohl ich nun die Bildung der tuffartigen Meteoriten jetzt ebenso wie früher auf einen vulkanischen Vorgang zurückführe, so glaube ich doch die Form der Kügelchen nicht mehr von einer Zerreibung fester Gesteinsmassen ableiten zu sollen, vielmehr möchte ich es für wahrscheinlicher halten, dass die Kügelchen erstarrte Tropfen seien, dass also bei den vorausgesetzten vulkanischen Vorgängen eine dünnflüssige Schmelze in Tropfen zerstäubt wurde, welche nach ihrer raschen Erstarrung die Hauptmasse des Tuffes bildeten, der nun als chondritisches Gemenge vorliegt.

Der untersuchte Stein enthält auch ungemein viele zerbrochene Kügelchen, was in anderen Meteoriten seltener zu beobachten ist, ferner zeigt derselbe in der Textur und Farbe der Gemengteile eine ungewöhnliche Mannigfaltigkeit, so dass der Stein In mehrfacher Beziehung als ein merkwürdiger zu bezeichnen ist.

Die mikroskopische Untersuchung lässt als Mineralgemengteile vor allem Olivin erkennen, der öfter nette Krystalle bildet, außerdem Bronzit samt den Übergängen zum Enstatit überdies Augit, Magnetkies und Nickeleisen.

Ob ein feldspatähnliches Mineral vorhanden sei, konnte nicht mit Sicherheit bestimmt werden, weil es möglich ist, dass in den weißen dichten Kügelchen und Splittern, welche hie und da auftreten, aber der mikroskopischen Prüfung unzugänglich sind, eine geringe Menge von einem solchen Mineral vorkommt. Im Olivin und Bronzit sind Einschlüsse von braunem Glas mit fixen Libellen häufig.

Der ausführliche Bericht über den Meteoritenfall und die Untersuchung des Steines wird von mir und Herrn Professor Makowsky erstattet werden und in den Denkschriften der Akademie zum Abdrucke gelangen.
\clearpage
\end{document}
